% Cursus Onderzoekstechnieken
%
% Genereer PDF-versie met volgende procedure:
% 
% 1) latexmk -pdf "cursus-onderzoekstechnieken"
% 2) biber "cursus-onderzoekstechnieken"
% 3) latexmk -pdf "cursus-onderzoekstechnieken"
%
\documentclass[11pt,fleqn,a4paper]{book}

\input{structure}

\author{Dr. Jens Buysse, Anita Bernard, Bert Van Vreckem}
\title{Cursus Onderzoekstechnieken}
\date{Academiejaar 2016-2017}

\begin{document}

\thetitlepage

%----------------------------------------------------------------------------------------
%	COPYRIGHT PAGE
%----------------------------------------------------------------------------------------

\newpage
~\vfill
\thispagestyle{empty}

\noindent Copyright \copyright\ 2015-2017 Jens Buysse\\ % Copyright notice

\noindent \textsc{www.hogent.be}\\ % URL

\noindent \textit{Gegenereerd op \today} % Printing/edition date

%----------------------------------------------------------------------------------------
%	TABLE OF CONTENTS
%----------------------------------------------------------------------------------------

\usechapterimagefalse

\tableofcontents % Print the table of contents itself

\cleardoublepage % Forces the first chapter to start on an odd page so it's on the right

\setlength{\parindent}{0pt}

\def\R{\mathbb{R}}

\includecomment{solution}
%\excludecomment{solution}



\chapter*{Voorwoord}
Deze cursus werd geschreven in het kader van de lessenreeks Onderzoekstechnieken aan de Hogeschool Gent. Ik wil hierbij gebruik maken om volgende mensen te bedanken bij het nakijken en verbeteren van de cursus.
\begin{itemize}
	\item C\'edric Berlez
	\item J\"urgen Van Meerhaeghe
	\item Gianni Stubbe
	\item Jelle Elaut
	\item Thijs Van Der Burgt
	\item Lotte Potth\'e
	\item \"Ozg\"ur Akin
	\item Cedric Devylder
\end{itemize}

\bigskip \bigskip
{\raggedleft	% Lijn rechts uit
Jens Buysse\\
08 februari 2016\\
}

\chapter{Het onderzoeksproces}
\section{De wetenschappelijke methode}
Er zijn verschillende manieren om kennis te vergaren:

\begin{enumerate}
	\item Wetenschappelijke methode
	\item De niet-wetenschappelijke methode
\end{enumerate}

\paragraph{Niet wetenschappelijk}Er zijn verschillende versies van niet wetenschappelijk redeneren: 
\begin{description}
	\item [Autoritair] hier geldt iemand als autoriteit in een bepaald gebied en wordt als betrouwbaar bestempeld. Alles wat deze persoon beweert wordt aanzien als waarheid. 
	\item [Deductief] gegeven een set van veronderstellingen gaat men op een welbepaalde manier conclusies trekken. Alhoewel hier dus correcte conclusies kunnen behaald worden, hangt dit enkel en alleen af van de waarheid van de veronderstellingen. Maar deze veronderstellingen worden niet empirisch onderzocht.
\end{description}

\paragraph{Wetenschappelijk}
Een kenmerk van de \textsl{wetenschappelijke methode} is \textbf{Empirische validering}: gebaseerd op ervaring en directe observatie. Dus een uitspraak is geldig indien het overeen komt met wat geobserveerd wordt.


\begin{exercise}
Probeer nu vertrekkende van de niet-wetenschappelijke en wetenschappelijke manieren aan te tonen dat varkens kunnen vliegen. 
\end{exercise}


Aan de hand van zo'n empirisch onderzoek kunnen we verschillende doelen behalen:
\begin{enumerate}
	\item Exploratie: bestaat iets of gebeurt er iets?
	\item Beschrijving: wat zijn de eigenschappen van deze gebeurtenis
	\item Voorspelling: is een bepaalde gebeurtenis gerelateerd aan een andere en kan ik deze zo voorspellen?
	\item Controle: kan ik een gebeurtenis volledig voorspellen aan de hand van andere zaken?
\end{enumerate}

\paragraph{Onderzoeksdoelstellingen}

Er zijn twee grote onderzoeksdoelen die we willen behalen:

\begin{description}
	\item [Generalisatie] we gaan vaak maar een onderzoek doen op een bepaalde, beperkte groep van de totale groep (populatie). Indien we correcte conclusies kunnen trekken voor die subgroep, die ook gelden voor de totale groep dan hebben we een correcte generalisatie gevonden.
    \item[Specialisatie] Toepassen van algemene kennis op een specifiek domein of probleem. Toegepast onderzoek kan hier meestal onder geclassificeerd worden.
\end{description}

Er zijn twee soorten generalisaties
\begin{enumerate}
	\item Over 1 enkel fenomeen.
	\item Over verbanden tussen fenomenen.
\end{enumerate}
Er zijn drie redenen waarom verbanden zo belangrijk zijn:
\begin{enumerate}
	\item Volledig verstaan van een fenomeen. 
	\item Verbanden kunnen zorgen voor een voorspelling
	\item Causale verbanden: een van de fenomenen heeft dat andere fenomeen tot gevolg. 
\end{enumerate}

	
\section{Basisconcepten in onderzoek}
\paragraph{Meetniveaus}
In statistiek werken we met variabelen en waarden.

\begin{definition}[Variabele] 
    Algemene eigenschap van een object waardoor we objecten van elkaar kunnen onderscheiden. Vb.~lengte, gewicht, \ldots
\end{definition}  
\begin{definition}[Waarde]
    Specifieke eigenschap, invulling voor die variabele. Vb.~1.83m, 78 kg, \ldots
\end{definition}

Er worden meestal vier meetniveaus gebruikt in statische analyse. Het meetniveau bepaalt welke statische methodes bruikbaar zijn. 
\begin{description}
	\item [Nominaal meetniveau] \index{Nominaal}: er is slechts keuze uit een beperkt aantal categorie\"en, waarbij geen volgorde aanwezig is tussen de antwoorden.
	\item [Ordinaal meetniveau] \index{Ordinaal}: een variabele die is ingedeeld in categorie\"en, waar er echter wel een logische volgorde is tussen de categori\"en. 
	\item [Intervalniveau] \index{Intervalniveau}: variabelen die niet in categorie\"en voorkomen, en waarbij berekeningen kunnen mee uitgevoerd worden, maar zonder nulpunt.
	\item [Rationiveau] \index{Rationiveau}: intervalniveau met nulpunt. Je kunt hierdoor verhoudingen berekenen tussen verschillende waarden op de schaal.
\end{description}

\begin{exercise}
	Zoek zelf nu eens voorbeelden voor de verschillende meetniveaus.
\end{exercise}

\paragraph{Onderzoeksproces}
Het onderzoeksproces kan grotendeels opgedeeld worden in 6 grote delen:
\begin{enumerate}
	\item Formuleren van de probleemstelling: wat is de onderzoeksvraag
	\item Exacte informatiebehoefte defini\"eren: welke specifieke vragen moeten we stellen
	\item Uitvoeren van het onderzoek: enqu\^etes, simulaties, \dots
	\item Verwerken van de gegevens: statistische software
	\item Analyseren van de gegevens: uitvoeren van de statistische methodes
	\item Conclusies schrijven: schrijven van onderzoeksverslag
\end{enumerate}

\begin{definition}[Oorzakelijk verband]
\index{Oorzakelijk verband} Een variabele veroorzaakt een oorzakelijk verband wanneer een verandering in die variabele op een betrouwbare manier een geassocieerde verandering van een andere variabele tot gevolg heeft, op voorwaarde dat alle andere potenti\"ele oorzaken ge\"elimineerd zijn.
\end{definition}

Er is niet altijd verband zichtbaar en we moeten soms verder kijken dan naar de absolute waarden van de variabelen alvorens conclusies te trekken.

\begin{example}
	Bij het voobeeld van Pepsi versus cola zou je initeel kunnen denken dat Pepsi lekkerder is omdat er meer mensen ervan geproefd hebben (70 ten opzichte van 30). Maar dit zou een verkeerde manier van redeneren zijn. We moeten relatief ten op zichte van de  \textit{marginale} totalen kijken. Hier zien we dan dat 56 van de 70 ($\frac{56}{70} = 0.8$) mensen die Pepsi gedronken hebben het lekker vonden en 24 van de 30 ($\frac{24}{30}$=0.8) mensen vonden cola lekker. Dus is er geen verschil in waarden voor Cola en Pepsi voor de gemiddelde waarde van smaak. 
\end{example}

\chapter{Analyse op 1 variabele}
\label{ch:analyse1var}

%% TODO: volgorde van centrum-/ spreidingsmaten eens bekijken en overzichtelijker maken.

\begin{definition}[Beschrijvende statistiek]
  Met beschrijvende statistiek \index{beschrijvende statistiek} bedoelen we een verzameling van technieken om data synthetisch voor te stellen en samen te vatten.
\end{definition}

\section{Voorbeeld met superhelden}
\begin{table}
  \centering
  \begin{tabular}{|c|c|c|c|c|}
    \hline
    $x_{1}$ & $x_{2}$ & $x_{3}$ & $x_{4}$&  $x_{5}$ \\
    \hline
    141 & 198 & 143 & 201 & 184 \\
    \hline
  \end{tabular}
  \caption{Voorbeeldtabel superhelden vanuit slides}
  \label{tab:helden}
\end{table}

\section{Gemiddelde}
\label{sec:gemiddelde}

\begin{definition}[Gemiddelde]
  \index{Gemiddelde} Het gemiddelde (symbool $\mu$) van een set waarden is de som van al deze waarden gedeeld door het aantal waarden. De formule staat beschreven in \ref{eq:Mean}.
  \begin{equation}
    \mu = \frac{1}{n} \times \sum_{i=1}^{n} x_{i}
    \label{eq:Mean}
  \end{equation}

  Waarbij:
  \begin{itemize}
    \item $x_{i}$ de waarden zijn vanuit tabel \ref{tab:helden}.
    \item $n$ het aantal waarden is. In het voorbeeld van de superhelden zou dit 5 zijn, want we hebben 5 lengtes van superhelden.
  \end{itemize}
\end{definition}


\begin{exercise}
  Wat is de gemiddelde lengte van de superhelden?
\end{exercise}

\begin{exercise}
  Vraag: het gemiddelde van 15 cijfers is 12. Welk nummer moeten
  we aan de rij van cijfers toevoegen om een gemiddelde van 13 te bekomen?
\end{exercise}

Het rekenkundig gemiddelde is gevoelig aan outliers: een extreme waarde kan het rekenkundig gemiddelde zwaar be\"invloeden.

\section{Mediaan}

\begin{definition}[Mediaan]
  Indien we alle cijfers sorteren van klein naar groot, is de \index{Mediaan} mediaan het middelste cijfer, of het gemiddelde van de twee middelste cijfers indien het aantal cijfers even is.
\end{definition}

De mediaan is niet gevoelig aan outliers.

\section{Modus}
\begin{definition}[Modus]
  De \index{Modus} modus is het cijfer dat het meest voorkomt in een set van cijfers.
\end{definition}

\begin{itemize}
  \item Heeft niet veel zin als alle cijfers even veel voorkomen. (Zoals bij onze superhelden). Misschien is het nuttig om ze dan te groeperen.
  \item Er kunnen twee modi zijn: dit noemen we \index{Bimodaal} bimodaal;
  \item Er kunnen meerdere modi zijn: dit noemen we \index{Multimodaal} multimodaal.
\end{itemize}

\begin{example}
  Het groeperen kunnen we tonen bijvoorbeeld bij het aantal mensen gered door Batman de laatste acht jaar.
  \begin{itemize}
    \item $[0-9]$ mensen : 4, 7
    \item $[10-19]$ mensen: 11, 16
    \item $[20-29]$ mensen : 20, 22, 25, 26
    \item $[30-39]$ mensen: 33
  \end{itemize}
  Dus categorie $[20-29]$ komt het meest voor. We kunnen dus bv. kiezen om 25 als modus te gebruiken. Zo'n klasse noemen we dan een modale klasse.
\end{example}

\section{Range / Bereik}
\begin{definition}[Bereik]
  Het \index{Range} \index{Bereik} bereik in een set van getallen is de absolute waarde van het verschil tussen het laagste en grootste getal.
\end{definition}

\section{Kwartielen \& kwartielafstand}
\begin{definition}[Kwartielen \& Kwartielafstand]
  De \index{Kwartiel} kwartielen zijn de waarden die een gesorteerde lijst van nummers in 4 gelijke delen deelt. Elk deel vormt dus een kwart van de dataset. Men spreekt van een eerste, tweede en derde kwartiel ($Q_{1}$, $Q_{2}$, $Q_{3}$).
\end{definition}

Dus:
\begin{itemize}
  \item eerste kwartiel $Q_{1}$ is de getalswaarde die de laagste 25 \% van de reeks afscheidt.
  \item tweede kwartiel $Q_{2}$ is de getalwaarde die de laagste 50\% van de reeks afscheidt.
  \item derde kwartiel $Q_{3}$ is de getalwaarde die de laagste 75\% van de reeks afscheidt.
\end{itemize}

\begin{definition}
  \index{Kwartielafstand} Kwartielafstand is het verschil tussen $Q_{3}$ en $Q_{1}$ ( dus $Q_{3} - Q_{1}$).
\end{definition}

Methode om te berekenen (volgens \textcite{Moore2002}) (met $n$ oneven aantal getallen):
\begin{itemize}
  \item $Q_{1}$ komt overeen met cijfer $\frac{n+1}{4}$
  \item $Q_{3}$ komt overeen met cijfer $\frac{3n+3}{4}$
\end{itemize}

Methode om te berekenen (met $n$ even aantal getallen):
\begin{itemize}
  \item $Q_{1}$ komt overeen met cijfer $\frac{n+2}{4}$
  \item $Q_{3}$ komt overeen met cijfer $\frac{3n+2}{4}$
\end{itemize}

\begin{exercise}
  Met welke voorgaande statistiek komt $Q_{2}$ overeen?
\end{exercise}

\section{Variantie en standaardafwijking}
\label{sec:varEnSD}
\begin{definition}[Variantie]
  De \index{Variantie} variantie (symbool $\sigma^{2}$ - lees sigma kwadraat) is het gemiddelde van de kwadraten van de verschillen tussen de waarde van de dataset en het gemiddelde.
  \begin{equation}
    \sigma^{2} = \frac{1}{n} \times \sum_{i=1}^{n} \left( \mu - x_i \right)^{2}
    \label{eq:variantie}
  \end{equation}
\end{definition}


\begin{example}
  De variantie bij de lengtes van onze superhelden wordt als volgt berekend:

  \begin{equation}
    \begin{aligned}
      \sigma^{2} &=  \frac{(173.4-141)^{2} + (173.4 - 198 )^{2} + (173.4 - 143)^{2} + (173.4- 201)^{2} + (173.4  -184 )^{2}}{5} \\
      &=  \frac{(-32.4)^{2}+	(24.6)^{2}	+ (-30.4)^{2}+	(27.6)^{2}	+ (10.6)^{2}}{5}\\
      &= \frac{1049.76 + 	605.16	 + 924.16	 + 761.76 + 	112.36}{5}\\
      &= \frac{3453.2}{5} = 690.64
    \end{aligned}
  \end{equation}
\end{example}

\begin{definition}[Standaardafwijking]
  De \index{Standaardafwijking} standaardafwijking wordt dan gedefinieerd als de vierkantswortel van de variantie.
  \begin{equation}
    \sigma = \sqrt{\sigma^{2}}
    \label{eq:stdev}
  \end{equation}
\end{definition}


Dit geeft ons dus inzicht in wat normaal is en wat abnormaal is: een kleine standaardafwijking wijst erop dat de waarden dicht bij de centrummaat ($\mu$) liggen, terwijl een grote standaardafwijking duidt dat de waarden verspreid liggen over een groot bereik van waarden. In sommige gevallen wil men een grote standaardafwijking, in andere gevallen niet zoals hieronder beschreven.

\begin{example}
  Bij het vervaardigen van een schroevendraaier is de grootte van de kop belangrijk voor het goed functioneren van de schroevendraaier. Als we dus van 100 verschillende schroevendraaiers de kopgrootte meten, is het beter dat die grootte redelijk constant is en wensen we dus een kleine standaardafwijking.
\end{example}

\begin{example}
  Bij het onderzoek naar onze superhelden, wensen we te weten hoeveel ze ongeveer verdienen in hun normale job. We hebben een aantal rijke superhelden (bv. Batman) en een aantal minder rijke superhelden (bv. Spiderman). De spreiding op hun inkomen is dus groot, maar dat is niet per definitie slecht.
\end{example}


Een aangename eigenschap van de standaardafwijking is dat het uitgedrukt kan worden in dezelfde metriek als de gemeten data. Bij ons voorbeeld van de superhelden, wil dat zeggen dat de standaardafwijking 26.28 cm is.

Zoals het gemiddelde is de variantie en de standaarddeviatie gevoelig aan outliers (uitschieters). De variantie is eigenlijk gevoeliger dan het gemiddelde. Inderdaad, voor een outlier is de afstand tot het gemiddelde kleiner dan het kwadraat van deze afstand.

\section[Centrum- en spreidingsmaten toepassen]{Toepassing spreidingsmaten en maten centraliteit op verschillende soorten variabelen}

\begin{table}
  \centering
  \begin{tabular}{|l|l|l|l|l|}
    \hline
    \textbf{Analyse} & \textbf{Nominaal} & \textbf{Ordinaal} & \textbf{Interval} of \textbf{Ratio} \\
    \hline
    \textbf{Centrum} & Modus & Mediaan & Gemiddelde \\
    & Modale klasse & Modus & Mediaan \\
    & & Modale klasse & Modale klasse \\
    \hline
    \textbf{Spreiding} & & Range & Range \\
    & & Interkwartielafstand & Interkwartielafstand \\
    & & & Standaarddeviatie \\
    \hline
  \end{tabular}
  \caption{Meetniveaus en mogelijkheden op variabelen}
  \label{tab:Meetniveaus}
\end{table}

\section{Grafieken}

\subsection{Boxplot}

De \index{Boxplot} boxplot wordt gevormd door een rechthoek begrensd door de kwartielwaarden (25\% en 75\%). In deze rechthoek wordt ook de mediaan getekend. De stelen, die aan de rechthoek zitten, bevatten de rest van de waarnemingen op de uitschieters en extremen na.

\begin{itemize}
  \item Een \index{Uitschieter} uitschieter is een waarde die meer dan 1.5 keer de interkwartielafstand boven/onder het derde/eerste kwartiel ligt. Wordt aangeduid met een cirkeltje.
  \item Een \index{Extremum} extremum is een waarde die meer dan 3 keer de interkwartielafstand boven/onder het derde/eerste kwartiel ligt. Wordt aangeduid met een sterretje.
\end{itemize}

\section{R}
Zodra u een vector (of een lijst met nummers) in het geheugen hebt, zijn de meeste basiswerkingen beschikbaar. De meeste basiswerkzaamheden werken op een hele vector en kunnen gebruikt worden om snel een groot aantal berekeningen uit te voeren met een enkele opdracht. Indien je een operatie uitvoert op meerdere vectoren, is het vaak nodig dat de vectoren allemaal hetzelfde aantal elementen bevatten.

Hieronder zie je een set van eenvoudige operaties die je met vectoren kan doen. Let op dat de operaties allemaal op een element per element basis uitgevoerd worden. 
\lstinputlisting{data/operaties.R}

De volgende commando's kunnen worden gebruikt om het gemiddelde,de kwartielen, het  minimum, het maximum, de variantie en  de standaardafwijking van een reeks getallen te verkrijgen.

\lstinputlisting{data/basics.R}

\section{Oefeningen}

\subsection{Centrum- en spreidingsmaten}

\begin{definition}
	Een frequentietabel is tabel waarin opgesomd staat hoeveel keer een waarde voorkomt in de volledige dataset (= frequentie). Meestal zijn de tabellen verticaal georiënteerd.
\end{definition}

\begin{exercise}
  \label{ex:mean-stdev-freq}
  De formules voor gemiddelde $\mu$ en variantie $\sigma^2$ staan beschreven in secties \ref{sec:gemiddelde} en \ref{sec:varEnSD}, resp. Hoe moeten deze formules aangepast worden om $\mu$ en $\sigma^2$ te berekenen wanneer we te maken hebben met een frequentietabel? Doe dit voor de data in tabel \ref{tab:pinfreq}.
\end{exercise}

\begin{table}
  \centering
  
  \begin{tabular}{@{}ll@{}}
    \toprule
    Pinnen $x$ & Frequentie $f_{x}$ \\ \midrule
    0          & 2                  \\
    1          & 1                  \\
    2          & 2                  \\
    3          & 0                  \\
    4          & 2                  \\
    5          & 4                  \\
    6          & 9                  \\
    7          & 11                 \\
    8          & 13                 \\
    9          & 8                  \\
    10         & 8
  \end{tabular}
  \caption{Tijdens het spelen van een kegelspel is bijgehouden hoeveel pinnen telkens omver gegooid werden. Voor elke mogelijke score $x$ is bijgehouden hoeveel keer }
  \label{tab:pinfreq}
\end{table}

\begin{exercise}
  \label{ex:variance-formula}
  In de formule voor de variantie wordt het verschil tussen de meetpunten en het gemiddelde gekwadrateerd. Waarom? Zouden we geen eenvoudiger formule kunnen bedenken die een even goede maatstaf is voor de spreiding van een dataset? Hieronder vind je drie voorstellen (de derde is de ``echte'' formule).

  \begin{align}
    \sigma^{2}_{1} &= \frac{1}{n} \sum_{i=1}^{n} (\mu - x) \\
    \sigma^{2}_{2} &= \frac{1}{n} \sum_{i=1}^{n} \left| \mu - x\right| \\
    \sigma^{2}_{3} &= \frac{1}{n} \sum_{i=1}^{n} (\mu - x)^{2}
  \end{align}

  Pas elke formule toe op de twee datasets hieronder. Door het resultaat te vergelijken zou je moeten kunnen besluiten of de formules geschikt zijn als een spreidingsmaat.
  
  \begin{align*}
    X &= \left\{ 4,4,-4,-4 \right\} \\
    Y &= \left\{ 7,1,-6,-2 \right\}
  \end{align*}

\end{exercise}

\begin{exercise}
Zoek eens zelfstandig op wat de variatieco\"effici\"ent is. Hoe
wordt die gedefinieerd voor een volledige populatie en wat zou
je ermee kunnen doen?
\end{exercise}

\begin{exercise}
  \label{ex:ais}
Beschouw de volgende subsets uit het data frame \texttt{ais} (uit de library DAAG):
\begin{enumerate}
\item Ontleed de gegevens voor de roeiers.
\item Ontleed de gegevens voor de roeiers, de netballers en de tennissers.
\item Ontleed de gegevens voor de vrouwelijke basketballers en roeiers.
\end{enumerate}
\end{exercise}

\begin{exercise}
Gebruik de functies \texttt{mean} en \texttt{range} om het gemiddelde en bereik van:
\begin{itemize}
  \item de cijfers 1, 2, \dots, 21 
  \item 50 willekeurige normale waarden, die  worden gegenereerd vanuit een normale distributie met gemiddelde 0 en variantie 1 (functie \texttt{rnorm})
  \item de kolommen \texttt{height} en \texttt{weight} in de data frame \texttt{women} (standaard in R).
\end{itemize}
\end{exercise}

In vorige oefeningen hebben we de verschillende spreidingsmaten en centrummaten besproken. Zoals je merkt worden deze metrieken ook gebruikt in het onderzoek van~\textcite{Akin2016}. In de volgende oefeningen gaan we trachten de resultaten te reproduceren.

Hiervoor hebben we het bestand \texttt{android\_persistence\_cpu.csv} nodig. Je vindt die in de Github-repository van deze cursus, onder directory \texttt{oefeningen/data/oef\_3\_1variabele}

\begin{exercise}
  \label{oef:casus-akin2016-1var}
	Open de file met excel en bekijk de structuur van het document. Hoe ziet die er uit? Kan je de variabelen identificeren en hun type benoemen. 
\end{exercise}

We gaan het programma \texttt{R} gebruiken samen met \texttt{RStudio}. Open de file in \texttt{RStudio}.

\begin{lstlisting}
android_cpu <-  read.csv("android_persistence_cpu.csv", sep=";", dec=",")
attach(android_cpu)
\end{lstlisting}

We hebben nu de data ingeladen. We kunnen eens kijken wat de gemiddelde tijd, de standaarddeviatie, de kwartielen e.a. zijn. Gebruik hiervoor de commando's \texttt{mean}, \texttt{median}, \texttt{quantile}, \texttt{min}, \texttt{max}, \texttt{var}, \texttt{sd}. Je kan ook makkelijk gebruik maken van de methode \texttt{summary}.

\begin{exercise}
	Als je de vorige metrieken berekend hebt, wat kan je daar dan over zeggen. Kan je zinnige conclusies trekken uit de vorige resultaten. Zo ja vermeld ze, zo nee beschrijf waarom je dat denkt.
\end{exercise}

\subsection{Grafieken in R}

Een histogram is een eenvoudige plot. het toont de frequenties van de data die in een bepaald bereik voorkomen. 

\begin{lstlisting}
hist(android_cpu$Tijd,main="Verdeling van de tijd",xlab="De gemeten cpu tijd");
hist(android_cpu$Tijd,main="Verdeling van de tijd",xlab="De gemeten cpu tijd",breaks=2);
\end{lstlisting}
\begin{exercise}
	Wat concludeer je als je bovenstaande grafiek\footnote{Heb je wat problemen met het genereren van grafieken, volgende link \url{https://www.datacamp.com/community/tutorials/15-questions-about-r-plots\#gs.RK_ORsI} bevat een aantal goede tips and tricks om je op weg te helpen.} genereert? Is dit een zinnig resultaat? Wat gebeurt er als je de variabele breaks verhoogt?
\end{exercise}

Een boxplot toont de mediaan, de kwartielen, het maximum en het minimum van een dataset. Dit geeft ons een duidelijk impressie van hoe de data er uitziet.

\begin{lstlisting}
boxplot(x = android_cpu$Tijd);
boxplot(android_cpu$Tijd,main='Spreiding van de CPU tijd',ylab='Tijd in ms');
\end{lstlisting} 

\begin{exercise}
	De boxplot wordt standaard verticaal getekend. Gebruik het commando \texttt{help(boxplot)} om uit te zoeken hoe we de tekening horizontaal krijgen. 
\end{exercise}

Als je goed geantwoord hebt op de volgende vragen merk je natuurlijk dat het weinig zin heeft de volledige dataset te analyseren, aangezien de dataset verdeeld is over verschillende categorie\"en. We willen dus wel deze statistieken weten, maar per categorie. We kunnen dus een boxplot maken voor elke categorie.

\begin{lstlisting}
boxplot(android_cpu$Tijd~android_cpu$Datahoeveelheid,main='Spreiding van de CPU tijd t.o.v. datahoeveelheid',ylab='Tijd in ms');
\end{lstlisting}

\begin{exercise}
	\label{ex:boxplot}
	Interpreteer de resultaten die je behaalt uit deze grafiek. Zijn deze al wat zinniger?
\end{exercise}

We kunnen hetzelfde doen voor de verschillende soorten dataopslagmogelijkheden in android.

\begin{exercise}
	Zelfde vraag als \ref{ex:boxplot} Interpreteer de resultaten die je behaalt uit deze grafiek. Zijn deze al wat zinniger?
\end{exercise}

We kunnen eens kijken hoe de data eruit ziet over alle categorie\"en heen.

\begin{lstlisting}
boxplot(android_cpu$Tijd~android_cpu$PersistentieType*android_cpu$Datahoeveelheid,main='Spreiding van de CPU tijd',ylab='Tijd in ms');
\end{lstlisting}

Het blijkt dat we wel al een duidelijker zicht krijgen over de data over de categorie\"en heen, maar de figuur is op dit moment te druk. 

We moeten de data dus onderverdelen in categorie\"en namelijk onder \texttt{PersistentieType} en \texttt{Datahoeveelheid}. We gaan hiervoor de functie \texttt{which}\footnote{Je kan ook gebruik maken van de functie \texttt{subset}, wat misschien zelfs eenvoudiger is} gebruiken en kijken hoe de verschillende datahoeveelheden verschillen per datahoeveelheidcategorie. 

\begin{lstlisting}
greenDOA <- android_cpu[which(android_cpu$PersistentieType=='GreenDAO'),];
boxplot(greenDOA$Tijd~greenDOA$Datahoeveelheid);
\end{lstlisting}

\begin{exercise}
	Wat concludeer je uit de vorige grafiek?
\end{exercise}

\begin{exercise}
	Ga nu zelf na welke boxplots er interessant zijn om te maken, en kijken of jouw resultaten overeen met die van \textcite{Akin2016}. Welke conclusies trek je?
\end{exercise}


\subsection{Antwoorden op geselecteerde oefeningen}

\paragraph{Oefening \ref{ex:mean-stdev-freq}}

\begin{itemize}
  \item $\mu = 7$
  \item $\sigma^2 \approx 5.7333$
  \item $\sigma \approx 2.3944$
\end{itemize}

\paragraph{Oefening \ref{ex:ais}}

De opgave van deze oefening is zeer algemeen gesteld. Onder ``ontleed de gegevens'' wordt concreet bedoeld alle technieken voor de analyse van een variabele uit te proberen op de dataset. We geven hier een voorbeeld van de resultaten voor enerzijds kwantitatieve en anderzijds kwalitatieve variabelen.

Tabel \ref{tab:opl-ais-ht} geeft een overzicht met de belangrijkste centrum- en spreidingsmaten voor de variabele \texttt{ht} (height, lengte) over de gevraagde groepen.

\begin{table}
  \centering
  \begin{tabular}{@{}l|r|rrrr|rr@{}}
    \toprule
    & \textbf{(1)} & \multicolumn{4}{c}{\textbf{(2)}}                                                 & \multicolumn{2}{c}{\textbf{(3)}} \\ 
    & \textbf{Row} & \textbf{hele groep} & \textbf{Row} & \textbf{Netball} & \textbf{Tennis} & \textbf{B\_ball}  & \textbf{Row} \\ \midrule
    \textbf{gemiddelde} & 182.376      & 179.066                      & 182.376      & 176.087          & 174.164         & 182.269           & 178.859      \\
    \textbf{stdev}      & 7.798        & 7.936                        & 7.798        & 4.124            & 9.858           & 8.621             & 5.970        \\
    \textbf{min}        & 156.000      & 156.000                      & 156.000      & 168.600          & 157.900         & 169.100           & 156.000      \\
    \textbf{Q1}         & 179.300      & 174.200                      & 179.300      & 173.450          & 167.300         & 174.000           & 177.600      \\
    \textbf{mediaan}    & 181.800      & 179.500                      & 181.800      & 176.000          & 175.000         & 184.600           & 179.650      \\
    \textbf{Q3}         & 186.300      & 183.400                      & 186.300      & 179.150          & 180.750         & 188.700           & 181.200      \\
    \textbf{max}        & 198.000      & 198.000                      & 198.000      & 183.300          & 190.800         & 195.900           & 186.300      \\
    \textbf{IQR}        & 7.000        & 9.150                        & 7.000        & 5.700            & 13.450          & 14.700            & 3.600        \\ \bottomrule
  \end{tabular}
  \caption{Overzicht resultaten in oefening \ref{ex:ais} voor de variabele \texttt{ht} (height/lengte), met drie cijfers na de komma. In deeloefening 2 zijn de resultaten zowel gegeven voor de hele groep (roeiers, netballers én tennissers) als opgesplitst (via de functie \texttt{aggregate}).}
  \label{tab:opl-ais-ht}
\end{table}

In deeloefening 1 en 2 nemen we de variabele \texttt{sex} als voorbeeld. Zie tabel \ref{tab:opl-ais-sex} voor een overzicht. Over kwalitatieve variabelen valt minder te zeggen, we geven hier een frequentietabel waaruit we de modus kunnen afleiden.

In deeloefening 3 zijn enkel vrouwen geselecteerd, en voor deze oefening tonen we in tabel \ref{tab:opl-ais-sport} de frequenties van variabele \texttt{sport}.

\begin{table}
  \centering
  \begin{tabular}{@{}l|r|rrrr}
  	\toprule
  	               & \textbf{(1)} &                    \multicolumn{4}{c}{\textbf{(2)}}                     \\
  	               & \textbf{Row} & \textbf{hele groep} & \textbf{Row} & \textbf{Netball} & \textbf{Tennis} \\ \midrule
  	\textbf{f}     &           22 &                  52 &           22 &               23 &               7 \\
  	\textbf{m}     &           15 &                  19 &           15 &                0 &               4 \\
  	\textbf{modus} &            f &                   f &            f &                f &               f \\ \bottomrule
  \end{tabular}
  \caption{Overzicht resultaten in oefening \ref{ex:ais} (1) en (2) voor de variabele \texttt{sex}. Meer bepaald zijn hier de frequenties van de waarden opgegeven, en ook telkens de modus.}
  \label{tab:opl-ais-sex}
\end{table}

\begin{table}
  \centering
  \begin{tabular}{@{}l|l}
  	\toprule
  	                 & Frequenties \\ \midrule
  	\textbf{B\_ball} & 13          \\
  	\textbf{Row}     & 22          \\
  	\textbf{modus}   & Row         \\ \bottomrule
  \end{tabular}
  \caption{Overzicht resultaten in oefening \ref{ex:ais} (3) voor de variabele \texttt{sport}.}
  \label{tab:opl-ais-sport}
\end{table}

\chapter{Steekproefonderzoek}
\label{ch:steekproefonderzoek}

Een reden om kwantitatief onderzoek uit te voeren is het kunnen doen van uitspraken die een representatief beeld van de werkelijkheid geven. Hierbij wordt vaak gebruikgemaakt van een steekproef. Een steekproef is een selectie uit een totale populatie ten behoeve van een meting van bepaalde eigenschappen van die populatie.

\begin{figure}
\centering
  \begin{tikzpicture}[xscale=4,yscale=2]
    \draw (0,2) -- (0,0);
    \foreach \num/\label in {0/0, 0.2/20, .4/40, .6/60, .8/80, 1/100, 1.2/120, 1.4/140, 1.6/160, 1.8/180, 2/200}{%
      \draw (0, \num) -- (2.5, \num);
      \draw[shift={(0, \num)}] (1pt,0pt) -- (-1pt,0pt) node[left] {\scriptsize \label};
    }

    \node[anchor=north] (hero1) at (0.3,1.5)
    {\includegraphics[height=2.9cm]{images/les2-hero-1}};
    \node[anchor=north] (hero2) at (0.8,2.05)
    {\includegraphics[height=4cm]{images/les2-hero-2}};
    \node[anchor=north] (hero3) at (1.3,1.575)
    {\includegraphics[height=3.1cm]{images/les2-hero-3}};
    \node[anchor=north] (hero4) at (1.8,2.1)
    {\includegraphics[height=4.1cm]{images/les2-hero-4}};
    \node[anchor=north] (hero5) at (2.3,1.95)
    {\includegraphics[height=3.8cm]{images/les2-hero-5}};

    \node (size1) at (0.3, 1.5) {\scriptsize 141 cm};
    \node (size2) at (0.8, 2.1) {\scriptsize 198 cm};
    \node (size3) at (1.3, 1.51) {\scriptsize 143 cm};
    \node (size4) at (1.8, 2.15) {\scriptsize 201 cm};
    \node (size5) at (2.3, 1.95) {\scriptsize 184 cm};
  \end{tikzpicture}
  \caption{De superhelden die we onderzoeken}
  \label{fig:superheldenSteekproef}
\end{figure}

\begin{figure}
  \centering
  \includegraphics[width=0.85\textwidth]{images/les5-heroes.jpg}
  \caption{Onze superhelden die we onderzoeken vormen een steekproef uit de populatie van alle superhelden.}
  \label{fig:populatieHelden}
\end{figure}

\section{Populatie en Steekproeven}
\begin{definition}[Populatie]
  De verzameling van \textbf{alle} objecten of personen waar men in ge\"interesseerd is en onderzoek naar wil doen noemt men de \index{populatie} populatie. Een ander woord is ook wel onderzoeksgroep of doelgroep.
\end{definition}

\begin{definition}[Steekproef]
  Wanneer met een subgroep uit een populatie gaat onderzoeken, dan noemen we die groep een \index{steekproef} steekproef.
\end{definition}

\begin{definition}[Steekproefkader]
  Een \index{steekproefkader} steekproefkader is een lijst van alle leden van een te onderzoeken populatie.
\end{definition}

Er zijn een aantal redenen waarom een steekproef genomen wordt:
\begin{itemize}
  \item Populatie is te groot om een volledig onderzoek te doen.
  \item Kostbare metingen waardoor het onderzoek te duur wordt.
  \item Wanneer snelheid belangrijk is, is het vaak sneller een subgroep te onderzoeken;
  \item Gemakkelijker \dots
\end{itemize}

Om een steekproef op te zetten volg je volgende stappen:
\begin{enumerate}
  \item \textbf{Definitie populatie}: Wie is er deel van de populatie? Dit hangt nauw samen met de probleemstelling van het onderzoek. Dit is een zeer belangrijke stap waar je niet licht over mag gaan. Elementen die van belang zijn, zijn bijvoorbeeld sociale, demografische of fysieke kenmerken zoals geslacht, leeftijd, woonplaats \dots
  \item \textbf{Bepalen van steekproefkader}: Een populatie heeft verschillende segmenten zoals bijvoorbeeld rijke superhelden, arme superhelden, bekende en onbekende superhelden, superhelden met en zonder oudercomplex \dots In de praktijk is het meestal onmogelijk om de populatie als geheel te onderzoeken. Daarom beperken we ons vaak tot enkele redelijk homogene subpopulaties of segmenten. De populatiesegmenten die daadwerkelijk onderzocht worden noemen we de operationele populatie.
  \item \textbf{Budget en Tijd}: het aantal te onderzoeken objecten of personen zal ook afhankelijk zijn van budget en tijd.
\end{enumerate}

\section{Kiezen van steekproefmethode}
  Soms is de populatie die men wenst te bestuderen erg verschillend op een aantal belangrijke kenmerken. Daartoe wordt de populatie als geheel in een aantal elkaar niet-overlappende en homogene strata of klassen ingedeeld.

\begin{definition}[Gestratificeerde steekproef]
Een \textbf{gestratificeerde} \index{gestratificeerd} steekproef is proportioneel als het aandeel van de subpopulatie in de steekproef gelijk is aan het aandeel van de subpopulatie in de populatie als geheel.
\end{definition}

\begin{example}
  Indien we uit een populatie van de superhelden kijken naar de leeftijd van  mannen en vrouwen, zien we in tabel \ref{tab:heldenPopulatie1} de absolute waarden. We kunnen niet alle superhelden ondervragen, maar indien we een steekproef nemen waarbij de mannen en leeftijdscategorie\"en relatief equivalent zijn met de populatie, hebben we een gestratificeerde steekproef genomen (zie tabel \ref{tab:heldenPopulatie2}).
\end{example}

  \begin{table}
  \centering
    \begin{tabular}{l|cccc|c}
      & \multicolumn{4}{c|}{\textbf{Leeftijd}} & \\
      Geslacht & $\le 18$ & $]18,25]$ & $]25, 40]$ & $> 40$ & Totaal\\
      \hline
      Vrouw & 500 & 1500 & 1000 & 250 & 3250 \\
      Man   & 400 & 1200 & 800 & 160 & 2560\\
      \hline
      Totaal & 900 & 2700 & 1800 & 410 & 5810
    \end{tabular}
    \caption{Frequenties van de superhelden in de populatie volgens geslacht en leeftijdscategorie}
    \label{tab:heldenPopulatie1}
\end{table}

\begin{table}
  \centering
    \begin{tabular}{l|cccc|c}
      & \multicolumn{4}{c|}{\textbf{Leeftijd}} & \\
      Geslacht & $\le 18$ & $]18,25]$ & $]25, 40]$ & $> 40$ & Totaal\\
      \hline
      Vrouw & 50 & 150 & 100 & 25 & 325 \\
      Man   & 40 & 120 & 80 & 16 & 256\\
      \hline
      Totaal & 90 & 270 & 180 & 41 & 581
    \end{tabular}
      \caption{Steekproef van superhelden gestratificeerd volgens geslacht en leeftijdscategorie.}
    \label{tab:heldenPopulatie2}
  \end{table}

Nadat gestratificieerd is, moet bepaald worden op welke wijze binnen ieder stratum het aantal benodigde objecten of respondenten gekozen moet worden. Bij de toevals- of \textbf{aselecte} \index{aselect} steekproeven heeft elk element van de populatie een even mogelijke kans om in de steekproef te worden opgenomen. Dit heeft als gevolg dat je op basis van de data van een aselecte steekproef conclusies kan trekken ten aanzien van de kenmerken van een populatie, en dit in tegenstelling met een \textbf{niet-aselecte} steekproef. In een niet-aselecte steekproef kent men de kans niet dat elk lid van de populatie heeft om in de steekproef terecht te komen, met als gevolg dat je gegevens enkel gelden voor je onderzochte groep.

\subsection{Fouten bij steekproeven}
\subsubsection{Toevallige steekproeffouten}
Wanneer er puur door het toeval een verschil is in een waarde voor de populatie en de steekproef.

\subsubsection{Systematische steekproeffouten}
Een procedure in de steekproef die een fout oplevert die een systematische oorzaak heeft en dus niet te wijten is aan toevallige effecten. Bijvoorbeeld door systematisch een bevoordeeld deel van de populatie te ondervragen. Als we onze superhelden zouden ondervragen via het internet, sluiten we alle superhelden uit die geen internetverbinding hebben.

\subsubsection{Toevallige niet-steekproeffouten}
Hieronder vallen bijvoorbeeld verkeerd aangekruiste antwoorden of verschil in interpretatie van de vragen.

\subsubsection{Systematische niet-steekproeffouten}
Wanneer bijvoorbeeld respondenten met een sterke band met het onderzoek eerder geneigd zijn om een vragenlijst in te vullen, ga je positievere antwoorden krijgen - terwijl ze niet representatief zijn voor de gehele populatie.

\subsection{Aanpassing formules standaarddeviatie}
We noemen het gemiddelde van de steekproef het steekproefgemiddelde en gebruiken hiervoor het symbool $\overline{x}$ (dit hebben we stilzwijgend al een aantal keer gedaan in de vorige hoofdstukken).

Als we de standaardafwijking van een steekproef willen bepalen dan moeten we niet meer delen door $n$ (aantal metingen) maar door $(n-1)$. Waarom?

Aangezien de som van de afwijkingen $x_{i} - \overline{x}$ steeds 0 oplevert (zie hieronder in vergelijking \ref{eq:sumGemid}), kan de laatste afwijking gevonden worden uit de eerste $n-1$ afwijkingen. We berekenen dus niet het gemiddelde van $n$ getallen zonder verwantschap. Slecht $n-1$ van de gekwadrateerde afwijkingen kunnen vrij bewegen, daarom berekenen we het gemiddelde door het totaal te delen door $n-1$. Het getal $n-1$ noemt men het aantal vrijheidsgraden van de variantie of van de standaardafwijking.

\begin{equation}
 \sum_{i}^{n}(x_{i} - \overline{x}) = \sum_{i}^{n}x_{i} - \sum_{i}^{n}\overline{x} = \sum_{i}^{n}x_{i} - n (\frac{1}{n}\sum_{i}^{n} x_{i})
\label{eq:sumGemid}
\end{equation}

\section{Kansverdeling van een steekproef}
\subsection{Stochastisch experiment}
Een random (of stochastisch) experiment heeft volgende elementen nodig:

\begin{definition}[Universum of Uitkomstenruimte]\
 Het universum of uitkomstenruimte van een experiment
is de verzameling van alle mogelijke uitkomsten van dit experiment en
wordt genoteerd met $\Omega$.
\end{definition}
~\\
\textbf{Opmerkingen}
\begin{itemize}
\item De uitkomstenruimte moet \emph{volledig}\/ zijn: elke mogelijke
uitkomst van een experiment moet tot $\Omega$ behoren.
\item Bovendien moet elke
uitkomst van een experiment overeenkomen met \emph{juist \'e\'en}\/ element van
$\Omega$.
\item Samengevat: na het uitvoeren van een experiment is het  mogelijk  om eenduidig
aan te geven welk element van $\Omega$ zich heeft voorgedaan.
\end{itemize}

\begin{definition}[Gebeurtenis]
 Een gebeurtenis is een deelverzameling van de uitkomstenruimte. Een enkelvoudige of elementaire gebeurtenis is een singleton;   een samengestelde gebeurtenis heeft cardinaliteit groter dan 1.
\end{definition}
Gebeurtenissen die geen gemeenschappelijke uitkomsten hebben noemt men
{disjunct. \\
Disjuncte gebeurtenissen kunnen dus nooit samen voorkomen.\\
Wanneer de gebeurtenissen $A$ en $B$ disjunct zijn dan geldt
$A \cap B = \emptyset$.
Startend met
 de gebeurtenissen $A$ en $B$ kan men de volgende gebeurtenissen vormen:
 \begin{itemize}
  \item $A$ \textbf{of} $B$, of wiskundig genoteerd $A \cup B$;
  \item $A$ \textbf{en} $B$, of wiskundig genoteerd $A \cap B$;
  \item \textbf{niet} $A$, of wiskundig genoteerd $\overline{A}$.
\end{itemize}
\textbf{Opmerkingen}
\begin{itemize}
\item Door inductie leidt men gemakkelijk  af dat
de unie van $n$  gebeurtenissen $A_1$ t.e.m.~$A_n$ eveneens
een gebeurtenis is.
\item Idem voor de doorsnede van
gebeurtenissen.
\item Voor sommige toepassingen is het nodig om ook (aftelbaar) oneindige
unies en doorsnedes te beschouwen.
\end{itemize}

\begin{definition}[Kansruimte]
Het toekennen van kansen aan gebeurtenissen dient aan de volgende drie regels te voldoen.
\begin{enumerate}
\item Kansen zijn steeds positief:
 $P(A) \geq 0$ voor elke $A$.
  \item
  De uitkomstenruimte heeft kans 1:
  $P(\Omega) = 1.$
 \item Wanneer $A$ en $B$ \emph{disjuncte}\/ gebeurtenissen zijn dan is
 \[P(A\cup B) = P(A) + P(B). \]
 Dit noemt men de somregel.
\end{enumerate}
Wanneer de functie $P$ aan de bovenstaande eigenschappen (axioma's) voldoet
dan noemt men het drietal $(\Omega, \mathcal{P}(\Omega), P)$ een
kansruimte (met $\mathcal{P}(\Omega)$ de \emph{machtsverzameling} van $\Omega$, d.w.z.~de verzameling van alle deelverzamelingen van $\Omega$).
\end{definition}

\begin{example}
Beschouw een uitkomstenruimte $\Omega =  \left\{ 1,2,3,4,5,6 \right\} $ en
een kansfunctie $P(\omega)=\frac{1}{|\Omega|}$, dan zou dit een dobbelsteen
kunnen voorstellen met uitkomsten 1 tot en met 6 met een kans
$P(\omega) = \frac{1}{6}$ om een van de nummers te werpen.
\end{example}

In dit onderdeel van de  cursus gaan we ons bezig houden met \textbf{inductieve statistiek}: op basis van een getrokken steekproef uitspraken doen over de populatie.

\subsection{Kansverdeling}
\subsubsection{Discrete kansverdeling}
Als we het voorbeeld nemen van het gooien van een dobbelsteen, dan kunnen we de kans dat we een van de getallen $\Omega = \{1,2,3,4,5,6\}$ in een tabel zetten of kunnen er een histogram van maken.  Er zijn een aantal belangrijke opmerkingen hierbij:
\begin{enumerate}
  \item De kansen zijn allemaal groter of gelijk aan nul.
  \item De kans op een getal is gelijk aan de bijbehorende oppervlakte van de staaf.
  \item De totale oppervlakte van alle staven is 1.
\end{enumerate}

Een ander voorbeeld is het gooien van twee dobbelstenen met de mogelijke uitkomst. Je hebt volgens de productregel $6 \times 6 = 36$ mogelijke uitkomsten. Om bijvoorbeeld drie te gooien heb je twee mogelijkheden (kans $P(X=3) = \frac{2}{36}$). Zie voor de andere getallen tot en met 7 de tabel ($ P[X=n] = \frac{n-1}{36}$).

Als we dit nu in een histogram steken bekomen we een mooie trap naar boven tot 7 en dan weer naar beneden. Nu kan je makkelijk zien dat:
\begin{itemize}
  \item Voor de kans om 10 of meer te gooien moet je bijvoorbeeld die blauwe oppervlakte hebben.
  \item Voor de kans op een aantal meer dan 2 maar minder dan 7 moet je de rode oppervlakte hebben.
  \item Voor de kans op een aantal meer dan 7 maar minder dan 10 moet je de groene oppervlakte hebben.
  \item Dan is het ook logisch dat de totale oppervlakte 1 is: de kans dat 1 van al die mogelijkheden voorkomt is natuurlijk 100\%.
\end{itemize}

\subsubsection{Continue kansverdeling}

Continue kansverdelingen zijn verdelingen waarbij hetgeen we meten niet alleen een beperkt aantal waarden kan aannemen (nominaal en ordinaal meetniveau), maar ook alle er tussenliggende waarden (ratio- en intervalniveau). Neem bijvoorbeeld het gewicht van onze superhelden. Dat is continu, immers dat kan niet alleen $60$ of $70$ kilo zijn, maar ook (bij benadering)  $66,8735485653$ kilo. In principe zijn alle tussenliggende waarden mogelijk (al is dat in praktijk vaak niet te meten). Dat heeft een belangrijk gevolg voor de kansverdeling. Die bestaat nu (in theorie) niet meer uit losse staafjes, maar is een vloeiende kromme geworden. Dat betekent dat de kans op bijvoorbeeld precies $70$ kilogram een kans nul heeft. Bij precies $70$ kg hoort een verticaal lijntje, en een lijntje heeft oppervlakte nul. Nu is die kans natuurlijk ook nul. Als we zeggen $70$ kg, dan bedoelen we meestal tussen $69,5$ en $70,5$, of preciezer het interval $[69,5; 70,5[$. Als we zeggen $70,00000$ kg, dan bedoelen we iets als binnen $[70,000005; 69,999995[$ kg.

De twee regels voor kansverdelingen hierboven blijven gewoon geldig. Als zo'n kromme een goede kansverdeling is, dan moet de totale oppervlakte ervan 1 zijn, en dan kun je de kans op een gewicht dat bijvoorbeeld tussen de 60 en 70 kg ligt uitrekenen door de oppervlakte hiernaast te bepalen (merk op dat het uiteindelijk niet belangrijk is of die $60$ en $70$ zelf ook nog tot het interval behoren, die hebben toch kans nul!).

\section{De normale verdeling}

\begin{figure}[t]
\centering
\begin{tikzpicture}
\begin{axis}[
  domain=0:10, samples=100,
  axis lines*=left, xlabel=$x$, ylabel=$y$,
  every axis y label/.style={at=(current axis.above origin),anchor=south},
  every axis x label/.style={at=(current axis.right of origin),anchor=west},
  height=5cm, width=12cm,
  xtick={5,3.5,6.5}, ytick=\empty,
  enlargelimits=false, clip=false, axis on top,
  grid = major
  ]
  \addplot [fill=cyan!20, draw=none, domain=0:9] {gauss(5,1.5)} \closedcycle;
  \draw [yshift=-0.6cm, latex-latex](axis cs:3.5,0) -- node [fill=white] {$\sigma$} (axis cs:5.0,0);
\end{axis}
\end{tikzpicture}
\caption{De kansverdeling van de reactiesnelheid van superman. Deze grafiek noemen we de normaalverdeling met gemiddelde $\mu = 5$ ms en standaarddeviatie $\sigma = 1,5 ms.$}
\label{fig:verdelingReactievermogen}
\end{figure}


In figuur \ref{fig:verdelingReactievermogen} tonen we de kansverdeling van de reactiesnelheid X van superman. Deze grafiek noemen we de normaalverdeling met gemiddelde 5 ms en standaarddeviatie 1,5 ms. Symbolisch:
\[ X  \sim Nor(\mu = 5; \sigma = 1,5) \]

De functie die hiermee gepaard gaat is de volgende:

\begin{equation}
  f(x) = \frac{1}{\sigma \sqrt{2\pi}} e^{-\frac{1}{2} \frac{(x - \mu)^{2}}{\sigma^{2}}}
  \label{eq:normalFunction}
\end{equation}

De normale verdeling kent volgende eigenschappen:
\begin{itemize}
  \item Normale verdeling is klokvormig
  \item De normale verdeling is symmetrisch
  \item Vanwege symmetrie is gemiddelde, mediaan en modus aan elkaar gelijk
  \item De totale oppervlakte onder de klokvormige figuur is 1
  \item In gebied $\sigma$ onder $\mu$ en $\sigma$ boven $\mu$ (het zogenoemde sigma gebied) ligt ongeveer 68\% van de waarnemingen.
  \item In het gebied $2\sigma$ boven en onder $\mu$ ligt ongeveer 95\% van alle waarnemingen.
  \item Voor de verschillende gebieden zie figuur \ref{fig:standaardNormaleVerdeling}
\end{itemize}

\subsection{De standaardnormale verdeling}

Indien de toevalsveranderlijke $X \sim N(\mu,\sigma)$ verdeeld is dan is de toevalsvariabele $Z = \frac{X - \mu}{\sigma}$ normaal verdeeld: $Z \sim N(0,1)$. Dit noemen we de standaardnormale verdeling.

  % Bron: http://johncanning.net/wp/?p=1202
  \begin{center}
  \begin{figure}
  \centering
    \begin{tikzpicture}
      \begin{axis}[
          no markers, domain=0:10, samples=100,
          axis lines*=left,height=6cm, width=10cm,
          xtick={-3, -2, -1, 0, 1, 2, 3}, ytick=\empty,
          enlargelimits=false, clip=false, axis on top,
          grid = major
        ]
        \addplot [smooth,fill=cyan!20, draw=none, domain=-3:3] {gauss(0,1)} \closedcycle;
        \addplot [smooth,fill=orange!20, draw=none, domain=-3:-2] {gauss(0,1)} \closedcycle;
        \addplot [smooth,fill=orange!20, draw=none, domain=2:3] {gauss(0,1)} \closedcycle;
        \addplot [smooth,fill=blue!20, draw=none, domain=-2:-1] {gauss(0,1)} \closedcycle;
        \addplot [smooth,fill=blue!20, draw=none, domain=1:2] {gauss(0,1)} \closedcycle;
        \addplot[<->] coordinates {(-1,0.4) (1,0.4)};
        \addplot[<->] coordinates {(-2,0.3) (2,0.3)};
        \addplot[<->] coordinates {(-3,0.2) (3,0.2)};
        \node[coordinate, pin={68.3\%}] at (axis cs: 0, 0.35){};
        \node[coordinate, pin={95.4\%}] at (axis cs: 0, 0.25){};
        \node[coordinate, pin={99.7\%}] at (axis cs: 0, 0.15){};
        \node[coordinate, pin={34.1\%}] at (axis cs: -0.5, 0){};
        \node[coordinate, pin={34.1\%}] at (axis cs: 0.5, 0){};
        \node[coordinate, pin={13.6\%}] at (axis cs: 1.5, 0){};
        \node[coordinate, pin={13.6\%}] at (axis cs: -1.5, 0){};
        \node[coordinate, pin={2.1\%}] at (axis cs: 2.5, 0){};
        \node[coordinate, pin={2.1\%}] at (axis cs: -2.5, 0){};
      \end{axis}
    \end{tikzpicture}
    \caption{De standaardnormale verdeling met opdeling in zones}
    \label{fig:standaardNormaleVerdeling}
    \end{figure}
  \end{center}

In het algemeen kan dus bij een waarneming $x$ de zogenaamde $z$-score bepalen als volgt:

\begin{equation}
  z = \frac{x-\mu}{\sigma}
  \label{eq:zscore}
\end{equation}

Deze score geeft dus aan hoe extreem een waarneming is of anders gezegd, hoeveel standaarddeviaties is de waarneming $x$ van het gemiddelde $\mu$ verwijderd. Voor een willekeurige $x$-waarde kunnen we met formule \ref{eq:zscore} de bijhorende $z$-score bepalen. Voor deze $z$-scores heeft men tabellen opgesteld met de kansen dat een waarde kleiner dan $z$ getrokken wordt uit Z, de zgn.~linkerstaartkans\footnote{Er bestaan ook tabellen met de rechterstaartkans}: $P(Z<z)$.

R heeft eveneens functies voor het rekenen met kansen van normaal verdeelde variabelen. Deze worden samengevat in Tabel~\ref{rab:norm-prob-r}.

\begin{table}
  \centering
  \begin{tabular}{ll}
  	\textbf{Functie}      & \textbf{Betekenis}                                             \\ \hline
  	\verb|pnorm(x, m, s)| & Linkerstaartkans, $P(X<\mathtt{x})$                            \\
  	\verb|dnorm(x, m, s)| & Hoogte van de Gausscurve op punt \texttt{x}                    \\
  	\verb|qnorm(p, m, s)| & Onder welke grens zal \texttt{p}\% van de waarnemingen liggen? \\
  	\verb|rnorm(n, m, s)| & Genereer \texttt{n} normaal verdeelde random getallen
  \end{tabular}

  \caption{Kansberekeningsfuncties in R voor een normale verdeling met gemiddelde \texttt{m} en standaardafwijking \texttt{s}. Indien argumenten \texttt{m} en \texttt{s} weglaten worden, wordt de standaardnormaalverdeling verondersteld.}
  \label{rab:norm-prob-r}
\end{table}

We komen dan tot de volgende methode voor het berekenen van kansen met de normale verdeling:
\begin{enumerate}
  \item Bepaal de kansvariabele met de bijbehorende normale verdeling
  \item Bereken de $z$-score bij de bijhorende $x$-waarde.
  \item Schets de plaats van de gevraagde kans
  \item Herleid de gevraagde kans met behulp van de schets tot een linkerstaartkans en gebruik de $z$-tabel van de standaardnormale verdeling om deze te bepalen. Gebruik indien nodig de symmetrieregel en de regel van 100\% kans.
\end{enumerate}

\begin{example}
Hoe groot is de kans dat superman in minder dan 4 ms reageert?
\[ P(X < 4) = P(Z < -0,67) = 0,2514 \]
\end{example}
\begin{example}
Hoe groot is de kans dat hij in minder dan 7 ms reageert?
\[ P(X < 7) = P(Z < 1,33) = 0,9082 \]
\end{example}
\begin{example}
Hoe groot is de kans dat superman in minder dan 3 ms reageert?
\[ P(X<3) = P(Z < -1,33) = 0,0918 \]
\end{example}
\begin{example}
Hoe groot is de kans dat hij reageert tussen de 2 en 6,5 ms
\[ P( 2 < X < 6,5) = P(X<6,5) - P(X<2) = P(Z<1) - P(Z<-2) = 0,8186 \]
\end{example}

\subsection{Testen op normaliteit}
\label{sec:normtesting}

Er zijn verschillende methoden die kunnen gebruikt worden om na te gaan of een steekproef uit een normale verdeling komt.
\begin{enumerate}
  \item Construeer een histogram voor de gegevens en bekijk de vorm van de grafiek. Als de gegevens bij benadering een normale verdeling hebben, zal de vorm van het histogram een klokcurve vormen.
  \item Bereken de intervallen $\overline{x} \pm s$, $\overline{x} \pm 2s$, $\overline{x} \pm 3s$ en bepaal het percentage meetwaarden dat binnen elk van deze intervallen valt. Als de gegevens ongeveer normaal verdeeld zijn, zullen de percentages ongeveer gelijk zijn aan respectievelijk 68\%, 95\% en 99,7\%.
  \item Construeer een QQ-plot (normaliteitsplot, zie Definitie~\ref{def:qq-plot}) voor de gegevens. Als de gegevens ongeveer normaal verdeeld zijn, zullen de punten ongeveer op een rechte lijn liggen.
  \item Bereken de \emph{kurtosis} (``welving'' of ``platheid''): duidt aan hoe scherp de ``piek'' van de verdeling is.
    \begin{itemize}
      \item Een normale verdeling heeft een kurtosis = 0
      \item Een vlakke distributie heeft een negatieve kurtosis
      \item Een eerder piekvormige distributie heeft een positieve kurtosis
      \item Let op: bij de originele definitie van kurtosis (zoek die eens op!) heeft de normale verdeling een kurtosis van 3. Wij gebruiken hier een alternatieve definitie, meestal de ``excess kurtosis'' genoemd, waar men 3 aftrekt van de originele waarde, zodat je op 0 uitkomt.
    \end{itemize}
  \item Berken de \emph{Skewness} (scheefheid): duidt aan hoe symmetrisch de data is.
    \begin{itemize}
      \item Een symmetrische distributie heeft een skewness = 0
      \item Bijgevolg: een normale verdeling heeft een skewness = 0.
      \item Een distributie met een lange linkerstaart heeft een negatieve skewness
      \item Een distributie met een lange rechterstaart heeft een positieve skewness
      \item Vuistregel: absolute waarde van skewness $>1$, geen symmetrische distributie.
    \end{itemize}
\end{enumerate}

\begin{definition}[QQ-plot of normaliteitsplot]
  \label{def:qq-plot}
  Een normaliteitsplot of QQ-plot\footnote{Q staat hier voor \emph{quantile}, kwantiel} voor een gegevens\-verzameling is een spreidingsdiagram met de gesorteerde gegevenswaarden op de ene as en de bijbehorende verwachte $z$-waarden van een standaardnormale verdeling op de andere as. Zie figuur~\ref{fig:qqplot} voor enkele voorbeelden. De R-code voor het genereren van deze afbeeldingen is hieronder gegeven.
\end{definition}

\begin{figure}
  \begin{center}
    \includegraphics[width=.45\textwidth]{sampling-qqplot-good}
    \includegraphics[width=.45\textwidth]{sampling-qqplot-bad}
  \end{center}
  \caption{De QQ-plot links is gebaseerd op een steekproef van 50 observaties uit een normale distributie met gemiddelde 1000 en standaardafwijking 50. De rechterplot is gebaseerd op een Student-$t$ distributie met 15 vrijheidsgraden. Het aantal observaties, gemiddelde en standaardafwijking zijn hetzelfde als links.
    De lijnen in het rood duiden aan waar zich in theorie de observaties zouden moeten bevinden. Links is dat min of meer zo, maar rechts wijken de observaties af, vooral in de extremen.}
  \label{fig:qqplot}
\end{figure}

\lstinputlisting{data/qqplot.R}

%\subsection[Chi-kwadraatverdeling]{$\chi^{2}$ verdeling}

% TODO:  Deze sectie hoort hier niet (meer) thuis. Weglaten en controleren of de inhoud nog in het relevante hoofdstuk staat

%Laat $X_{1}, X_{2}, \dots X_{v}$ onafhankelijk standaardnormale variabelen zijn ($\sim N(0,1)$). De $\chi^{2}$ (chi-kwadraat) variabele wordt als volgt gedefinieerd:
%\[ \chi^{2}_{v} = X_{1}^{2} + X_{2}^{2} + \dots + X_{v}^{2} \]
%
%Het getal $v$ noemt men het aantal vrijheidsgraden van de variabele. $\chi^{2}$ is een continue toevalsveranderlijke, die positief is omdat ze de som is van kwadraten. Haar dichtheidsfunctie is de volgende:
%
%\[ f_{n}(x) = \frac{1}{2^{\frac{n}{2}}\Gamma(\frac{n}{2})} x^{\frac{n}{2} -1} e^{\frac{x}{2}} \]
%
%De verwachtingswaarde (= gemiddelde) is $v$ en zijn variantie is $2v$. Zijn modus voor $v \geq 2$ is $v-2$.
%
%De $\chi^{2}$ variabele komt niet in de natuur voor. Geen verschijnsel kan erdoor gemodelleerd worden. maar deze variabele zal zeer belangrijk zijn in het vervolg van de cursus.

\section{Centrale limietstelling}
\label{sec:centrale-limietstelling}

\begin{definition}[Lineaire combinatie van onafhankelijke, gelijk verdeelde stochasten]
Formeel: Een lineaire combinatie van onafhankelijke, gelijk verdeelde stochasten is steeds normaal verdeeld.

\[X_{i} \sim Nor(\mu_{i}, \sigma_{i}) \Rightarrow Y = \sum_{i} \alpha_{i} X_{i} \textnormal{ ook normaal verdeeld} \]

Bijgevolg zal ook het steekproefgemiddelde van een steekproef uit een populatie met een willekeurige verdeling, nagenoeg normaal verdeeld zijn voor een voldoende grote $n$.
\end{definition}

Wanneer men dus een aselecte steekproef neemt van onafhankelijke variabelen met een normale verdeling, dan zegt de centrale limietstelling dat het gemiddelde van deze steekproef bij benadering normaal verdeeld zal zijn. Dus als men steeds opnieuw een steekproef neemt met dezelfde grootte, en telkens het gemiddelde optekent, bekomt men bij benadering de grafiek van een normale verdeling. Hoe groter de steekproef, hoe beter de benadering. Het steekproefgemiddelde is dus normaal verdeeld, onafhankelijk van de onderliggende verdeling van de grootheid waarvan men een steekproef neemt. Algemeen kunnen we volgende stelling poneren:

\begin{definition}[Centrale limietstelling]
Beschouw een aselecte steekproef van $n$ waarnemingen die uit een populatie met verwachtingswaarde $\mu$ en standaardafwijking $\sigma$ wordt genomen. Als $n$ groot genoeg is zal de kansverdeling van het steekproefgemiddelde $\overline{x}$ een normale verdeling benaderen met verwachting $\mu_{\overline{x}} = \mu$ en standaardafwijking $\sigma_{\overline{x}} = \frac{\sigma}{\sqrt{n}}$. Hoe groter de steekproef is, des te beter zal de kansverdeling van $\overline{x}$ de verwachtingswaarde van de populatie benaderen.

\end{definition}

Bij het afnemen van een steekproef is zelden de onderliggende verdeling gekend, en toch kan men uitspraken doen over de gemiddelde waarde. Dit is volledig te danken aan de centrale limietstelling, die dit gemiddelde een regel oplegt los van de onderliggende kansverdeling. De centrale limietstelling houdt het steekproefgemiddelde in bedwang, sluit het op in de Gaussische kooi waaruit het nooit kan ontsnappen. Dit, en alleen dit, laat wetenschappers toe het nauwkeurig te bestuderen, te observeren en stelt hen in staat te concluderen.

Want, mocht de verdeling van het steekproefgemiddelde afhankelijk zijn van de onderliggende verdeling, een resultaat dat men tot op zekere hoogte zelfs zou verwachten, zou het onmogelijk zijn om concrete uitspraken te doen over vele wetenschappelijke resultaten. In de theoretische statistiek duiken vrijwel constant limieten van steekproefgemiddeldes op, en deze kunnen dankzij de centrale limietstelling zonder verpinken vervangen worden door een normale verdeling. Zou dit niet mogelijk zijn, dan zou de ganse theorie rond het schatten van parameters in elkaar storten wat dan weer rampzalig zou zijn voor de praktijk. Onderzoeken vergelijken zou herleid worden tot een quasi onmogelijke opgave, en de statistiek in het algemeen zou veel lastiger en ingewikkelder worden.

\subsection{Toepassing van de centrale limietstelling}
Bij het trekken van een aselecte steekproef van omvang $n$ uit een populatie met (onbekend) gemiddelde $\mu$ en standaarddeviatie $\sigma$ is de kansverdeling van het steekproefgemiddelde een kansvariabele $M \sim N (\overline{x}, \frac{\sigma}{\sqrt{n}})$, op voorwaarde dat de steekproefomvang voldoende groot is.

\begin{example}
  We bekijken nu de reactiesnelheid van al onze superhelden en uit onze steekproef met $n = 100$ en $\overline{x} = 90, \sigma = 60$ (miliseconden). Dan kunnen we ons de vraag stellen: wat is de kans dat de gemiddelde reactiesnelheid van een superheld minder is dan $104 ms$?


  \begin{enumerate}
    \item De kansvariabele hier is de gemiddelde reactiesnelheid $\overline{x}$ in een steekproef van $n=100$ superhelden. Daarom geldt wegens de centrale limietstelling:
    \[ \overline{x} \sim Nor(\mu = 90, \sigma_{\overline{x}} = \frac{60}{\sqrt{100}} = 6) \]
    \item We kunnen hierbij de passende $z$-score bepalen:
    \[ z = \frac{104-90}{\frac{60}{\sqrt{100}}} = \frac{104-90}{6} = 2,33 \]
    Dus geldt : $P(\overline{x} < 104) = P(Z < 2,33) = 1 - 0,0099 \approx 0,99$
  \end{enumerate}
\end{example}

\subsection{Schatten van een parameter}
Indien we nu een steekproef onderzoeken, willen we uit de berekening op de steekproef een aantal conclusies kunnen trekken met betrekking tot de populatie. We willen bijvoorbeeld de gemiddelde kracht kennen van een superheld of de fractie superhelden die rijk zijn. Als we een schatting geven voor dergelijke onbekende parameter, noemen we dat ook een puntschatter. We gebruiken bijvoorbeeld $\overline{x}$ als schatter om $\mu$ te schatten.

\begin{definition}[puntschatter]
  Een puntschatter voor een populatieparameter is een regel of een formule die ons zegt hoe we uit de steekproef een getal moeten berekenen om de populatieparameter te schatten. Een puntschatter is dus een steekproefgrootheid.
\end{definition}

\subsection{Betrouwbaarheidsinterval populatiegemiddelde bij grote steekproef}
\label{ssec:betrouwbaarheidsinterval-grote-steekproef}

In het geval het schatten van een gemiddelde van een populatie uit een steekproef hebben we totaal geen idee over hoe correct onze schatting is. Daarvoor gaan we op zoek naar een interval waarvan we met een bepaalde zekerheid, bv. 95\%, kunnen zeggen dat het de te schatten karakteristiek bevat.

\begin{definition}[Betrouwbaarheidsinterval]
Een betrouwbaarheidsinterval is een regel of een formule die ons zegt hoe we uit de steekproef een interval moeten berekenen dat de waarde van de parameter met een bepaalde hoge waarschijnlijkheid bevat.
\end{definition}

Een eerste goede schatting voor populatiegemiddelde zou het steekproefgemiddelde zijn:

\[ \overline{x} = \frac{1}{n} \sum_{i} x_{i} \]

Natuurlijk is deze schatting niet de werkelijke waarde van de populatie. Daarom wordt vaak rondom $\overline{x}$ een interval geconstrueerd dat de waarden bevat die aannemelijk zijn voor $\mu$. Hiervoor kunnen we gebruik maken van de centrale limietstelling: het gemiddelde in een te trekken steekproef van omvang $n$ is normaal verdeeld met karakteristieken $\mu$ en $\frac{\sigma}{\sqrt{n}}$.  Als we nu het gemiddelde standaardiseren krijgen we:

\[ Z = \frac{\overline{x} - \mu}{\frac{\sigma}{\sqrt{n}}} \]

Deze uitdrukking hangt van $\mu$ af maar we weten wel dat deze standaardnormaal verdeeld is. We kunnen daarom getallen $-z$ en $z$ vinden, onafhankelijk van $\mu$, waartussen $Z$ met een gekozen kans $1 - \alpha$ ligt. Deze kans $1 - \alpha$ wordt het \emph{betrouwbaarheidsniveau}\index{betrouwbaarheidsniveau}\index{niveau!betrouwbaarheids-} genoemd. We nemen hier $1 - \alpha= 0,95$.

\[P(-z < Z < z) = 1 - \alpha = 0,95 \]

Hieruit halen we dat $\alpha = 0,05$. Door het toepassen van de symmetrieregel weten we dus dat we volgende term moeten berekenen:

\[ P( Z < z) = 0,025 \]

Kijken we in de Z-tabel dan vinden we voor de rechterstaartkans $0,025$ de z-score van $1,96$.

Dus vinden we :

\[ P( -1,96 < \frac{\overline{x} - \mu}{\frac{\sigma}{\sqrt{n}}} < 1,96 ) \]
en dus
\[ P ( \overline{x} -1,96 \frac{\sigma}{\sqrt{n}} <\mu < \overline{x} + 1,96 \frac{\sigma}{\sqrt{n}}) \]

Op die manier kunnen we dus grenzen bepalen die een interval aanduidt waar 95\% kans is dat $\mu$ gevonden wordt. Formeel: als je herhaalde steekproeven zou nemen en telkens op basis van het gerealiseerde steekproefgemiddelde $\overline{x}$ een betrouwbaarheidsinterval zou maken, dan zal bij 95\% van de intervallen $\mu$ binnen de intervalgrenzen liggen.

Opgelet, we gaan er hier van uit dat we de standaarddeviatie van de populatie kennen, wat meestal niet zo is. Indien de steekproef groot genoeg is, kunnen we de steekproefstandaarddeviatie nemen als schatter voor de standaarddeviatie voor de populatie.

\[ P ( \overline{x} -1,96 \frac{\sigma_{\overline{x}}}{\sqrt{n}} < \mu < \overline{x} + 1,96 \frac{\sigma_{\overline{x}}}{\sqrt{n}}) \]


\begin{figure}[t]
\centering
\begin{tikzpicture}
\begin{axis}[
  domain=-3:3, samples=100,
  axis lines*=left, xlabel=$z$,
  every axis y label/.style={at=(current axis.above origin),anchor=south},
  every axis x label/.style={at=(current axis.right of origin),anchor=west},
  height=5cm, width=12cm,
  xtick={-1.96,0,1.96}, ytick=\empty,
  enlargelimits=false, clip=false, axis on top,
  grid = major
  ]
  \addplot [fill=cyan!20, draw=none, domain=-3:3] {gauss(0,1)} \closedcycle;
  \draw [yshift=-0.6cm, latex-latex](axis cs:-1.96,0) -- node [fill=white] {$\sigma$} (axis cs:1.96,0);
\end{axis}
\end{tikzpicture}
\caption{Standaardnormale verdeling die 95\% betrouwbaarheidsinterval aanduidt.}
\label{fig:verdelingStandaardnormaal}
\end{figure}

\subsection{Betrouwbaarheidsinterval populatiegemiddelde bij een kleine steekproef}
\label{ssec:betrouwbaarheidsinterval-kleine-steekproef}

Bij kleine steekproeven kunnen we niet langer veronderstellen dat de kansverdeling van $\overline{x}$ bij benadering
normaal verdeel is, omdat de centrale limietstelling alleen normaliteit garandeert voor grote steekproeven ($n >30$). De vorm
van de kansverdeling van het steekproefgemiddelde $\overline{x}$ hangt nu af van de vorm van de verdeling van de populatie waaruit de
steekproef genomen wordt. Alhoewel nog steeds geldt dat $\sigma_{\overline{x}} = \frac{\sigma}{\sqrt{n}}$ kan
de standaardafwijking $s$ een slechte benadering zijn voor $\sigma$ als de steekproef klein is.

Als oplossing kunnen we een nieuwe grootheid bepalen. In plaats van

\[ z = \frac{\overline{x} - \mu}{\frac{\sigma}{\sqrt{n}}} \]

construeren we

\[ t = \frac{\overline{x} - \mu}{\frac{s}{\sqrt{n}}} \]

Deze heeft een kansverdeling die beschreven wordt door een Student-t verdeling. Deze lijkt zeer goed op de normale verdeling: klokvormig, symmetrisch en met verwachtingswaarde 0.

De precieze vorm van de kansverdeling $t$ hang af van de steekproefomvang $n$. We zeggen dat de t-verdeling $(n-1)$ vrijheidsgraden heeft (afgekort $df$).
Merk op dat:
\begin{itemize}
  \item $(n-1)$ ook gebruikt werd om $s^{2}$ te berekenen
  \item als $n \rightarrow \infty$ we de standaardnormale verdeling verkrijgen.
\end{itemize}

Indien we nu een betrouwbaarheidsinterval willen bepalen voor een steekproef met een klein aantal waarden moeten we het volgende doen:

\begin{definition}[Betrouwbaarheidsinterval kleine steekproef]
  Om een betrouwbaarheidsinterval voor het gemiddelde te bepalen op basis van een klein steekproef bepalen we:
  \[ \overline{x} \pm t_{\frac{\alpha}{2}}(\frac{s}{\sqrt{n}}) \]
  waarbij $t_{\frac{\alpha}{2}}$ gebaseerd is op $(n-1)$ vrijheidsgraden. We veronderstellen wel dat we een aselecte steekproef genomen hebben uit
  een populatie die bij benadering normaal verdeeld is.
\end{definition}

\begin{table}
  \centering
  \begin{tabular}{ll}
  	\textbf{Functie} & \textbf{Betekenis}                                             \\ \midrule
  	\verb|pt(x, df)| & Linkerstaartkans, $P(X<\mathtt{x})$                            \\
  	\verb|dt(x, df)| & Hoogte van de curve op punt \texttt{x}                         \\
  	\verb|qt(p, df)| & Onder welke grens zal \texttt{p}\% van de waarnemingen liggen? \\
  	\verb|rt(n, df)| & Genereer \texttt{n} random getallen volgens deze verdeling
  \end{tabular}

  \caption{Kansberekeningsfuncties in R voor de Student-$t$ verdeling met \texttt{df} vrijheidsgraden, verwachte waarde 0 en standaardafwijking 1.}
  \label{tab:t-prob-r}
\end{table}

\subsection{Betrouwbaarheidsinterval voor populatiefractie bij een grote steekproef}
\label{ssec:betrouwbaarheidsinterval-populatiefractie}

Indien je een variabele wil meten als een fractie, bijvoorbeeld \% mensen die ja geantwoord heeft op een bepaalde vraag, dan willen we in feite de kans $p$ op succes in een bernouilli experiment schatten, waarbij $p$ de kans is dat een willekeurig geselecteerde respondent (of element van de populatie) een succes is (succes in termen van binomiaal experiment). We kunnen $p$ dan schatten door bijvoorbeeld:

\[ \overline{p} = \frac{\textnormal{aantal successen}}{n} \]

Om nu de betrouwbaarheid van de schatter $\overline{p}$ te bepalen moeten we de kansverdeling kennen van $\overline{p}$. Dit kunnen we beredeneren door toepassing van de centrale limietstelling op het gemiddelde aantal successen in de steekproef van omvang $n$. Indien succes = 1 en faling = 0, dan hebben we een steekproef van $n$ elementen, ieder met dezelfde verdeling (kans op 1 is $p$ en kans op 0 is $q=1-p$).  Het gemiddelde $\overline{p}$ heeft dan bij benadering een normale verdeling. Of dus:

\begin{itemize}
  \item Verwachting van kansverdeling van $\overline{p}$ is $p$.
  \item De standaardafwijking van kansverdeling $\overline{p} = \sqrt{\frac{pq}{n}}$
  \item Voor grote steekproeven is $\overline{p}$ bij benadering normaal verdeeld.
\end{itemize}

Aangezien $\overline{p}$ een steekproefgemiddelde is van het aantal successen, stelt dit ons in staat een betrouwbaarheidsinterval te berekenen analoog als die voor de intervalschatting van $\mu$ voor grote steekproeven.

\begin{definition}[Betrouwbaarheidsinterval voor $p$ gebaseerd op grote steekproef]
  \[ \overline{p} \pm z_{\frac{\alpha}{2}} \sqrt{\frac{\overline{p}\overline{q}}{n}} \]
  met $\overline{p} = \frac{x}{n}$ en $\overline{q} = 1- \overline{p}$
\end{definition}


\section{R}
We kijken naar enkele basisoperaties die verband houden met enkele distributies. Er zijn een groot aantal verdelingen beschikbaar, maar we kijken maar naar een paar. Als u wilt weten welke distributies beschikbaar zijn, kunt u een zoekopdracht uitvoeren met behulp van de opdracht

\begin{lstlisting}
> help.search ("distribution").
\end{lstlisting}


Hier geven we details over de commando's die verband houden met de normale distributie en vermelden kort de commando's voor andere distributies. De functies voor verschillende verdelingen zijn zeer vergelijkbaar.

De prefixen zijn als volgt:
\begin{description}
	\item[d] geeft de hoogte van de respectievelijke kansdichtheidsfunctie
	\item[p] geeft de cumulatieve kansdichtheidsfunctie
	\item[q] geeft de omgekeerde cumulatieve dichtheidsfunctie
	\item[r] geeft een willekeurige waarde
\end{description}

\subsection{De normale verdeling}
Er zijn vier functies die kunnen worden gebruikt om de waarden geassocieerd met de normale distributie te genereren.
\subsubsection{dnorm}

De eerste functie waarnaar we kijken, is \texttt{dnorm}. Gegeven een waarde geeft het de hoogte van de kansverdeling op elk punt terug. Als u alleen de punten zonder gemiddelde en standaardafwijking ingeeft wordt een gemiddelde van nul en standaardafwijking van 1 beschouwd. Er zijn opties om verschillende waarden voor de gemiddelde en standaardafwijking te gebruiken.

\lstinputlisting{data/norm.R}

\subsubsection{pnorm}

Dit is de cumulatieve kansdichtheidsfunctie, of anders gezegd de linkerstaartkans: \texttt{pnorm(x)} is $P(Z < x)$.

\subsubsection{qnorm}
De volgende functie die we bekijken is \texttt{qnorm}, die de inverse van \texttt{pnorm} is. Het idee achter \texttt{qnorm} is dat je het een kans $\alpha$ geeft, en het geeft het getal weer waarvan de cumulatieve distributie overeenkomt met de waarschijnlijkheid $\alpha$.

\subsubsection{rnorm}

\lstinputlisting{data/qnorm.R}
De laatste functie die we onderzoeken is de \texttt{rnorm} functie die willekeurige getallen kan genereren waarvan de distributie normaal is. Het argument dat je ingeeft is het aantal willekeurige getallen dat u wilt, met optionele argumenten om de gemiddelde en standaardafwijking op te geven:

\lstinputlisting{data/rnorm.R}

\section{Oefeningen}
\label{sec:steekproefonderzoek-oefeningen}

\begin{exercise}
  Een onderzoeker wil zo correct mogelijk de consumptiegewoontes van de inwoners van 18 jaar en ouder in een bepaalde gemeente, met 3 woonkernen, onderzoeken.  Hij onderscheidt 4 leeftijdsgroepen zodat hij uiteindelijk aan 12 deelgroepen komt. Hij vraagt de procentuele samenstelling van de bevolking op in de gemeente en berekent daaruit hoeveel bevragingen hij per deelgroep moet uitvoeren.  Dit noemen we een \emph{quotasteekproef}.

  Vragen:
  \begin{enumerate}[label=\alph*.]
    \item Wat zijn de voor- en nadelen?
    \item Welke soort fouten kunnen hier gemaakt worden?
    \item Welke andere parameters zouden kunnen gebruikt worden bij het opsplitsen in deelgroepen?
  \end{enumerate}
\end{exercise}

\begin{exercise}
  Een onderzoeksbureau wil het aankoopgedrag van wasproducten nagaan. Men beslist een aantal vragen te stellen aan vrouwen tussen de 25 en 55 jaar omdat men ervan uitgaat dat de relevante populatie uit deze categorie consumenten bestaat.

  Vraag:

  \begin{enumerate}[label=\alph*.]
    \item Welke fout wordt hier gemaakt?
    \item Hoe groot is de impact van deze fout?
  \end{enumerate}
\end{exercise}

\begin{exercise}
  	De vakbonden willen een onderzoek doen naar de werkomstandigheden van de werknemers van een IT-bedrijf. Dat bedrijf heeft in totaal 3200 werknemers die verdeeld zijn over
  12 vestigingen. Omdat het aantal werknemers groot is worden aselect 40 werknemers gekozen per vestiging. De steekproefomvang is dus $n = 480$.
  \begin{enumerate}[label=\alph*.]
    \item Welk bezwaar kan tegen deze steekproefprocedure worden gebracht?
    \item Wanneer zou dit geen bezwaar zijn?
  \end{enumerate}
\end{exercise}

\begin{exercise}
  We willen een onderzoek voeren naar onze studenten aan de Hogeschool Gent, faculteit Bedrijf en Organisatie. Hiervoor worden de aanwezige studenten in een bepaald opleidingsonderdeel bevraagd.

  \begin{enumerate}[label=\alph*.]
    \item Welke kritiek kan je op deze methode geven?
    \item Stel dat de aanwezige docent een kernvak geeft, zeer streng is en tijdens de bevraging rondloopt. Welk bezwaar kan hier gegeven worden?
    \item Stel dat de bevraging niet tijdens een les, maar na een examen gehouden wordt. Welke kritiek kan je op deze methode geven?
  \end{enumerate}
\end{exercise}

\begin{exercise}
  \label{ex:prob-norm-dist}
  Bereken ook elke keer het gevraagde gebied.
  \begin{enumerate}[label=\alph*.]
    \item $P(Z < 1.33)$
    \item $P(Z > 1.33)$
    \item $P(Z < -1.33)$
    \item $P(Z > -1.33)$
    \item $P(Z < 0.45)$
    \item $P(Z > -1.05)$
    \item $P(Z < 0.65)$
    \item $P(-0.45 < Z < 1.20)$
    \item $P(-1.35 < Z < -0.10)$
    \item $P(-2.10 < Z < -0.90)$
  \end{enumerate}
\end{exercise}

\begin{exercise}
	Bepaal de dichtheid en de cumulatieve waarschijnlijkheidscurve voor een normale verdeling met een gemiddelde $\mu$
	van 2,5 en $\sigma = 1,5$. Bepaal de oppervlakte voor het gebied onder de dichtheidscurve tussen
	$x = 0.5$ en $x = 4$. Controleer uw antwoord door de berekening te doen.
\end{exercise}

\begin{exercise}
	Bepaal de dichtheid en de cumulatieve waarschijnlijkheidscurve voor een t-verdeling met $df = 3$. Teken ook een normale verdeling met een $\mu = 0$  en $\sigma = 1$.
\end{exercise}

\begin{exercise}
Gebruik de functie \verb|rnorm()| een willekeurige steekproef van 25 waarden uit een normale verdeling te tekenen met een gemiddelde van 0 en een standaardafwijking gelijk aan 1,0. Gebruik een histogram, met \verb|probability = TRUE|.

Maak een overlay over het histogram met: (a) de theoretische dichtheidscurve voor een normale verdeling met gemiddelde 0 en standaardafwijking gelijk aan 1,0; (b) een ``geschatte'' dichtheidscurve op basis van het gemeten steekproefgemiddelde en -standaardafwijking.

Herhaal dit voor een steekproef van 100 en 500 waarden.
\end{exercise}

\begin{exercise}
  In de  Hogeschool zijn er twee klassen voor het vak onderzoekstechnieken. De studenten werden willekeurig over de klassen verdeeld, zodat we mogen veronderstellen dat de ene klas niet slimmer is dan de andere. In de A-klas geeft mevr. X les, in de B-klas geeft mr. Y les. X is nogal streng en op het einde van het schooljaar behaalt haar klas een gemiddelde van 54 op 100 met een standaardafwijking van 11.

  Y is iets losser en stimuleert de leerlingen al gauw met een puntje meer. Op het einde van het schooljaar behaalt zijn klas een gemiddelde van 62 op 100 en een standaardafwijking van 7.

  Wouter zit in de A-klas en heeft $\frac{63}{100}$ voor wiskunde. Stijn zit in de B-klas en behaalt $\frac{67}{100}$. Wie heeft volgens jou het beste gescoord binnen de eigen klas?
\end{exercise}

\begin{exercise}
  Een gezondheidsonderzoek tussen 1988 en 1994 gaf aan dat de gemiddelde cholesterolwaarde bij vrouwen tussen 20 en 29 jaar 183 mg/dl bedroeg, met een standaardafwijking gelijk aan 36. We nemen nu een aselecte steekproef van 81 vrouwen. Los volgende vragen op:

  \begin{enumerate}[label=\alph*.]
    \item Schets de kansdichtheidsfunctie voor de populatie en de kansverdeling van het steekproefgemiddelde $\overline{x}$.
    \item Bepaald de kans dat $\overline{x}$ kleiner is dan 185.
    \item Bepaal de kans dat $\overline{x}$ tussen 175 en 185 ligt.
    \item Bepaal de kans dat $\overline{x}$ groter is dan 190.
  \end{enumerate}
\end{exercise}

\begin{exercise}
  Een aselecte steekproef van 64 stuks wordt getrokken uit een populatie met onbekende verdeling. De verwachting en de standaardafwijking van de populatie
  zijn wel gekend: $\mu = 20$ en $\sigma=16$. Los volgende vragen op:

  \begin{enumerate}[label=\alph*.]
    \item Bepaal de verwachting en standaardafwijking van het steekproefgemiddelde.
    \item Beschrijf de vorm van de verdeling van het steekproefgemiddelde. In hoeverre hangt je antwoord af van de grootte van de steekproef?
    \item Bereken de $z$ score bij $\overline{x_{1}} = 15.5$ en $\overline{x_{2}} = 23$.
    \item Bepaal kans dat $\overline{x} <16$.
    \item Bepaal kans dat $\overline{x} > 23$.
    \item Bepaal kans dat $16< \overline{x}< 22$.
  \end{enumerate}
\end{exercise}

\begin{exercise}
  Verkeersdrempels zijn bedoeld om de snelheid van automobilisten te be\"invloeden. Afhankelijk van de gewenste snelheid in een straat worden de drempels steiler of minder steil gemaakt. Drempel A is zo ontworpen dat 85 \% van de automobilisten de drempel passeert met een snelheid van minder dan 50 km per uur. In de praktijk blijkt dat de passeersnelheid bij een drempel normaal verdeeld is. Bij drempel A werd een gemiddelde passeersnelheid van 43,1 km/h gevonden met standaardafwijking 6,6 km/h.

  \begin{enumerate}[label=\alph*.]
    \item Toon aan dat 85\% van de automobilisten niet harder dan 50 km/h rijdt.
    \item Bij hoeveel van de 1200 metingen kan, op grond van eerdere ervaringen, een snelheid van meer dan 55 km/h worden verwacht?
  \end{enumerate}
\end{exercise}

\begin{exercise}
  Gegeven 20 examenresultaten in Tabel~\ref{tab:examen}. Uit resultaten van de laatste jaren blijkt dat $\sigma = 2.45$.

  \begin{enumerate}[label=\alph*.]
    \item Wat is $\sigma_{\overline{x}}$ , de standaardafwijking van $\overline{x}$?
    \item Geef het 92\% betrouwbaarheidsinterval voor $\mu$.
    \item Kunnen we er zeker van zijn dat het gemiddeld resultaat minder dan 12.5 bedraagt?
  \end{enumerate}
\end{exercise}

\begin{table}
  \centering
  \begin{tabular}{llllllllll}
    11.5 & 16.5 & 11 & 17.3 & 10.8 & 5.6  & 13.1 & 11.5 & 14.2 & 12.9 \\
    8.7  & 9.2  & 15 & 14.4 & 10   & 10.3 & 18.3 & 12.9 & 14.2 & 8.7
  \end{tabular}
  \caption{Examenresultaten}
  \label{tab:examen}
\end{table}

\begin{exercise}
  Een schoenhandelaar voert een marktonderzoek uit bij 500 klanten. Daaruit blijkt dat 30\% van hen minstens éénmaal per jaar sportschoenen koopt.  Op basis van secundaire informatie weet hij dat het nationaal gemiddelde op 26\% ligt.  Hij vraagt zich nu af in hoeverre zijn zaak in dat opzicht afwijkt van de nationale norm? (We werken met $\alpha= 5\%$, tweezijdig.)
\end{exercise}

\begin{exercise}
  Een conservenfabrikant krijgt de laatste tijd klachten over de netto inhoud van zijn conserven met wortelen en erwtjes, die volgens de verpakking netto 1 liter zouden moeten bevatten. Daarom laat hij een steekproef nemen waarin de netto inhoud van 40 willekeurig gekozen blikjes wordt gecontroleerd. De resultaten worden samengevat in Tabel~\ref{tab:Steekproefwaarden}.

Vraag A:
\begin{itemize}
  \item Vul de tabel aan met de cumulatieve absolute frequentie
  \item Vul de tabel aan met de relatieve frequentie
  \item Vul de tabel aan met de cumulatieve relatieve frequentie.
\end{itemize}
Vraag B:

\begin{itemize}
  \item Bereken het gemiddelde
  \item Bereken de standaardafwijking
  \item Hoeveel procent van de blikken bevatten te weinig wortelen en erwtjes.
  \item Teken een histogram van de absolute frequentie.
  \item Zijn de gegevens normaal verdeeld?  Hoe zie je dat?
\end{itemize}

\end{exercise}

  \begin{table}
  \centering
  \begin{tabular}{lr}
    \toprule
    Inhoud & $n_{i}$ \\
    \midrule
    $[970,980[$ & 3 \\
    $[980,990[$ & 5 \\
    $[990,1000[$ & 13 \\
    $[1000,1010[$ & 11 \\
    $[1010,1020[$ & 5 \\
    $[1020,1030[$ & 3 \\
    \bottomrule
  \end{tabular}
  \caption{Steekproefwaarden}
  \label{tab:Steekproefwaarden}
\end{table}

\begin{exercise}
  Een webhostingfirma heeft een Service Level Agreement met een klant voor een gegarandeerde uptime van ``five nines'' (99,999\%).  Die wordt aan het einde van elk jaar gecontroleerd en als de minimale uptime niet gehaald wordt, moet de hostingfirma een boete betalen.

  Om de uptime te meten, voert een monitoringsysteem elke minuut een \texttt{HTTP GET /} uit en controleert het resultaat a.h.v.
  de HTTP return code. In de maand januari is er één enkele HTTP request onsuccesvol geweest.

  \begin{itemize}
    \item Als deze trend zich voortzet, wat is de kans dat de SLA niet gehaald wordt aan het einde van het jaar? Gebruik de formule voor de kansverdeling van een fractie.
    \item De gebruikte formule is eigenlijk niet geschikt in dit specifieke geval en geeft een vertekend beeld. Wat zou de reden kunnen zijn?
  \end{itemize}
\end{exercise}

\section{Antwoorden op geselecteerde oefeningen}
\label{sec:oplossingen-steekproefonderzoek}

\paragraph{Oefening \ref{ex:prob-norm-dist}}

\begin{enumerate}[label=\alph*.]
  \item $0,908$
  \item $0,092$
  \item $0,092$
  \item $0,908$
  \item $0,674$
  \item $0,853$
  \item $0,742$
  \item $0,559$
  \item $0,372$
  \item $0,166$
\end{enumerate}

\chapter{Toetsingsprocedures}

In de voorbije hoofdstukken hebben we gezien hoe we aan de hand van steekproefonderzoek bepaalde kerngetallen over een populatie kunnen berekenen, bijvoorbeeld aan de hand van puntschatters of betrouwbaarheidsintervallen. We kunnen deze informatie ook gebruiken om bepaalde hypothesen over een populatie te toetsen. Een hypothese is een veronderstelling waarvan nog bewezen moet worden dat ze correct is. Het doel van een toetsingsprocedure is het testen van een hypothese omtrent de waarden van 1 of meerdere populatieparameters.

\begin{definition}[Statistische hypothese.]
  Een statische hypothese\index{hypothese!statistische} is een uitspraak over de numerieke waarde van een populatieparameter.
\end{definition}

Voorbeelden van hypothesen:

\begin{itemize}
  \item Gemiddeld redt een superheld minstens 3,3 mensen per dag.
  \item De gemiddelde lengte van een superheld is minstens 120 cm.
  \item \dots
\end{itemize}

In dit hoofdstuk gaan we de algemene theorie over toetsen formuleren aan de hand van het testen van hypothesen over het populatiegemiddelde $\mu$, de $z$-toets. Naast de $z$-toets bestaan er echter nog vele andere statistische hypothesetoetsen die in specifieke situaties gebruikt kunnen worden. De meest geschikte statistische toets hang o.a.~af van de populatiegrootheid in kwestie (gemiddelde, standaardafwijking, enz.), en veronderstellingen over de onderliggende stochastische verdeling van de populatie (normaal verdeeld of niet, enz,).

\section{Elementen van een hypothesetoets}
\label{sec:elementen-hypothesetoets}

Algemeen gezien bestaat een toetsingsprocedure uit 4 elementen:
\begin{enumerate}
  \item \textbf{Nulhypothese}\index{nulhypothese}\index{hypothese!nul-} $H_{0}$: Deze hypothese proberen we te ontkrachten door een redenering in het ongerijmde. We gaan deze hypothese accepteren, tenzij de observaties uit de steekproef overtuigend wijzen op het tegendeel.
  \item \textbf{Alternatieve hypothese}\index{hypothese!alternatieve} $H_{1}$: Dit is meestal de hypothese die de onderzoeker wil bewijzen. Deze hypothese zal echter alleen worden geaccepteerd als de observaties uit de steekproef overtuigend wijzen op de juistheid ervan.
  \item \textbf{Teststatistiek}: De veranderlijke die berekend wordt uit de steekproef
  \item Aanvaardings- en kritiek gebied:
  \begin{itemize}
    \item  \textbf{Aanvaardingsgebied\index{aanvaardingsgebied}\index{gebied!aanvaardings-}}: Het gebied van waarden die de nulhypothese $H_{0}$ ondersteunt
    \item \textbf{Verwerpingsgbied\index{verwerpingsgebied}\index{gebied!verwerpings-}}: gebied dat waarden bevat die de nulhypothese verwerpen. Ook kritiek gebied\index{gebied!kritiek} genoemd.
  \end{itemize}
\end{enumerate}

Een alternatief voor de laatste stap is het berekenen van de \emph{overschrijdingskans} (zie verder).

De beslissing om de nulhypothese $H_{0}$ te verwerpen of te aanvaarden is gebaseerd op informatie uit een steekproef, getrokken uit de populatie waarover de hypothese is geformuleerd. De steekproefwaarden worden gebruikt om 1 enkele waarde van een teststatistiek te berekenen die de beslissing zal bepalen. Daartoe worden alle waarden die de teststatistiek kan aannemen, verdeeld in twee gebieden, \begin{inparaenum}[(i)] \item het aanvaardingsgebied en \item het verwerpingsgebied\end{inparaenum}. Indien de waarde van de teststatistiek ligt in het verwerpingsgebied, dan wordt de nulhypothese verworpen en de alternatieve hypothese aanvaard. Indien de waarde van de teststatistiek in het aanvaardingsgebied valt dan wordt de nulhypothese aanvaard.

\section{Toetsingsprocedure voor de \texorpdfstring{$z$}{z}-toets}
\label{sec:toetsingsprocedure-z-toets}

In de eerste toetsingsprocedure die we in deze cursus uitwerken, gaan we een uitspraak over het populatiegemiddelde $\mu$ verifiëren. Deze is algemeen bekend als de $z$-toets\index{$z$-toets}\index{toets!$z$-}.

\begin{enumerate}
  \item De vermoedens over de populatie worden vastgelegd in twee hypothesen $H_{0}$ en $H_{1}$.
  \item Het significantieniveau\index{significantieniveau}\index{niveau!significantie-} $\alpha$ en steekproefomvang $n$ worden vastelegd. Je kan $\alpha$ in principe zelf kiezen (bv. 0,05)\footnote{Merk op dat het significantieniveau gerelateerd is aan het betrouwbaarheidsniveau $1-\alpha$. Zie Sectie~\ref{ssec:betrouwbaarheidsinterval-grote-steekproef}}. Hoe dichter het significantieniveau bij 0 ligt, hoe minder twijfel er is over het resultaat van de toets. Maar langs de andere kant wordt het ook moeilijker om de nulhypothese te verwerpen.
  \item De waarde van de toetsingsgrootheid in de steekproef wordt berekend. De uitkomst is bepalend voor de beslissing of we de nulhypothese $H_{0}$ kunnen verwerpen of niet. Vaak kies je de voor de hand liggende grootheid, bijvoorbeeld het steekproefgemiddelde bij een hypothese over populatiegemiddelde. We weten in dat geval dat de kansverdeling van het steekproefgemiddelde $M \sim Nor( \mu, \frac{\sigma}{\sqrt{n}})$.
  \item Het kritieke gebied bepalen, of meer bepaald de grens tussen het aanvaardings- en het verwerpingsgebied. Deze kritieke grenswaarde\index{kritieke grenswaarde} $g$ wordt berekend als:
  
  \begin{equation}
    g = \mu \pm z \times \frac{\sigma}{\sqrt{n}}
    \label{eq:kritieke-grenswaarde}
  \end{equation}
  
  In deze formule hangt de waarde $z$ af van het gekozen significantieniveau. Meer bepaald is $P(Z > z) = \alpha$, of $P(Z < z) = 1 - \alpha$.
  
  Alle waarden die \emph{binnen} het door $g$ bepaalde gebied vormen het aanvaardingsgebied. Waarden erbuiten, die dus ver van het $H_0$ veronderstelde populatiegemiddelde liggen, zijn het verwerpingsgebied.
\end{enumerate}

\begin{example}
  \label{ex:hypothesetoets-dagelijkse-reddingen}
  Algemeen wordt aangenomen dat superhelden stellen gemiddeld 3,3 mensen per dag redt. De onderzoekers krijgen echter gevoel dat dat niet zo is: ze hebben de indruk dat een superheld \emph{meer} dan $3,3$ mensen per dag redt.
  
  Ze gaan dit onderzoeken en voeren een steekproef uit bij $n = 30$ superhelden. In deze steekproef is het gemiddelde $\overline{x} = 3,483$ is. De standaardafwijking in de populatie is verondersteld gekend en is $\sigma = 0,55$.
  
  Kan hieruit besloten worden dat superhelden gemiddeld meer dan 3,3 mensen per dag redt?

  \begin{enumerate}
    \item We veronderstellen dat het aantal mensen dat een superheld redt normaal verdeeld  is en formuleren twee hypothesen omtrent de parameter $\mu$.
    \begin{itemize}
      \item $H_{0}$ = de nulhypothese (hetgeen we willen weerleggen). In dit geval \[ H_{0} : \mu = 3,3 \]
      \item $H_{1}$ = alternatieve hypothese (vermoeden dat we willen aantonen). In dit geval \[H_{1}= \mu > 3,3 \]
    \end{itemize}
  
    We veronderstellen in de redenering initieel dat de nulhypothese $H_{0}$ waar is. Indien het gemiddelde aantal mensen gered per dag $\overline{x}$ van de steekproef sterk afwijkt van de veronderstelde waarde, verwerpen we de nulhypothese $H_{0}$ en aanvaarden we de alternatieve hypothese $H_{1}$.
    
    Wat betekent nu ``sterk afwijken''? Zou je uit een populatie met gemiddelde van $3,3$ gemakkelijk een steekproef kunnen trekken met gemiddelde $3,483$? De centrale limietstelling (zie Sectie~\ref{sec:centrale-limietstelling}) laat ons toe de kans hiertoe te berekenen.
    
    \item Vastleggen significantieniveau $\alpha$ en steekproefomvang $n$. We willen een significantieniveau van 5\% kiezen, dus $\alpha = 0,05$. De steekproefomgang is gegeven en is hier $n = 30$.
    
    \item De waarde van de toetsingsgrootheid in de steekproef bepalen. We nemen hier het steekproefgemiddelde: $\overline{x} = 3,483$
    
    We veronderstellen in de redenering dat de nulhypothese $H_{0}$ waar is en dat we $\sigma$ goed kunnen schatten hebben ($\sigma = 0,55$). Dan geldt voor het gemiddelde $M$ volgens de centrale limietstelling dat:
    
    \[M \sim  Nor(\mu = 3,3; \sigma = \frac{0,55}{\sqrt{30}})\]
    
    De waarde $\overline{x} = 3,483$ bevindt zich erg rechts (zie Figuur~\ref{fig:hypothesetoets-reddingen-per-dag}). $\overline{x}$ ligt zelfs zo ver naar rechts dat de kans (indien $H_{0}$ waar is) om dergelijke geobserveerde waarde te krijgen of groter, zeer klein is. Een dergelijke geobserveerde waarde onder de nulhypothese kan dus moeilijk verklaard worden door louter toeval. Intu\"itief voelen we dus aan dat hoe verder de geobserveerde waarde $\overline{x}$ zich bevindt in de rechtse richting, hoe meer we geneigd zijn om de nulhypothese te verwerpen. Maar wat is te ver en wat niet?
    
    \item De kritieke grenswaarde berekenen. De $z$-waarde voor een significantieniveau van $0,05$ is 1,645\footnote{In R kan je dit berekenen met \texttt{qnorm(1 - 0.05)}}.
    
    \[ g = \mu + z \times \frac{\sigma}{\sqrt{n}} = 3,3 + 1,645 \times \frac{0,5}{\sqrt{30}} \approx 3,45 \]
    
    Het steekproefgemiddelde $\overline{x} = 3,483$ ligt nog verder van $\mu = 3,3$ dan de grenswaarde $g = 3,45$. De kans is heel klein dat zo'n steekproef getrokken wordt uit een populatie met dit gemiddelde. Slechts in 34 steekproeven op 1000 zal een dergelijke gebeurtenis optreden. Met andere woorden, de steekproefwaarde ligt in het verwerpingsgebied. We kunnen dus $H_0$ verwerpen en besluiten met dat superhelden inderdaad \emph{meer} dan 3,3 mensen per dag redden.
  \end{enumerate}

\end{example}

\begin{exercise}
  Kunnen we in Voorbeeld~\ref{ex:hypothesetoets-dagelijkse-reddingen} zomaar veronderstellen dat het gemiddelde normaal verdeeld is? Waarom (niet)?
\end{exercise}

\begin{figure}
  \centering
  \begin{tikzpicture}
    \begin{axis}[
        domain=3:3.6, samples=100,
        axis lines*=left, xlabel=$z$,
        every axis y label/.style={at=(current axis.above origin),anchor=south},
        every axis x label/.style={at=(current axis.right of origin),anchor=west},
        height=5cm, width=12cm,
        xtick={3.3,3.483}, ytick=\empty,
        enlargelimits=false, clip=false, axis on top,
        grid = major
      ]
      \addplot [fill=cyan!20, draw=none, domain=3:3.6] {gauss(3.3,0.101328673)} \closedcycle;
    \end{axis}
  \end{tikzpicture}
  \caption{Verdeling van het aantal mensen dat gemiddeld per dag gered wordt door een superheld (Voorbeeld~\ref{ex:hypothesetoets-dagelijkse-reddingen}). De kansverdeling voor het steekproefgemiddelde is normaal verdeeld met $\mu = 3,3$ en $\sigma = 0,5$. Het steekproefgemiddelde $\overline{x} =3,483$.}
  \label{fig:hypothesetoets-reddingen-per-dag}
\end{figure}

\section{Kritieke gebied}
\label{sec:kritieke-gebied}

De formule voor de berekening van de grenswaarde (zie Formule~\ref{eq:kritieke-grenswaarde}) is gebaseerd op de centrale limietstelling, en meer bepaald betrouwbaarheidsintervallen.

De kritieke grenswaarde vormt een betrouwbaarheidsinterval rond $\mu$ met een gekozen zekerheidsniveau. Als we bijvoorbeeld stellen dat $\alpha = 0.05$, weten we vanuit de centrale limietstelling dat we kunnen verwachten dat als we herhaaldelijk voldoende steekproeven uit deze populatie nemen, in 95\% van de gevallen het steekproefgemiddelde binnen dit betrouwbaarheidsgeval zal liggen.

Als we de redenering omkeren, en een steekproef genomen hebben waar het gemiddelde $\overline{x}$ \emph{niet} binnen dit betrouwbaarheidsinterval ligt, dan is de kans heel klein (kleiner dan 5\%) dat deze uit een populatie getrokken is met het veronderstelde gemiddelde $\mu$. In dat geval kunnen we de nulhypothese dus verwerpen.

In Voorbeeld~\ref{ex:hypothesetoets-dagelijkse-reddingen} is de kritieke grenswaarde het getal $g$ waarvoor geldt dat

\[ P(M > g) = \alpha \]

wat dan wordt verder uitgewerkt als:

\[ P(Z > \frac{g - \mu}{\frac{\sigma}{\sqrt{n}}}) = \alpha \]

waaruit volgt dat:

\begin{equation}
  \label{eq:kritieke-waarde-rechtszijdig}
  g = \mu + z \times \frac{\sigma}{\sqrt{n}}
\end{equation}

\section{Overschrijdingskans}
\label{sec:overschrijdingskans}

Een karakteristiek die gebruikt wordt om weer te geven hoe sterk de geobserveerde waarde afwijkt van $H_{0}$, is de overschrijdingskans (probability value of ook $p$-waarde\index{$p$-waarde}). Dit vormt een alternatieve manier om te bepalen of de nulhypothese al dan niet verworpen kan worden.

\begin{definition}[overschrijdingskans]
  De \emph{overschrijdingskans}\index{overschrijdingskans} is de kans, indien de nulhypothese waar is, om een waarde te verkrijgen van de toetsingsgrootheid die minstens even extreem is als de geobserveerde waarde.
\end{definition}

\begin{definition}[statistische significantie]
  In een statistische hypothesetoets heeft men een \emph{statistisch significant}\index{significant} resultaat behaald waneer de geobserveerde overschrijdingskans $p$ van de teststatistiek lager is dan het significantieniveau $\alpha$. De $p$-waarde wordt onder het gekozen significantieniveau beschouwd als te extreem om de veronderstelling dat de nulhypothese waar is aan te houden.
\end{definition}

\begin{example}
  In het onderzoek naar het aantal dagelijkse reddingen door superhelden (Voorbeeld~\ref{ex:hypothesetoets-dagelijkse-reddingen}) kan de overschrijdingskans als volgt berekend worden:
  
\[ P(M > 3,483) = P \left(Z> \frac{3,483 - 3,3}{\frac{\sigma}{\sqrt{n}}}\right) = P (Z > 1,822) = 0,0344 \]
\end{example}

Als de overschrijdingskans of $p$-waarde kleiner is dan de onbetrouwbaarheidsdrempel dan moet $H_{0}$ verworpen worden, is de $p$-waarde gelijk of groter dan $\alpha$ dan mag je $H_{0}$ niet verwerpen. In ons geval is de $p$-waarde $0,0344$ en die is kleiner dan $\alpha = 0,05$ dus moeten we $H_{0}$ verwerpen.

\begin{itemize}
  \item $p$-waarde $< \alpha \Rightarrow$ $H_{0}$, verwerpen want de gevonden waarde voor $\overline{x}$ is te extreem;
  \item $p$-waarde $\geq \alpha \Rightarrow$ $H_{0}$ niet verwerpen, want de gevonden waarde voor $\overline{x}$ kan nog verklaard worden door toeval.
\end{itemize}

\section{Eenzijdig of tweezijdig toetsen}
\label{sec:eenzijdig-of-tweezijdig}

In Voorbeeld~\ref{ex:hypothesetoets-dagelijkse-reddingen} gaat het om een hypothese waar we vermoeden dat het populatiegemiddelde \emph{hoger} ligt dan een bepaalde waarde. We twijfelen dus aan de de nulhypothese als ons steekproefgemiddelde significant boven het vooropgestelde gemiddelde $\mu = 3,3; \alpha = 0,05$ ligt. Het kritieke gebied om $H_{0}$ te verwerpen ligt dus aan de rechterzijde van de curve en we noemen deze toets dan ook rechtszijdig.

We zouden ook een toets kunnen maken waar we denken dat de superhelden gemiddeld \emph{minder} mensen redden per dag. Dan ligt het kritieke gebied aan de linkerzijde en noemen we de toets linkszijdig.

\begin{exercise}
  \label{ex:kritieke-waarde-linkszijdig}
  Wat zou je in vergelijking \ref{eq:kritieke-waarde-rechtszijdig} moeten veranderen opdat je de correcte kritieke waarde zou berekenen voor een linkszijdige $z$-toets?
\end{exercise}

Soms kan het ook zijn dat er tweezijdig moet getoetst worden. De alternatieve hypothese wordt dan geformuleerd als zijnde dat het populatiegemiddelde verschillend is van de opgegeven waarde. Er moeten dan twee kritieke grenswaarden berekend worden namelijk de linker- en de rechter grenswaarden.

\begin{equation}
  g = \mu \pm z \times \frac{\sigma}{\sqrt{n}}
  \label{eq:kritieke-waarde-tweezijdig}
\end{equation}

De totale oppervlakte van het kritieke gebied moet $1 - \alpha$ zijn, en je moet er rekening mee houden dat zowel links als rechts een gebied met telkens oppervlakte $\alpha / 2$ samen het aanvaardingsgebied vormen. Je moet dan ook de overeenkomstige $z$-waarde kiezen. Als we opnieuw significantieniveau $\alpha = 0,05$ nemen, zoeken we dus de $z$ waarde waarvoor geldt dat;

\[P(Z < -z) + P(Z > z) = \alpha \Leftrightarrow 2 P(Z>z) = \alpha \Leftrightarrow P(Z < z) = 1-\frac{\alpha}{2} = 0,975\]

De overeenkomstige $z$-waarde is dan ongeveer 1.96 (op te zoeken in de $z$-tabel of in R met \texttt{qnorm(.975)}).

De drie vormen van de $z$-toets worden samengevat in tabel~\ref{tab:toetsingsprocedures}.

\begin{table}
  \centering
  \begin{tabular}{l|ccc}
    \toprule
    Doel              & \multicolumn{3}{l}{\parbox{.5\textwidth}{Test op gemiddelde waarde $\mu$ van de populatie aan de hand van een steekproef van $n$ onafhankelijke steekproefwaarden}} \\
    \midrule
    Voorwaarde        & \multicolumn{3}{l}{\parbox{.5\textwidth}{De populatie is willekeurig verdeeld, $n$ voldoende groot}} \\
    \midrule
    Type test         & Tweezijdig           & Eenzijdig links & Eenzijdig rechts \\
    \midrule
    $H_{0}$           & $\mu = \mu_{0}$      & $\mu = \mu_{0}$ & $\mu = \mu_{0}$  \\
    $H_{1}$           & $\mu \neq \mu_{0}$   & $\mu < \mu_{0}$ & $\mu > \mu_{0}$  \\
    Verwerpingsgebied & $\left|z\right| > g$ & $z< -g $        & $z>g$            \\
    Teststatistiek    & \multicolumn{3}{c}{$z = \frac{\overline{x} - \mu_{0}}{\frac{\sigma}{\sqrt{n}}}$} \\
    \bottomrule
  \end{tabular}
  \caption{Samenvatting verschillende vormen van de $z$-toets}
  \label{tab:toetsingsprocedures}
\end{table}

\section{De \texorpdfstring{$z$}{z}-toets in R}
\label{sec:z-toets-R}

Het codevoorbeeld hieronder is de uitwerking van Voorbeeld~\ref{ex:hypothesetoets-dagelijkse-reddingen} in R.

\lstinputlisting{data/z-toets.R}

\begin{figure}
  \centering
  \includegraphics[width=\textwidth]{z-toets-reddingen}
  \caption{Plot in R van de situatie van Voorbeeld~\ref{ex:hypothesetoets-dagelijkse-reddingen}}
\end{figure}

\section{Voorbeelden}

\begin{example}
  Bij een aselecte steekproef van 50 waarnemingen vinden we we volgende grootheden:
  \begin{itemize}
    \item $\overline{x} = 25$
    \item $s = \sqrt{55} = 7,41$
  \end{itemize}
  
  We willen weten of er reden is om aan te nemen dat $\mu$ van de populatie kleiner is dan 27.
  
  \begin{enumerate}
    \item Bepalen van de hypothesen: 
    
    $H_{0} : \mu = 27$ en $H_{1}: \mu < 27$.
    
    \item Vastleggen significantieniveau $\alpha$ en steekproefomvang $n$:
    
    $\alpha = 0,05$ en $n=50$.
    
    \item Waarde toetsingsgrootheid bepalen. 
    
    We kiezen hiervoor het steekproefgemiddelde $M$. Volgens de centrale limietstelling geldt:
    
    \[ M \sim Nor(\mu = 27, \frac{\sigma}{\sqrt{n}}) \]
    De toetsingsgrootheid is
    \[ Z = \frac{\overline{x} - \mu}{\frac{\sigma}{\sqrt{n}}} = \frac{25-27}{\sqrt\frac{55}{50}} \approx -1,91\]
    
    \item Overschrijdingskans berekenen.
    
    We vinden een overschrijdingskans van het gemiddelde van \texttt{pnorm(-1.91)} of ongeveer $0,0281$. Bij een significantieniveau van 0,05 duidt dit er op dat we $H_{0}$ mogen verwerpen.
    
    \item Bereken en teken kritiek gebied.
    
    \[ g = \mu - z \times \frac{\sigma}{\sqrt{n}} \]
    en dus
    
    \[ g = 27 - 1,645 \times \sqrt{\frac{\sigma}{n}} \]
    \[ g =  25,27470944 \]
    
    We vinden dus dat $\overline{x} < g$ komen tot hetzelfde besluit, nl.~dat we $H_{0}$ kunnen verwerpen.
    
  \end{enumerate}
\end{example}

\begin{example}
  In een onderzoek naar het kleingeld dat in de zakken van van onze superhelden zit, stellen de onderzoekers dat zij gemiddeld 25 euro op zak hebben. Ze gaan ervan uit de spreiding $\sigma = 7$ is. Verder zijn de gegevens van de aselecte steekproef van omvang $n=64$ beschikbaar met gemiddeld zakgeld $\overline{x}$ van 23 euro. Voor het significantieniveau kiezen ze $\alpha = 0,05$.
  
  \begin{enumerate}
    \item Bepalen van de hypothesen.
    
    $H_{0} : \mu = 25$ en $H_{1}: \mu \neq 25$.
    
    \item Vastleggen significantieniveau $\alpha$ en steekproefomvang $n$.
    
    $\alpha = 0,05$ en $n=64$.
    
    \item Bepalen van de kritieke grenzen.
    
    \[ g_{1} = \mu - z \times \frac{\sigma}{\sqrt{n}} = 23,28 \]
    
    \[ g_{2} = \mu + z \times \frac{\sigma}{\sqrt{n}} = 26,72 \]
    
    \item Kritiek gebied.
    
    We vinden dat $\overline{x}$ in het kritieke gebied ligt (want $\overline{x} = 23 < g_1 = 23,25$), dus mogen we $H_{0}$ verwerpen.
    
  \end{enumerate}
\end{example}

\section{Fouten in hypothesetoetsen}

Bij het uitvoeren van een hypothesetoets kunnen altijd nog fouten optreden. Indien we $H_{0}$ verwerpen wanneer ze in werkelijkheid juist is, spreken we van een fout van type I en wanneer we $H_{0}$ ten onrechte aanvaarden van een fout van type II.

Het significatieniveau $\alpha$ bepaalt bij het uitvoeren van een hypothesetoets wanneer de nulhypothese precies verworpen kan worden. Stel dat we een significatieniveau van 5\% kiezen. Als de nulhypothese waar is, dan is de kans dat we een steekproef trekken met een toetsingswaarde die in het verwerpingsgebied terecht komt 5\%. M.a.w. de kans om de nulhypothese te verwerpen terwijl ze waar is, is 5 \% of in het algemeen: het significantieniveau van een toets is gelijk aan de kans op het maken van een fout van type I.

Het is vanzelfsprekend dat we de kans op een fout van type I zo klein mogelijk willen houden. Jammer genoeg is dit ten koste van de kans op een type II fout, aangeduid met $\beta$, die hierdoor groter wordt. Het verband tussen $\alpha$ en $\beta$ is niet triviaal en we gaan hier in deze cursus niet verder op in.

In vele gevallen is het maken van een fout van type I erger dan een van type II. Denk maar aan een rechtszaak waarbij de nulhypothese is dat de persoon onschuldig is. Indien we toetsen op een 5\% significantieniveau is de kans op een type I fout 5 op 100. M.a.w. er is een betrouwbaarheid van 95\% dat de juiste beslissing wordt genomen indien $H_{0}$ correct is. Daarom vermijden we liever de conclusie dat $H_{0}$ geaccepteerd wordt, maar eerder dat de steekproef onvoldoende bewijs bevat om $H_{0}$ bij een bepaald significantieniveau te verwerpen.

\begin{table}
  \centering
    \begin{tabular}{@{}l|cc@{}}
      \toprule
      & \multicolumn{2}{c}{\textbf{Werkelijke stand van zaken}} \\
      \textbf{Conclusies}          & \textbf{$H_{0}$ correct} & \textbf{$H_{1}$ correct}     \\
      \midrule
      \textbf{$H_{0}$ geaccepteerd}& Juist                    & Fout van type II \\
      \textbf{$H_{0}$ verworpen}   & Fout van type I          & Juist            \\
      \bottomrule
    \end{tabular}
  \caption{Conclusies en consequenties bij toetsen van een hypothese; types van fouten.}
  \label{tab:hypfouten}
\end{table}

\section{De \texorpdfstring{$t$}{t}-toets}
\label{sec:t-toets}

Bij de $z$-toets gaan we uit van een aantal veronderstellingen waar we rekening moeten mee houden:

\begin{itemize}
  \item De steekproef moet voldoende groot zijn ($n \ge 30$);
  \item De variatie van de toetsingsgrootheid moet normaal verdeeld zijn;
  \item We veronderstellen dat de standaardafwijking van de populatie, $\sigma$, gekend is.
\end{itemize}

De eerste drie voorwaarden maken dat de centrale limietstelling kan toegepast worden.

Soms zijn deze veronderstellingen niet geldig en mogen we dan ook de $z$-toets \emph{niet} gebruiken! In deze gevallen kunnen we wel gebruik maken van de Student-$t$ verdeling. In de $t$-toets\index{$t$-toets}\index{toets!$t$-} wordt er wel van uit gegaan dat de onderzochte variabele normaal verdeeld is.

De formule voor de kritieke grenswaarde wordt dan aangepast als:

\begin{equation}
g = \mu \pm t \times \frac{s}{\sqrt{n}}
\label{eq:kritieke-waarde-t-toets}
\end{equation}

Voor het bepalen van de $t$-waarde hebben we het aantal vrijheidsgraden nodig, $n-1$. Om de standaardafwijking te schatten, gebruiken we de steekproefstandaardafwijking, $s$.

\begin{example}
  \label{ex:t-toets-dagelijkse-reddingen}
  Stel dat de onderzoekers van de superhelden uit Voorbeeld~\ref{ex:hypothesetoets-dagelijkse-reddingen} door tijdsdruk niet in staat waren om een voldoende grote steekproef te nemen en slechts $n = 20$ observaties gedaan hebben, met hetzelfde steekproefgemiddelde $\overline{x} = 3,483$. De standaardafwijking in deze steekproef bleek $s = 0,55$.
  
  Kunnen we in deze omstandigheden, met eenzelfde significantieniveau $\alpha = 0,05$, het besluit dat superhelden dagelijks \emph{meer} dan 3,3 mensen redden aanhouden?
  
  \begin{enumerate}
    \item Bepalen van de hypothesen.
    
      $H_{0} : \mu = 3,3$ en $H_{1}: \mu > 25$.
    
    \item Vastleggen significantieniveau $\alpha$ en steekproefomvang $n$.
    
    $\alpha = 0,05$ en $n=25$.
    
    \item Bepalen van de kritieke grenswaarde.
    
    \[ g_{2} = \mu + t \times \frac{s}{\sqrt{n}} \approx 3,3 + 1,711 \times \frac{0,55}{\sqrt{25}} \approx 3.488 \]
    
    De waarde voor $t$ wordt in R berekend met \texttt{qt(1-a, df = n - 1)} (met \texttt{a} het significantieniveau en \texttt{df} het aantal vrijheidsgraden.)
    
    \item Conclusie.
    
    We vinden dat $\overline{x} = 3,483$ kleiner is dan de kritieke grenswaarde en dus in het aanvaardingsgebied ligt. Met andere woorden, we mogen $H_{0}$ \emph{niet} verwerpen.
  \end{enumerate}

  Met andere woorden, ook al krijgen we gelijkaardige resultaten in onze steekproef, kunnen we niet hetzelfde besluit trekken. Omdat onze steekproef te klein is, is er grotere onzekerheid of de waarde van het steekproefgemiddelde extreem genoeg is om de nulhypothese te verwerpen.
  
  Hieronder vind je de uitwerking van dit voorbeeld in R.
\end{example}

\lstinputlisting{data/t-toets.R}

\begin{figure}
  \centering
  \includegraphics[width=\textwidth]{t-toets-reddingen}
  \caption{Plot in R van de situatie van Voorbeeld~\ref{ex:t-toets-dagelijkse-reddingen}}
\end{figure}

\begin{example}
  Een uitbraak van een door Salmonella veroorzaakte ziekte werd toegeschreven aan vanille-ijs van een bepaalde fabriek~\autocite{Lindquist}. Wetenschappers hebben het niveau van Salmonella gemeten in 9 willekeurig genomen steekproeven.
  
  De niveaus (in MPN/g\footnote{Most Probable Number. Zie bv.~\url{http://www.microbiologie.info/mpn-methode.html} voor meer uitleg over deze methode.}) zijn de volgende:
  
    \begin{center}
    \begin{tabular}{|l|l|l|l|l|}
      \hline
      0,593 & 0,142 & 0,329 & 0,691 & 0,231 \\ \hline
      0,793 & 0,519 & 0,392 & 0,418 &       \\ \hline
    \end{tabular}
  \end{center}

  Is er reden om aan te nemen dat het Salmonella-niveau in het ijs significant groter is dan 0,3 MPN/g? We zullen gebruik maken van de R-functie \texttt{t.test} om deze vraag te beantwoorden. Lees zelf de help-pagina van deze functie om de mogelijke opties te leren kennen.
  
  \begin{enumerate}
    \item Bepalen van de hypothesen
    
    $H_0: \mu = 0.3, H_1: \mu > 0,3$
    
    \item Vastleggen significantieniveau $\alpha = 0.05$ (in R moet je het betrouwbaarheidsniveau $1-\alpha$ opgeven, dus 0,95) en steekproefomvang $n = 9$
    
    \item Bepalen overschrijdingskans. Het gaat hier over een rechtszijdige toets, wat aangegeven wordt met de optie \texttt{alternative="greater"}. Het gekozen betrouwbaarheidsniveau is de standaardwaarde voor deze functie en moet niet expliciet meegegeven worden.
    
\begin{lstlisting}
x <- c(0.593, 0.142, 0.329, 0.691, 0.231, 0.793, 0.519, 0.392, 0.418)
t.test(x, alternative = "greater", mu = 0.3)
\end{lstlisting}
    
    Het resultaat is:
    
\begin{verbatim}
One Sample t-test

data:  x
t = 2.2051, df = 8, p-value = 0.02927
alternative hypothesis: true mean is greater than 0.3
95 percent confidence interval:
0.3245133       Inf
sample estimates:
mean of x 
0.4564444 
\end{verbatim}
    
    \item Conclusie. De overschrijdingskans $p = 0,029 < \alpha = 0,05$. We kunnen dus de nulhypothese verwerpen; er is met ander worden en vrij sterke aanwijzing dat het gemiddelde Salmonella-niveau in het ijs groter is dan 0,3 MPN/g.
    
    Je kan uit de uitvoer van de \texttt{t.test}-functie ook het kritieke gebied aflezen: $[0,3245133; +\infty[$. Het steekproefgemiddelde $0,4564444$ ligt in het kritieke gebied, wat eveneens leidt tot de conclusie dat de nulhypothese kan verworpen worden.
  \end{enumerate}
\end{example}

\section{De \texorpdfstring{$t$}{t}-toets voor twee steekproeven}
\label{sec:t-toets-twee-steekproeven}

De $t$-toets kan ook gebruikt worden om twee steekproeven met elkaar te vergelijken. Je kan er dan mee nagaan of het steekproefgemiddelde van beide steekproeven \emph{significant} verschillend is.

Men maakt onderscheid tussen twee gevallen:

\begin{itemize}
  \item Beide steekproeven zijn onafhankelijk, zijn afzonderlijk genomen. Een voorbeeld is een onderzoek naar een medische behandelingsmethode waar een contolegroep de behandeling \emph{niet} krijgt en een testgroep de behandeling wel krijgt.
  \item De steekproeven zijn afhankelijk, of gepaard. Een voorbeeld is twee metingen uitvoeren op hetzelfde lid van de populatie, zoals de koorts nemen voor en na het innemen van een medicijn om het effect ervan te meten.
\end{itemize}

In R kan je eveneens de functie \texttt{t.test} gebruiken voor het uitvoeren van een toets met twee steekproeven. We geven hieronder twee voorbeelden, één voor elk geval.

\begin{example}
  In een klinisch onderzoek wil men nagaan of een nieuw medicijn als bijwerking een verminderde reactiesnelheid heeft~\autocite{Lindquist}.
  
  Zes deelnemers kregen een medicijn toegekend (interventiegroep) en zes anderen een placebo (controlegroep). Vervolgens werd hun reactietijd op een stimulus gementen (in ms). We willen nagaan of er significante verschillen zijn tussen de interventie- en controlegroep.
  
  \begin{itemize}
    \item Controlegroep: 91, 87, 99, 77, 88, 91
    \item Interventiegroep: 101, 110, 103, 93, 99, 104
  \end{itemize}
  
  We noteren $\mu_1$ voor het populatiegemiddelde van de patiënten die het medicijn nemen en $\mu_2$ voor het gemiddelde van de niet behandelde populatie.
  
  De hypothesen worden formeel als volgt genoteerd:
  
  $H_0: \mu_1 - \mu_2 = 0$ en $H_1: \mu_1 - \mu_2 < 0$
  
  Het gaat hier dus over een linkszijdige test, wat weergegeven wordt door de optie \texttt{alternative = "less"}. In de nulhypothese veronderstellen we dat het verschil tussen de populatiegemiddelden 0 is, wat met de optie \texttt{mu = 0} wordt aangeduid. Merk op dat dit de standaardwaarde is voor deze parameter en dus in principe niet moet worden opgegeven.
  
\begin{lstlisting}
controle <-  c(91, 87, 99, 77, 88, 91)
interventie <- c(101, 110, 103, 93, 99, 104)
t.test(controle, interventie, alternative="less", mu=0)
\end{lstlisting}

  Het resultaat van de toets:
  
\begin{verbatim}
t.test(controle, interventie, alternative="less")

Welch Two Sample t-test

data:  controle and interventie
t = -3.4456, df = 9.4797, p-value = 0.003391
alternative hypothesis: true difference in means is less than 0
95 percent confidence interval:
-Inf -6.044949
sample estimates:
mean of x mean of y 
88.83333 101.66667
\end{verbatim}

  De $p$ waarde, 0,003391, ligt duidelijk onder het significantieniveau (niet expliciet opgegeven, dus werd de standaardwaarde $\alpha = 0,05$ gebruikt.)
  
  De teststatistiek $t = -3,4456$ ligt binnen het verwerpingsgebied $]-\infty; -6,044949]$.
  
  We mogen dus de nulhypothese verwerpen en besluiten dat volgens de resultaten van deze steekproef het medicijn inderdaad een significant effect heeft op de reactiesnelheid van patiënten.
\end{example}

\begin{example}
  In een studie werd nagegaan of auto's die rijden op benzine met additieven ook een lager verbruik hebben. Tien auto's werden eerst volgetankt met ofwel gewone benzine, ofwel benzine met additieven (bepaald door opgooien van een munt), waarna het verbruik werd gemeten (uitgedrukt in mijl per gallon). Vervolgens werden de auto's opnieuw volgetankt met de andere soort benzine en werd opnieuw het verbruik gemeten. De resultaten worden gegeven in Tabel~\ref{tab:benzineverbruik-additieven}.
  
  \begin{table}
    \centering
    \begin{tabular}{|l|c|c|c|c|c|c|c|c|c|c|}
      \hline 
      Auto & 1 & 2 & 3 & 4 & 5 & 6 & 7 & 8 & 9 & 10 \\ 
      \hline 
      Gewone benzine & 16 & 20 & 21 & 22 & 23 & 22 & 27 & 25 & 27 & 28 \\ 
      \hline 
      Additieven & 19 & 22 & 24 & 24 & 25 & 25 & 25 & 26 & 28 & 32 \\ 
      \hline 
    \end{tabular} 
  \caption{Verbruik in mijl per gallon met 2 soorten benzine.}
  \label{tab:benzineverbruik-additieven}
  \end{table}
  
  We gaan door middel van een \emph{gepaarde $t$-test} na of auto's significant zuiniger rijden met benzine met additieven.
  
  De optie \texttt{paired=TRUE} geeft aan dat het hier om een gepaarde $t$-toets gaat.
  
\begin{lstlisting}
gewone    <- c(16, 20, 21, 22, 23, 22, 27, 25, 27, 28)
additieven <-c(19, 22, 24, 24, 25, 25, 26, 26, 28, 32)
t.test(additieven, gewone, alternative="greater", paired=TRUE)
\end{lstlisting}

  Resultaat:
  
\begin{verbatim}
 Paired t-test

data:  additieven and gewone
t = 4.4721, df = 9, p-value = 0.0007749
alternative hypothesis: true difference in means is greater than 0
95 percent confidence interval:
 1.180207      Inf
sample estimates:
mean of the differences 
                      2 
\end{verbatim}

  De $p$-waarde, 0,0007749, ligt onder het significantieniveau, dus we kunnen de nulhypothese verwerpen. Volgens deze steekproef rijden auto's inderdaad zuiniger met benzine met additieven.
  
  De teststatistiek $t = 4,4721$ ligt binnen het verwerpingsgebied $[1,180207; +\infty]$.
\end{example}

% TODO: later evt. toevoegen
% - ANOVA, variantie-analyse
% - Kolmogorov-Smirnov test

\vspace{1cm} %% Workaround voor kop van volgende sectie die onderaan een pagina terecht komt

\section{Oefeningen}
\label{sec:toetsingsprocedures-oefeningen}

\begin{exercise}
  Betrouwbaarheidsintervallen.
  
  \begin{enumerate}
    \item Wat is de onder- en bovengrens van een betrouwbaarheidsinterval van 99\%?
    \item Een betrouwbaarheidsinterval van 99\% is breder dan een van 95\%. Waarom is dit zo?
    \item Hoe zou het betrouwbaarheidsinterval voor 100\% er uit zien?
  \end{enumerate}
  
\end{exercise}

\begin{exercise}
  \label{oef:bindend-studieadvies}
  
  Er wordt gezegd dat het invoeren van een bindend studieadvies (BSA) een rendementsverhoging tot gevolg heeft in slaagkans. Voor het invoeren van het BSA was in de studentenpopulatie het gemiddelde aantal behaalde studiepunten per jaar per student gelijk aan 44 met een standaardafwijking van 6,2. Na invoering van het BSA wijst een onderzoek uit onder 72 studenten dat deze een gemiddeld aantal studiepunten haalden van 46,2.
  
  \begin{enumerate}
    \item Toets of er bewijs is dat het invoeren van een BSA leidt tot een rendementsverhoging. Gebruik methode van kritieke grenswaarde. ($\sigma = 6,2, \alpha = 2,5\%$).
    \item Toon hetzelfde aan met de methode van de overschrijdingskans.
    \item Geef een interpretatie wat de betekenis is van $\alpha = 2,5 \%$.
  \end{enumerate}
\end{exercise}

\begin{exercise}
  \label{oef:prijsverschil-autos}
  
  Eén van de motieven voor het kiezen van een garage is de inruilprijs voor de oude auto. De importeur van Ford wil graag dat de verschillende dealers een gelijk prijsbeleid voeren. De importeur vindt dat het gemiddelde prijsverschil tussen de dichtstbijzijnde Ford-dealer en de dealer waar men de auto gekocht heeft hoogstens \euro{300} mag bedragen. De veronderstelling is dat als het verschil groter is, potentiële klanten eerder geneigd zullen zijn om bij hun vorige dealer te blijven.
  
  In een steekproef worden volgende verschillen genoteerd:
  
  \begin{center}
    \begin{tabular}{|l|l|l|l|l|l|l|}
      \hline
      400 & 350 & 400 & 500 & 300 & 350 & 200 \\ \hline
      500 & 200 & 250 & 250 & 500 & 350 & 100 \\ \hline
    \end{tabular}
  \end{center}

  Toets of er reden is om aan te nemen dat het gemiddelde prijsverschil in werkelijkheid significant groter is dan \euro{300}.
  
\end{exercise}

\begin{exercise}
  \label{oef:casus-akin2016-toets}
  
  In Oefening~\ref{oef:casus-akin2016-1var} en volgende hebben we de resultaten van performantiemetingen voor persistentiemogelijkheden in Android geanalyseerd~\autocite{Akin2016}. Er werden experimenten uitgevoerd voor verschillende combinaties van hoeveelheid data (klein, gemiddeld, groot) en persistentietype (GreenDAO, Realm, SharedPreferences, SQLite). Voor elke hoeveelheid data hebben we kunnen bepalen welk persistentietype het beste resultaat gaf.
  
  Nu gaan we uitzoeken of het op het eerste zicht beste persistentietype ook \emph{significant} beter is dan de concurrentie.
  
  Concreet: ga aan de hand van een toets voor twee steekproeven voor elke datahoeveelheid na of het gemiddelde van het best scorende persistentietype \emph{significant lager} is dan het gemiddelde van \begin{inparaenum}[(i)] \item het \emph{tweede} beste en \item het slechtst scorende type \end{inparaenum}.
  
  Kunnen we de conclusie aanhouden dat voor een gegeven datahoeveelheid één persistentietype het beste is, d.w.z.~significant beter is dan gelijk welk ander persistentietype?
\end{exercise}

\begin{exercise}
  Een groot aantal studenten heeft deelgenomen aan een test die in verschillende opeenvolgende sessies werd georganiseerd. Omdat het opstellen van een aparte opgave voor elke sessie praktisch onhaalbaar was, is telkens dezelfde opgave gebruikt. Eigenlijk bestaat er dus het gevaar dat studenten na hun sessie info konden doorspelen aan de groepen die nog moesten komen. De latere groepen hebben dan een voordeel ten opzichte van de eerste. Blijkt dit ook uit de cijfers?
  
  Het bestand \texttt{puntenlijst.csv} bevat alle resultaten van de test. Elke groep wordt aangeduid met een letter, in de volgorde van de sessie.
  
  \begin{itemize}
    \item Dag 1: sessies A, B
    \item Dag 2: sessies C, D, E
    \item Dag 3: sessies F, G, H
  \end{itemize}

  Sessies A en B zijn doorgegaan op een andere campus, dus er zou kunnen verondersteld worden dat er weinig tot geen communicatie is met de studenten van de andere sessies.
  
  Als er info met succes doorgespeeld werd, dan verwachten we dat de scores van de groepen die later komen significant beter zijn dan de eerste.
  
  Merk op dat de omgekeerde redenering niet noodzakelijk geldt: als blijkt dat het resultaat van de latere sessies inderdaad significant beter blijkt, dan betekent dat niet noodzakelijk dat de oorzaak (enkel) het doorspelen van informatie is. Er kunnen ook andere oorzaken zijn (bv.~``zwakkere'' klasgroepen zijn toevallig eerder geroosterd).
  
  \begin{enumerate}
    \item Ga op verkenning in de data. Bereken de gepaste centrum- en spreidingsmaten voor de dataset als geheel en voor elke sessie afzonderlijk.
    
    \item Maak een staafgrafiek van de gemiddelde score per sessie. Is dit voldoende om een beeld te vormen van de resultaten? Waarom (niet)?
    
    \item Maak een boxplot van de scores opgedeeld per groep. Vergelijk onderling de hieronder opgesomde sessies. Denk je dat er een significant verschil is tussen de resultaten? Wordt ons vermoeden dat er informatie doorgespeeld wordt bevestigd?
    
    \begin{itemize}
      \item A en B
      \item C, D en E
      \item F, G en H
      \item C en H
      \item A en H
    \end{itemize}
  
    \item Ga door middel van een geschikte statistische toets voor na of de verschillen tussen die hierboven opgesomde groepen ook \emph{significant} is. Kunnen we concluderen dat de latere groepen beter scoren of niet?
  \end{enumerate}
\end{exercise}

\section{Antwoorden op geselecteerde oefeningen}
\label{sec:toetsingsprocedures-antwoorden}

\paragraph{Oefening~\ref{ex:kritieke-waarde-linkszijdig}:}

\begin{equation}
g = \mu - z \times \frac{\sigma}{\sqrt{n}}
\label{eq:kritiekeRechtseWaarde2}
\end{equation}

want

\[ P(M < g) = P\left(Z < \frac{g - \mu}{\frac{\sigma}{\sqrt{n}}}\right) = 0,05 \]
Wegens de symmetrieregel kunnen we zeggen
\[ P\left(Z > - \left( \frac{g - \mu}{\frac{\sigma}{\sqrt{n}}} \right) \right) = 0,05 \]
De z-waarde die ermee overeen komt is 1,645 dus hebben we
\[ z = \frac{-g + \mu}{\frac{\sigma}{\sqrt{n}}} \]
\[ \Leftrightarrow -g = \frac{\sigma}{\sqrt{n}} z - \mu \]
\[ \Leftrightarrow g = -\frac{\sigma}{\sqrt{n}} z + \mu \]

\paragraph{Oefening~\ref{oef:bindend-studieadvies}}

\begin{enumerate}
  \item $g \approx 45,4 < \overline{x} = 46,2$.
  
  $\overline{x}$ ligt in het kritieke gebied, dus we mogen de nulhypothese verwerpen. We hebben dus redenen om aan te nemen dat bindend studieadvies inderdaad het studierendement significant verhoogt.
  
  \item $P(M > 46.2) \approx 0,01 < \alpha = 0,025$. De overschrijdingskans is kleiner dan het significantieniveau, dus we mogen de nulhypothese verwerpen.
  
  \item  $\alpha$ is de kans dat je $H_{0}$ ten onrechte verwerpt. Er is m.a.w.~een kans van 2,5\% dat je ten onrechte de conclusie trekt dat het studierendement hoger is geworden.
\end{enumerate}

\paragraph{Oefening~\ref{oef:prijsverschil-autos}}

In deze situatie ($n = 14 < 30$) mogen we geen $z$-toets gebruiken, maar vallen we terug op de $t$-toets.

\begin{itemize}
  \item $\overline{x} \approx 332,143$
  \item $s \approx 123,424$
  \item $g \approx 358,42$
  \item Het steekproefgemiddelde ligt niet in het kritieke gebied, dus we kunnen $H_0$ \emph{niet} verwerpen.
\end{itemize}

Er is op basis van deze steekproef dus geen reden om aan te nemen dat het gemiddelde prijsverschil op de inruilprijs van oude wagens significant groter is dan door de importeur aanbevolen.

\chapter{Analyse op 2 variabelen}
\label{ch:analyse2var}

In de vorige hoofdstukken hebben we telkens één variabele tegelijkertijd onderzocht. Vaak hebben onderzoeksvragen echter te maken met \emph{verbanden} (en dan vooral oorzakelijke) tussen variabelen. In dit hoofstuk gaan we hier verder op in.

Wanneer we een verband beschrijven tussen variabelen, onderscheiden we:

\begin{itemize}
  \item De \emph{afhankelijke variabele}\index{variabele!afhankelijke}, waarover we een voorspelling willen doen;
  \item De \emph{onafhankelijke variabele}\index{variabele!onafhankelijke}, op basis van dewelke we de voorspelling doen.
\end{itemize}

Als de onafhankelijke variabele op een bepaalde manier verandert, verwachten we dat de waarde van de afhankelijke variabele op een voorspelbare manier mee verandert.

\begin{example}
  Een voorbeeld waarbij verbanden kunnen gevonden tussen variabelen vind je bijvoorbeeld bij Ant Colony Optimization (ACO). Dit is een techniek die gebruikt wordt in verschillende computationele problemen. Men baseert zich hier op hoe mieren voedsel zoeken en  vinden en dat communiceren aan de groep. Mieren verspreiden feromonen als ze op pad gaan op zoek naar eten. Hoe langer het pad, hoe minder feromonen het pad zal bevatten, hoe korter het pad, hoe groter de kans dat er een grote concentratie aan feromonen te vinden is. Mieren worden aangetrokken door deze feromonen en zullen dus proberen de meest bewandelde paden te gebruiken om naar een bepaalde voedselbron te gaan. Nu kan je onderzoeken of de tijd voor het vinden van een pad, afhangt van een aantal variabelen:

  \begin{itemize}
    \item De mate waarin feromonen verspreid worden
    \item De mate waarin een feromoon verdwijnt
    \item Het aantal obstakels tussen het nest en de voedselbron
    \item De vorm van de obstakels tussen nest en voedselbron (vinden ze sneller het pad indien er geen hoeken aan de obstakels zijn bv.)
  \end{itemize}
\end{example}

Om een vergelijking te maken kunnen we \begin{inparaenum}[(i)]
\item de bekende statistieken zoals gemiddelde e.a. berekenen en analyseren of \item grafische voorstellingen maken van deze statistieken. \end{inparaenum}

Welke soort van grafieken we kunnen gebruiken hangt af van het meetniveau:
@
\begin{itemize}
  \item Interval of ratio:

    \begin{itemize}
      \item Staafdiagram van de gemiddelden
      \item Boxplot per groep
    \end{itemize}
  \item Ordinaal of nominaal

    \begin{itemize}
      \item Kruistabel
      \item Geclusterd staafdiagram
      \item Rependiagram
    \end{itemize}
\end{itemize}

Bij de vraag of er samenhang is tussen twee variabelen kunnen we volgende grafieken/statistieken gebruiken:

\begin{itemize}
  \item Nominaal x Nominaal:

    \begin{itemize}
      \item Kruistabel met Cramér's V
    \end{itemize}
  \item Ordinaal x Ordinaal
    \begin{itemize}
      \item Geclusterd staafdiagram
      \item Rependiagram
    \end{itemize}
  \item Ratio x Ratio

    \begin{itemize}
      \item Spreidingsdiagram
      \item Regressie en correlatie met correlatiecoëfficiënt.
    \end{itemize}
\end{itemize}

\section{Kruistabellen en Cramér's V}

\begin{definition}[Kruistabel]
  In een kruistabel\index{kruistabel} (zie bv.~Figuur~\ref{tab:kruistabel0}) worden de frequenties van twee variabelen samengevat.
  
  Elke cel van de laatste kolom bevat de som van de overeenkomstige rij en elke cel van de laatste rij bevat de som van de overeenkomstige kolom. Dit worden de \emph{marginale totalen}\index{totaal!marginaal}\index{marginaal totaal} genoemd.
\end{definition}

In R kan je een kruistabel (Eng.: \emph{contingency table} of \emph{cross table}) opstellen met de functie \texttt{table}. Een voorbeeld i.v.m.~de oefening over Android persistentietypes (zie Oefening~\ref{oef:casus-akin2016-1var}):

\begin{lstlisting}
> table(Datahoeveelheid, PeristentieType)
        PersistentieType
DataHoeveelheid GreenDAO Realm Sharedpreferences SQLLite
Large                 30    30                 0      30
Medium                30    30                 0      30
Small                 30    30                30      30
\end{lstlisting}

\begin{table} \centering
  \begin{tabular}{@{}rrrr}
    \toprule
                & Vrouw & Man & Totaal \\ \midrule
           Goed &     9 &   8 &     17 \\
      Voldoende &     8 &  10 &     18 \\
    Onvoldoende &     5 &   5 &     10 \\
         Slecht &     0 &   4 &      4 \\
         Totaal &    22 &  27 &     49 \\ \bottomrule
  \end{tabular}
  \caption{Een kruistabel voor de waardering door mannen en vrouwen van een bepaald assortiment producten.}
  \label{tab:kruistabel0}
\end{table}

In een gewone kruistabel kunnen we geen directe conclusies trekken, aangezien het analyseren of er samenhang bestaat tussen variabelen niet goed gaat op basis van de celfrequenties. Niet alle metingen zijn even groot! Daarom moeten we percenteren. Nog even snel de regel van percenteren:

\begin{itemize}
  \item Om te weten hoeveel percent $x$ is van $y$, deel je $x$ door $y$ en vermenigvuldig je met 100: $perc = 100 \times \frac{x}{y}$.
  \item Om te weten hoeveel $x$ \% is van $y$ : $ \frac{x \times y}{100}$
\end{itemize}

\begin{example}
  \label{vb:kruistabel}
  In Tabel~\ref{tab:kruistabel0} vinden we de data waar er gekeken wordt naar het verschil in waardering van een assortiment tussen mannen en vrouwen. We percenteren per geslacht en vinden bijvoorbeeld dat 41\% van de vrouwen een waardering goed heeft (zie Tabel~\ref{tab:kruistabel1}). Nu kunnen we ons de vraag stellen of de waarderingskeuze afhangt van het geslacht van de persoon.
\end{example}

In ons voorbeeld kunnen we besluiten dat 30\% van de mannen tevreden is en 15\% van de mannen ontevreden. Maar hoe goed is die samenhang tussen de verschillende variabelen (geslacht en tevredenheid)? Dat kunnen we bepalen aan de hand van Cramér's V. Voordat we die definitie kunnen geven, moeten we echter eerst de waarde $\chi^2$ introduceren.

\begin{table} \centering
  \begin{tabular}{@{}rrrrrrr@{}} \toprule
    & Vrouw & Man & Totaal & Vrouw \% & Man\%   & Totaal  \\ \midrule
    Goed        & 9     & 8   & 17     & 41\%  & 30\% & 35\% \\
    Voldoende   & 8     & 10  & 18     & 36\%  & 37\%    & 37\% \\
    Onvoldoende & 5     & 5   & 10     & 23\%  & 18\% & 20\% \\
    Slecht      & 0     & 4   & 4      & 0\%      & 15\% & 8\%  \\
    Totaal      & 22    & 27  & 49     & 100\%    & 100\%   & 100\%   \\
    \bottomrule
  \end{tabular}
  \caption{De kruistabel waarbij we de waarden gepercenteerd hebben.}
  \label{tab:kruistabel1}
\end{table}

\section{\texorpdfstring{$\chi^{2}$}{chi-kwadraat} test voor associatie}
\index{$\chi^{2}$waarde}

De $\chi^{2}$ (\emph{chi-kwadraat}) waarde is een grootheid die gebruikt wordt om te bepalen of er een significant verband bestaat tussen twee variabelen. Meer hierover volgt later in Hoofdstuk~\ref{ch:chikwadraat}. De berekening ervan leggen we alvast hier uit.

\begin{enumerate}
  \item Stel de kruistabel op samen met marginale totalen (zie tabel \ref{tab:kruistabel1}).
  \item Stel voor elke cel een schatter op voor de theoretische kans om in die cel te geraken. Deze schatter kan je bereken als volgt: (kans op in de rij van deze cel te komen) $\times$ (kans om in de kolom van de cel te komen). In het voorbeeld is dit dus voor cel$_{1,2}$:
  
  \[P[rij_{1}] \times P[kolom_{2}] = \frac{17}{49} \times \frac{27}{49} = 0.191170346 \]
  Algemeen kan je dus stellen dat de verwachte theoretische waarde $e$ als volgt kan berekend worden:

  \begin{equation}
    e = (\frac{rijtotaal}{n} \times \frac{kolomtotaal}{n}) \times n = \frac{rijtotaal \times kolomtotaal}{n}
  \end{equation}

  Met:

  \begin{itemize}
    \item $e$ verwachte waarde bij onafhankelijkheid
    \item $rijtotaal$ totaal van de rij van de betreffende cel
    \item $kolomtotaal$ totaal van de kolom van de betreffende cel
  \end{itemize}

  Voor cel$_{1,2}$ is dit dus 9.36.
  
  \item Dan bereken je het verschil tussen geobserveerde (notatie $a$) en verwachte frequentie ($e$). (Zie tabel \ref{tab:kruistabel2})
  
  \item De laatste stap houdt in dat we een berekening gaan maken voor de maat van afwijking voor elke cel. Opnieuw gaan we hier een kwadraat nemen om het teken kwijt te spelen. We gaan ook de afwijking delen door de verwachte theoretische waarde om hen relatief even belangrijk te maken. Bijvoorbeeld: een afwijking van 5 op een verwachte frequentie van 20 is groter dan bv. een afwijking op een verwachte waarde van 200. Dit geeft dan volgende berekening (zie Tabel~\ref{tab:kruistabel3}):
  
  \begin{equation}
    \frac{(a-e)^{2}}{e}
  \end{equation}
  
  \item Deze gekwadrateerde deviaties gaan we dan optellen en vormt de $\chi^{2}$ \footnote{Let op dat er afgerond wordt.}
  
  \begin{equation}
    \chi^{2} = \sum \frac{(a-e)^{2}}{e}
  \end{equation}
\end{enumerate}

\begin{table} \centering
  \begin{tabular}{@{}rrrrrrr@{}} \toprule
    & Vrouw & Man & Totaal & Vrouw \% & Man\%   & Totaal  \\ \midrule
    Goed        & $9 -\textcolor{red}{7.63}$     & $8 - \textcolor{red}{9.36}$   & $17$     & $41$\%  & $30$\% & $35$\% \\
    Voldoende   & $8 - \textcolor{red}{8.08}$   & $10 - \textcolor{red}{9.91}$  & $18$     & $36$\%  & $37$\%    & $37$\% \\
    Onvoldoende & $5 - \textcolor{red}{4.48}$    & $5 - \textcolor{red}{5.51}$  & $10$     & $23$\%  & $18$\% & $20$\% \\
    Slecht      & $0 - \textcolor{red}{1.79}$    & $4 - \textcolor{red}{2.20}$  & $4$      & $0$\%      & $15$\% & $8$\%  \\
    Totaal      & $22$    & $27$  & $49$     & $100$\%    & $100$\%   & $100$\%   \\
    \bottomrule
  \end{tabular}
  \caption{De kruistabel waarbij we de schatter $e$  (hetgeen we verwachten bij geen samenhang) bepaald hebben voor elke cel en die aftrekken van de geobserveerde waarde.}
  \label{tab:kruistabel2}
\end{table}

\begin{table} \centering
  \begin{tabular}{@{}rrrrrrr@{}} \toprule
    & Vrouw                   & Man                     & Totaal & Vrouw \% & Man\%   & Totaal  \\ \midrule
    Goed        & $\textcolor{blue}{0.2}$ & $\textcolor{blue}{0.2}$ & $17$   & $41$\%   & $30$\%  & $35$\% \\
    Voldoende   & $\textcolor{blue}{0}$   & $\textcolor{blue}{0}$   & $18$   & $36$\%   & $37$\%  & $37$\% \\
    Onvoldoende & $\textcolor{blue}{0.1}$ & $\textcolor{blue}{0}$   & $10$   & $23$\%   & $18$\%  & $20$\% \\
    Slecht      & $\textcolor{blue}{1.8}$ & $\textcolor{blue}{1.5}$ & $4$    & $0$\%    & $15$\%  & $8$\%  \\
    Totaal      & $22$                    & $27$                    & $49$   & $100$\%  & $100$\% & $100$\%   \\
    \bottomrule
  \end{tabular}
  \caption{De kruistabel waarbij we het verschil gekwadrateerd en genormeerd hebben.}
  \label{tab:kruistabel3}
\end{table}

Met deze statistiek kunnen de waarde Cramér's V\index{Cramér's V} berekenen:

\begin{definition}[Cramér's V]
  \begin{equation}
    V = \sqrt{\frac{\chi^{2}}{n(k-1)}}
    \label{eq:Craemer}
  \end{equation}
  met
  \begin{itemize}
    \item $\chi^{2}$ de berekende chi-kwadraatwaarde.
    \item $n$ het aantal waarnemingen.
    \item $k$ = de kleinste waarde van het aantal kolommen of het aantal rijen van de tabel.
  \end{itemize}

\end{definition}

Cramér's V is de $\chi^{2}$, gecorrigeerd voor steekproefomvang en het aantal categorieën in de variabelen. Het resultaat is altijd een getal tussen 0 en 1. Tabel~\ref{tab:interpretatie-cramers-v} geeft aan hoe je het resultaat kan interpreteren.

\begin{table}
  \centering
  \begin{tabular}{ll}
    $V = 0$ & geen samenhang \\
    $V \approx 0,1$ & zwakke samenhang \\
    $V \approx 0,25$ & redelijk sterke samenhang \\
    $V \approx 0,50$ & sterke samenhang \\
    $V \approx 0,75$ & zeer sterke samenhang \\
    $V = 1$ & volledige samenhang \\
  \end{tabular}
  \caption{Interpretatie van de waarde van Cramérs'V}
  \label{tab:interpretatie-cramers-v}
\end{table}

Voor ons voorbeeld waarbij gekeken wordt naar de samenhang tussen geslacht en waardering van het assortiment vinden we een $\chi^{2}= 3.811$ en dus een Cramér's V van $0.279$ (want $n = 49$ en $k = 2$), wat duidt op redelijk sterke samenhang tussen de variabelen. Met andere woorden, de resultaten van de bevraging geven aan dat er een verschil is in de waardering die vrouwen en mannen geven over het assortiment.

Hieronder vind je de uitwerking van Voorbeeld~\ref{vb:kruistabel} in R.

\lstinputlisting{data/kruistabellen.R}

\begin{example}
  In Tabel~\ref{tab:autovoorkeur} worden de voorkeuren van vrouwen en mannen voor de gegeven automerken opgesomd. We zien dat nog steeds dertig van de honderd respondenten een voorkeur hebben voor de Mercedes, maar dat tweederde van deze dertig vrouwen zijn. We zouden  ook kunnen zeggen dat de helft van de vrouwen een voorkeur heeft voor de Mercedes. Evenzo blijkt dat een derde van de mannen een voorkeur heeft voor een Alfa Romeo, tegenover geen van de vrouwen. Het lijkt alsof de onderscheiden automerken niet gelijkelijk gewaardeerd worden door mannen en vrouwen. Om dit te staven bepalen we $\chi^{2}$ en Cramér's V. Probeer dit zelf, hetzij in R, hetzij met een rekenblad (Excel, Numbers, LibreOffice Calc)! We vinden:
  \[ \chi^{2} = 22.619 \]
  \[ V = \sqrt{\frac{22.169}{100 . (2-1)}}  = 0.476\]

  We vinden dus tussen een sterke tot zeer sterke samenhang.
\end{example}

\begin{table} \centering
  \begin{tabular}{@{}rrrrrr@{}}
  	\toprule
  	        & Mercedes &  BMW & Porsche & Alfa Romeo & Totaal \\ \midrule
  	 Mannen &     $10$ & $10$ &    $20$ &       $20$ &   $60$ \\
  	Vrouwen &     $20$ &  $5$ &    $15$ &        $0$ &   $40$ \\
  	 Totaal &     $30$ & $15$ &    $35$ &       $20$ &  $100$ \\ \bottomrule
  \end{tabular}
  \caption{Tabel die uitdrukt hoeveel vrouwen en hoeveel mannen een voorkeur voor een bepaald automerk hebben.}
  \label{tab:autovoorkeur}
\end{table}

\section{Regressie}
\label{sec:regressie}

Bij \index{Regressie} regressie gaan we proberen een consistente en systematische koppeling tussen de variabelen te vinden. Dat betekent concreet: ``als we de waarde van de onafhankelijke variabele kennen, kunnen we dan ook de waarde van de afhankelijke variabele voorspellen?'' We kennen twee soorten verbanden:
\begin{description}
  \item [Monotoon:] een monotoon verband is een verband waarbij de onderzoeker de algemene richting van de samenhang tussen de twee variabelen kan aanduiden, hetzij stijgend, hetzij dalend. De richting van het verband verandert nooit.
  \item [Niet-monotoon:] bij een niet-monotoon verband wordt de aanwezigheid (of afwezigheid) van de ene variabele systematisch gerelateerd aan de aanwezigheid (of afwezigheid) van een andere variabele. De richting van het verband kan echter niet aangeduid worden.
\end{description}

Bij lineaire regressie gaan we ons beperken tot een lineair verband: een rechtlijnige samenhang tussen een onafhankelijke en afhankelijke variabele, waarbij kennis van de onafhankelijke variabele kennis over de afhankelijke variabele geeft.

Bij een lineair verband zijn er drie karakteristieken:

\begin{enumerate}
  \item Aanwezigheid: is er wel een verband tussen de twee variabelen?
  \item Richting: is er een dalend of een stijgend verband?
  \item Wat is de sterkte van het verband: sterk, gematigd of niet-bestaand?
\end{enumerate}

Een voorbeeld van een linear verband $y = \beta_{0} + \beta_{1}x$  vind je bijvoorbeeld in figuur \ref{fig:regressieFig}.

\begin{figure}[t]
  \begin{tikzpicture}
    \begin{axis}[
        axis x line=middle,
        axis y line=middle,
        enlarge y limits=true,
        width=\textwidth, height=8cm,     % size of the image
        grid = major,
        grid style={dashed, gray!30},
        ylabel=$y$,
        xlabel=$x$,
        legend style={at={(0.1,-0.1)}, anchor=north}
      ]
      \addplot[only marks] table  {data/regressie.dat};
      \addplot [no markers, thick, red] table [y={create col/linear regression={y=y}}] {data/regressie.dat};
    \end{axis}
  \end{tikzpicture}
  \caption{Een voorbeeld van een lineair verband}
  \label{fig:regressieFig}
\end{figure}

Zo'n verband kunnen we vinden aan de hand van de \index{Kleinste kwadraten methode} kleinste kwadraten methode van Gauss. Dit wordt als volgt gedaan.

\begin{theorem}

  Een lineair verband wordt weergegeven als volgt:

  \begin{equation}
    y = \beta_{0} + \beta_{1} x
    \label{eq:lineair}
  \end{equation}
  met
  \begin{itemize}
    \item $y$ de afhankelijke
    \item $x$ de onafhankelijke
  \end{itemize}

  We willen hier de som van de kwadraten minimaliseren van de afwijkingen $e_{i} = y_{i} - (\beta_{0} + \beta_{1}x_{i})$. Zo'n afwijking kan ook geschreven worden als (stel $X_{i} = x_{i} - \overline{x}$ en $Y_{i} = y_{i} - \overline{y}$):

  \begin{eqnarray}
    e_{i} & = & y_{i} - \beta_{1} x_{i} - \beta_{0} \\
    e_{i} & = & (y_{i} - \overline{y}) - \beta_{1}(x_{i} - \overline{x}) - (\beta_{0} - \overline{y} + \beta_{1} \overline{x}) \\
    \label{eq:regressie-bewijs}
    e_{i} & = & Y_{i} - \beta_{1} X_{i} - (\beta_{0} - \overline{y} + \beta_{1} \overline{x})
  \end{eqnarray}

  In stap~\ref{eq:regressie-bewijs} doen we eigenlijk $+\overline{x}-\overline{x}+\overline{y}-\overline{y}$, wat een nuloperatie is. Dit is een gedachtensprong die niet meteen voor de hand ligt, maar onthou dat dit een ``shortcut'' is naar de oplossing en dat het ``echte'' bewijs een stuk ingewikkelder is.

  We willen de som van de kwadraten van $e_i$  minimaliseren:

  \begin{eqnarray}
    \sum_{i}^{n} e_{i}^{2} & =& \sum_{i}^{n} (y_{i} - (\beta_{0} + \beta_{1}x_{i}))^{2}\\
    & = & \sum_{i}^{n} ((Y_{i} - \beta_{1} X_{i}) - (\beta_{0} - \overline{y} + \beta_{1}\overline{x}))^{2}\\
    & = & \sum_{i}^{n}(Y_{i} - \beta_{1} X_{i})^2 - 2 \sum_{i}^{n}(Y_i - \beta_1 X_i)(\beta_0 - \overline{y})+ \beta_1\overline{x}) + (\beta_{0} - \overline{y} + \beta_{1}\overline{x}))^{2} \label{eq:stap1}\\
    & = & \sum_{i}^{n}(Y_{i} - \beta_{1} X_{i})^{2} + n(\beta_{0} - \overline{y} + \beta_{1} \overline{x})^{2} \label{eq:stap2}
  \end{eqnarray}
	
	We kunnen de stap maken van \ref{eq:stap1} naar \ref{eq:stap2} door volgende uit te werken:
	
\[ \sum_{i}^{n}X_i = \sum_{i}^{n} (x_i - \overline{x}) = 0 \]
	en equivalent
\[ \sum_{i}^{n}Y_i = \sum_{i}^{n} (y_i - \overline{y}) = 0 \]
daardoor is
\[ \sum_{i}^{n}(Y_i - \beta_1 X_i) = \sum_{i}^{n}Y_i - \beta_1 \sum_{i}^{n}X_i = 0 \]
en bijgevolg dus ook
\[ 2 \sum_{i}^{n}(Y_i - \beta_1 X_i)(\beta_0 - \overline{y}) \]
	

Nu is $e^{2}_{i}$ geschreven als een som van twee positieve uitdrukkingen. Deze som is minimaal als beide uitdrukkingen minimaal zijn.

  \begin{equation}
    \begin{cases}
      \sum_{i}^{n}( Y_{i} - \beta_{1} X_{i})^{2} \textnormal{ is minimaal.}\\
      n(\beta_{0} - \overline{y} + \beta_{1} \overline{x})^{2} \textnormal{ is minimaal}
    \end{cases}
    \label{eq:vgl}
  \end{equation}
	

Voor de eerste uitdrukking vinden we eigenlijk een kwadratische functie in $\beta_1$.
  \begin{eqnarray}
		& \sum_{i}^{n}( Y_{i} - \beta_{1} X_{i})^{2} \textnormal{ is minimaal.} \label{eq:uitdrukking}\\
		\Leftrightarrow & \sum_i^n (Y_i^2 - 2X_iY_i\beta_1 + X_i^2\beta_1^2 \textnormal{ is minimaal.} \\
		\Leftrightarrow & \beta_1^2 \sum_i^n X_i^2 - 2\beta_1 \sum_i^n X_iY_i + \sum_i^nY_i^2 \textnormal{ is minimaal.} \\
		\Leftrightarrow & \textnormal{is minimaal als } \beta_{1} = \frac{\sum_{i}^{n} X_{i}Y_{i}}{\sum_{i}^{n} X_{i}^{2}}
	\end{eqnarray}

Voor de tweede uitdrukking vinden we

  \begin{eqnarray}
		& n(\beta_{0} - \overline{y} + \beta_{1} \overline{x})^{2} \textnormal{ is minimaal}
		\Leftrightarrow & n(\beta_{0} - \overline{y} + \beta_{1} \overline{x})^{2} = 0 \\
		\Leftrightarrow & \beta_{0} - \overline{y} + \beta_{1} \overline{x} = 0 \\
		\Leftrightarrow & \beta_{0} = \overline{y} - \beta_{1}\overline{x} 
	\end{eqnarray}

	
  met als oplossing

  \begin{equation}
    \begin{cases}
      \beta_{1} = \frac{\sum_{i}^{n} X_{i}Y_{i}}{\sum_{i}^{n} X_{i}^{2}}\\
      \beta_{0} = \overline{y} - \beta_{1}\overline{x}
    \end{cases}
    \label{eq:vgl2}
  \end{equation}

  en dus

  \begin{eqnarray}
    \beta_{1} & = & \frac{\sum_{i}^{n} (x_{i} - \overline{x})(y_{i} - \overline{y})}{\sum_{i}^{n} (x_{i} - \overline{x})^{2}} \\
    \beta_{0} & = & \overline{y} - \beta_{1} \overline{x}
    \label{eq:regressie}
  \end{eqnarray}
\end{theorem}


\begin{table} \centering
  \begin{tabular}{@{}rr@{}} \toprule
    Eiwitgehalte\%& Gewichtstoename (gram)  \\
    \midrule
    0		&	177 \\
    10 	&	231	\\
    20	& 249	\\
    30	& 348 \\
    40	& 361 \\
    50	& 384 \\
    60	& 404 \\
    \bottomrule
  \end{tabular}
  \caption{De data die verzameld geweest is door de kerstman: per eiwitpercentage wordt de gewichtstoename beschouwd.}
  \label{tab:rendieren}
\end{table}



\begin{table} \centering
  \begin{tabular}{@{}llllll@{}}
    \toprule
    $x$   & $y$     & $x-\overline{x}$    & $y - \overline{y}$        & $(x-\overline{x})(y - \overline{y})$       &  $(x-\overline{x})^{2}$    \\ \midrule
    0  & 177 & -30 & -130,71 & 3921,3 & 900  \\
    10 & 231 & -20 & -76,71  & 1534,2 & 400  \\
    20 & 249 & -10 & -58,71  & 587,1  & 100  \\
    30 & 348 & 0   & 40,29   & 0      & 0    \\
    40 & 361 & 10  & 53,29   & 532,9  & 100  \\
    50 & 384 & 20  & 76,29   & 1525,8 & 400  \\
    60 & 404 & 30  & 96,29   & 2888,7 & 900  \\
    &     &     &         & 10990  & 2800 \\ \bottomrule
  \end{tabular}
  \caption{Berekeningen die nodig zijn voor het toepassen van de kleinste kwadratenmethode.}
  \label{tab:rendieren2}
\end{table}

\begin{figure}
  \begin{tikzpicture}
    \begin{axis}[
        axis x line=middle,
        axis y line=middle,
        enlarge y limits=true,
        width=\textwidth, height=8cm,     % size of the image
        grid = major,
        grid style={dashed, gray!30},
        ylabel=gewichtstoename (g),
        xlabel=eiwitgehalte (\%),
        legend style={at={(0.1,-0.1)}, anchor=north}
      ]
      \addplot[only marks] table  {data/santa.txt};
      \addplot [no markers, thick, red] table [y={create col/linear regression={y=y}}] {data/santa.txt};
    \end{axis}
  \end{tikzpicture}

  \caption{Lineair verband tussen eiwitgehalte en gewichtstoename}
  \label{fig:rendierenFiguur}
\end{figure}

\begin{example}
  \label{vb:rendieren}
  We kijken naar het voorbeeld van de Kerstman en zijn rendieren. Hij wil zien of er een lineair verband bestaat tussen het eiwitgehalte van het voeder en de gewichtstoename van de rendieren. Hij voert een aantal proeven uit en bekomt de data in tabel \ref{tab:rendieren}. Door toepassing van de formules die hierboven staan bekomt men (zie tabel \ref{tab:rendieren2}):
  \[ \beta_{1} = \frac{\sum_{i=1}^{n} (x_{i}-\overline{x})(y_{i} - \overline{y})}{\sum_{i=1}^{n} (x-\overline{x})^{2}} = \frac{10990}{2800} = 3.925 \]
  \[ \beta_{0} = \overline{y} - \beta_{1} \overline{x} = 307.7143 - 3.925 \times 30 = 189.96 \]
  Men heeft dus een lineair verband gevonden die de kwadraten van de residuen minimaliseert. Let wel, er wordt niets gezegd over de sterkte of validiteit van dit verband. Dit verband wordt getekend in figuur \ref{fig:rendierenFiguur}.
\end{example}

Voorbeeld~\ref{vb:rendieren} uitgewerkt in R (met plot van de regressierechte):

\lstinputlisting{data/regressie.R}

% TODO: Klopt dit??? Dit lijkt Van een ander voorbeeld te komen,
% de waarden komen niet overeen met cov(x, y) in R.
%We vinden verdere statistieken:
%\begin{itemize}
%  \item Pearson-correlatieco\"effici\"ent $R = 0.725$. Duidt op een sterke lineaire samenhang. Zie Sectie~\ref{sec:correlatie}.
%  \item Determinatieco\"effici\"ent $R^{2}=0.526$. Duidt aan dat 52\% van de variantie in de bestedingen kan verklaard worden door de variantie in het aantal dagen dat iemand het restaurant bezoekt.
%\end{itemize}

\section{Correlatie}
\label{sec:correlatie}

\subsection{Pearsons product-momentcorrelatiecoëfficiënt}
We kunnen twee statistieken bepalen die de sterkte van een lineair verband uitdrukken.

\begin{definition}[Pearsons product-momentcorrelatiecoëfficiënt]
   Pearsons product momentcorrelatiecoëfficiënt\index{Pearsons product-momentcorrelatiecoëfficiënt} $R$ (of kortweg correlatiecoëfficiënt\index{correlatiecoëfficiënt}) is een maat voor de sterkte van de lineaire samenhang tussen X en Y. De waarde kan vari\"eren van -1 tot 1.

  \begin{itemize}
    \item Een waarde van +1 duidt een positief lineair verband aan.
    \item Een waarde van -1 duidt een negatief lineair verband aan.
    \item Een waarde van 0 wil zeggen dat er totaal geen lineaire samenhang is.
  \end{itemize}
  
  Hoe dichter de correlatiecoëfficiënt bij 1 of -1, hoe beter de kwaliteit van het lineair model.
\end{definition}

\subsection{Determinatieco\"effici\"ent}
\begin{definition}
  De \index{Determinatieco\"effici\"ent}determinatieco\"effici\"ent ($R^{2}$) is het kwadraat van de correlatieco\"effici\"ent en verklaart het percentage van de variantie van de waargenomen waarden t.o.v. de regressierechte.

  \begin{itemize}
    \item $R^{2}$ is de verklaarde variantie
    \item $1-R^{2}$ is de onverklaarde variantie
  \end{itemize}
\end{definition}

\begin{figure}[t]
  \begin{tikzpicture}
    \begin{axis}[
        axis x line=middle,
        axis y line=middle,
        enlarge y limits=true,
        width=\textwidth, height=8cm,     % size of the image
        grid = major,
        grid style={dashed, gray!30},
        ylabel=gezinsgrootte moeder,
        xlabel=gezinsgrootte,
        legend style={at={(0.1,-0.1)}, anchor=north}
      ]
      \addplot[only marks] table  {data/families.txt};
      \addplot [no markers, thick, red] table [y={create col/linear regression={y=y}}] {data/families.txt};
    \end{axis}
  \end{tikzpicture}
  \caption{Linear verband tussen grootte van een gezin en de grootte van de familie van de moeder}
  \label{fig:moederVerband}
\end{figure}

\tikzset{small dot/.style={fill=black, circle,scale=0.2}}
\tikzset{every pin/.style={draw=black,fill=yellow!10}}

\begin{figure}[t]%
  \begin{tikzpicture}
    \begin{axis}[
        axis x line=middle,
        axis y line=middle,
        enlarge y limits=true,
        width=\textwidth, height=8cm,     % size of the image
        grid = major,
        grid style={dashed, gray!30},
        ylabel=gezinsgrootte moeder,
        xlabel=gezinsgrootte,
        legend style={at={(0.1,-0.1)}, anchor=north}
      ]
      \draw (axis cs:3,0)--(axis cs:3,8);
      \draw (axis cs:0,4.3)--(axis cs:6,4.3);
      \node[small dot, pin=120:{$III$}] at (axis cs:1.6,7) {};
      \node[small dot, pin=120:{$I$}] at (axis cs:5.5,7) {};
      \node[small dot, pin=120:{$II$}] at (axis cs:1.6,2) {};
      \node[small dot, pin=120:{$IV$}] at (axis cs:5.5,2) {};
      \addplot[only marks] table  {data/families.txt};
    \end{axis}
  \end{tikzpicture}
  \caption{De figuur opgedeeld in 4 kwadranten}%
  \label{fig:kwadranten}%
\end{figure}


\subsubsection{Bepaling van $R$ en $R^{2}$}
\label{sec:determinatiecoef}
Beschouw het voorbeeld in figuur \ref{fig:moederVerband}:  de grootte van een gezin vs. de grootte gezin moeder. We zien duidelijk dat er een linear verband is. Indien we de gemiddelde berekenen en de figuur in 4 kwadranten (kwadrant $I$, $II$, $III$, $IV$) volgens de gemiddelden verdelen krijgen we de figuur in \ref{fig:kwadranten}.  Dan kunnen we volgende situaties bekijken.

\begin{itemize}
  \item Neem een element uit gebied I. Voor dit element is $x_{i} - \overline{x}$ positief en $y_{i} - \overline{y}$ ook. Dus is hun product. $(x_{i} - \overline{x}) (y_{i} - \overline{y}) > 0$.
  \item Neem een element uit gebied II. Voor dit element is $x_{i} - \overline{x}$ negatief en $y_{i} - \overline{y}$ ook. Dus is hun product. $(x_{i} - \overline{x}) (y_{i} - \overline{y}) > 0$.
  \item Neem een element uit gebied III. Voor dit element is $x_{i} - \overline{x}$ negatief en $y_{i} - \overline{y}$ positief. Dus is hun product. $(x_{i} - \overline{x}) (y_{i} - \overline{y}) < 0$.
  \item Neem een element uit gebied IV. Voor dit element is $x_{i} - \overline{x}$ positief en $y_{i} - \overline{y}$ negatief. Dus is hun product. $(x_{i} - \overline{x}) (y_{i} - \overline{y}) < 0$.
\end{itemize}

Aangezien dat er meer punten in gebieden I en II zijn dan in gebieden III en IV zal de som $\sum_{i} (x_{i} - \overline{x}) (y_{i} - \overline{y})$ een positief getal zijn. Hoe meer punten in I en II, hoe groter het getal. We merken dus een sterk positief lineair verband.

Indien de punten ongeveer gelijk verdeeld zouden zijn over de vier gebieden vinden we dat deze soms dicht bij nul zal zijn. Omgekeerd, indien er een negatief lineair verband zou zijn vinden we een negatief getal.

We hebben dus een maat gevonden om het verband tussen twee variabelen te meten:

\begin{itemize}
  \item Stijgende gecorreleerde verbanden is $\sum_{i} (x_{i} - \overline{x}) (y_{i} - \overline{y})$ positief en groot.
  \item Dalende gecorreleerde verbanden is $\sum_{i} (x_{i} - \overline{x}) (y_{i} - \overline{y})$ negatief en groot (in absolute waarde).
  \item Met niet gecorreleerde variabelen is $\sum_{i} (x_{i} - \overline{x}) (y_{i} - \overline{y})$ klein in absolute waarde.
\end{itemize}

We kunnen deze maat onafhankelijk maken van de grootte van de steekproef door te delen door de steekproefgrootte $n$. Dit noemen we de co-variantie en wordt gedefinieerd als gemeenschappelijke spreiding:

\begin{equation}
  Cov(X,Y) = \frac{\sum_{i}^{n}(x_{i} - \overline{x}) (y_{i} - \overline{y})}{n}
  \label{eq:covariantie}
\end{equation}

Dit geeft ons de gemiddelde afwijking per meetpunt.

Om opnieuw te normaliseren (een variatie in X is niet per se van dezelfde grootteorde als een variatie in Y) gaan we de maatstaf voor het gezamelijk vari\"eren onafhankelijk maken van het aantal waarnemingen en de orde van grootte van de getalswaarden. Zo kunnen we deze waarden universeel vergelijkbaar maken. Daarom delen we de co-variantie door het product van de standaardafwijkingen en noemen we de relatieve co-variantie of Pearson's correlatieco\"effici\"ent ook bekend als product-moment-correlatieco\"effici\"ent of kortweg als correlatieco\"effici\"ent.

\begin{eqnarray}
  R &=&\frac{COV(X,Y)}{\sigma_{x}\sigma_{y}} \\
  &=& \frac{COV(X,Y)}{\sqrt{\frac{\sum(x_{i} - \overline{x})^{2}}{n}} \times \sqrt{\frac{\sum(y_{i} - \overline{y})^{2}}{n}}} \\
  &=& \frac{\sum_{i}^{n}(x_{i}-\overline{x})(y_{i} - \overline{y})}{\sqrt{\sum_{i}^{n} (x_{i}-\overline{x})^{2}} \sqrt{\sum_{i}^{n} (y_{i}-\overline{y})^{2}}}
  \label{eq:relCovar}
\end{eqnarray}

De correlatieco\"effici\"ent is onafhankelijk van de meeteenheid terwijl de covariantie afhankelijk is van de meeteenheid.

\subsubsection{$R^{2}$ interpretatie}
Als we aannemen dat $x$ niet bijdraagt aan de voorspelling van $y$ dan is de beste voorspelling voor een waarde van $y$ het steekproefgemiddelde $\overline{y}$, dat in figuur \ref{fig:rendierenFiguur3} als een horizontale lijn wordt weergegeven. De verticale lijnstukken zijn de afwijkingen van de waargenomen punten $y$ van deze voorspelling (het steekproefgemiddelde). De som van de kwadraten van deze afwijkingen is:

\[ SS_{yy} = \sum(y_{i} - \overline{y})^{2} \]

Indien we aannemen dat $x$ wel een rol speelt bij de voorspelling van $y$, berekenen we de regressielijn bij dezelfde gegevensverzameling en de afwijkingen van de punten ten opzichte van de lijn zoals in figuur \ref{fig:rendierenFiguur2}.

\[ SSE_{yy} = \sum(y_{i} - \widehat{y})^{2} \]

 Als we nu de afwijkingen vergelijken met elkaar zien we het volgende:
\begin{enumerate}
  \item Als $x$ weinig of niet bijdraagt in de voorspelling zullen de sommen van de kwadraten van de afwijkingen van de twee lijnen nagenoeg dezelfde zijn:
    \[ SS_{yy} = \sum(y_{i} - \overline{y})^{2} \] en
    \[ SSE_{yy} = \sum(y_{i} - \widehat{y})^{2} \]
    waarbij $\widehat{y}$ de voorspelde waarde is.
  \item Als $x$ wel bijdraagt tot de voorspelling van $y$ zal $SSE$ kleiner zijn dan $SS_{yy}$. In feite zal
    \[	SSE_{yy} = \sum(y_{i} - \widehat{y})^{2} \]
    gelijk zijn aan nul als alle punten perfect voorspeld worden (en dus op de regressierechte liggen).
\end{enumerate}

De vermindering in de som van de kwadraten die toegeschreven kan worden aan het opnemen van $x$ in het model is dan (uitgedrukt in fractie van $SS_{yy}$)
\[ \frac{SS_{yy} - SSE_{yy}}{SS_{yy}} \]
We noemen $SS_{yy}$ de totale steekproefvariantie van de meetwaarden rond het steekproefgemiddelde $\overline{y}$ en $SSE_{yy}$ de overblijvende niet-verklaarde steekproefvariantie, na het schatten van de lijn $\widehat{y} = \beta_{0} + \beta_{1}x$. Dus dan is $(SS_{yy} - SSE_{yy})$ de verklaarde variantie die toe te schrijven is aan de lineaire relatie met $x$.

Er kan nu worden aangetoond dat bij enkelvoudige lineaire regressie deze fractie
\[ \frac{SS_{yy} - SSE_{yy}}{SS_{yy}} = \frac{\textnormal{verklaarde variantie}}{\textnormal{totale steekproefvariantie}} \]
gelijk is aan het kwadraat van de pearsoncorrelatieco\"effici\"ent.  (= het deel van de totale variantie dat verklaard wordt door de lineaire rechte).


\begin{figure}[t]
  \begin{tikzpicture}
    \begin{axis}[
        axis x line=middle,
        axis y line=middle,
        enlarge y limits=true,
        width=\textwidth, height=8cm,     % size of the image
        grid = major,
        grid style={dashed, gray!30},
        ylabel=eiwitgehalte,
        xlabel=gewichtstoename(gram),
        legend style={at={(0.1,-0.1)}, anchor=north}
      ]
      \addplot[only marks] table  {data/santa.txt};
      \addplot [no markers, thick, red] table [y={create col/linear regression={y=y}}] {data/santa.txt};
      \addplot [mark=none, color=red] coordinates {
        (0,177) (0,189.9643)
      };
      \addplot [mark=none, color=red] coordinates {
        (10,231) (10,229.2143)
      };
      \addplot [mark=none, color=red] coordinates {
        (20,249) (20,268.4643)
      };
      \addplot [mark=none, color=red] coordinates {
        (30,348) (30,307.7143)
      };
      \addplot [mark=none, color=red] coordinates {
        (40,361) (40,346.9643)
      };
      \addplot [mark=none, color=red] coordinates {
        (50,384) (50,386.2143)
      };
      \addplot [mark=none, color=red] coordinates {
        (60,404) (60,425.4643)
      };

    \end{axis}
  \end{tikzpicture}
  \caption{Deviaties tot de regressierechte: aanname $x$ geeft extra informatie voor het voorspellen van $y$.}
	\label{fig:rendierenFiguur2}
\end{figure}

\begin{figure}[t]
  \begin{tikzpicture}
    \begin{axis}[
        axis x line=middle,
        axis y line=middle,
        enlarge y limits=true,
        width=\textwidth, height=8cm,     % size of the image
        grid = major,
        grid style={dashed, gray!30},
        ylabel=eiwitgehalte,
        xlabel=gewichtstoename(gram),
      ]
      \addplot[only marks] table  {data/santa.txt};
      \addplot [mark=none, color=black] coordinates {
        (0,307.71) (60,307.71)
      };
      \addplot [mark=none, color=red] coordinates {
        (0,177) (0,307.71)
      };
      \addplot [mark=none, color=red] coordinates {
        (10,231) (10,307.71)
      };
      \addplot [mark=none, color=red] coordinates {
        (20,249) (20,307.71)
      };
      \addplot [mark=none, color=red] coordinates {
        (30,348) (30,307.71)
      };
      \addplot [mark=none, color=red] coordinates {
        (40,361) (40,307.71)
      };
      \addplot [mark=none, color=red] coordinates {
        (50,384) (50,307.71)
      };
      \addplot [mark=none, color=red] coordinates {
        (60,404) (60,307.71)
      };

    \end{axis}
  \end{tikzpicture}
  \caption{Deviaties tot de gemiddelde van y: aanname $x$ geeft geen informatie voor het voorspellen van $y$ ($\overline{y} =307.71$).}
	  \label{fig:rendierenFiguur3}
\end{figure}


\section{Conclusie}

Er bestaan verschillende soorten verbanden tussen variabelen. Wij zijn geïnteresseerd in monotone en lineaire verbanden. We beschikken hier over een correlatieco\"effici\"ent en lineaire regressie. Deze technieken mogen niet met nominale en ordinale variabelen gebruikt worden. Een kleine waarde $(=0)$ voor een maat voor verband betekent alleen dat het overeenkomend verband afwezig is: er kan een ander soort verband aanwezig zijn. Het gebruik van een spreidingsdiagram is dus altijd aan te raden.

Het feit dat twee variabelen gecorreleerd zijn betekent niet dat de ene de oorzaak is van de andere.

\section{Samenvatting}

In dit hoofdstuk zijn verschillende technieken voorgesteld om na te gaan of er een verband bestaat tussen twee variabelen. De ene variabele noemen we de \emph{onafhankelijke}, de andere de \emph{afhankelijke} variabele. Wat we willen uitzoeken is of de waarde van de onafhankelijke variabele een impact heeft op die van de afhankelijke.

De technieken die we kunnen gebruiken (hetzij rekenkundige, hetzij voor visualisatie), hangen af van het meetniveau van de onderzochte variabelen. Tabel~\ref{tab:overzicht-2-variabelen} geeft een overzicht.

\begin{table}

  \begin{tabular}{llll}
    \toprule
    \multicolumn{2}{c}{\textbf{Meetniveau variabele}}             & \textbf{}                   & \textbf{}                                             \\
    \textbf{Onafhankelijke}       & \textbf{Afhankelijke}         & \textbf{Numeriek}           & \textbf{Visualisatie}                                 \\
    \midrule
    \multirow{3}{*}{Kwalitatief}  & \multirow{3}{*}{Kwalitatief}  & $\chi^2$                    & mozaïekdiagram                                        \\
    &                               & Cramér's V                  & geclusterd staafdiagram                               \\
    &                               &                             & rependiagram                                          \\
    \midrule
    \multirow{2}{*}{Kwalitatief}  & \multirow{2}{*}{Kwantitatief} & t-toets voor 2 steekproeven & boxplot                                               \\
    &                               &                             & \parbox{4.5cm}{(evt. staafdiagram gemiddelde met standaardafwijking)} \\
    \midrule
    \multirow{3}{*}{Kwantitatief} & \multirow{3}{*}{Kwantitatief} & covariantie                 & spreidings-/XY-diagram \\
    &                               & correlatiecoëfficiënt       & regressierechte                                       \\
    &                               & determinatiecoëfficiënt     &                                                      \\
    \bottomrule
  \end{tabular}
  
  \caption{Overzicht technieken voor de analyse van twee variabelen.}
  \label{tab:overzicht-2-variabelen}
\end{table}

\section{Oefeningen}
\label{sec:analyse op 2 variabelen-oefeningen}

De databestanden voor deze oefeningen zijn te vinden op Github (in de directory \emph{oefeningen/data/hfst6\_2variabelen}).

\begin{exercise}
  \label{oef:Kruistabellen, Cramér's V en chi-kwadraat - handmatig} % $\chi^{2}$ - handmatig}

  Marktonderzoek toont aan dat achtergrondmuziek in een supermarkt invloed kan hebben op het aankoopgedrag van de klanten. In een onderzoek werden drie methoden met elkaar vergeleken: geen muziek, Franse chansons en Italiaanse hits. Telkens werd het aantal verkochte flessen Franse, Italiaanse en andere wijnen geteld~\autocite{Ryan1998}.
  
  De onderzoeksdata bevindt zich in het csv-bestand MuziekWijn.

  Vragen:
  \begin{enumerate}
    \item Stel de correcte kruistabel op. Gebruik hiervoor het R-commando \textit{table} om de frequentietabel te bekomen.
    \item Bepaal de marginalen.
    \item Bepaal de verwachte resultaten.
    \item Bereken manueel de $\chi^{2}$ toetsingsgrootheid.  
    \item Bereken manueel de Cramér's V. Wat kan je hieruit besluiten?
  \end{enumerate}
\end{exercise}

\begin{exercise}
	Gebruik dezelfde data.
		\begin{enumerate}
		\item Stel de percentages verkochte wijnen voor in een staafdiagram met de  muziekconditie= Geen.
		\item Stel de percentages verkochte wijnen voor in een geclusterd staafdiagram (clustered bar chart).
		\item Stel de percentages verkochte wijnen voor in rependiagram (stacked bar chart).
	\end{enumerate}
\end{exercise}

\begin{exercise}
	\label{oef:Kruistabellen, Cramér's V en chi-kwadraat met R } %$\chi^{2}$ met R}
Lees het databestand ``Aardbevingen.csv'' in. 	
		\begin{enumerate}
		\item Maak een histogram en een boxplot van de variabele ``Magnitudes''.
		\item Maak een lijngrafiek met het aantal aardbevingen per maand.
		\item Onderzoek of er een verband bestaat tussen de variabelen ``Type'' en ``Source''. Bereken ook de Cramér's V-waarde. Wat is de conclusie?
	\end{enumerate}
\end{exercise}

\begin{exercise}
	\label{oef:Lineaire regressie, correlatie en determinatiecoëfficiënt - handmatig}
	In onderstaande tabel vindt men voor elke rij (= persoon) het resultaat van een test en zijn examenscore. Gevraagd:
\begin{itemize}
	\item Bepaal handmatig de regressierechte $\beta_{0} + \beta_{1} x$. 
	\item Bepaal handmatig de correlatie- en determinatieco\"effici\"ent ($R^{2}, R$) 
	\item Geef uitleg bij de gevonden statistieken.
\end{itemize}

% \begin{table}
	\centering
	\begin{tabular}{@{}rr@{}} \toprule
	Resultaat Test ($X$) & Examenresultaat ($Y$) \\
		\midrule
		10 & 11 \\
		12 & 14 \\
		8 & 9 \\
		13 & 13 \\
		9 & 9 \\
		10 &  9 \\
		7 & 8 \\
		14 & 14 \\
		11 & 10 \\
		6 & 6  \\
		\bottomrule
	\end{tabular}
	\captionof{table}{Scores test en examen voor aantal personen}
	\label{tab:testExamen}
% \end{table}	
\end{exercise}

\begin{exercise}
	Gegeven 6 scatterplots in volgende figuur en onderstaande correlatieco\"effici\"enten. Match de co\"effici\"enten met de scatterplots. Er is dus één scatterplot waarvan geen correlatie gegeven staat hieronder.
\begin{itemize}
	\item $r_{1}$ = 0.6
	\item $r_{2}$ = 0
	\item $r_{3}$ = -0.9
	\item $r_{4}$ = 0.9
	\item $r_{5}$ = 0.3
\end{itemize}
%\begin{figure}[h!]
%	\centering
	\includegraphics[width=1.10\textwidth]{images/correlaties.png}
	\captionof{figure}{Correlaties}
	\label{fig:correlaties}
%\end{figure}
\end{exercise}

\begin{exercise}
	\label{oef:Lineaire regressie, correlatie en determinatiecoëfficiënt met R}
	Lees het databestand ``Cats.csv'' in. 
		\begin{enumerate}
		\item Voer een lineaire regressieanalyse uit op de variabelen Lichaamsgewicht (Bwt) en Gewicht hart (Hwt).
		\item Maak een spreidingsdiagram van beide variabelen.
		\item Bereken en teken de regressielijn.
		\item Bereken de correlatie- en de determinatiecoëfficiënt.
		\item Geef een interpretatie van deze resultaten.
	\end{enumerate}
\end{exercise}

\begin{exercise}
	Gebruik dezelfde data als in vorige oefening.
		\begin{enumerate}
		\item Voer een lineaire regressieanalyse uit op de variabelen Lichaamsgewicht (Bwt) en Gewicht hart (Hwt) per geslacht.
		\item Maak een spreidingsdiagram van beide variabelen voor elk van de geslachten.
		\item Bereken en teken telkens de regressielijn.
		\item Bereken de correlatie- en de determinatiecoëfficiënt.
		\item Geef een interpretatie aan deze resultaten.
	\end{enumerate}
\end{exercise}

\begin{exercise}
	Lees het databestand ``Pizza.csv'' in.
		\begin{enumerate}
		\item Voer een volledige lineaire regressieanalyse uit op de variabelen Rating en CostPerSlice. Trek hieruit de juiste conclusies en ga deze ook grafisch na.
		\item Onderzoek een mogelijk verband tussen Rating en Neighbourhood. Welke methode kan je hiervoor gebruiken? Kan je de gegevens van Rating hiervoor in dezelfde vorm gebruiken?
		\item Geef een interpretatie aan deze resultaten.
		\item Stel de kruistabel grafisch voor met een staafdiagram.  Voorzie een legende.
	\end{enumerate}
\end{exercise}

\chapter{De \texorpdfstring{$\chi^{2}$}{Chi-kwadraat} toets}
\label{ch:chikwadraat}

\section{\texorpdfstring{$\chi^{2}$}{Chi-kwadraat} toets voor verdelingen}


\section{Oefeningen}
\label{sec:chi-kwadraat-oefeningen}

\begin{exercise}
  \label{ex:chisq-survey}
  Voor deze oefening maken we gebruik van de dataset \texttt{survey} die is meegeleverd met R. De dataset is samengesteld uit een bevraging onder studenten. Om deze te laden, doe het volgende:
  
  \begin{lstlisting}
  library(MASS)
  View(survey)  # Toont de "survey" dataset
  ?survey       # Help-pagina voor deze dataset met uitleg over de inhoud
  \end{lstlisting}
  
  Als je een foutboodschap krijgt bij het laden van de bibliotheek (eerste regel), betekent dit dat de package \texttt{MASS} nog niet geïnstalleerd is. Dit kan je alsnog doen via Tools > Install Packages en het invullen van de package-naam in het tekstveld.
  
  We willen de relatie onderzoeken tussen enkele discrete (nominale of ordinale) variabelen in deze dataset. Voor elke hieronder opgesomde paren, volg deze stappen:
  
  \begin{enumerate}[label=(\alph*)]
    \item Denk eerst eens na welke uitkomt je precies verwacht voor de opgegeven combinatie van variabelen.
    \item Stel een frequentietabel op voor de twee variabelen. De (vermoedelijk) onafhankelijke variabele komt eerst.
    \item Plot een grafiek van de data, bv.~geclusterde staafgrafiek, gestapelde staafgrafiek van relatieve frequenties, of een ``mozaïekgrafiek'' (eenvoudig met \texttt{plot(table(data\$col1, data\$col2))}).
    \item Als je de grafiek bekijkt, verwacht je dan een eerder hoge of eerder lage waarde voor de $\chi^2$-statistiek? Waarom?
    \item Bereken de $\chi^2$-statistiek en de kritieke grenswaarde $g$ (voor significantieniveau $\alpha = 0.05$)
    \item Bereken de $p$-waarde
    \item Moeten we de nulhypothese aanvaarden of verwerpen? Wat betekent dat concreet voor de relatie tussen de twee variabelen?
  \end{enumerate}

  Hieronder zijn de te onderzoeken variabelen opgesomd. De vermoedelijke onafhankelijke variabele komt telkens eerst.
  
  \begin{enumerate}
    \item \texttt{Exer} (sporten) en \texttt{Smoke} (rookgedrag)
    \item \texttt{W.Hnd} (de hand waarmee je schrijft) en \texttt{Fold} (de hand die bovenaan komt als je de armen kruist)
    \item \texttt{Sex} (gender) en \texttt{Smoke}
    \item \texttt{Sex} en \texttt{W.Hnd}
  \end{enumerate}
\end{exercise}

\begin{exercise}
  \label{ex:chisq-aids2}
  Laad de dataset \texttt{Aids2} uit package \texttt{MASS} (zie Oefening~\ref{ex:chisq-survey}) die informatie bevat over 2843 patiënten die vóór 1991 in Australië met AIDS besmet werden. Deze dataset werd in detail besproken door~\textcite{Ripley2007}. Onderzoek of er een relatie is tussen de variabele geslacht (\texttt{Sex}) en de manier van besmetting (\texttt{T.categ}).
  
  \begin{enumerate}
    \item Ga op de gebruikelijke manier te werk: visualiseren van de data, $\chi^2$, $g$ en $p$-waarde berekenen ($\alpha = 0,05$), en tenslotte een conclusie formuleren.
    \item Bepaal de gestandaardiseerde residuën om te bepalen welke categorieën extreme waarden bevatten.
  \end{enumerate}
  
\end{exercise}

\begin{exercise}
  \label{ex:chisq-digimeter}
  
  Elk jaar voert Imec (voorheen iMinds) een studie uit over het gebruik van digitale technologieën in Vlaanderen, de Digimeter~\autocite{Vanhaelewyn2016}. In deze oefening zullen we nagaan of de steekproef van de Digimeter 2016 ($n = 2164$) representatief is voor de bevolking wat betreft de leeftijdscategorieën van de deelnemers.
  
  In Tabel~\ref{tab:digimeter2016} worden de relatieve frequencies van de deelnemers weergegeven. De absolute frequenties voor de verschillende leeftijdscategorieën van de Vlaamse bevolking worden samengevat in Tabel~\ref{tab:leeftijd-vlaanderen}. Deze gegevens zijn ook te vinden in bijgevoegd CSV-bestand \texttt{oefeningen/data/bestat-vl-ages.csv}.
  
  \begin{enumerate}
    \item De tabel met leeftijdsgegevens van de Vlaamse bevolking als geheel heeft meer categorieën dan deze gebruikt in de Digimeter. Maak een samenvatting zodat je dezelfde categorieën overhoudt dan deze van de Digimeter. Tip: dit gaat misschien makkelijker in een rekenblad dan in R.
    \item Om de goodness-of-fit test te kunnen toepassen hebben we de absolute frequenties nodig van de geobserveerde waarden in de steekproef. Bereken deze.
    \item Bereken ook de verwachte percentages ($\pi_{i}$) voor de populatie als geheel.
    \item Voer de goodness-of-fit test uit over de verdeling van leeftijdscategorieën in de steekproef van de Digimeter. Is de steekproef in dit opzicht inderdaad representatief voor de Vlaamse bevolking?
  \end{enumerate}
\end{exercise}

\begin{table}
  \caption{Frequenties van de leeftijd van deelnemers aan de iMec Digimeter 2016 en de Vlaamse bevolking.}
  \label{tab:frequenties-leeftijden}
  \centering
  \begin{tabular}{cc}
    \textbf{Leeftijdsgroep} & \textbf{Percentage} \\ \midrule
    15-19 & 6,6\% \\
    20-29 & 14,2\% \\
    30-39 & 15,0\% \\
    40-49 & 16,3\% \\
    50-59 & 17,3\% \\
    60-64 & 7,3\% \\
    64+   & 23,2\% \\
  \end{tabular}
  \subcaption{Percentage van deelnemers aan de Digimeter 2016 van iMec ($n = 2164$), opgedeeld per leeftijdscategorie. \autocite{Vanhaelewyn2016}}
  \label{tab:digimeter2016}
  
  \centering
  \begin{tabular}{cc}
    \textbf{Leeftijdsgroep} & \textbf{Aantal} \\ \midrule
              –5            &     352017      \\
              5-9           &     330320      \\
             10-14          &     341303      \\
             15-19          &     366648      \\
             20-24          &     375469      \\
             25-29          &     387131      \\
             30-34          &     401285      \\
             35-39          &     409587      \\
             40-44          &     458485      \\
             45-49          &     493720      \\
             50-54          &     463668      \\
             55-59          &     413315      \\
             60-64          &     379301      \\
             65-69          &     299152      \\
             70-74          &     279789      \\
             75-79          &     249260      \\
             80-84          &     182352      \\
             85-89          &     104449      \\
             90-94          &      29888      \\
             95-99          &      7678       \\
             100+           &       923
  \end{tabular}
  \subcaption{Absolute frequentie van de Vlaamse bevolking per leeftijdscategorie. Bron: BelStat (\url{https://bestat.economie.fgov.be/bestat/}, C01.1: Bevolking volgens verblijfplaats (provincie), geslacht, positie in het huishouden (C), burgerlijke staat en leeftijd (B)).}
  \label{tab:leeftijd-vlaanderen}
  

\end{table}

\section{Antwoorden op geselecteerde oefeningen}
\label{sec:chi-kwadraat-oplossingen}

\paragraph{Oefening~\ref{ex:chisq-survey}}

\begin{enumerate}
  \item \texttt{Exer}/\texttt{Smoke}: $\chi^2 = 5.4885$, $g = 12.59159$, $p = 0.4828422$
  \item \texttt{W.Hnd}/\texttt{Fold}: $\chi^2 = 1.581399$, $g = 5.9915$, $p = 0.454$
  \item \texttt{Sex}/\texttt{Smoke}: $\chi^2 = 3.554$, $g = 7.8147$, $p = 0.314$
  \item \texttt{Sex}/\texttt{W.Hnd}: $\chi^2 = 0.236$, $g = 3.8415$, $p = 0.627$
\end{enumerate}

\paragraph{Oefening~\ref{ex:chisq-aids2}} $\chi^2 = 1083.372914$, $g = 14.067140$, $p \approx 1.157 \times 10^{-229}$

\paragraph{Oefening~\ref{ex:chisq-digimeter}} $\chi^2 \approx 6.6997$ ($df = 6$), $g \approx 12.5916$, $p \approx 0.3495$

\chapter{Tijdreeksen}

\section{Tijdreeksen \& voorspellingen}

\begin{definition}[Tijdsreeks]
	Een tijdreeks is een opeenvolging van observaties van een willekeurige variabele in functie van de tijd.
\end{definition}

Een tijdreeks is dus eens stochastisch proces. Denk hierbij maar aan:
\begin{itemize}
	\item maandelijkse vraag naar melk
	\item jaarlijkse instroom van studenten bij de Hogeschool
	\item dagelijks debiet van een rivier
	\item verkoop van een bepaalde order bij een bedrijf
\end{itemize}

Het voorspellen van tijdreeksen is een belangrijk onderdeel van onderzoek omdat ze vaak de basis vormen voor beslissingsmodellen. Voorbeelden hiervan zijn :

\begin{itemize}
	\item algemene ontwikkeling van toekomstplannen (investeringen, capaciteit \dots)
	\item plannen van budgettering om tekortkomingen te vermijden (operationeel budget, marketing budget \dots)
	\item competitieve leveringstijden van een bedrijf
	\item ondersteuning van financi\"ele objectieven
	\item onzekerheid vermijden
	\item de mogelijkheid om ontwikkelingen in de verkeersveiligheid
kwantitatief te modelleren
\end{itemize}

Tijdreeksen modelleren is een statistisch probleem: we gaan ervan uit dat de observaties vari\"eren volgens een bepaalde kansdichtheidsfunctie in functie van de tijd. Vaak gaan we ervan uit dat de observaties in een tijdsreeks gecorreleerd zijn en dus niet uit een random sample komen. 

Er zijn verschillende types modellen in gebruik voor het analyseren van tijdreeksen. Deze modellen hebben met elkaar gemeen dat ze in principe niet alleen de ontwikkeling in een geobserveerde tijdreeks kunnen beschrijven, maar dat we ze ook kunnen gebruiken om
\begin{inparaenum}[(i)]
	\item verklaringen te vinden voor die ontwikkeling en
	\item om de toekomstige waarden van de tijdreeks te voorspellen
\end{inparaenum}
Hun geschiktheid voor het verwezenlijken van deze doelstellingen loopt echter sterk uiteen. In dit hoofdstuk beperken we ons tot het gebruik van tijdreeksen met een geschiedenis om tijdsafhankelijke modellen te bepalen. Een voorbeeld van een tijdsreeks is bijvoorbeeld de leeftijd van de opeenvolgende koningen van Engeland startend van Willem De Veroveraar \autocite{Hipel194}.
\begin{lstlisting}
kings <- scan(file = 'Documents/education/onderzoekstechnieken-cursus/cursus/data/tijdsreeksen/kings.data', skip = 3)
kingstimeseries <- ts(kings)
plot.ts(kingstimeseries, ylab='leeftijd', xlab="tijd")
grid(lty=2,lwd=1,col='black')
\end{lstlisting}

\begin{figure}[htbp]
	\centering
	\includegraphics[width=\textwidth]{images/tijdsreeksen/tijdsreekskings.png}
	\caption{De tijdsreeks die de leeftijden van de koningen voorstelt.}
	\label{fig:tijdreeks11}
\end{figure}


\section{Tijdreeksmodellen}
\subsection{Wiskundig model}
Ons doel is het opstellen van een model dat een verklaring vindt voor de geobserveerde data en dat toelaat om observaties in de toekomst zo goed mogelijk te voorspellen. Het simpelste model dat je kan bedenken is een model waarbij een constante $b$ gebruikt wordt met variaties rond $b$ bepaald door een willekeurige variabele $\epsilon_{t}$ zoals in vergelijking \ref{eq:constante}. 

\begin{equation}
	X_{t} = b + \epsilon_{t}
\label{eq:constante}
\end{equation}

\begin{description}
	\item [$X_{t}$] stelt een \textit{variabele} voor dat de onbekende is op tijdstip $t$.
	\item [$x_{t}$] stelt een \textit{observatie} voor op tijdstip $t$ (en is dus gekend). 
	\item [$\epsilon_{t}$] noemt met de \textit{storing} (engels \textsl{noise}) en wordt geacht een gemiddelde van $0$ te hebben met variantie $\sigma^{2}$ en normaal verdeeld. 
\end{description}




We kunnen ook ervan uit gaan dat er een lineair verband is:

\begin{equation}
	X_{t} = b_{0} + b_{1} \times t + \epsilon_{t}
\label{eq:lineair8}
\end{equation}

De vergelijking in \ref{eq:constante} en \ref{eq:lineair8} zijn speciale gevallen van het polynomiaal geval:

\begin{equation}
	X_{t} = b_{0} + b_{1} t + b_{2} t^{2} + \dots + b_{n} t^{n} + \epsilon_{t} 
\label{eq:polynomiaal}
\end{equation}

\begin{exercise}
	Wat zou volgende tijdreeks kunnen voorstellen?
	\begin{equation}
		X_{t} = b_{0} + b_{1} \sin(\frac{2\pi t}{4}) + b_{1} \cos(\frac{2\pi t}{4}) + \epsilon_{t}
	\label{eq:seasonal}
\end{equation}
\end{exercise}

\begin{solution}
Antwoord: dit is een cyclische tijdreeks met periode $= 4$. Dit zou bijvoorbeeld kunnen gebruikt worden bij een tijdreeks voor seizoenen. 

\begin{lstlisting}
f <- function(a, b,t){
	return(a + b * sin((2 * pi*4)/4) + b * cos((2 * pi*4)/4) + rnorm(1))
}
t <- seq(from = 1, to = 100, by = 1)
X <- lapply(t,f,a=5,b=5)
plot(x = t, y=X, type = 'l')
\end{lstlisting}
	
\end{solution}



\subsubsection{Algemeen}

In elk model beschouwd is de tijdreeks een functie van tijd en parameters van het model. We kunnen algemeen stellen dat:

\begin{equation}
	X_{t} = f(b_{0}, b_{1}, b_{2}, \dots , b_{t}, t) + \epsilon_{t}
\label{eq:general}
\end{equation}

We aanvaarden vervolgens nog volgende stellingen:
\begin{itemize}
	\item Het model gaat uit van twee componenten van variabiliteit: het gemiddelde van de voorspellingen verandert met de tijd en de variaties tot dit gemiddelde vari\"eren willekeurig.
	\item De residuen van het model ($X_{t} - x_{t}$) zijn homoscedastisch : dat wil zeggen in de tijd een constante variantie hebben.
\end{itemize}

Eenmaal het model gekozen, rest enkel nog het  probleem van het schatten van de parameters voor vergelijking \ref{eq:general}. Dit is wat in de volgende stukken besproken zal worden.

\section{Schatten van de parameters}
Eenmaal een model geselecteerd wordt, is het aan de onderzoeker om de parameters te gaan schatten, i.e. parameters die ervoor zorgen dat het model de geobserveerde waarden zo goed mogelijk benaderen. Meestal gaan we ervan uit dat alle waarden gelijkwaardig zijn, maar dat is niet zo bij tijdreeksen. Aangezien onze onafhankelijke parameter de tijd is moeten we methoden bekomen die ervoor zorgen dat recentere data belangrijker zijn dat oude data of omgekeerd. 

In wat volgt beschrijven we de tijdreeksen met geschatte waarden voor de parameters. We zetten hievoor een hoedje op de parameters:

\[ \widehat{b}_{1}, \widehat{b}_{2} \dots \widehat{b}_{n} \] 

\subsection{Voorbeeld - moving average}

\begin{table}[t]
\centering
    \begin{tabular}{|l|l|l|l|l|l|l|l|l|l|}
    \hline
    4 & 16 & 12 & 25 & 13 & 12 & 4 & 8  & 9 & 14 \\ \hline
    3 & 14 & 14 & 20 & 7  & 9  & 6 & 11 & 3 & 11 \\ \hline
    8 & 7  & 2  & 8  & 8  & 10 & 7 & 16 & 9 & 4  \\ \hline
    \end{tabular}
    \caption{Voorbeeld data van vraag voor product, zie figuur \ref{fig:tijdreeks11}}
    \label{tab:data}
\end{table}

Stel dat de statisticus de data in tabel \ref{tab:data} tot het twintigste datapunt beschikbaar heeft (bekende data). De onderzoeker kent de datapunten vanaf het twintigste datapunt niet en moet deze gaan voorspellen. Een eerste model dat gebruikt zou kunnen worden is het constante model zoals in \ref{eq:constante}. 

Met dit model, zijn de waarden random waarden uit een populatie met gemiddelde $b$. De beste schatter voor $b$ is het gemiddelde van deze twintig data punten. 

\begin{lstlisting}
	data <- c(4 , 16 , 12 , 25 , 13 , 12 , 4 , 8  , 9 , 14, 
	+           3 , 14 , 14 , 20 , 7  , 9  , 6 , 11 , 3 , 11, 
	+           8 , 7  , 2  , 8  , 8  , 10 , 7 , 16 , 9 , 4 )
	mean(data[1:20])
\end{lstlisting}

\[ \widehat{b} = \frac{1}{20} \sum_{1}^{20} x_{t}= 10.75 \] 

Dit is de beste schatter vertrekkende van de 20 datapunten. We merken wel op dat $x_{1} =  4$ evenveel \textit{waarde} heeft als $x_{20} = 11$, of ander verwoord: de co\"effici\"ent van  $x_{1}$ is dezelfde als die van $x_{20}$, namelijk $\frac{1}{20}$.

Indien we dit als schatter zouden gebruiken dan zien we dat dit in figuur \ref{fig:tijdreeks21} geen goed idee is.

\begin{lstlisting}
AV20 <- matrix(10.75,30,1)
plot.ts(data, col="blue", type='b', xlab='tijd', ylab='Data')
lines(AV20,col='red', type='l')
legend(x= 'topright',legend = c("Data","Average 10.75"), lty = c(1,1), lwd = c(2.5,2.5), col=c('blue','red'))
\end{lstlisting}

\begin{figure}[htbp]
	\centering
		\includegraphics[width=1.00\textwidth]{images/tijdsreeksen/tijdsreeks20.png}
	\caption{Tijdreeks met constant gemiddelde $10.75$}
	\label{fig:tijdreeks21}
\end{figure}


Indien we veronderstellen dat de data verandert met de tijd is het beter om oude data minder waarde te geven en de recente data meer waarde. Een mogelijkheid is om enkel recente data te gebruiken, bijvoorbeeld de 10 en 5 laatste datapunten (zie figuur \ref{fig:tijdreeks31}).

\[ \widehat{b} = \frac{1}{10} \sum_{10}^{20} x_{t} = 10.18 \] en
\[ \widehat{b} = \frac{1}{5} \sum_{15}^{20} x_{t} = 7.83 \]

\begin{lstlisting}
sma10 <- SMA(x =data,n=10)
sma5 <- SMA(x=data,n=5)
plot.ts(x = data, col = 'blue',type = 'l')
lines(sma10, col='red', type = 'b')
lines(sma5, col='purple', type = 'b')
\end{lstlisting}

\begin{figure}
	\centering
		\includegraphics[width=1.00\textwidth]{images/tijdsreeksen/tijdsreekssma.png}
		\caption{Tijdreeks met moving average $m = 10$ en $m=5$}. 
	\label{fig:tijdreeks31}
\end{figure}




Dit worden \textit{moving averages} genoemd \index{moving average}. 

Welke schatter is nu de beste? We kunnen dit nu nog niet zeggen. 
\begin{itemize}
	\item De schatter die alle datapunten gebruikt is de beste indien de tijdreeks het model volledig volgt.
	\item De schatter met de recentere datapunten is de beste indien de tijdreeks verandert met de tijd.
\end{itemize}

\begin{definition}
	Algemeen is het moving average het gemiddelde van de $m$ laatste observaties.
	\begin{equation}
		\widehat{b} = \sum_{i=k}^{t} \frac{x_{i}}{m}
	\label{eq:movingAverage}
	\end{equation}
	met $k = t-m+1$. $m$ is de time range en is de parameter van de methode.
\end{definition}

\pagebreak
\subsection{Meten van de nauwkeurigheid van voorspellingen}

\begin{table}[h]
	\begin{tabular}{|lllllllllll|}
		\hline
		~         & 11   & 12   & 13   & 14   & 15   & 16   & 17   & 18   & 19   & 20   \\
		Data      & 3    & 14   & 14   & 20   & 7    & 9    & 6    & 11   & 3    & 11   \\
		Schatting & 11.7 & 11.6 & 11.4 & 11.6 & 11.1 & 10.5 & 10.2 & 10.4 & 10.7 & 10.1 \\
		Error     & -8.7 & 2.4  & 2.6  & 8.4  & -4.1 & -1.5 & -4.2 & 0.6  & -7.7 & 0.9  \\ \hline
	\end{tabular}
	\caption{Voorspellingsfout voor een moving average $m = 10$}
	\label{tab:error}
\end{table}


Een methode om de voorspelling te meten is het gemiddelde van de deviaties ($MAD$): gemiddelde absolute verschil tussen het voorspelde en de werkelijke waarden van de tijdsreeks.

\begin{definition}[$MAD$]
\begin{equation}
	MAD = \frac{1}{n} \sum_{1}^{n} \left| e_{i} \right|  
\label{eq:MAD}
\end{equation}
\end{definition}

Je kan dit ook percenteren om zo tot de gemiddelde absolute procentuele afwijking ($MAPE$) te komen.

\begin{definition}[$MAPE$]
\begin{equation}
	MAPE = \frac{1}{n} \sum_{1}^{n} \left| \frac{e_{i}}{X_i} \right|  
\label{eq:MAD2}
\end{equation}
\end{definition}


Je kan ook de variantie ervan bepalen:

\begin{definition}[$VAR$]
\begin{equation}
	s^{2}_{e} = \frac{1}{m} \sum_{1}^{n} (e_{i} - \overline{e})^{2}
\label{eq:varError}
\end{equation}
\end{definition}

Als laatste interessante parameter kan gekeken worden naar de wortel uit de gemiddelde kwadratische afwijking ($RMSE$), als de wortel uit het gemiddelde kwadratische verschil tussen de voorspelde en de werkelijke waarden van de tijdsreeks.

\begin{definition}[$RMSE$]
\begin{equation}
	RMSE_{e} = \sqrt{\frac{1}{m} \sum_{1}^{n} (e_{i})^{2}}
\label{eq:varError2}
\end{equation}
\end{definition}

\pagebreak
\section{Exponenti\"ele smoothing}
Bij een moving average krijgen alle voorgaande observaties een gelijk gewicht. Bij exponentieel smoothing worden kleinere gewichten toegekend aan oudere observaties. M.a.w.: recente observaties krijgen relatief meer gewicht dan oudere observaties.

In het geval van moving average zijn de gewichten hetzelfde, namelijk $\frac{1}{m}$.

\subsection{Enkelvoudige exponenti\"ele smoothing}
Exponenti\"ele effening (of smoothing) is een gewogen gemiddlede dat positieve gewichten toekent aan de huidige waarden en waarden uit het verleden van de tijdsreeks. Een enkel gewicht, $0\leq \alpha \leq1$ of de exponentie\"ele effeningsconstante wordt hiervoor gekozen. 
Voor een tijdseenheid $T$ wordt het enkelvoudige exponenti\"ele smoothing gevonden door vergelijking \ref{eq:singleExpSmooting}.

\begin{definition}[Exponenti\"ele smoothing]
\begin{equation}
	X_{T} = \alpha x_{t} + (1-\alpha)X_{t-1}, 0 \leq \alpha \leq 1, t \geq 3
\label{eq:singleExpSmooting}
\end{equation}
\end{definition}

$\alpha$ wordt de smoothing constante genoemd. Met andere woorden,$X_{T}$ is een gewogen gemiddelde van de huidige waarneming $x_t$ en de vorige exponenti\"ele smooting $X_{t-1}$.


\subsubsection{Inti\"ele setting}
Het bepalen van $X_{2}$ is een belangrijke parameter. Men kan kiezen om:
\begin{enumerate}
	\item $X_{2} = x_{1}$ te stellen
	\item $X_{2}$ gelijk te stellen aan een bepaald objectief
	\item Een gemiddelde te nemen van de eerste $x$ observaties
	\item \dots
\end{enumerate}

\begin{exercise}
	Waarom wordt dit een exponenti\"ele methode genoemd?
\end{exercise}
Antwoord: als we zouden substitueren vinden we bv. voor $X_{t-1}$:

\[ X_{t} = \alpha x_{t} + (1-\alpha)\left[\alpha x_{t-1} + (1-\alpha)X_{t-2}\right] \] 
\[ X_{t} = \alpha x_{t-1} + \alpha (1-\alpha)x_{t-1} + (1-\alpha)^{2} X_{t-2} \]
of dus algemeen gesteld :
\[ X_{t} = \alpha \sum_{i=0}^{t-2}(1-\alpha)^{i-1}x_{t-i} + (1-\alpha)^{t-2} X_{2}, t \geq 2 \]

Zo merk je dat oudere componenten een exponentieel kleiner gewicht verkrijgen. 

\subsubsection{Waarde voor $\alpha$}
De snelheid waarmee de oude observaties ''vergeten`` worden hang af van $\alpha$. Met een $\alpha$ dicht bij 1 vergeet je snel, terwijl een $\alpha$ dicht bij nul ervoor zorgt dat vergeten minder snel gaat (zoals aangetoond in tabel \ref{tab:alpha}). Vaak wordt een waarde gebruikt tussen $0.10$ en $0.30$.

\begin{table}
\centering
    \begin{tabular}{l|llll}
    $\alpha$ & $(1-\alpha)$ & $(1-\alpha)^{2}$ & $(1-\alpha)^{3}$ & $(1-\alpha)^{4}$ \\ \hline
    0.9   & 0.1       & 0.01             & 0.001                      & 0.0001           \\
    0.5   & 0.5       & 0.25             & 0.125                      & 0.062            \\
    0.1   & 0.9       & 0.81             & 0.729                      & 0.6561           \\
    \end{tabular}
		\caption{Waarden voor $\alpha$ en $(1-\alpha)^{n}$}
		\label{tab:alpha}
\end{table}


Bijvoorbeeld, het bestand \texttt{precip.data} bevat totale jaarlijkse neerslag in inches voor Londen, vanaf 1813-1912. Laten we dit eens analyseren met R.
\begin{lstlisting}
rain <- scan("Documents/education/onderzoekstechnieken-cursus/cursus/data/tijdsreeksen/precip.data",skip=1)
rainseries <- ts(rain,start=c(1813))
plot.ts(rainseries)
plot(rainseriesforecasts)
\end{lstlisting} 

%TODO figuur maken voor verschillende alpha's

\subsubsection{Voorspelling met exponenti\"ele effening}
Stel dat het doel is om de volgende waarde $X_{t+1}$ te voorspellen, dan wordt dit gelijk gesteld aan de smoothing waarde op tijdstip $t$.

\begin{equation}
	X_{t+1} = EMA_t = X_t
	\label{eq:EMA}
\end{equation}
Met $X_t$ de laatst voorspelde waarde. 

 We kunnen dit eenvoudig uitvoeren in R. Je krijgt hierbij een prediction interval: 
 Een prediction interval geeft een interval waarin we verwachten dat de voorspelde waarde met een bepaalde waarschijnlijkheid zal liggen. Standaard krijg je een 80\% en een 95\% interval. 
\begin{lstlisting} 
library('forecast')
rainseriesforecasts2 <- forecast.HoltWinters(rainseriesforecasts, h=8)
plot.forecast(rainseriesforecasts2)
\end{lstlisting} 

We zouden correlaties mogen zien tussen de voorspellingsfouten voor opeenvolgende voorspellingen. Met andere woorden, als er sprake is van een correlatie tussen prognosefouten voor opeenvolgende voorspellingen, is het eerder waarschijnlijk dat de simpele exponentiële effening kan worden verbeterd door een andere voorspellingstechniek te gebruiken.

Om te achterhalen of dit het geval is, kunnen we een correlogram verkrijgen van de in-sample voorspellingsfouten voor.

We weten nog uit hoofdstuk \ref{ch:analyse2var} dat de covariantie of correlate de lineaire relatie beschrijft tussen twee variabelen. . De autocovariantie en autocorrelatie meten de lineaire relatie tussen \textit{lagged} waarden voor een tijdsreeks. Met lagged bedoelen we een aantal stappen terug in de tijd.

\begin{definition}[Autocovariantie]
	We defini\"eren de autocovariantie bij lag $k$ door $c_k$.
	\[ c_k = \sum_{t=k+1}^{T} (y_t - \overline{y})(y_{t-k} - \overline{y}) \]
\end{definition}

\begin{definition}[Autocorrelatie]
	We defini\"eren de autocorrelatie bij lag $k$ door $r_k$.
	\[ r_k = \frac{c_k}{c_0} \]
\end{definition}

 We kunnen een correlogram die de autocorrelaties tekent, berekenen van de voorspellingsfouten met behulp van de functie 'acf ()' in R. Om de maximale lag te bepalen die we willen bekijken, gebruiken we de parameter 'lag.max' in acf ().

Bijvoorbeeld, om een ​​correlogram te berekenen van de in-steekprognosefouten voor de London-regenvalgegevens voor lags 1-20, typen we:

\begin{lstlisting}
	acf(rainseriesforecasts2$residuals, lag.max=20, na.action = na.pass)
\end{lstlisting}

Om te testen of er significant bewijs is voor significante correlaties bij lags 1-20, kunnen we een Ljung-Box test uitvoeren. Dit kan in R worden gedaan met de functie "Box.test ()". De maximale vertraging die we willen bekijken, wordt gespecificeerd met behulp van de parameter "Lag" in de Box.test () functie.

De test volledig uitleggen is buiten het bereik van deze cursus, maar de test gaat uit van volgende $H_0$ en $H_1$. De teststatistieken kunnen dan gewoon ge\"intepreteerd worden zoals alle andere hypothesetesten die beschreven geweest zijn in vorige hoofdstukken. 

\begin{itemize}
	\item $H_0$ De gegevens zijn onafhankelijk verdeeld (d.w.z. de correlaties in de populatie waaruit de sample wordt genomen, zijn 0, zodat elke waargenomen correlatie in de data voortvloeien uit willekeurigheid).
	\item $H_1$ De gegevens zijn niet onafhankelijk verdeeld: ze tonen een linaire correlatie.
\end{itemize}

 Bijvoorbeeld, om te testen of er geen nul autocorrelaties zijn op lags 1-20, voor de in-sample voorspellingen fouten voor Londen regenval data, typen we:
 \begin{lstlisting}
 Box.test(rainseriesforecasts2$residuals, lag=20, type="Ljung-Box")
 Box-Ljung test
 data:  rainseriesforecasts2$residuals
 X-squared = 17.4008, df = 20, p-value = 0.6268
 \end{lstlisting}
 
 Als laatste moeten we ook kijken naar de distributie van de errors van de voorspelling. Zoals boven vermeld gaan we ervan uit dat de errors normaal verdeeld zijn met een gemiddelde $\mu = 0$ en een standaardafwijking die constant is. Om te controleren of de voorspellingsfouten normaal verdeeld zijn met gemiddelde nul, kunnen we een histogram van de prognosefouten plotten, met een overlappende normale curve met gemiddelde nul en dezelfde standaardafwijking heeft als de verdeling van de voorspellingsfouten. Om dit te kunnen doen, kunnen we een R-functie "plotForecastErrors ()" definiëren. Het is ook aangewezen de methodes zoals beschreven in sectie \ref{sec:normtesting}.
 
 \begin{lstlisting}
 plotForecastErrors <- function(forecasterrors)
 {
 # make a histogram of the forecast errors:
 mybinsize <- IQR(forecasterrors)/4
 mysd   <- sd(forecasterrors)
 mymin  <- min(forecasterrors) - mysd*5
 mymax  <- max(forecasterrors) + mysd*3
 # generate normally distributed data with mean 0 and standard deviation mysd
 mynorm <- rnorm(10000, mean=0, sd=mysd)
 mymin2 <- min(mynorm)
 mymax2 <- max(mynorm)
 if (mymin2 < mymin) { mymin <- mymin2 }
 if (mymax2 > mymax) { mymax <- mymax2 }
 # make a red histogram of the forecast errors, with the normally distributed data overlaid:
 mybins <- seq(mymin, mymax, mybinsize)
 hist(forecasterrors, col="red", freq=FALSE, breaks=mybins)
 # freq=FALSE ensures the area under the histogram = 1
 # generate normally distributed data with mean 0 and standard deviation mysd
 myhist <- hist(mynorm, plot=FALSE, breaks=mybins)
 # plot the normal curve as a blue line on top of the histogram of forecast errors:
 points(myhist$mids, myhist$density, type="l", col="blue", lwd=2)
 }
 \end{lstlisting}

\subsection{Dubbele exponenti\"ele smoothing}
Enkelvoudige smoothing wordt gebruikt wanneer er geen trend zichtbaar is. Wanneer er een trend (stijgend of dalend) is dan kan er iets fout gaan. Zie bijvoorbeeld de data in tabel \ref{tab:trend} en figuur \ref{fig:tijdreeks61}.

\begin{table}[h]
\centering
    \begin{tabular}{|ll|}
    \hline
    Data & Enkelvoudige smoothing \\
    6.4  & ~                      \\
    5.6  & 6.4                    \\
    7.8  & 6.2                    \\
    8.8  & 6.7                    \\
    11.0 & 7.3                    \\
    11.6 & 8.4                    \\
    16.7 & 9.4                    \\
    15.3 & 11.6                   \\
    21.6 & 12.7                   \\
    22.4 & 15.4                   \\ \hline
    \end{tabular}
		\caption{Enkelvoudige smoothing met $\alpha = 0.3$}
		\label{tab:trend}
\end{table}

\begin{figure}[h]
	\centering
		\includegraphics[width=1.00\textwidth]{images/tijdsreeksen/tijdsreeks61.jpg}
	\caption{Exponenti\"ele smoothing bij een trend}
	\label{fig:tijdreeks61}
\end{figure}

Daarom voegen we een extra constante toe om deze trap te overbruggen:

\begin{definition}[Holt-voorspelling of dubbele exponenti\"ele voorspelling]
\begin{eqnarray}
	X_{t} = \alpha x_{t} + (1-\alpha)(X_{t-1} + b_{t-1}) & 0 \leq \alpha \leq 1 \\
	b_{t} = \gamma(X_{t}-X_{t-1}) + (1-\gamma)b_{t-1} & 0 \leq \gamma \leq 1 
\label{eq:doubleSmoothing}
\end{eqnarray}
\end{definition}

\subsubsection{Initi\"ele waarde}
Net zoals in enkelvoudige smoothing kan je verschillende methodes kiezen om initi\"ele waardes voor $X_{t}$ en $b_{t}$ te kiezen:
\begin{itemize}
	\item $X_{1} = x_{1}$
	\item $b_{1} = x_{2} - x_{1}$
	\item $b_{1} = \frac{1}{3}\left[ (x_{2} - x_{1}) + (x_{1} - x_{2}) + (x_{4} - x_{3}) \right]$
	\item $b_{1} = \frac{x_{n} - x_{1}}{n-1}$
\end{itemize}

\subsubsection{Voorspelling}
Een voorspelling maken met dubbele exponenti\"ele smoothing gebeurt dan iets anders (noem $F_{t+1}$ de voorspelling voor tijd $T+1$):

\[ F_{t+1} = X_{t} + b_{t} \]
of
\[ F_{t+m} = X_{t} + m b_{t} \]

Als we nu de tekening maken met enkelvoudige smoothing ($\alpha = 0.977$) en dubbele smoothing ($\alpha = 0.3623, \gamma = 1.0, X_{1} = x_{1} = 6.4$ en $b_{1} = \frac{1}{3}\left[ (x_{2} - x_{1}) + (x_{1} - x_{2}) + (x_{4} - x_{3}) \right] = 0.8$ vinden we volgende waarden in tabel \ref{tab:doubleSingle} en figuur \ref{fig:tijdreeks71}:

\begin{table}
\centering
    \begin{tabular}{|llll|}
    \hline
    Data & Enkelvoudige smoothing $X_{t}$ & Double smoothing $X_{t}$ & $F_{t}$ \\
    6.4  & ~                      & 6.4              & ~                             \\
    5.6  & 6.4                    & 6.6              & 7.2                           \\
    7.8  & 5.6                    & 7.2              & 6.8                           \\
    8.8  & 6.7                    & 8.1              & 7.8                           \\
    11.0 & 8.8                    & 9.8              & 9.1                           \\
    11.6 & 10.9                   & 11.5             & 11.4                          \\
    16.7 & 11.6                   & 14.5             & 13.2                          \\
    15.3 & 16.6                   & 16.7             & 17.4                          \\
    21.6 & 15.3                   & 19.9             & 18.9                          \\
    22.4 & 21.5                   & 22.8             & 23.1                          \\ \hline
    \end{tabular}
		\caption{Tabel met enkelvoudige en dubbele smoothing}
		\label{tab:doubleSingle}
\end{table}

\begin{figure}
	\centering
		\includegraphics[width=1.00\textwidth]{images/tijdsreeksen/tijdsreeks71.jpg}
	\caption{Enkelvoudige en dubbele smoothing}
	\label{fig:tijdreeks71}
\end{figure}

De manier om dit met R op te lossen is gelijkaardig als bij exponenti\"ele smoothing, alleen moet de parameter gamma $\gamma$ niet op NULL gezet worden. Het maken van de correlogram, de Ljung–Box test en het testen van de normaliteit van de errors gebeurt om dezelfde manier. 

\subsection{Driedubbele exponenti\"ele smoothing}
Wanneer dubbele smoothing niet werkt kan driedubbele smoothing gebruikt worden, ofwel Holt-Winters methode genoemd.

\begin{eqnarray}
	X_{t} = \alpha \frac{x_{t}}{c_{t-L}} + (1-\alpha) (X_{t-1} + b_{t-1}) & \textnormal{Smoothing}\\
	b_{t} = \gamma (X_{t} - X_{t-1}) + (1-\gamma)b_{t-1} & \textnormal{Trend smoothing} \\
	c_{t} = \beta \frac{x_{t}}{X_{t}} + (1-\beta)c_{t-L} & \textnormal{Seasonal smoothing} \\
	F_{t+m} = (X_{t} + mb_{t})c_{t-L+m} \mod L & \textnormal{Voorspelling}
\label{eq:HoltWinters}
\end{eqnarray}
 met 
\begin{itemize}
	\item $x_{t}$ de observatie op tijdstip $t$
	\item $X_{t}$ is de smoothed observatie op tijdstip $t$
	\item $b_{t}$ is de trendfactor op tijdstip $t$
	\item $c_{t}$ is de seizoensindex op tijdstip $t$
	\item $F_{t}$ is de voorspelling op tijdstip $t$
	\item $L$ is de periode (bv. van de seizoenen)
\end{itemize}

$\alpha, \beta, \gamma$ zijn constanten die geschat moeten worden. 


\section{Oefeningen}
\label{sec:tijdreeksen-oefeningen}

\begin{exercise}
In bijgevoegd bestand \emph{Budget.csv} vind je vanaf 1981 tot 2005 per kwartaal de omzet, het advertentiebudget en het BNP van een middelgroot bedrijf.  en voeg een kolom 'Kwartaalnummer' toe.
\begin{enumerate}
	\item Bereken het voortschrijdend gemiddelde \emph{(simple moving average)} over de periodes 4 en 12 voor deze data. Gebruik hiervoor de methode SMA. Maak een lijngrafiek van $X$, $MA(4)$ en $MA(12)$.
	\item Welke techniek die we eerder gezien hebben (in het deel over beschrijvende statistiek) is ook geschikt om voorspellingen te maken over de waarden van $X$? Werk dit uit aan de hand van de daarvoor bestemde functie en plot het resultaat in de grafiek.
	\item Gebruik de methode \emph{forecast} om voorspellingen voor de 10 volgende periodes met elk van voorgaande methoden (dus moving average 4 en 10 en regressie) te maken. Teken deze eveneens op de grafiek.
\item Is het gebruik van één van deze technieken interessant om voor deze data voorspellingen te maken? 
\item Maak van de data een tijdreeks via de methode \emph{ts}. Gebruik de methode \emph{decompose} om de tijdreeks op te delen en zo een idee te krijgen van de trend en de seizoenschommeling.
 \item Bereken het exponentieel voortschrijdend gemiddelde \emph{(exponential moving average, EMA)} door gebruik te maken van de methode \emph{HoltWinters}. Maak opnieuw via de methode \emph{forecast} een voorspelling voor 20 periodes. Gebruik als startwaarden $s_1 = x_1$ en $\alpha $ de door R gegenereerde waarde. Plot het resultaat op een nieuwe grafiek samen met $X$.
\item Doe nu hetzelfde met $\alpha=0.1$. 
\item Hoe zien de voorspellingen er nu uit?
\item Doe nu hetzelfde met \emph{dubbele} exponentiële afvlakking. Gebruik als startwaarden $s_1 = x_1$ en $b_1 = \frac{x_n - x_1}{n - 1}$, $\alpha =  0.05$ en $\beta = 0.2$. Plot het resultaat op de grafiek.
\item Gebruik dubbele exponentiële afvlakking om voorspellingen te berekenen voor 20 periodes. Plot de waarden op de grafiek. Is deze techniek beter of slechter dan de vorige voor deze dataset?
\item Speel met de waarden voor $\alpha$ en $\beta$ en bekijk het resultaat, zowel voor enkele als dubbele exponentiële afvlakking.
	\item Gebruik de \emph{HoltWinters}-methode zonder trend.  M.a.w. we stellen $\beta=0$. Gebruik als startwaarden $\alpha =  0.05$ en $\gamma = 0.9$. Plot het resultaat op de grafiek.
\item Bereken opnieuw voorspellingen voor 20 periodes. Plot de waarden op de grafiek. Is deze techniek beter of slechter dan de vorige voor deze dataset?
\item Speel met de waarden voor $\alpha$, $\beta$ en $\gamma$ en bekijk het resultaat.
	\item Gebruik de \emph{HoltWinters}-methode met de door R-gegeneerde waarden zonder trend.  M.a.w. we stellen $\beta=0$.  Plot het resultaat op de grafiek.
\item Bereken opnieuw voorspellingen voor 20 periodes maar gebruik nu de methode \emph{predict}. Plot de waarden op de grafiek. Is deze techniek beter of slechter dan de vorige voor deze dataset?
\end{enumerate}	
	
\end{exercise}

\begin{exercise}
	In bijgevoegd bestand \emph{Passagiers2.csv} vind je vanaf januari 1949 tot december 1960 het aantal passagiers van een luchtvaartmaatschappij. 
	\begin{enumerate}
		\item Bereken het voortschrijdend gemiddelde \emph{(simple moving average)} over de periodes 4 en 12 voor deze data. Gebruik hiervoor de methode \emph{ma}. Maak een lijngrafiek van $X$, $MA(4)$ en $MA(12)$.
		\item Welke techniek die we eerder gezien hebben (in het deel over beschrijvende statistiek) is ook geschikt om voorspellingen te maken over de waarden van $X$? Werk dit uit aan de hand van de daarvoor bestemde functie en plot het resultaat in de grafiek.
		\item Gebruik de methode \emph{forecast} om voorspellingen voor de 10 volgende periodes met elk van voorgaande methoden (dus moving average 4 en 10 en regressie) te maken. Teken deze eveneens op de grafiek. Conclusie?
			\item Is het gebruik van één van deze technieken interessant om voor deze data voorspellingen te maken? 
		\item Gebruik de methode \emph{decompose} om de tijdreeks op te delen en zo een idee te krijgen van de trend en de seizoenschommeling.
			\item Bereken het exponentieel voortschrijdend gemiddelde \emph{(exponential moving average, EMA)} door gebruik te maken van de methode \emph{ses} met $\alpha=0.2$. Maak opnieuw via de methode \emph{forecast} een voorspelling voor 20 periodes. Plot het resultaat op een nieuwe grafiek samen met $X$.
		\item Doe nu hetzelfde met $\alpha=0.6$ en $\alpha=0.89$. 
		\item Hoe zien de voorspellingen er nu uit?
			\item Doe nu hetzelfde met \emph{dubbele} exponentiële afvlakking. Gebruik hiervoor de methode \emph{holt}  $\alpha =  0.8$ en $\beta = 0.2$. Plot het resultaat op de grafiek.
		\item Gebruik dubbele exponentiële afvlakking om voorspellingen te berekenen voor 20 periodes. Plot de waarden op de grafiek. Is deze techniek beter of slechter dan de vorige voor deze dataset?
		\item Gebruik in de methode de optie $exponential=TRUE$. Teken het resultaat.  Wat is het verschil?
			\item Gebruik de \emph{hw}-methode met de door R gegeneerde waarden. Plot het resultaat op de grafiek.
		\item Bereken opnieuw een aantal voorspellingen via de methode \emph{predict}. Plot de waarden op de grafiek. Is deze techniek beter of slechter dan de vorige voor deze dataset?
		\item Speel met de waarden voor $\alpha$, $\beta$ en $\gamma$ en bekijk het resultaat.
	\end{enumerate}
	
\end{exercise}	

\begin{appendices}
\chapter{Logistisch regressie}

\section{Inleiding}

In dit onderzoek gaan we een andere vorm van verband zoeken tussen variabelen waarbij de afhankelijke variabele twee waarden kan aannemen. 

\begin{example}
	\label{ex:slagen}
	Stel dat je wil nagaan of het student al dan niet zal slagen voor het examen onderzoekstechnieken. We zijn dus ge\"interesseerd in de voorspelling (door
	onafhankelijke variabelen) van de kans dat een student in de categorie 'examen slagen' of in de categorie 'niet slagen' valt. 
\end{example}

In bovenstaand voorbeeld zal een 'gewone' lineaire regressie analyse 
algemeen wel de juiste richting van de $\beta$-co\"efficiënten opleveren. Maar de schatting is niet helemaal correct, omdat enkele belangrijke regressie assumpties geschonden worden, zoals de normaliteitsassumptie en de assumptie van homoscedasticiteit. Het grootste probleem is evenwel dat de door lineaire regressie voorspelde kansen groter kunnen zijn dan 1 en kleiner dan 0 en dat is niet te interpreteren.

Bij logistische regressie gaan we werken met kansverhoudingen. In voorbeeld \ref{ex:slagen} hebben we een kansverdeling dat een student wel slaagt $(y = 1)$ met kans $p$ gedeeld door de kans om niet te slagen $(y=0)$ met kans $q = 1-p$:
\[ 
	\textnormal{verhouding} = \frac{p}{1-p}
\]

We wensen dat de waarden van de verhouding gaan van $- \infty$ tot $\infty$. Daarom gaan we de natuurlijke logaritme nemen van de verhouding. Om de functie te tekenen van de logaritmische functie kan je onderstaande code gebruiken. 

\lstinputlisting{data/logcurve.R}

Als we de onafhankelijke variabelen $X_1$, $X_2$  \dots $X_n$ noemen,dan ziet het logistische model er in formulevorm als volgt uit:
\[ 
	log(\frac{p}{1-p}) = \beta_0 + \beta_1 X_1 + \cdots + \beta_n X_n 
\]

We kunnen het kansmodel ook herschrijven (afzonderen van de $p$):

\begin{eqnarray}
	p = \frac{e^{\beta_0 + \beta_1 X_1 + \cdots + \beta_n X_n }}{1+ e^{\beta_0 + \beta_1 X_1 + \cdots + \beta_n X_n }}
	\label{eq:prob}
\end{eqnarray} 



We kunnen het kansmodel dan ook herschrijven (afzonderen van de $(1-p)$):
\[ 
1-p = \frac{1}{1+ e^{\beta_0 + \beta_1 X_1 + \cdots + \beta_n X_n }}
\]

Aan deze formules is af te lezen dat de kansen $p$ en $1-p$ bij elkaar opgeteld gelijk zijn aan \'e\'en.
Verder is te zien dat de kansen $p$ en $1-p$ afhankelijk zijn van de variabelen $X_1, X_2 \cdots X_n$, maar dat deze afhankelijkheid niet lineair is. Een logistische regressielijn ziet er dus niet als een rechte lijn
uit, maar als een S-vormige curve. (TODO: hier zou een tekening moeten komen van de sigmo\"ide functie).

Bij logistische regressie gaan we dus op zoek naar goede waarden voor $\beta_0 \cdots \beta_n$ die het model zo goed mogelijk beschrijven zodat we ook voorspellingen kunnen doen. Dit kan in R makkelijk door de methode \texttt{glm}.

Om de logistische functie te tekenen kunnen we gebruik maken van onderstaande code (twee parameters).

\lstinputlisting{data/sigmoid.R}

\subsection{Intu\"itie rond de oplossingsmethode}
Om de waarden van $\beta_0, \beta_1 \cdots \beta_n$ te bepalen gaan we deze keer niet gebruik maken van de kleinste kwadratenmethode (zie sectie \ref{sec:regressie}), maar wel van een meer algemene methode : maximum likelihood methode \index{maximum likelihood}. Hierbij proberen we waarden voor de $\beta_i$ te vinden die ervoor zorgen dat in de trainingsdataset (de dataset die we gebruiken om de parameters $\beta_i$ te bepalen) de elementen die een label 1 krijgen zo goed mogelijk benaderd worden door 1 in vergelijking \ref{eq:prob} en de elementen die een label 0 krijgen zo goed mogelijk benaderd worden door 0. Dit doen we door volgende vergelijking te maximaliseren. 

\begin{equation}
	\Pi_{i: y_i=1} p(x_i) \Pi_{j= y_j = 0} (1 - p(x_j)) 
\end{equation}

De oplossingsmethode wordt ge\"implementeerd in R en is buiten de scope van deze cursus. We refereren de ge\"interesseerde lezer naar \cite{Hastie2009} voor meer informatie rond deze methode. 


\subsection{Performantie van het model}
Er zijn een aantal performantiematen die in rekening moeten gebracht worden wanneer aan logistische regressie gedaan wordt. 

\subsubsection{Akaike Information Criteria}
\index{Akaike Information Criteria}
Dit is een statistiek die wat overeenkomt met $R^2$ vanuit sectie \ref{sec:determinatiecoef}. Het geeft aan hoe goed de opgenomen variabelen in ons model het resultaat weergeven en we wensen die AIC zo laag mogelijk te houden. Het geeft ons dus een inkijk in het gebruik van de variabelen en zorgt ervoor dat we niet te veel variabelen in ons model opnemen. 

De waarde van de AIC is op zichzelf niet van belang, maar wordt vooral gebruikt wanneer de verschillende modellen wilt vergelijken: dan neem je best het model met de laagste AIC. 

\subsubsection{Null deviance}
\index{null deviance}
Dit is een indicatie hoe goed het model de data fit waarbij alleen gebruik gemaakt wordt van de intercept. Hoe lager deze waarde hoe beter.

\subsubsection{Residual deviance}
\index{residual deviance}
Dit is een indicatie hoe goed het model de data fit, waarbij de onafhankelijke variabelen toegevoegd zijn. Hier geldt ook, hoe lager deze waarde hoe beter.

Bij de output in R krijg je bovenstaande waarden. Waar je als onderzoeker vooral ge\"interesseerd in bent is een lage AIC en een een significante daling van the Null Deviance naar de Residual deviance. 


\section{Logistische regressie in R}

We gaan het voorbeeld nemen dan in Kaggle \footnote{\href{https://www.kaggle.com/c/titanic/data}{https://www.kaggle.com/c/titanic/data}} gegeven wordt. Het bevat de informatie rond de mensen die de reis van de titanic ondernomen hebben en het overleefd hebben of niet. De analyse komt uit het blog artikel \cite{michy}, maar is wat aangepast aangezien niet alle conclusies in dit artikel kloppen. 

\subsection{Data cleaning}

Importeer de data, en zorg ervoor dat de juiste types voor de juiste variabelen gekozen zijn (Sex is bijvoorbeeld een \texttt{factor} variabele)

We gaan de data opruimen en kijken welke parameters er in het model kunnen zitten. We gaan dit na door te kijken welke parameters in de dataset niet voldoende aanwezig zijn. 

\begin{lstlisting}
sapply(train,function(x) sum(is.na(x)))
sapply(train, function(x) length(unique(x)))
missmap(train, main = "Missing values vs observed")
\end{lstlisting}
Hierbij zien we dat de variabelen \texttt{cabin} te weinig waarden bevat. Ook \texttt{tickets} laten we vallen aangezien dit weinig invloed zal hebben. 
We nemen bijgevolg een subset van de data en gaan hiermee aan de slag. 
 
\begin{lstlisting}
data <- subset(train,select=c(2,3,5,6,7,8,10,12))
\end{lstlisting} 

We moeten ervoor zorgen dat de andere data elementen die er te kort zijn zinvol ingevuld worden. Je hebt hier verschillende methodieken voor. Je kan vervangen door:
\begin{itemize}
	\item het gemiddelde
	\item de mediaan
	\item de modus
	\item een elementen uit een bepaalde distributie
\end{itemize} 

We gaan voor de optie om de \texttt{NA} elementen te vervangen door hun gemiddelde. 

\begin{lstlisting}
data$Age[is.na(data$Age)] <- mean(data$Age,na.rm=T)
\end{lstlisting}

Voor de nominale en ordinale variabelen kunnen we kijken hoe ze gecodeerd worden door R. 
\begin{lstlisting}
contrasts(data$Sex)
\end{lstlisting}

\subsection{Fitten van de data in R}
We gaan de data opsplitsen in een trainingsset en een testset. We gaan hiervoor de library \texttt{caTools} gebruiken. 

\begin{lstlisting}
install.packages('caTools')
library(caTools)
\end{lstlisting}

Nu kunnen we het model laten opbouwen door R.

\begin{lstlisting}
model <- glm(Survived ~.,family=binomial(link='logit'),data=train)
summary(model)
\end{lstlisting}

Je krijgt volgende output na het uitvoeren van dit commando:
\begin{description}
	\item[Coefficient] De schatting voor de co\"effici\"ent in het model
	\item[Std. error] De standard errors op de co\"effci\"ent. 
	\item[z-statistic] Dit komt overeen met de $\frac{\beta_i}{SE(\beta_i)}$. 
	\item[P-value] De p-waarde geassocieerd met de null-hypothese van de co\"effci\"ent. 
\end{description}
Deze laatste twee getallen hebben wat verduidelijking nodig. Voor elke $\beta_i$ wordt een null-hypothese $H^i_0$ opgesteld. Deze stelt dat 
\[ 
	p(X_i) = \frac{e^{\beta_0 + \cdots \beta_{i-1} + \beta_{i+1} + \cdots \beta_n}}{1+e^{\beta_0 + \cdots \beta_{i-1} + \beta_{i+1} + \cdots \beta_n}}
\]
wat eigenlijk neerkomt dat het model niet afhangt van $X_i$. Wanneer de $|z|$ groot genoeg is en bijgevolg de $p$-waarde klein is mag de $H^i_0$ verworpen worden en kunnen we stellen dat $X_i$ wel degelijk van belang is in het model. 

Om een betrouwbaarheidsinterval te bouwen rond de geschatte parameter $\beta_i$ kan je gewoonweg volgende formule gebruiken:
\[
	\beta_i +z_i  \times SE(\beta_i)
\]

Als output krijgen we:
\begin{lstlisting}
Coefficients:
(Intercept)      Pclass2      Pclass3    Sexfemale          Age       SibSp0       Parch1         Fare    EmbarkedC    EmbarkedQ  
1.36178     -0.96344     -2.19975      2.67728     -0.04503     -0.49519      0.08984      0.00105      0.37631      0.68404  

Degrees of Freedom: 666 Total (i.e. Null);  657 Residual
Null Deviance:	    887.4 
Residual Deviance: 582.5 	AIC: 602.5
\end{lstlisting}

En met \texttt{summary} van het model bekomen we volgende output:

\begin{lstlisting}
Call:
glm(formula = Survived ~ ., family = binomial(link = "logit"), 
data = dresstrain)

Deviance Residuals: 
Min       1Q   Median       3Q      Max  
-2.4971  -0.6377  -0.3730   0.6240   2.5854  

Coefficients:
Estimate Std. Error z value Pr(>|z|)    
(Intercept)  1.361775   0.495174   2.750  0.00596 ** 
Pclass2     -0.963440   0.339019  -2.842  0.00449 ** 
Pclass3     -2.199753   0.335770  -6.551  5.7e-11 ***
Sexfemale    2.677281   0.224722  11.914  < 2e-16 ***
Age         -0.045028   0.009096  -4.951  7.4e-07 ***
SibSp0      -0.495189   0.252907  -1.958  0.05023 .  
Parch1       0.089835   0.304233   0.295  0.76778    
Fare         0.001050   0.002259   0.465  0.64190    
EmbarkedC    0.376309   0.277878   1.354  0.17566    
EmbarkedQ    0.684037   0.364547   1.876  0.06060 .  
---
Signif. codes:  0 '***' 0.001 '**' 0.01 '*' 0.05 '.' 0.1 ' ' 1

(Dispersion parameter for binomial family taken to be 1)

Null deviance: 887.35  on 666  degrees of freedom
Residual deviance: 582.46  on 657  degrees of freedom
AIC: 602.46

Number of Fisher Scoring iterations: 5
\end{lstlisting}

Hieruit kunnen we volgende dingen zeggen:
\begin{itemize}
	\item SibSp, Parch1, Fare, EmbarkedC en EmbarkedQ zijn niet statisch significant. 
	\item We zien dat Sexfemale erg significant. De positieve co\"effici\"ent voor sexFemale toont aan dat vrouw zijn ervoor zorgt dat je meer kans hebt op overleven.  
\end{itemize}

Bij de \texttt{anova} wordt getoond wat het effect is van een variabele een per een toe te voegen aan het model. \textbf{TODO: dit nog eens deftig interpreteren.}


De volledige code kan je hier nog eens bekijken:
\lstinputlisting{data/titanicregression.R}

\section{Oefeningen}

\begin{exercise}
	
	\begin{itemize}
		\item Beschouw de dataset \texttt{Smarker} van de package \texttt{ISLR}. 
		Deze dataset bestaat uit
		het rendement voor de S \& P 500 aandelenindex over 1250 dagen, van 
		begin 2001 tot eind 2005. Voor elke datum hebben we het  retourneer percentage opgenomen voor elk van de vijf vorige handelsdagen (Lag1 t.e.m. Lag5). We hebben ook het Volume opgenomen (het aantal verhandelde aandelen) en het percentage van vandaag. Daarnaast hebben we ook opgenomen of de markt daalde of steeg.
		\item Schrijf de algemene statistieken uit van de verschillende variabelen. 
		\item Probeer eens een plot te maken die aanduid of het volume stijgt of daalt met de jaren. 
		\item We gaan proberen een logistisch model op te stellen dat het stijgen of dalen in functie van lag1 t.e.m. lag5 en volume uitzet. Gebruik hiervoor het commando glm.
		\item Analyseer de co\"effici\"enten. Wat kan je erover zeggen?
		\item Kijk nu eens hoe goed het model de dataset zelf voorspelt. Dit kan je doen door aan het predict commando geen dataset mee te geven. 
		\item Zet de voorspelde probabiliteit om in juiste labels ($\geq 0.5$ up)
		\item  Cree\"er een matrix die de vals positieven en ware positieven e.a. uitzet t.o.v. elkaar. Gebruik hiervoor de methode table. 
		\item Wat kom je hier nu voor uit?
	\end{itemize}
\end{exercise}

\begin{exercise}
	
	\begin{itemize}
		\item Beschouw dezelfde dataset als hierboven, maar train nu de dataset met de elementen van voor 2005 en gebruik als testset de elementen boven 2005. Wat kom je nu uit?
		\item Probeer nu het model aan te passen door de juiste variabelen te kiezen om mee te nemen in het model. 
		\item Wanneer je tevreden bent met het model, probeer dan een voorspelling te doen van een willekeurige dataset.
	\end{itemize}
\end{exercise}


\include{multivariate}




\chapter{Notatie}
\label{app:notatie}

\begin{table}
  \centering
  \begin{tabular}{p{.25\textwidth}p{.75\textwidth}}
  	\toprule
  	\textbf{Notatie}                                        & \textbf{Betekenis}                                                                                                                     \\
  	\midrule
  	$\widehat{a}, \widehat{b}, \ldots$                      & Het ``hoedje'' geeft aan dat het gaat om een schatter.                                                                                 \\
    $d$                                                     & Effectgrootte, of Cohen's $d$ \\
  	$M \sim Nor(\mu_{\overline{x}}, \sigma_{\overline{x}})$ & De kansverdeling van het steekproefgemiddelde (cfr.~de centrale limietstelling, Sectie~\ref{sec:centrale-limietstelling})              \\
  	$N$                                                     & De populatieomvang                                                                                                                     \\
  	$n$                                                     & De steekproefgrootte                                                                                                                   \\
  	$R$                                                     & Pearson's product-momentcorrelatiecoëfficiënt (kort: correlatiecoëfficiënt). In de literatuur soms ook $\rho$ (rho).                         \\
  	$R^2$                                                   & Determinatiecoëfficiënt. In de literatuur soms ook $\rho^2$.                                                                           \\
  	$s$                                                     & De standaardafwijking van een steekproef                                                                                                \\
  	$s^2$                                                   & De variantie van een steekproef                                                                                                         \\
  	$t \in \mathbb{N}$                                      & Een tijdstip                                                                                                                           \\
  	$X = \left\{x_1, x_2, \ldots, x_n \right\}$             & Een stochastische variabele $X$ met $n$ waarnemingen $x_i$ (voor $i: 1 \ldots n$)                                                      \\
  	$X \sim Nor(\mu, \sigma)$                               & De variabele $X$ is \emph{normaal verdeeld} met gemiddelde $\mu$ en standaardafwijking $\sigma$                                        \\
  	$\overline{x}$                                          & Het gemiddelde over de \emph{steekproef}                                                                                               \\
  	$Z \sim Nor(0, 1)$                                      & $Z$ is een variabele met een kansverdeling die de \emph{standaardnormaalverdeling} volgt, dus met gemiddelde 0 en standaardafwijking 1 \\
  	$\alpha$ (alfa)                                         & Een significantieniveau (voor een statistische toets)                                                                                  \\
  	$1 - \alpha$                                            & Een betrouwbaarheidsniveau (voor een betrouwbaarheidsinterval)                                                                         \\
  	$\epsilon$                                              & Storing in een tijdreeks (typisch een klein getal)                                                                                     \\
  	$\mu$ (mu)                                              & Het gemiddelde (ook: verwachtingswaarde) over heel de \emph{populatie}.                                                                \\
  	$\mu_{\overline{x}}$                                    & De verwachtingswaarde bij de kansverdeling van het steekproefgemiddelde                                                                \\
  	$\sigma$ (sigma)                                        & De standaardafwijking over heel de populatie                                                                                           \\
  	$\sigma^2$                                              & De variantie over heel de populatie                                                                                                    \\
  	$\sigma_{\overline{x}}$                                 & De standaardafwijking bij de kansverdeling van het steekproefgemiddelde                                                                \\
  	\bottomrule
  \end{tabular}
  \caption[Overzicht gebruikte symbolen.]{\textbf{Overzicht gebruikte symbolen.} De symbolen zijn alfabetisch gesorteerd, met eerst het latijnse en daarna het Griekse alfabet (zie Tabel~\ref{tab:griekse-alfabet}).}
  \label{tab:notatie}
\end{table}

\begin{table}
  \centering
  \begin{tabular}{lll}
  	\toprule
  	\textbf{Grieks}              & \textbf{Naam} & \textbf{Klank, uitspraak}   \\
  	\midrule
  	$A, \alpha$                  & alfa          & a                           \\
  	$B, \beta$                   & bèta          & b                           \\
  	$\Gamma, \gamma$             & gamma         & g                           \\
  	$\Delta, \delta$             & delta         & d                           \\
  	$E, \epsilon$                & epsilon       & e                           \\
  	$Z, \zeta$                   & zèta          & dz                          \\
  	$H, \eta$                    & èta           & ei                          \\
  	$\Theta, \theta$             & thèta         & th                          \\
  	$I, \iota$                   & iota          & i                           \\
  	$K, \kappa$                  & kappa         & k                           \\
  	$\Lambda, \lambda$           & lambda        & l                           \\
  	$M, \mu$                     & mu            & m                           \\
  	$N, \nu$                     & nu            & n                           \\
  	$\Xi, \xi$                   & xi            & ks                          \\
  	$O, o$                       & omikron       & o (kort)                    \\
  	$\Pi, \pi$                   & pi            & p                           \\
  	$P, \rho$                    & rho           & r                           \\
  	$\Sigma, \sigma (\varsigma)$ & sigma         & s                           \\
  	$T, \tau$                    & tau           & t                           \\
  	$\Upsilon, \upsilon$         & upsilon       & u                           \\
  	$\Phi, \phi (\varphi)$       & phi           & f                           \\
  	$X, \chi$                    & chi           & ch (zoals in \emph{chemie}) \\
  	$\Psi, \psi$                 & psi           & ps                          \\
  	$\Omega, \omega$             & omega         & o (lang)                    \\
  	\bottomrule
  \end{tabular}
  \caption[Het Griekse alfabet]{\textbf{Het Griekse alfabet.} In wiskundige en statistische teksten worden vaak Griekse letters gebruikt. Ter info vind je hier een overzicht van het Griekse alfabet. Telkens is de hoofd- en kleine letter gegeven. Soms staat tussen haakjes een variant van de letter die soms ook voorkomt.}
  \label{tab:griekse-alfabet}
\end{table}

\clearpage
\addcontentsline{toc}{chapter}{\textcolor{maincolor}{\IfLanguageName{dutch}{Bibliografie}{Bibliography}}}
\printbibliography

\clearpage
\addcontentsline{toc}{chapter}{\textcolor{maincolor}{Index}}
\printindex
\end{appendices}
\end{document}
