% !TeX root = ./cursus-onderzoekstechnieken.tex
%=============================================================================
% Cursus Onderzoekstechnieken HOGENT toegepaste informatica
%=============================================================================

\documentclass{hogent-report}

%-----------------------------------------------------------------------------
% Packages en instellingen specifiek voor dit document
%-----------------------------------------------------------------------------

%% Titelpagina
\usepackage{hogent-titlepage-image}

%% Figuren
\usepackage{pgfplotstable}
\usepackage{pgfplots}
\pgfplotsset{compat=1.13}
\usetikzlibrary{arrows,shapes,backgrounds,positioning,shadows}
\usetikzlibrary{pgfplots.statistics}

\pgfmathdeclarefunction{gauss}{2}{%
  \pgfmathparse{1/(#2*sqrt(2*pi))*exp(-((x-#1)^2)/(2*#2^2))}%
}

%% Titel
\title{Cursus~\\Onderzoekstechnieken}
\author{Dr. Jens Buysse, Wim {De Bruyn}, Pieter-Jan Maenhaut , Bert {Van Vreckem}}
\date{Academiejaar 2019-2020}

\hypersetup{
  pdftitle={\thetitle},
  pdfauthor={\theauthor}
}

\begin{document}

\inserttitlepage{achtergrond-ozt.png}

%-----------------------------------------------------------------------------
% Copyright
%-----------------------------------------------------------------------------

\newpage

\thispagestyle{empty}

\vspace*{20cm}

\noindent Copyright \copyright\ 2015-{\the\year} Jens Buysse % Copyright notice

\noindent \textsc{www.hogent.be} % URL

\noindent \textit{Gegenereerd op \today} % Printing/edition date

%-----------------------------------------------------------------------------
% Inhoudstafel
%-----------------------------------------------------------------------------

\usechapterimagefalse
\tableofcontents % Print the table of contents itself

\cleardoublepage % Forces the first chapter to start on an odd page so it's on the right

\def\R{\mathbb{R}}

%-----------------------------------------------------------------------------
% Corpus
%-----------------------------------------------------------------------------

\include{0_dankwoord}

\chapter{Aan de slag}
\label{ch:aan-de-slag}

\section{Studiewijzer}

De studiewijzer geeft een overzicht van de belangrijkste informatie over deze cursus, o.a.~leerdoelen, lesmateriaal, weekplanning en leeraanwijzingen. Lees alles aandachtig door!

\subsection{Doel en plaats van de cursus in het curriculum}

Deze cursus is een inleiding op wat tegenwoordig vaak \emph{data science} genoemd wordt. Het doel is om je wegwijs te maken in het correct verzamelen, verwerken en analyseren van numerieke data en daar een onderbouwd onderzoeksverslag over te schrijven.

In de eerste plaats is dit een voorbereiding op de bachelorproef, waar je deze technieken in de praktijk zal moeten omzetten. Maar ook na je afstuderen blijft de kennis die je in deze cursus opdoet waardevol. Succesvolle bedrijven nemen beslissingen, niet op basis van buikgevoel of intuïtie, maar door het verzamelen en analyseren van data. Aan de hand van de technieken die hier toegelicht worden, heb je voldoende achtergrond om vragen te beantwoorden als:

\begin{itemize}
  \item Is een (web)applicatie snel genoeg voor de gebruikers? Is de gebruikerservaring consistent, of zit er grote variatie op responstijden?
  \item Als je twee systemen moet vergelijken, zij het software of hardware, welke van de twee is het meest performant? Is het verschil tussen beide significant, of kunnen verschillen in de metingen te wijten zijn aan het toeval of andere factoren?
  \item Wanneer moeten aankopen van nieuwe apparatuur (bv.~harde schijven, servers, geheugen, enz.) ingepland worden, op basis van historische gebruiksgegevens?
\end{itemize}

De competenties die je in deze cursus verwerft hebben ook buiten de informatica hun nut. Je leert immers kritisch omgaan met data en informatie, en hoe die correct te analyseren en interpreteren. In het politieke en maatschappelijke debat worden moedwillig beweringen gedaan die aantoonbaar fout zijn of die de waarheid proberen ``om te buigen.'' De term die in dat verband vaak de kop opsteekt is `Fake News'. Een manier om je hiertegen te wapenen is kritisch omgaan met de informatie die verspreid wordt. Daardoor kan vaak de achterliggende reden van die desinformatie duidelijk gemaakt worden.

Statistiek  en data science zijn dan ook onontbeerlijk om (i) te data correct te analyseren en tot onderbouwde conclusies te komen en (ii) zelf onderzoeken uit te voeren waarbij je onderbouwde conclusies de wereld kan in sturen. 

\subsection{Leerdoelen en competenties}

\begin{itemize}
  \item Kan begrippen, formules, stellingen en de uitwerking ervan uit de beschrijvende en inductieve statistiek benoemen en verklaren
  \item Kan formules, stellingen uit de beschrijvende en inductieve statistiek in onderzoeksvraagstukken correct toepassen
  \item Kan data analyseren met statistische software
  \item Kan een gestructureerd wetenschappelijk document schrijven en voorzien van referenties in \LaTeX{}
  \item Kan de wetenschappelijke methode vergelijken met niet-wetenschappelijke onderzoeksmethodes en daarbij voor- en nadelen opsommen 
\end{itemize}

Deze vind je ook terug in de studiefiche.

\subsection{Leerinhoud}

Verder in dit hoofdstuk vind je instructies voor het installeren van de nodige software, en een korte inleiding op het werken met R, een programmeertaal voor data-analyse.

Hoofdstuk~\ref{ch:onderzoeksproces} geeft een inleiding op het verloop van een typisch onderzoeksproces en introduceert enkele basisconcepten van data-analyse, waaronder het nemen van een steekproef uit een populatie, variabelen en meetniveaus.

Hoofdstuk~\ref{ch:analyse1var} behandelt de analyse van een enkele variabele, meer bepaald centrum- en spreidingsmaten, samen met geschikte visualisatietechnieken voor elke soort variabele.

Hoofdstuk ~\ref{ch:centrale-limietstelling} herhaalt kort de stof rond kansverdelingen, waarna de centrale limietstelling aan bod komt met een onmiddelijke toepassing via betrouwbaarheidsintervallen. 

Hoofdstuk~\ref{ch:toetsingsprocedures} gaat hierop verder met de algemene werkwijze voor het voeren van statistische toetsen, en specifiek met toetsen voor uitspraken over het gemiddelde van een populatie: de $z$-toets en de $t$-toets.

Waar de vorige hoofdstukken telkens één variabele apart beschouwden, bekijkt Hoofdstuk~\ref{ch:analyse2var} verschillende technieken om verbanden tussen twee variabelen te leggen, afhankelijk van het variabeletype.

Hoofdstuk~\ref{ch:tijdreeksen} geeft een inleiding op het analyseren van hoe de waarde van een variabele evolueert in de tijd aan de hand van wiskundige modellen die onder bepaalde voorwaarden ook toelaten om voorspellingen te doen.

\subsection{Leermateriaal}

Het belangrijkste leermateriaal voor dit opleidingsonderdeel is deze cursus, die ook de oefeningenopgaven bevat. Die wordt ter beschikking gesteld via Chamilo als PDF. Op Chamilo vind je ook de PDF's met de slides gebruikt tijdens de lessen.

Daarnaast krijgen studenten toegang tot een GitHub-repository met de broncode voor:

\begin{itemize}
  \item Deze cursus
  \item De slides gebruikt tijdens de lessen
  \item Broncodevoorbeelden in R voor alle technieken die in de cursus aan bod komen.
\end{itemize}

\textbf{Errata en wijzigingen} aan de cursus worden in GitHub aangebracht. De PDF's op Chamilo zullen niet noodzakelijk bijgewerkt worden. Studenten kunnen zelf de laatste versies van alle documenten met \LaTeX{} genereren.

De software die nodig is voor dit opleidingsonderdeel is gratis/open source. Instructies voor de installatie kan je vinden in Sectie~\ref{sec:installatie-software}.

\subsection{Werkvormen}

\textbf{Studenten afstandsleren} kunnen vragen stellen tijdens de contactmomenten. Dit zijn echter geen lesmomenten! Het rooster vind je in de Chamilo-cursus ``Informatie voor studenten TILE.''

\textbf{Studenten dagonderwijs} krijgen drie uur per week les, waarvan één uur klassikale instructie en hoorcollege, en twee uur oefeningen en begeleiding bij het groepswerk.

\subsection{Werk- en leeraanwijzingen}

Het opleidingsonderdeel \emph{Onderzoekstechnieken} wordt door veel studenten als moeilijk ervaren. Dat is begrijpelijk, want het onderwerp ligt dan ook buiten de comfortzone van de doorsnee informaticastudent en we weten allemaal dat wiskundige vakken niet de populairste van onze opleiding zijn.

Er zijn twee manieren om hier mee om te gaan. Je kan de weg van de minste weerstand nemen: je concentreren op de vakken die je graag doet en een dag voor het examen de cursus doornemen in de hoop dat je voldoende punten bij elkaar sprokkelt om een tien te halen. De ervaring leert dat deze strategie niet succesvol is, wat blijkt uit het lage slagingspercentage in de eerste zittijd (in academiejaar 2016-2017 was dat ca. 35\% voor het dagonderwijs en 10\% voor afstandsleren). In de tweede zittijd zien we vaak een veel hoger slagingspercentage, wat naar onze mening suggereert dat wanneer je voldoende inspanning levert voor dit vak, het zeker haalbaar is.

Enkele tips om w\'el meteen te slagen voor dit vak:

\begin{itemize}
  \item Kom naar de les en \emph{neem actief nota's}~\parencite{Lundin2020};
  \item Werk ook voor dit vak \emph{buiten de contactmomenten}. Herhaal de geziene theorie en werk oefeningen af waarmee je nog niet klaar was. Noteer zaken die je niet begrijpt of waar je vast zit, en stel je vraag tijdens het eerstvolgende college.
  \item Gebruik goede \emph{leertechnieken}. Je vindt een goed overzicht van leertechnieken waarvan het effect wetenschappelijk aangetoond is via de website van \emph{The Learning Scientists}\footnote{\url{http://www.learningscientists.org/}}.
  \begin{itemize}
    \item \emph{Spaced practice:} Studeer in meerdere kleine sessies (minstens één keer per week) en niet in grote blokken. Blokkeer een vast moment in je weekagenda/lesplanning.
    \item \emph{Retrieval practice:} Neem een leeg blad papier en probeer zoveel mogelijk zaken over een bepaald onderwerp op te schrijven vanuit je herinnering (dus zonder in de cursus te kijken). Controleer dit daarna aan de hand van je lesnota's en in de cursus.
    \item \emph{Elaboration:} Stel jezelf vragen over hoe dingen (bv. formules, toetsingsprocedures \ldots) in elkaar zitten en waarom dat zo is. Overleg met medestudenten. Vraag je lector om meer uitleg indien nodig. Leg verbanden tussen verschillende onderwerpen in de cursus (bv. vergelijk toetsingsprocedures).
    \item \emph{Interleaving:} Wissel onderwerpen af tijdens het studeren.
    \item Gebruik \emph{concrete voorbeelden} om abstracte idee\"en te begrijpen. In de cursus worden reeds enkele voorbeelden gegeven, probeer er zelf andere te bedenken. Overleg met medestudenten en vraag eventueel feedback aan je lector.
    \item \emph{Dual coding:} Combineer woord en beeld, probeer de leerstof die je instudeert visueel voor te stellen.
  \end{itemize}
\end{itemize}

Uiteindelijk komt het er op neer dat je voldoende tijd en inspanning investeert om te studeren voor dit vak.

\subsection{Studiebegeleiding en planning}

Studenten \textbf{afstandsleren} die vragen hebben over de leerstof kunnen in de eerste plaats terecht op het forum in Chamilo. Op de contactmomenten voor afstandsleren is er ook gelegenheid voor het stellen van vragen.

Studenten \textbf{dagonderwijs} kunnen vragen stellen tijdens de werkcolleges of  op het forum.

%In Tabel~\ref{tab:weekplanning} vind je een overzicht van de lesplanning voor het dagonderwijs die ook als leidraad kan dienen voor de studieplanning van studenten afstandsleren.

%\begin{table}
%  \begin{center}
%    \begin{tabular}{cll}
%       \hline
%       \textbf{Week} & \textbf{Theorie}     & \textbf{Oefeningen}            \\
%       \hline
%       1  & Intro, Onderzoeksproces         & Software installeren, \LaTeX{} \\
%       2  & Analyse van 1 variabele         & Wetenschappelijk schrijven     \\
%       3  & Steekproefonderzoek             & Analyse van 1 variabele        \\
%       4  & Steekproefonderzoek             & Steekproefonderzoek            \\
%       5  & Toetsingsprocedures ($z$-toets) & Steekproefonderzoek            \\
%       6  & Toetsingsprocedures ($t$-toets) & Toetsingsprocedures            \\
%       7  & Analyse van 2 variabelen        & Toetsingsprocedures            \\
%      --- & \textbf{Paasvakantie}           & ---                            \\
%       8  & Analyse van 2 variabelen        & Analyse van 2 variabelen       \\
%       9  & $\chi^2$-toets                  & Analyse van 2 variabelen       \\
%      10  & Tijdreeksen                     & $\chi^2$-toets                 \\
%      11  & Toelichting bachelorproef       & Tijdreeksen                    \\
%      12  & Herhaling                       & Herhaling                      \\
%      \hline
%    \end{tabular}
%    \caption[Weekplanning]{Weekplanning van de cursus.}
%    \label{tab:weekplanning}
%  \end{center}
%\end{table}

\subsection{Evaluatie}



\begin{itemize}
  \item Eerste examenperiode:
  \begin{itemize}
    \item 70\% periodegebonden evaluatie: schriftelijk examen, bestaande uit een deel gesloten boek (theorie) en een deel met voorbereiding op pc (oefeningen)
    \item 30\% niet-periodegebonden evaluatie, groepsopdracht: het voeren van een mini-onderzoek in groep, bestaande uit een literatuurstudie, opzetten van een reproduceerbaar experiment, verzamelen van meetgegevens en die statistisch analyseren, en er een verslag over schrijven
  \end{itemize}
  \item Tweede examenperiode:
  \begin{itemize}
    \item 70\% periodegebonden evaluatie: schriftelijk examen, bestaande uit een deel gesloten boek (theorie) en een deel met voorbereiding op pc (oefeningen)
    \item 30\% niet-periodegebonden evaluatie: er wordt geen tweede examenkans georganiseerd voor dit onderdeel. Wanneer een student in de eerste examenkans niet geslaagd was voor het opleidingsonderdeel wordt het resultaat voor de groepsopdracht ongewijzigd overgenomen.
  \end{itemize}
\end{itemize}

\section{Installatie software}
\label{sec:installatie-software}

Voor de cursus onderzoekstechnieken maak je gebruik van verschillende softwarepakketten. Hier vind je wat uitleg over de installatie en hoe je er mee aan de slag kan.

\begin{itemize}
  \item Git client (versiebeheersysteem);
  \item \LaTeX{} compiler;
  \item \LaTeX{} editor;
  \item Jabref (bibliografische databank);
  \item R (statistische analysesoftware);
  \item Rstudio (IDE voor R);
  \item Lettertypes voor de HOGENT-huisstijl.
\end{itemize}

Sommige van deze applicaties nemen veel schijfruimte in, dus zorg dat je voldoende ruimte vrij hebt.

In vele andere cursussen rond statistiek of onderzoekstechnieken wordt gebruik gemaakt van commerci\"ele software: SPSS of SAS voor data-analyse, MS Office voor de opmaak van documenten. In deze cursus wordt er expliciet voor gekozen om open source of gratis software te gebruiken. Het grootste voordeel is dat je die ook na je afstuderen nog kan gebruiken zonder dat jij of je bedrijf/organisatie softwarelicenties moet aankopen.

Bovendien zijn de tools die we zullen gebruiken kwalitatief minstens even goed dan hun commerci\"ele tegenhangers. R, een programmeertaal voor statistische analyse, wordt wereldwijd gebruikt in academische én professionele context. De kans is dus niet onbestaande dat je het in je professionele loopbaan nog zal tegenkomen, of het zal kunnen toepassen voor het oplossen van datagerelateerde problemen. Feedback die we kregen van oud-studenten bevestigt dit.

\LaTeX{} is een markuptaal en tekstzetsysteem voor de professionele vormgeving van documenten. De bedoeling is dat de auteur zich vooral moet bezig houden met het logisch structureren van een tekst, en dat het vormgeven op papier wordt overgenomen door de software. Het aanleren van de markuptaal vraagt wat inspanning, maar het is een investering die rendeert wanneer je een lang document (zoals een scriptie) op een professionele, strakke manier wil opmaken. Er zijn in het verleden nog zelden of nooit bachelorproeven ingediend die in MS Word geschreven waren en die een voldoende goede opmaak hadden. Het lijkt veel eenvoudiger om een tekst op te stellen in Word, maar het is zo goed als onmogelijk om in een lang document een consistente en professioneel ogende opmaak te realiseren.

\subsection{Windows}

Omdat het hier toch gaat om een vrij groot aantal applicaties, kunnen Windows-gebruikers beter gebruik maken van de Chocolatey package manager\footnote{\url{https://chocolatey.org/}} in plaats van alles manueel te downloaden en installeren.

Na installatie van Chocolatey\footnote{\url{https://chocolatey.org/install}}, voer je volgende commando's uit als Administrator in een CMD of PowerShell terminal:

\begin{verbatim}
choco install -y git
choco install -y miktex
choco install -y texstudio
choco install -y JabRef
choco install -y r.project
choco install -y r.studio
\end{verbatim}

Wie toch de ``klassieke'' werkwijze wil hanteren, vindt hier de verschillende softwarepakketten:

\begin{itemize}
  \item Git client: \url{https://git-scm.com/download/win}
  \item \LaTeX{} compiler: \url{https://miktex.org/download}
  \item TeXStudio: \url{http://www.texstudio.org/}
  \item Jabref: \url{https://www.fosshub.com/JabRef.html}
  \item R: \url{https://lib.ugent.be/CRAN/}
  \item Rstudio: \url{https://www.rstudio.com/products/rstudio/download/#download}
\end{itemize}

\subsection{macOS}

macOS gebruikers installeren de nodige software best via de Homebrew\footnote{\url{https://brew.sh/}} package manager\footnote{\textbf{Let op!} Deze werkwijze is nog niet getest. Feedback van Mac-gebruikers is welkom!}:

\begin{verbatim}
brew install git
brew cask install mactex
brew cask install texstudio
brew cask install jabref
brew install Caskroom/cask/xquartz
brew install --with-x11 r
brew cask install --appdir=/Applications rstudio
\end{verbatim}

Wie toch alles manueel wil installeren kan de applicaties hier downloaden:

\begin{itemize}
  \item Git client: \url{https://git-scm.com/download/mac}
  \item \LaTeX{} compiler: \url{https://www.tug.org/mactex/mactex-download.html}
  \item TeXStudio: \url{http://www.texstudio.org/}
  \item Jabref: \url{https://www.fosshub.com/JabRef.html}
  \item R: \url{https://lib.ugent.be/CRAN/}
  \item Rstudio: \url{https://www.rstudio.com/products/rstudio/download/#download}
\end{itemize}

\subsection{Linux}
\label{ssec:installatie-linux}

Op RStudio na zijn alle nodige softwarepakketten beschikbaar in de repositories van de meest gebruikte Linux-distributies. We geven hier command-line instructies voor enerzijds Ubuntu (Xenial/16.04) en Debian 9 en anderzijds Fedora.

\paragraph{Ubuntu/Debian} 

Controleer eerst de URL naar de laatste versie van RStudio via de website. Er is een aparte versie voor Debian-gebruikers, zij kopi\"eren dus best de URL van de link op de website i.p.v. deze te gebruiken die hieronder gegeven is.

\begin{verbatim}
sudo apt install biber git jabref r-base texlive-bibtex-extra \
  texlive-extra-utils texlive-fonts-recommended texlive-lang-european \
  texlive-latex-base texlive-latex-extra texlive-latex-recommended \
  texlive-pictures texstudio ttf-mscorefonts
wget https://download1.rstudio.org/desktop/bionic/amd64/rstudio-1.2.5033-amd64.deb
sudo dpkg -i ./rstudio-1.2.5033-amd64.deb
\end{verbatim}

\paragraph{Fedora}

Controleer eerst de link naar de laatste versie van RStudio via de website. Dit is één lang commando:

\begin{verbatim}
sudo dnf install git texstudio R \
  java-1.8.0-openjdk-openjfx texlive-collection-latex \
  texlive-texliveonfly texlive-babel-dutch \
  msttcore-fonts-installer.noarch \
  https://download1.rstudio.org/desktop/fedora28/x86_64/rstudio-1.2.5033-x86_64.rpm
\end{verbatim}

Je kan JabRef ook installeren vanuit de Fedora package repository, maar dan krijg je een verouderde versie. Je kan dan beter de ``Platform Independent Runnable Jar'' downloaden via de projectwebsite\footnote{\url{https://jabref.org/}}. Die kan je dan opstarten vanuit de shell met het commando (hier voorbeeld voor versie 4.3.1):

\begin{verbatim}
java -jar JabRef-4.3.1.jar
\end{verbatim}

\section{Configuratie}

\subsection{Git, GitHub}

Wellicht heb je Git al geconfigureerd voor enkele van je andere vakken. Kijk eventueel alles nog eens na! Als alles ok is, kan je deze sectie overslaan.

\emph{Wij raden aan om Git via de command line te gebruiken.} Zo krijg je het beste inzicht in de werking. Het commando \texttt{git status} geeft op elk moment een goed overzicht van de toestand van je lokale repository en geeft aan met welk commando je een stap verder kan zetten of de laatste stap ongedaan kan maken. Voor wie toch een GUI verkiest, raden we GitKraken~\footnote{\url{https://www.gitkraken.com/}} aan.

Als je nog geen GitHub-account hebt, kies dan een gebruikersnaam die je na je afstuderen nog kan gebruiken (dus bv.~niet je HoGent login). De kans is erg groot dat je tijdens je carrière nog van GitHub gebruik zult maken. Koppel ook je HoGent-emailadres aan je GitHub account (je kan meerdere adressen registreren). Op die manier kan je aanspraak maken op het GitHub Student Developer Pack\footnote{\url{https://education.github.com/pack}}, wat je gratis toegang geeft tot een aantal in principe betalende producten en diensten.

Windows-gebruikers voeren volgende instructies uit via Git Bash, macOS- en Linux-gebruikers via de standaard (Bash) terminal.

\begin{verbatim}
git config --global user.name 'Pieter Stevens'
git config --global user.email 'pieter.stevens.u12345@student.hogent.be'
git config --global push.default simple
\end{verbatim}

Maak ook een SSH-sleutel aan om het synchroniseren met GitHub te vereenvoudigen (je moet dan geen wachtwoord meer opgeven bij push/pull van/naar een private repository).

\begin{verbatim}
ssh-keygen
\end{verbatim}

Volg de instructies op de command-line, druk gewoon ENTER als je gevraagd wordt een wachtwoordzin (pass phrase) in te vullen. In de home-directory van je gebruiker (bv. \verb|c:\Users\Pieter| op Windows, \verb|/Users/pieter| op Mac, \verb|/home/pieter| op Linux) is nu een directory met de naam \verb|.ssh/| aangemaakt met twee bestanden: \verb|id_rsa| (je private key) en \verb|id_rsa.pub| (je public key). Open dit laatste bestand met een teksteditor en kopieer de volledige inhoud naar het klembord. Ga vervolgens naar je GitHub profiel en kies in het menu links voor SSH and GPG keys. Klik rechtsboven op de groene knop met ``New SSH Key'' en plak de inhoud van je publieke sleutel in het veld ``Key''. Bevestig je keuze.

Test nu of je de code van de cursus Onderzoekstechnieken kan downloaden. Ga in de Bash shell naar een directory waar je dit project lokaal wil bijhouden en voer uit:

\begin{verbatim}
git clone git@github.com:HoGentTIN/onderzoekstechnieken-cursus.git
\end{verbatim}

Als dit lukt, is er nu een directory aangemaakt met dezelfde naam als de repository. Je mag indien gewenst de directory verplaatsen en zelfs de naam wijzigen. Doe tijdens het semester regelmatig \texttt{git pull} (binnen deze directory) om de laatste wijzigingen in het cursusmateriaal bij te werken. Pas zelf geen bestanden aan binnen deze repository, dit zal leiden tot conflicten.

\subsection{Lettertypes}

De HOGENT-huisstijl, zoals toegepast in de slides, maakt gebruik van lettertypes die niet standaard geïnstalleerd zijn. Als je een PDF wil genereren van de slides, heb je deze lettertypes nodig.

Gebruikers van Linux downloaden ook best de gekende Microsoft-fonts (Arial, Courier, Times New Roman, enz.). Als je de installatie-instructies in Sectie~\ref{ssec:installatie-linux} gevolgd hebt, dan is dat al in orde.

De benodigde lettertypes zijn:

\begin{itemize}
  \item Montserrat: \url{https://fonts.google.com/specimen/Montserrat}
  \item Code Pro Black: o.a. via \url{https://www.wfonts.com/font/code-pro-black}
  \item Fira Math: \url{https://github.com/firamath/firamath}
  \item Inconsolata: \url{https://fonts.google.com/specimen/Inconsolata}
\end{itemize}

De Google Fonts kan je als volgt downloaden: volg de link naar het lettertype, klik op ``Select this font'' en vervolgens rechtsonder op de zwarte balk met de tekst ``1 Family selected''. In de pop-up zie je rechtsboven een download-icoon. Klik hier op om het lettertype te downloaden.

\subsection{TeXstudio}

Controleer deze instellingen via menu-item \emph{Options > Configure TeXstudio}:

\begin{itemize}
  \item Build:
  \begin{itemize}
    \item Default Compiler: XeLaTeX
    \item Default Bibliography tool: Biber
  \end{itemize}
  \item Commands:
  \begin{itemize}
    \item \texttt{xelatex -synctex=1 -interaction=nonstopmode  -shell-escape \%.tex}
    
    (voeg de optie \texttt{-shell-escape} toe)
  \end{itemize}
  \item Editor:
  \begin{itemize}
    \item Indentation mode: Indent and Unindent Automatically
    \item Replace Indentation Tab by Spaces: Aanvinken
    \item Replace Tab in Text by spaces: Aanvinken
    \item Replace Double Quotes: English Quotes: \verb|``''|
  \end{itemize}

\end{itemize}

Om te testen of TeXstudio goed werkt, kan je het bestand \texttt{cursus/cursus-onderzoekstech\-nie\-ken.tex} openen. Kies \emph{Tools > Build \& View} (of druk F5) om de cursus te compileren in een PDF-bestand. Controleer of er achteraan het document een bibliografie en/of index te vinden is. Indien niet, volg dan nog volgende stappen:

\begin{enumerate}
  \item Kies \emph{Tools > Index} voor het genereren van de index;
  \item Kies \emph{Tools > Bibliography} (of druk F8) voor het genereren van de bibliografie;
  \item Kies \emph{Tools > Index} (geen shortcut) om de zoekindex te genereren;
  \item Voer daarna opnieuw \emph{Build \& View} (F5) uit.
\end{enumerate}

Veel functionaliteiten van \LaTeX{} zitten in aparte packages die niet noodzakelijk standaard geïnstalleerd zijn. De eerste keer dat je een bestand compileert, is het dan ook mogelijk dat er extra packages moeten gedownload worden. MiK\TeX{} zal een pop-up tonen om je toestemming te vragen, bevestig dit. Op Linux is het mogelijk dat je deze packages nog manueel moet installeren. De eerste keer compileren kan enkele minuten duren zonder dat je feedback krijgt over wat er gebeurt. Even geduld, dus!

Indien er zich fouten voordoen bij de compilatie, kan je onderaan in het tabblad Log een overzicht krijgen van de foutboodschappen.

\subsection{JabRef}

JabRef\footnote{\url{http://www.jabref.org/}} is een GUI voor het bewerken van Bib\TeX{}-bestanden, een soort database van bronnen uit de wetenschappelijke- of vakliteratuur voor een \LaTeX{}-document.

Kies in het menu voor \emph{Options > Preferences > General} en kies onderaan voor de optie ``Default bibliography mode'' voor ``biblatex''. Dit maakt de bestandsindeling van de bibliografische databank compatibel met dat van de cursus en het aangeboden \LaTeX{}-sjabloon voor de bachelorproef.

Kies in het \emph{Preferences}-venster voor de categorie \emph{File} en geef een directory op voor het bijhouden van PDFs van de gevonden bronnen onder \emph{Main file directory}. Het is heel interessant om alle gevonden artikels te downloaden en onder die directory bij te houden. Nog beter is om als naam van het bestand de Bib\TeX{} key te nemen (typisch naam van de eerste auteur + jaartal, bv. \texttt{Knuth1998.pdf}). Je kan het bestand dan makkelijk openen vanuit JabRef.

Voor meer gedetailleerde informatie over het bijhouden van bibliografische referenties, zie de bachelorproefgids~\autocite{VanVreckem2017}.

\section{Gebruik van R}

R is een softwarepakket voor het bewerken, analyseren en visualiseren van data. Het heeft onder meer:

\begin{enumerate}
  \item een effectieve gegevensbeheer- en opslagfaciliteit,
  \item een reeks operatoren voor berekeningen op arrays, in het bijzonder matrices,
  \item een grote verzameling van instrumenten voor data-analyse,
  \item grafische faciliteiten voor data-analyse en weergave en
  \item een goed ontwikkelde, eenvoudige en effectieve programmeertaal (genaamd 'S').
\end{enumerate}

R heeft een ingebouwde hulpfaciliteit die vergelijkbaar is met die van UNIX man-pages. Voor meer informatie over elke specifieke functie, bijvoorbeeld \texttt{solve}, kan je volgend commando oproepen

\begin{lstlisting}
> help (solve)
\end{lstlisting}

Een alternatief is
\begin{lstlisting}
> ?solve
\end{lstlisting}

Er is online heel veel informatie terug te vinden over R. Er is een erg levendige en open \textit{community} van mensen wereldwijd die professioneel bezig zijn met R.

Je zal ook merken dat er meerdere manieren zijn om in R eenzelfde taak uit te voeren, bv. een databestand inlezen of een grafiek plotten. Er zijn meer bepaald twee grote ``families'' van werkwijzen die vaak aangeduid worden met enerzijds \textit{Base-R} en anderzijds \textit{the tidyverse}. \textit{Base-R} omvat de commando's en functies die al van oudsher in R aanwezig zijn, maar die door verschillende auteurs kunnen geschreven zijn, met onderling verschillende codeerstijl en API. \textit{The tidyverse} is een verzameling van codebibliotheken met een gemeenschappelijke filosofie en codeerstijl met als doel makkelijk leesbare code en krachtige functionaliteit.

In deze cursus hebben we niet echt een doorgedreven keuze gemaakt voor een van de twee. Voor alle taken die we verwachten dat studenten ze kunnen uitvoeren met R is er echter voorbeeldcode voorzien, daar kan je je op baseren om zelf aan de slag te gaan.

\subsection{Een omgeving opzetten voor bijhouden van oefeningen}

Voor de oefeningen waarvoor je R nodig hebt, kan je best een omgeving opzetten waar je alle databestanden en code kan bijhouden. Het is ook interessant om daar een Git repository van te maken.

\begin{enumerate}
  \item In RStudio, kies voor \textit{File > New Project}
  \item Selecteer \textit{New Directory}
  \item Selecteer \textit{New Project}
  \item Kies een naam voor de directory (bv. \texttt{ozt-oefeningen}) en de directory waaronder je het nieuwe project wil aanmaken (kies je zelf). Vink ook \textit{Create a git repository} aan.
  \item Maak nu een directory \texttt{datasets} aan onder \texttt{ozt-oefeningen} (dat kan via bestandsbeheer of in RStudio rechtsonder in het tabblad \textit{Files}). Kopieer in deze directory alle databestanden uit de Github-repository van de cursus, meer bepaald uit de directory \texttt{oefeningen/datasets} en \texttt{cursus/data}. Uit die laatste directory heb je niet de \texttt{.R}-bestanden nodig, maar wel de \texttt{.csv}, \texttt{.txt} en \texttt{.sav}-bestanden.
  \item Gebruik tenslotte de command-line of de Git-functionaliteit binnen R (bovenaan rechts in het tabblad Git) om een eerste commit aan te maken met de datasets, en de bestanden die door Rstudio zijn aangemaakt: \texttt{.gitignore} en een \texttt{.Rproj}-bestand.
\end{enumerate}

\subsection{Commando's opslaan en output uitvoeren}

Je slaat de code die je schrijft om oefeningen op te lossen best op in een R-script. Dat is een tekstbestand met de extensie \texttt{.R}. Je kan een nieuw R-script aanmaken met \textit{File > New File > R-script (Ctrl+Shift+N)}. Kies zelf zinvolle namen en/of mappenstructuren voor je oefeningen (bv. \texttt{hst2/oef-2-5.R}).

Als je de functionaliteiten van de \textit{tidyverse} wil gebruiken, start dan het script met:

\begin{lstlisting}
library(tidyverse)
\end{lstlisting}

Als de commando's in een extern bestand worden opgeslagen, bv. \texttt{commands.R} in de werkmap, dan kunnen deze in een R-sessie op elk moment uitgevoerd worden op de console (linksonder, tabblad \textit{Console}) met de opdracht:

\begin{lstlisting}
> source("commands.R")
\end{lstlisting}

Dit zal je echter niet zo vaak nodig hebben. Je kan ook op een regel in het script gaan staan en \textit{Ctrl+Enter} indrukken. Dat zal het commando op die regel uitvoeren en verder gaan naar de volgende regel. Door achtereenvolgens \textit{Ctrl+Enter} te blijven indrukken, voer je dus regel per regel het script uit.

\subsection{R omgeving en workspace}

De entiteiten die R cre\"eert en manipuleert staan bekend als objecten. Deze kunnen variabelen zijn, arrays van cijfers, reeksen, functies of meer algemene structuren die uit dergelijke componenten zijn gebouwd. Tijdens een R-sessie worden objecten gemaakt en opgeslagen op naam. Je kan deze in Rstudio rechtsboven in het tabblad \textit{Environment} terugvinden. Hetzelfde R commando

\begin{lstlisting}
> objects()
\end{lstlisting}

geeft een overzicht van alle objecten die gemaakt zijn tot op dat moment.
De verzameling van objecten die momenteel zijn opgeslagen, heet de werkruimte.
Om objecten te verwijderen kan je in het \textit{Environment}-tabblad op het bezem-icoontje klikken. In de console is de functie \texttt{rm} beschikbaar, waarmee je individuele objecten kan verwijderen:

\begin{lstlisting}
> rm (x, y, z, inkt, junk, temp, foo, bar)
\end{lstlisting}

Alle objecten die tijdens een R-sessie zijn aangemaakt, kunnen permanent in een bestand worden opgeslagen voor gebruik in de toekomstige sessies. Wanneer deze optie geactiveerd is, worden de objecten weggeschreven naar een bestand met extensie \texttt{.RData}.

In dit hoofdstuk onderzoeken we hoe je een dataset definieert in R. Er worden slechts twee commando's onderzocht. Het eerste is voor het eenvoudig toewijzen van gegevens, en het tweede is voor het inlezen van een databestand. Er zijn verschillende manieren om gegevens in een R-sessie te lezen, maar we richten ons op slechts twee om het eenvoudig te houden.

\subsection{Toewijzing}

De meest directe manier om een lijst met nummers op te slaan is via een opdracht met behulp van het \texttt{c}-commando. (C staat voor "combineren.") Het idee is dat een lijst met nummers onder een bepaalde naam wordt opgeslagen, en de naam wordt gebruikt om te verwijzen naar de gegevens. Een lijst wordt gespecificeerd met de opdracht \texttt{c}, en de toewijzing wordt geduid met de symbolen "<-". Een andere term die gebruikt wordt om de lijst met nummers te omschrijven is \texttt{vector}.

De cijfers binnen de \texttt{c}-opdracht worden gescheiden door komma's. Als voorbeeld kunnen we een nieuwe variabele maken, genaamd "\texttt{x}".

\begin{lstlisting}
> x <- c(10.4, 5.6, 3.1, 6.4, 21.7)
\end{lstlisting}

Wanneer je dit commando invoert, mag je geen uitvoer zien behalve een nieuwe opdrachtregel. Het commando maakt een lijst met nummers genaamd "x". Om te zien welke elementen zijn opgenomen in x, typ zijn naam en druk op de enter-toets.

Om met \'e\'en van de nummers te werken, kan je toegang krijgen tot de variabele en vervolgens vierkante haakjes noteren die aangeven welk nummer u wilt beschouwen:

\begin{lstlisting}
> x[2]
[1] 5.6
\end{lstlisting}

\subsection{Een csv-bestand lezen}

We gaan ervan uit dat het gegevensbestand een CSV-bestand is (\textit{Comma-Separated Values} of waarden gescheiden door komma's). Dat wil zeggen, elke regel bevat een rij met waarden die getallen of letters kunnen zijn, en elke waarde wordt gescheiden door een komma. We gaan ervan uit dat de eerste rij een lijst met labels bevat. Het idee is dat de labels in de bovenste rij gebruikt worden om te verwijzen naar de verschillende variabelen per rij.

In de \textit{tidyverse} kan je een CSV-bestand inlezen met de functie \texttt{read\_csv()}. \textbf{Let op:} deze functie veronderstelt dat de dataset is ingedeeld volgens Engelstalige conventies: het decimaalteken is de punt (\texttt{.}) en kolommen worden gescheiden door komma's (\texttt{,}). Voor datasets in het Nederlands met komma (\texttt{,}) als decimaalteken en kommapunt (\texttt{;}) als kolomscheidingsteken, is er de functie \texttt{read\_csv2()}.

Het resultaat van de \texttt{read\_csv}-functies is een object van het type \texttt{tibble}\footnote{In \textit{Base R} heb je een gelijkaardige datastructuur, de \texttt{data.table}}. Je kan deze meteen toewijzen aan een variabele, bv.:

\begin{lstlisting}
aardbevingen <- read_csv("datasets/Aardbevingen.csv")
android_persistence
  <- read_csv2("datasets/android_persistence.csv")
\end{lstlisting}

\begin{exercise}
  Probeer een databestand te lezen vanuit RStudio:
  
  \begin{enumerate}
    \item Gebruik de help-functie in Rstudio om na te gaan welke parameters er nog mogelijk zijn in \texttt{read\_csv}.
    \item Importeer het databestand \texttt{computers.csv} en wijs het toe aan de variabele \texttt{computers}. Kijk in het \textit{Environment}-tabblad na wat de waarde is van deze variabele.
  \end{enumerate}
  
Ga met behulp van het help commando na wat de parameters zijn van het commando. Probeer daarna het bestand \texttt{computers.csv} in te lezen. Je vindt het in \texttt{cursus/data}.
\end{exercise}

Het databestand \texttt{computers.csv} komt uit de publicatie van \autocite{Stengos2005}. Deze dataset bevat data van 1993 tot 1995 over de prijzen van computers. Je kan nagaan wat het effect van de toevoeging van een cd-rom-station is op de prijs van de computer, of wat het effect is van de kloksnelheid op de prijs. 

Via het \texttt{names()} commando kan je achterhalen welke kolommen gedefinieerd zijn:

\begin{lstlisting}[breaklines=true]
> names(computers)
[1] "price"   "speed"   "hd"      "ram"     "screen"  "cd"      "multi"   "premium" "ads"     "trend"
\end{lstlisting}

Voor de uitvoering van het commando \texttt{read.csv} gebruikt R een specifiek soort variabele, dat een dataframe heet. Alle gegevens worden opgeslagen in het dataframe als afzonderlijke kolommen. Als u niet zeker weet wat voor variabele u hebt, dan kunt u de opdracht \texttt{attributes} gebruiken. Hiermee worden alle dingen vermeld die R gebruikt om de variabele te beschrijven:

\begin{lstlisting}[breaklines=true]
attributes(computers)
$names
[1] "price"   "speed"   "hd"      "ram"     "screen"  "cd"      "multi"   "premium" "ads"     "trend"  

$class
[1] "tbl_df"     "tbl"        "data.frame"

$row.names
[1]    1    2    3    4    5    6    7    8    9   10   11   12   13   14   15   16   17   18   19   20   21   22   23   24   25   26   27
[28]   28   29   30   31   32   33   34   35   36   37   38   39   40   41   42   43   44   45   46   47   48   49   50   51   52   53   54
...
[ reached getOption("max.print") -- omitted 5259 entries ]

$spec
cols(
price = col_integer(),
speed = col_integer(),
hd = col_integer(),
ram = col_integer(),
screen = col_integer(),
cd = col_character(),
multi = col_character(),
premium = col_character(),
ads = col_integer(),
trend = col_integer()
)

\end{lstlisting}

Met de functie \texttt{glimpse()} kan je een vluchtige blik werpen op de inhoud van de dataset:

\begin{lstlisting}
glimpse(computers)
Observations: 6,259
Variables: 10
$ price   <dbl> 1499, 1795, 1595, 1849, 3295, 3695, 1720, ...
$ speed   <dbl> 25, 33, 25, 25, 33, 66, 25, 50, 50, 50, 33...
$ hd      <dbl> 80, 85, 170, 170, 340, 340, 170, 85, 210, ...
$ ram     <dbl> 4, 2, 4, 8, 16, 16, 4, 2, 8, 4, 8, 8, 4, 8...
$ screen  <dbl> 14, 14, 15, 14, 14, 14, 14, 14, 14, 15, 15...
$ cd      <chr> "no", "no", "no", "no", "no", "no", "yes",...
$ multi   <chr> "no", "no", "no", "no", "no", "no", "no", ...
$ premium <chr> "yes", "yes", "yes", "no", "yes", "yes", "...
$ ads     <dbl> 94, 94, 94, 94, 94, 94, 94, 94, 94, 94, 94...
$ trend   <dbl> 1, 1, 1, 1, 1, 1, 1, 1, 1, 1, 1, 1, 1, 1, ...
\end{lstlisting}

\subsection{Data types}

We kijken naar enkele manieren waarop R gegevens kan opslaan en organiseren. Dit is echter een inleiding dus beschouwen we maar een kleine subset van de verschillende datatypes die door R worden herkend. 

\subsubsection{Numbers}

De meest eenvoudige manier om een nummer op te slaan is om een variabele van een enkel getal te nemen:

\begin{lstlisting}
> a <- 3
\end{lstlisting}

Met deze variable kunnen we enkele basisoperaties doen en opslaan:

\begin{lstlisting}
> b <- sqrt(a*a+3)
> b
[1] 3.464102
\end{lstlisting}

Het \texttt{numeric} commando kan gebruikt worden om een lijst met nummers te initialiseren. Onderstaande opdracht maakt bijvoorbeeld een lijst van 10 nummers. Het \texttt{typeof} commando geeft het type terug van de variabele.

\begin{lstlisting}
> a <- numeric(10)
> a
[1] 0 0 0 0 0 0 0 0 0 0
> typeof(a)
[1] "double"
\end{lstlisting}

\subsubsection{Strings}

Een tekenreeks wordt gespecificeerd door gebruik te maken van aanhalingstekens. Zowel enkelvoudige als dubbele aanhalingstekens zijn mogelijk:

\begin{lstlisting}
> a <- "hello"
> a
[1] "hello"
> b <- c("hello","there")
> b
[1] "hello" "there"
> b[1]
[1] "hello"
\end{lstlisting}

\subsubsection{Factors}

In een volgend hoofdstuk zullen we leren dat elke variabele een zgn.~\textit{meetniveau}\index{meetniveau} heeft (zie Sectie~\ref{sec:onderzoeksproces-basisconcepten}). Eén van deze meetniveaus zijn de zgn.~\textit{kwalitatieve variabelen} die maar een beperkt aantal mogelijke waarden heeft, niet noodzakelijke numeriek. In R wordt dit soort variabelen een \textit{factor}\index{factor} genoemd.

Je geeft aan dat een variabele een factor is met behulp van het \texttt{factor} commando. 

\subsubsection{Data frames}

Data kan worden opgeslagen aan de hand van \textit{data frames}\index{data frame} (\textit{Base R}) of \textit{tibbles}\index{tibble} (\textit{tidyverse}). Beide kunnen meestal door elkaar gebruikt worden en je kan er meestal dezelfde bewerkingen op uitvoeren.

Dit is een manier om verschillende vectoren van verschillende types te nemen en ze op te slaan in dezelfde variabele. De vectoren kunnen van alle soorten zijn. Een dataframe kan bijvoorbeeld verschillende vectoren bevatten en elke lijst kan een vector zijn van factoren, strings of nummers.

Er zijn verschillende manieren om data frames te maken en te manipuleren. In deze cursus zullen we deze meestal aanmaken door een CSV-bestand in te lezen. De meeste andere vallen buiten het bereik van deze inleiding. Ze worden hier alleen genoemd om een meer volledige beschrijving te geven. 

\lstinputlisting{data/dataframe.R}

\subsubsection{Logische variabelen}

Een ander belangrijk gegevenstype is het logische type. Er zijn twee vooraf gedefinieerde variabelen, \texttt{TRUE} en \texttt{FALSE}.

\subsubsection{Tables}

Een andere  manier om informatie op te slaan is in een tabel.  We kijken alleen maar naar het maken en defini\"eren van tabellen. 

\lstinputlisting{data/tables.R}

Als je rijen wilt toevoegen aan de tabel, voeg dan nog een vector toe als argument van de tabelopdracht. In het onderstaande voorbeeld hebben wij twee vragen. In de eerste vraag staan de reacties  'Never', 'Sometimes' of 'Always'. In de tweede vraag staan de reacties 'Yes', 'No' of 'Maybe'. De set van vectoren 'a', en 'b' bevatten het antwoord voor elke meting. Het derde punt in 'a' is hoe de derde persoon op de eerste vraag reageerde en het derde punt in 'b' is hoe de derde persoon op de tweede vraag reageerde.

\lstinputlisting[breaklines=true]{data/twotables.R}

\subsubsection{Matrix}

Een matrix is een verzameling van gegevens die zijn aangebracht in een tweedimensionale rechthoekige indeling. Een voorbeeld van een matrix is bijvoorbeeld als volgt:

\[
\begin{bmatrix}
2 & 3 \\ 
4 & 5  
\end{bmatrix}
\]

\lstinputlisting{data/matrix.R}

\section{Oefeningen}

\begin{exercise}
    In deze oefening werken we met het ingebouwde data frame \texttt{mtcars}. 
    \begin{enumerate}
        \item   Gebruik ingebouwde R-functies om informatie weer te geven over deze dataset
        \item   Geef de waarde terug voor de eerste rij, tweede kolom
        \item   Geef het aantal rijen en het aantal kolommen
        \item   Geef enkel de kolom terug met de definities van de cylinders
    \end{enumerate}

    Om een data frame te bekomen met de twee kolommen \texttt{mpg} en \texttt{hp}, 
    pakken we de kolomnamen in een indexvector in met single square bracket operator. 
    Probeer ook eens op te zoeken hoe je een rijrecord van de ingebouwde data set \texttt{mtcars} bepaalt.
\end{exercise}

\begin{exercise}
  Maak zelf een willekeurige datafile aan in Excel en probeer deze in te lezen in R. Zijn er nog dataformaten die ondersteund worden door R?
\end{exercise}

\begin{exercise}
  Genereer een $4 \times 5$ array en noem die $x$. Geneer daarna een $3 \times 2$ array $i$ waarin de eerste kolom de rij-index kan zijn van $x$ en de tweede kolom een kolomindex voor $x$. Vervang de elementen gedefinieerd door de index in $i$ in $x$ door 0. 
\end{exercise}

\begin{exercise}
  Genereer een vector waar een voornaam en een achternaam in komen. Benoem ook de naam van de kolommen. Geef daarna de voornaam terug van het eerste element van de array. 
\end{exercise}

\begin{exercise}
  Importeer het bestand \texttt{rainforest.csv} in R.
  Je kan dit bestand terugvinden in de Github-repository van deze cursus, in de map \texttt{oefeningen/datasets}. 
  Een beschrijving van dit data frame is terug te vinden in dezelfde map, bestand \texttt{rainforest.html}.
    
  Je kan dit bestand importeren met behulp van volgende code:
  \begin{lstlisting}
rainforest <- read.csv("../path/to/rainforest.csv", sep = ",")
  \end{lstlisting}
  
  Probeer voor deze datafile te tellen hoeveel rijen er zijn per species die volledig en compleet zijn (dus geen n.a. bevatten). 
  Je kan hiervoor \texttt{with}, \texttt{table} en \texttt{complete.cases} gebruiken. 
\end{exercise}

\begin{exercise}
	Genereer een vector met de waarden van $e^x cos(x)$ voor $x= 3, 3.1, 3.2, \dots ,6$
\end{exercise}

\begin{exercise}
	Bereken: $\sum_{i=1}^{100}(i^3 + 4i^2)$
\end{exercise}

\include{2_onderzoeksproces}
\chapter{Univariate analyse}
\label{ch:analyse1var}

Wanneer onderzoekers een aantal metingen uitgevoerd hebben in een steekproef voor een bepaalde populatie, dan willen we weten welke eigenschappen deze groep als geheel heeft. Het is onpraktisch om alle metingen apart te beschouwen of op te sommen. Om inzichten te krijgen in de verzamelde data, hebben we technieken nodig om deze te visualiseren en samen te vatten. Het geheel van deze technieken wordt \emph{beschrijvende statistiek} genoemd.

\begin{definition}[Beschrijvende statistiek]
  Met beschrijvende statistiek \index{beschrijvende statistiek} bedoelen we een verzameling van technieken om data synthetisch voor te stellen en samen te vatten.
\end{definition}

In dit hoofdstuk gaan we variabelen afzonderlijk beschouwen. In Hoofdstuk \ref{ch:analyse2var} zullen we ook zoeken naar verbanden tussen twee variabelen onderling.

\section{Leerdoelen}
\label{sec:analyse1var-leerdoelen}

Na dit hoofdstuk moet je in staat zijn om:

\begin{itemize}
  \item Voor elk meetniveau de geschikte centrum- en spreidingsmaten te benoemen;
  \item De formules voor gemiddelde, variantie en standaardafwijking van een steekproef te reproduceren en te begrijpen;
  \item Van een gegeven variabele de centrum- en spreidingsmaten te berekenen;
  \item Voor elk meetniveau geschikte visualisatietechnieken te benoemen;
  \item Voor een gegeven variabele geschikte visualisatietechnieken toe te passen;
  \item Een gegeven grafiek te interpreteren, o.a.~het grafiektype benoemen, centrum- en spreidingsmaten afleiden.
\end{itemize}

\section{Centrum- en spreidingsmaten}

Stel je voor dat een groep onderzoekers een studie wil voeren over het verschijnsel superhelden. Vragen die men kan stellen zijn:

\begin{itemize}
  \item Hoe groot zijn superhelden doorgaans?
  \item Hoeveel wegen ze?
  \item Hoe succesvol zijn ze in het veilig maken van hun woonomgeving?
  \item Hoeveel mensen redden ze?
  \item enz.
\end{itemize}

Om de resultaten van metingen samen te vatten, zijn er verschillende methoden en die hangen af van het meetniveau van de betreffende variabele. Enerzijds gaan we op zoek naar een waarde die representatief is voor de hele groep, een \index{centrummaat}\emph{centrummaat}, en anderzijds naar een waarde die aangeeft hoe groot de onderlinge verschillen zijn binnen de groep, een \index{spreidingsmaat}\emph{spreidingsmaat}. Een centrummaat en de bijhorende spreidingsmaat worden vaak als samenvatting gebruikt voor een reeks metingen.

Tabel~\ref{tab:centrum-spreidingsmaten} geeft een overzicht van geschikte centrum- en spreidingsmaten per meetniveau. Deze worden verderop in dit hoofdstuk gedefinieerd.

\begin{table}
  \centering
  \begin{tabular}{lll}
  \toprule
 	\textbf{Meetniveau} & \textbf{Centrummaten} & \textbf{Spreidingsmaten}      \\
  \midrule
 	Kwalitatief         & Modus                 & ---                           \\
  \midrule
 	Kwantitatief        & Gemiddelde            & Variantie, standaardafwijking \\
	                    & Mediaan               & Bereik, interkwartielafstand  \\
  \bottomrule
  \end{tabular}
  \caption[Geschikte centrum- en spreidingsmaten voor elk meetniveau.]{Geschikte centrum- en spreidingsmaten voor elk meetniveau. De centrummaat en bijhorende spreidingsmaat worden vaak gebruikt als samenvatting voor de resultaten van de groep als geheel.}
  \label{tab:centrum-spreidingsmaten}
\end{table}

\section{Kwalitatieve variabelen}

Bij kwalitatieve variabelen is de waarde niet noodzakelijk een getal. De variabele `gender' heeft als waarden onder andere \emph{man} en \emph{vrouw}. Hiermee is het moeilijk te rekenen. Daarom zijn de mogelijkheden om de waarden samenvatten beperkt. Als centrummaat definiëren we de \emph{modus}, maar een bijhorende spreidingsmaat is er niet echt. Om de spreiding over de verschillende voorkomende waarden te tonen, kan je eventueel wel gebruik maken van een frequentietabel die voor elke waarde aangeeft hoe vaak die voorkomt in de dataset.

\subsection{Modus}

\begin{definition}[Modus]
  De \index{modus} modus (Eng. \index{mode}\emph{mode})is de waarde die het meest voorkomt in een verzameling metingen.
\end{definition}

\begin{itemize}
  \item Er kunnen twee modi zijn: in dit geval spreken we van een \index{bimodaal}\emph{bimodale} variabele;
  \item Er kunnen ook meerdere modi zijn: dit noemen we \index{multimodaal} \emph{multimodaal}.
  \item De modus niet veel zin bij een kwantitatieve variabele, waar elke meting typisch uniek is. Soms is het in dat geval nuttig om ze te groeperen in categorieën (zie Voorbeeld~\ref{ex:modale-klasse}).
\end{itemize}

\begin{example}
  \label{ex:modale-klasse}
  De onderzoekers hebben bijgehouden hoeveel mensen Batman elk jaar gered heeft. Hieronder zijn de cijfers voor de laatste negen jaar, onderverdeeld in 4 categorieën.
  
  \begin{itemize}
    \item $[0-9]$ mensen : 4, 7
    \item $[10-19]$ mensen: 11, 16
    \item $[20-29]$ mensen : 20, 22, 25, 26
    \item $[30-39]$ mensen: 33
  \end{itemize}

  Categorie $[20-29]$ komt het meest voor. Dit is de modale klasse. Batman redt dus doorgaans tussen 20 en 29 mensen per jaar.
\end{example}

\section{Kwantitatieve variabelen}

Bij kwantitatieve variabelen is er meer keuze qua centrum- en spreidingsmaten. Voor dit soort variabelen kan je gebruik maken van:

\begin{itemize}
  \item gemiddelde en standaardafwijking;
  \item mediaan en interkwartielafstand.
\end{itemize}

Stel dat de onderzoekers een aantal metingen uitgevoerd hebben van de lengte van superhelden. Je vindt de resultaten in Figuur~\ref{gfx:helden}.

\begin{figure}
  \centering
  \begin{tikzpicture}[xscale=4,yscale=2]
  \draw (0,2) -- (0,0);
  \foreach \num/\label in {0/0, 0.2/20, .4/40, .6/60, .8/80, 1/100, 1.2/120, 1.4/140, 1.6/160, 1.8/180, 2/200}{%
    \draw (0, \num) -- (2.5, \num);
    \draw[shift={(0, \num)}] (1pt,0pt) -- (-1pt,0pt) node[left] {\scriptsize \label};
  }
  
  \node[anchor=north] (hero1) at (0.3,1.5)
  {\includegraphics[height=2.9cm]{les2-hero-1}};
  \node[anchor=north] (hero2) at (0.8,2.05)
  {\includegraphics[height=4cm]{les2-hero-2}};
  \node[anchor=north] (hero3) at (1.3,1.575)
  {\includegraphics[height=3.1cm]{les2-hero-3}};
  \node[anchor=north] (hero4) at (1.8,2.1)
  {\includegraphics[height=4.1cm]{les2-hero-4}};
  \node[anchor=north] (hero5) at (2.3,1.95)
  {\includegraphics[height=3.8cm]{les2-hero-5}};
  
  \node (size1) at (0.3, 1.5) {\scriptsize 141 cm};
  \node (size2) at (0.8, 2.1) {\scriptsize 198 cm};
  \node (size3) at (1.3, 1.51) {\scriptsize 143 cm};
  \node (size4) at (1.8, 2.15) {\scriptsize 201 cm};
  \node (size5) at (2.3, 1.95) {\scriptsize 184 cm};
  \end{tikzpicture}
  \caption{De lengte van enkele superhelden.}
  \label{fig:helden}
\end{figure}

\subsection{Gemiddelde}
\label{sec:gemiddelde}

\begin{definition}[Gemiddelde]
  \index{Gemiddelde} Het rekenkundig gemiddelde (symbool $\overline{x}$, Eng. \index{mean}\emph{mean}, \index{average}\emph{average}) van een verzameling waarden is de som van al deze waarden gedeeld door het aantal waarden.
  \begin{equation}
    \overline{x} = \frac{1}{n} \sum_{i=1}^{n} x_{i}
    \label{eq:Mean}
  \end{equation}

  Waarbij:
  \begin{itemize}
    \item $x_{i}$ de verschillende waarden zijn (in het voorbeeld van Figuur~\ref{fig:helden} zijn dit 141, 198, 143, enz.)
    \item $n$ het aantal waarden is (in het voorbeeld is $n = 5$).
  \end{itemize}
\end{definition}

\begin{remark}[!!]
  Merk op dat we met het symbool $\overline{x}$ specifiek het gemiddelde van een \emph{steekproef} aanduiden. Het gemiddelde van een \emph{populatie} wordt aangeduid met de Griekse letter mu, $\mu$.
  
  Zie Appendix~\ref{app:notatie} voor een overzicht van de gebruikte symbolen en notaties in deze cursus.
\end{remark}

\begin{exercise}
  Wat is de gemiddelde lengte van de superhelden?
\end{exercise}

\begin{exercise}
  Vraag: het gemiddelde van 15 cijfers is 12. Welk nummer moeten
  we aan de rij van cijfers toevoegen om een gemiddelde van 13 te bekomen?
\end{exercise}

\begin{exercise}
  Men zegt dat het rekenkundig gemiddelde gevoelig is aan uitschieters: een extreme waarde kan het rekenkundig gemiddelde zwaar beïnvloeden. 
  
  Stel je voor dat Kabouter Wesley (10 cm) toegevoegd wordt aan het team superhelden. Wat wordt dan de gemiddelde lengte?
\end{exercise}

\subsection{Variantie en standaardafwijking}
\label{sec:varEnSD}

De spreidingsmaat die typisch geassocieerd wordt met het gemiddelde is de standaardafwijking. Voordat we die definiëren, geven we eerst die van de \emph{variantie}.

\begin{definition}[Variantie]
  De \index{variantie} variantie van een steekproef (symbool $s^{2}$, Eng. \index{variance}\emph{sample variance}) is de som van de gekwadrateerde verschillen tussen de waarde van de dataset en het gemiddelde, gedeeld door het aantal waarden min één:
  \begin{equation}
  s^{2} = \frac{1}{n-1} \sum_{i=1}^{n} \left(\overline{x} - x_i \right)^{2}
  \label{eq:variantie}
  \end{equation}
\end{definition}

Het is je misschien opgevallen dat in de formule gedeeld wordt door $n-1$ en niet door $n$, wat je zou kunnen verwachten. Er bestaat ook effectief een variant van de formule met $n$ in de noemer. Dit noemen we de populatievariantie, aangeduid met $\sigma^2$ (de Griekse letter sigma).

De variantie van een steekproef wordt in de praktijk gebruikt als een schatting voor de (onbekende) populatievariantie. De formule met $n-1$ in de noemer is een zgn.~\textit{zuivere schatter} is, wat betekent dat bij veel herhalingen het gemiddelde van de schattingen convergeert naar de te schatten populatievariantie. Je kan dit wiskundig bewijzen, maar dat valt buiten het bestek van deze cursus.

\begin{example}
  De variantie bij de lengtes van onze superhelden wordt als volgt berekend:
  
  \begin{align*}
  	s^{2} & =  \frac{(173,4 - 141)^{2} + (173,4 - 198 )^{2} + (173,4 - 143)^{2} + (173,4 - 201)^{2} + (173,4 - 184 )^{2}}{4} \\
  	      & =  \frac{(-32,4)^{2} + (24,6)^{2} + (-30,4)^{2} + (27,6)^{2} + (10,6)^{2}}{4}                                    \\
  	      & = \frac{1049,76 + 605,16 + 924,16 + 761,76 + 112,36}{4}                                                          \\
  	      & = \frac{3453,2}{4} = 863,3
  \end{align*}
\end{example}

Bij de berekening van de steekproefvariantie wordt er gedeeld door $n-1$ en niet door $n$. Waarom? Aangezien de som van de afwijkingen $x_{i} - \overline{x}$ steeds 0 oplevert (zie hieronder in vergelijking \ref{eq:sumGemid}), kan de laatste afwijking gevonden worden uit de eerste $n-1$ afwijkingen. We berekenen dus niet het gemiddelde van $n$ getallen zonder verwantschap. Slecht $n-1$ van de gekwadrateerde afwijkingen kunnen vrij bewegen, daarom berekenen we het gemiddelde door het totaal te delen door $n-1$. Het getal $n-1$ noemt men het aantal \index{vrijheidsgraden}\emph{vrijheidsgraden} van de variantie.

\begin{equation}
 \sum_{i}^{n}(x_{i} - \overline{x}) = \sum_{i}^{n}x_{i} - \sum_{i}^{n}\overline{x} = \sum_{i}^{n}x_{i} - n \left(\frac{1}{n}\sum_{i}^{n} x_{i}\right)
\label{eq:sumGemid}
\end{equation}

\begin{definition}[Standaardafwijking]
  De \index{standaardafwijking} standaardafwijking (Eng. \index{standard deviation}\emph{standard deviation}) wordt dan gedefinieerd als de vierkantswortel van de variantie.
  \begin{equation}
  s = \sqrt{s^{2}} = \sqrt{\frac{1}{n-1} \sum_{i=1}^{n} \left(\overline{x} - x_i \right)^{2}}
  \label{eq:stdev}
  \end{equation}
\end{definition}

\begin{remark}[!!]
  Ook hier hebben we specifiek de variantie en standaardafwijking van een \emph{steekproef} gedefinieerd. De standaardafwijking van een \emph{populatie} wordt aangeduid met de Griekse letter sigma, $\sigma$.
\end{remark}

Dit geeft ons dus inzicht in wat normaal is en wat abnormaal is: een kleine standaardafwijking wijst erop dat de waarden dicht bij de centrummaat ($\overline{x}$) liggen, terwijl een grote standaardafwijking aangeeft dat de waarden verder verspreid liggen. In sommige gevallen wil men een grote standaardafwijking, in andere gevallen niet zoals hieronder beschreven.

\begin{example}
  Bij het vervaardigen van een schroevendraaier is de grootte van de kop belangrijk voor het goed functioneren van de schroevendraaier. Als we dus van 100 verschillende schroevendraaiers de kopgrootte meten, is het beter dat die grootte redelijk constant is en wensen we dus een kleine standaardafwijking.
\end{example}

\begin{example}
  Bij het onderzoek naar onze superhelden, wensen we te weten hoeveel ze ongeveer verdienen in hun normale job. We hebben een aantal rijke superhelden (bv. Batman) en een aantal minder rijke superhelden (bv. Spiderman). De spreiding op hun inkomen is dus groot, maar dat is niet per definitie een probleem.
\end{example}

Een aangename eigenschap van de standaardafwijking is dat het uitgedrukt kan worden in dezelfde metriek als de gemeten data. Bij ons voorbeeld van de superhelden, wil dat zeggen dat de standaardafwijking (ongeveer) 29,38 cm is.

Zoals het gemiddelde zijn de variantie en de standaarddeviatie gevoelig aan uitschieters. De variantie is eigenlijk gevoeliger dan het gemiddelde. Inderdaad, voor een uitschieter is de afstand tot het gemiddelde kleiner dan het kwadraat van deze afstand. % TODO: Wat wordt hiermee bedoeld?

\subsection{Mediaan}

De mediaan is een alternatieve centrummaat die als voordeel ten opzichte van het gemiddelde heeft dat die een stuk minder gevoelig is voor uitschieters.

\begin{definition}[Mediaan]
  Indien we alle metingen sorteren van klein naar groot, is de \index{mediaan} mediaan (Eng. \index{median}\emph{median}) het middelste cijfer. Als het aantal cijfers even is, neemt men het gemiddelde van de twee middelste cijfers.
\end{definition}

\begin{exercise}
  Wat is de mediaan van de lengtes van de superhelden?
\end{exercise}

\begin{exercise}
  Wat wordt de mediaan als Kabouter Wesley er bij komt? Is de impact groter of kleiner dan bij het gemiddelde?
\end{exercise}

\subsection{Bereik}

\begin{definition}[Bereik]
  Het \index{bereik}bereik (Eng. \index{range}\emph{range}) van een variabele is de absolute waarde van het verschil tussen de grootste en kleinste waarde.
  \begin{equation}
    | max_i(x_i) - min_i(x_i) |
  \end{equation}
\end{definition}

Het bereik van een variabele is vaak niet zo interessant als spreidingsmaat omdat ze zeer gevoelig is voor uitschieters.

\subsection{Kwartielen \& kwartielafstand}

De interkwartielafstand is een betere spreidingsmaat die veel minder gevoelig is voor uitschieters. Om te begrijpen hoe ze werkt, moeten we eerst het begrip \emph{kwartielen} definiëren.

\begin{definition}[Kwartielen]
  De \index{kwartiel} kwartielen zijn de waarden die een gesorteerde lijst van waarden in 4 gelijke delen deelt. Elk deel vormt dus een kwart van de dataset. Men spreekt van een eerste, tweede en derde kwartiel ($Q_{1}$, $Q_{2}$, $Q_{3}$).
\end{definition}

Dus:

\begin{itemize}
  \item het eerste kwartiel $Q_{1}$ is de waarde die de laagste 25 \% van de reeks afscheidt.
  \item het tweede kwartiel $Q_{2}$ is de waarde die de laagste 50\% van de reeks afscheidt.
  \item het derde kwartiel $Q_{3}$ is de waarde die de laagste 75\% van de reeks afscheidt.
\end{itemize}

Om te bepalen welke waarden in een gesorteerde rij precies de kwartielen zijn, ga je als volgt te werk (volgens \textcite{Moore2002}):

Als $n$ oneven is:

\begin{itemize}
  \item $Q_{1}$ is het $\frac{n+1}{4}$e getal
  \item $Q_{3}$ is het $\frac{3n+3}{4}$e getal
\end{itemize}

Als $n$ even is:

\begin{itemize}
  \item $Q_{1}$ is het $\frac{n+2}{4}$e getal
  \item $Q_{3}$ is het $\frac{3n+2}{4}$e getal
\end{itemize}

\begin{exercise}
  Met welke hiervoor gedefinieerde statistiek komt $Q_{2}$ overeen?
\end{exercise}

\begin{definition}[Interkwartielafstand]
  De \index{interkwartielafstand}interkwartielafstand (Eng. \index{interquartile range}InterQuartile Range, $IQR$) is het verschil tussen het derde en eerste kwartiel.
  
  \begin{equation}
    IQR = Q_3 - Q_1
  \end{equation}
\end{definition}

De definitie van kwartielen kan worden veralgemeend naar \index{deciel}\emph{decielen} (waarden die de dataset in tien gelijke delen verdelen) en \index{percentiel}\emph{percentielen} (honderd gelijke delen). Het 65e percentiel, bijvoorbeeld is dan het getal in de gesorteerde rij waarvoor geldt dat 65\% van de waarden \emph{kleiner} is dan dat getal.

\section{Visualisatie van één variabele}

Ook bij het visualiseren van één variabele hangt het meest geschikte grafiektype af van het meetniveau. In Tabel~\ref{tab:grafiektypes-1-variabele} vind je een overzicht.

\begin{table}
  \centering
  \begin{tabular}{ll}
  	\toprule
  	\textbf{Meetniveau} & \textbf{Grafiektype}          \\
  	\midrule
  	Kwalitatief         & Staafdiagram van frequenties  \\
  	\midrule
  	Kwantitatief        & Boxplot, histogram, dichtheid \\
  	\bottomrule
  \end{tabular}
  \caption{Geschikte grafiektypes per meetniveau voor het visualiseren van één variabele.}
  \label{tab:grafiektypes-1-variabele}
\end{table}

\subsection{Staafdiagram}

Bij het visualiseren van een kwalitatieve variabele wordt eerst een frequentietabel opgesteld van alle voorkomende waarden.

\begin{definition}[Frequentietabel]
  Een \index{frequentietabel}\emph{frequentietabel} is tabel waarin opgesomd staat hoeveel keer een waarde voorkomt in de volledige dataset (= frequentie). Meestal zijn de tabellen verticaal georiënteerd.
\end{definition}

Om de een \index{staafdiagram}\emph{staafdiagram} (zie Figuur~\ref{fig:staafdiagram}) te tekenen worden de verschillende waarden opgesomd onder de X-as en voor elke waarde wordt een staaf getekend waarvan de hoogte bepaald wordt door het aantal keer dat de overeenkomstige waarde voorkomt.

\begin{figure}
  \centering
  \includegraphics[width=.7\textwidth]{staafdiagram}
  \caption[Voorbeeld van een staafdiagram]{\textbf{Voorbeeld van een staafdiagram.} De labels op de x-as duiden de mogelijke waarden van de gevisualiseerde kwalitatieve variabele aan. De hoogte van elke staaf geeft de frequentie aan van elke waarde binnen de steekproef.}
  \label{fig:staafdiagram}
\end{figure}

\subsection{Boxplot}

De \index{boxplot}\emph{boxplot} (zie Figuur~\ref{fig:boxplot}) wordt gevormd door een rechthoek begrensd door de kwartielwaarden (25\% en 75\%). In deze rechthoek wordt ook de mediaan getekend. De stelen, die aan de rechthoek zitten, bevatten de rest van de waarnemingen, op de uitschieters en extremen na.

\begin{itemize}
  \item Een \index{uitschieter}\textit{uitschieter} is een waarde die meer dan 1,5 keer de interkwartielafstand boven/onder het derde/eerste kwartiel ligt en wordt aangeduid met een cirkeltje.
  \item Een \index{extremum}\textit{extremum} is een waarde die meer dan 3 keer de interkwartielafstand boven/onder het derde/eerste kwartiel ligt en wordt in een boxplot aangeduid met een sterretje.
\end{itemize}

Een boxplot wordt horizontaal of verticaal georiënteerd, op basis van wat het duidelijkst is.

\begin{figure}
  \centering
  % Source: http://mirrors.ibiblio.org/CTAN/graphics/pgf/contrib/pgfplots/doc/pgfplots.pdf
  % p.430
  \begin{tikzpicture}
    \begin{axis}[x=3cm,xticklabels={},xmax=2.3]
      \addplot+[
      boxplot prepared={
        draw direction=y,
        lower whisker=5,
        lower quartile=7,
        median=8.5,
        upper quartile=9.5,
        upper whisker=10,
      },
      ]
      table[row sep=\\,y index=0] {
        data\\ 1\\ 3\\
      }
      [right,color=hgorange]
      node at (boxplot box cs: 1,.6) {uitschieter}
      node at (boxplot box cs: \boxplotvalue{lower quartile},1) {$Q_1$}
      node at (boxplot box cs: \boxplotvalue{median},1)         {$Q_2$, mediaan}
      node at (boxplot box cs: \boxplotvalue{upper quartile},1) {$Q_3$}
      node at (boxplot box cs: \boxplotvalue{upper whisker},1)  {max}
      ;
    \end{axis}
  \end{tikzpicture}
  \caption[Voorbeeld van een boxplot]{\textbf{Voorbeeld van een boxplot.} De blauwe rechthoek duidt het interval aan waarbinnen de helft van de waarnemingen zich bevinden. De grenzen zijn het eerste kwartiel onderaan ($Q_1$) en het derde bovenaan ($Q_3$). De lijn in het midden van de rechthoek is de mediaan (of het tweede kwartiel, $Q_2$). De andere blauwe horizontale strepen duiden in principe de kleinste en grootste waarnemingen aan. In dit geval zijn er echter enkele waarnemingen die erg ver van de mediaan liggen. Deze worden uitschieters genoemd en worden apart geplot als een punt.}
  \label{fig:boxplot}
\end{figure}

\subsection{Histogram}

Een \index{histogram}\emph{histogram} (zie Figuur~\ref{fig:histogram}) is een soort staafdiagram, maar dan aangepast naar kwantitatieve variabelen. Het bereik van de variabele wordt onderverdeeld in een door de onderzoeker gekozen aantal, typisch even grote, intervallen of klassen. Voor elke klasse wordt er geteld hoeveel waarnemingen er binnen vallen en op basis daarvan wordt de hoogte van de staven bepaald.

\begin{figure}
  \centering
  \includegraphics[width=.7\textwidth]{histogram}
  \caption[Voorbeeld van een histogram]{\textbf{Voorbeeld van een histogram.} De X-as is hier onderverdeeld in een aantal even grote intervallen. De hoogte van elke staaf is gebaseerd op het aantal waarnemingen binnen elk interval.}
  \label{fig:histogram}
\end{figure}

\subsection{Dichtheidsgrafiek}

De duidelijkheid van een histogram hangt in grote mate af van de keuze van de intervallen. Als deze te groot gekozen zijn verlies je precisie, en als ze te klein zijn zullen verschillende intervallen geen observaties bevatten. Als alternatief voor een histogram kan je ook een \index{dichtheidsgrafiek}\index{kansdichtheidsgrafiek}(kans)dichtheidsgrafiek (zie Figuur~\ref{fig:dichtheidsgrafiek}) plotten. De X-as is dan niet onderverdeeld in klassen. Er wordt een continue curve getekend waarvan de hoogte overeenkomt met de hoeveelheid waarnemingen ``in de buurt''. Een dichtheidsgrafiek toont typisch nog beter hoe de waarnemingen gespreid zijn.

\begin{figure}
  \centering
  \includegraphics[width=.7\textwidth]{dichtheidsgrafiek}
  \caption[Voorbeeld van een dichtheidsgrafiek]{\textbf{Voorbeeld van een (kans)dichtheidsgrafiek.} De grafiek toont dezelfde data als het histogram in Figuur~\ref{fig:histogram}.}
  \label{fig:dichtheidsgrafiek}
\end{figure}





\section{Oefeningen}

\subsection{Centrum- en spreidingsmaten}

\begin{exercise}
  \label{ex:mean-stdev-freq}
  De formules voor het steekproefgemiddelde $\overline{x}$, de steekproefvariantie $s^2$ en de standaarddeviatie $s$ staan beschreven in secties \ref{sec:gemiddelde} en \ref{sec:varEnSD}.
  Hoe moeten deze formules aangepast worden om $\overline{x}$, $s^2$ en $s$ te berekenen wanneer we te maken hebben met een frequentietabel? 
  Doe dit voor de data in Tabel~\ref{tab:pinfreq}.
\end{exercise}

\begin{table}
  \centering
  \begin{tabular}{cc}
  	\toprule
  	Pinnen $x$ & Frequentie $f_{x}$ \\
  	\midrule
  	    0      &         2          \\
  	    1      &         1          \\
  	    2      &         2          \\
  	    3      &         0          \\
  	    4      &         2          \\
  	    5      &         4          \\
  	    6      &         9          \\
  	    7      &         11         \\
  	    8      &         13         \\
  	    9      &         8          \\
  	    10     &         8          \\
  	\bottomrule
  \end{tabular}
  \caption{Tijdens het spelen van een kegelspel is bijgehouden hoeveel pinnen telkens omver gegooid werden. Voor elke mogelijke score $x$ is bijgehouden hoeveel keer.}
  \label{tab:pinfreq}
\end{table}

\begin{exercise}
  \label{ex:variance-formula}
  In de formule voor de steekproefvariantie wordt het verschil tussen de meetpunten en het gemiddelde gekwadrateerd. Waarom? Zouden we geen eenvoudiger formule kunnen bedenken die een even goede maatstaf is voor de spreiding van een dataset? Hieronder vind je drie voorstellen (de derde is de ``echte'' formule).

  \begin{align}
    s^{2}_{1} &= \frac{1}{n-1} \sum_{i=1}^{n} (\overline{x} - x) \\
    s^{2}_{2} &= \frac{1}{n-1} \sum_{i=1}^{n} \left| \overline{x} - x\right| \\
    s^{2}_{3} &= \frac{1}{n-1} \sum_{i=1}^{n} (\overline{x} - x)^{2}
  \end{align}

  Pas elke formule toe op de twee datasets hieronder. Door het resultaat te vergelijken zou je moeten kunnen besluiten of de formules geschikt zijn als een spreidingsmaat.
  
  \begin{align*}
    X &= \left\{ 4,4,-4,-4 \right\} \\
    Y &= \left\{ 7,1,-6,-2 \right\}
  \end{align*}

\end{exercise}

\begin{exercise}
  Zoek eens zelfstandig op wat de variatieco\"effici\"ent voor een steekproef is. Hoe wordt die gedefinieerd voor een volledige populatie en wat zou je ermee kunnen doen?
\end{exercise}

\begin{exercise}
  \label{ex:ais}
  Importeer het bestand \texttt{ais.csv} in R (terug te vinden in de Github-repository van deze cursus, in de map \texttt{oefeningen/datasets}). 
  Een beschrijving van dit data frame is terug te vinden in dezelfde map, bestand \texttt{ais.html}.
  
  Je kan dit bestand importeren met behulp van volgende code:
  \begin{lstlisting}
  ais <- read.csv("ais.csv", sep = ",")
  attach(ais)
  \end{lstlisting}
  
  Beschouw de volgende deelverzamelingen uit dit data frame, 
  en bereken voor elk de geschikte centrum- en spreidingsmaten van de variabelen \texttt{sex} en \texttt{ht}.
  
  \begin{enumerate}
    \item de roeiers.
    \item de roeiers, de netballers en de tennissers samen.
    \item de vrouwelijke basketballers en roeiers.
  \end{enumerate}
\end{exercise}

\begin{exercise}
Gebruik de functies \texttt{mean} en \texttt{range} om het gemiddelde en bereik van:
\begin{itemize}
  \item de cijfers 1, 2, \dots, 21 
  \item 50 willekeurige normale waarden, die  worden gegenereerd vanuit een normale distributie met gemiddelde 0 en variantie 1 (functie \texttt{rnorm})
  \item de kolommen \texttt{height} en \texttt{weight} in de data frame \texttt{women} (standaard in R).
\end{itemize}
\end{exercise}

In vorige oefeningen hebben we de verschillende spreidingsmaten en centrummaten besproken. Zoals je merkt worden deze metrieken ook gebruikt in het onderzoek van~\textcite{Akin2016}. In de volgende oefeningen gaan we trachten de resultaten te reproduceren.

Hiervoor hebben we het bestand \texttt{android\_persistence\_cpu.csv} nodig. Je vindt deze eveneens terug in de folder \texttt{oefeningen/datasets}.

\begin{exercise}
  \label{oef:casus-akin2016-1var}
	Open de file met excel en bekijk de structuur van het document. Hoe ziet die er uit? Kan je de variabelen identificeren en hun type benoemen. 
\end{exercise}

We gaan het programma \texttt{R} gebruiken samen met \texttt{RStudio}. Open de file in \texttt{RStudio}.

\begin{lstlisting}
android_cpu <-  read.csv("android_persistence_cpu.csv", sep=";", dec=",")
attach(android_cpu)
\end{lstlisting}

We hebben nu de data ingeladen. We kunnen eens kijken wat de gemiddelde tijd, de standaarddeviatie, de kwartielen e.a. zijn. Gebruik hiervoor de commando's \texttt{mean}, \texttt{median}, \texttt{quantile}, \texttt{min}, \texttt{max}, \texttt{var}, \texttt{sd}. Je kan ook makkelijk gebruik maken van de methode \texttt{summary}.

\begin{exercise}
	Als je de vorige metrieken berekend hebt, wat kan je daar dan over zeggen. Kan je zinnige conclusies trekken uit de vorige resultaten. Zo ja vermeld ze, zo nee beschrijf waarom je dat denkt.
\end{exercise}

% TODO: Oefening over percentielen toevoegen, bv. gebaseerd op 95th percentile bandwith metering
% https://www.semaphore.com/95th-percentile-bandwidth-metering-explained-and-analyzed/
% http://aboutcolocation.info/95th-percentile-monitoring-explained/

\subsection{Grafieken in R}

% TODO: verwijderen uit de cursus, verplaatsen naar labos

Een histogram is een eenvoudige plot. het toont de frequenties van de data die in een bepaald bereik voorkomen. 

\begin{lstlisting}
hist(android_cpu$Tijd,main="Verdeling van de tijd",xlab="De gemeten cpu tijd");
hist(android_cpu$Tijd,main="Verdeling van de tijd",xlab="De gemeten cpu tijd",breaks=2);
\end{lstlisting}
\begin{exercise}
	Wat concludeer je als je bovenstaande grafiek\footnote{Heb je wat problemen met het genereren van grafieken, volgende link \url{https://www.datacamp.com/community/tutorials/15-questions-about-r-plots\#gs.RK_ORsI} bevat een aantal goede tips and tricks om je op weg te helpen.} genereert? Is dit een zinnig resultaat? Wat gebeurt er als je de variabele breaks verhoogt?
\end{exercise}

Een boxplot toont de mediaan, de kwartielen, het maximum en het minimum van een dataset. Dit geeft ons een duidelijk impressie van hoe de data er uitziet.

\begin{lstlisting}
boxplot(x = android_cpu$Tijd);
boxplot(android_cpu$Tijd,main='Spreiding van de CPU tijd',ylab='Tijd in ms');
\end{lstlisting} 

\begin{exercise}
	De boxplot wordt standaard verticaal getekend. Gebruik het commando \texttt{help(boxplot)} om uit te zoeken hoe we de tekening horizontaal krijgen. 
\end{exercise}

Als je goed geantwoord hebt op de volgende vragen merk je natuurlijk dat het weinig zin heeft de volledige dataset te analyseren, aangezien de dataset verdeeld is over verschillende categorie\"en. We willen dus wel deze statistieken weten, maar per categorie. We kunnen dus een boxplot maken voor elke categorie.

\begin{lstlisting}
boxplot(android_cpu$Tijd~android_cpu$Datahoeveelheid,main='Spreiding van de CPU tijd t.o.v. datahoeveelheid',ylab='Tijd in ms');
\end{lstlisting}

\begin{exercise}
	\label{ex:boxplot}
	Interpreteer de resultaten die je behaalt uit deze grafiek. Zijn deze al wat zinniger?
\end{exercise}

We kunnen hetzelfde doen voor de verschillende soorten dataopslagmogelijkheden in android.

\begin{exercise}
	Zelfde vraag als \ref{ex:boxplot} Interpreteer de resultaten die je behaalt uit deze grafiek. Zijn deze al wat zinniger?
\end{exercise}

We kunnen eens kijken hoe de data eruit ziet over alle categorie\"en heen.

\begin{lstlisting}
boxplot(android_cpu$Tijd~android_cpu$PersistentieType*android_cpu$Datahoeveelheid,main='Spreiding van de CPU tijd',ylab='Tijd in ms');
\end{lstlisting}

Het blijkt dat we wel al een duidelijker zicht krijgen over de data over de categorie\"en heen, maar de figuur is op dit moment te druk. 

We moeten de data dus onderverdelen in categorie\"en namelijk onder \texttt{PersistentieType} en \texttt{Datahoeveelheid}. We gaan hiervoor de functie \texttt{which}\footnote{Je kan ook gebruik maken van de functie \texttt{subset}, wat misschien zelfs eenvoudiger is} gebruiken en kijken hoe de verschillende datahoeveelheden verschillen per datahoeveelheidcategorie. 

\begin{lstlisting}
greenDOA <- android_cpu[which(android_cpu$PersistentieType=='GreenDAO'),];
boxplot(greenDOA$Tijd~greenDOA$Datahoeveelheid);
\end{lstlisting}

\begin{exercise}
  Wat concludeer je uit de vorige grafiek?
\end{exercise}

\begin{exercise}
  Ga nu zelf na welke boxplots er interessant zijn om te maken, en kijken of jouw resultaten overeen met die van \textcite{Akin2016}. Welke conclusies trek je1?
\end{exercise}

\begin{exercise}[Retrieval practice]
  Gebruik de procedure voor retrieval practice uit oefening~\ref{ex:retrieval-practice-meetniveaus} om de \emph{analyse- en visualisatietechnieken voor één variabele} in te studeren.
  
  Geef per meetniveau:
  
  \begin{itemize}
    \item De geschikte centrum- en spreidingsmaten (naam + definities en evt.~formules)
    \item Geschikte grafiektypes
  \end{itemize}
\end{exercise}

\subsection{Antwoorden op geselecteerde oefeningen}

\paragraph{Oefening \ref{ex:mean-stdev-freq}}

\begin{itemize}
  \item $\overline{x} = 7$
  \item $s^2 \approx 5.830508$
  \item $s \approx 2.414645$
\end{itemize}

\paragraph{Oefening \ref{ex:ais}}

Tabel \ref{tab:opl-ais-ht} geeft een overzicht met de belangrijkste centrum- en spreidingsmaten voor de variabele \texttt{ht} (height, lengte) over de gevraagde groepen.

\begin{table}
  \centering
  \begin{tabular}{@{}l|r|rrrr|rr@{}}
    \toprule
    & \textbf{(1)} & \multicolumn{4}{c}{\textbf{(2)}}                                                 & \multicolumn{2}{c}{\textbf{(3)}} \\ 
    & \textbf{Row} & \textbf{hele groep} & \textbf{Row} & \textbf{Netball} & \textbf{Tennis} & \textbf{B\_ball}  & \textbf{Row} \\ \midrule
    \textbf{gemiddelde} & 182.376      & 179.066                      & 182.376      & 176.087          & 174.164         & 182.269           & 178.859      \\
    \textbf{stdev}      & 7.798        & 7.936                        & 7.798        & 4.124            & 9.858           & 8.621             & 5.970        \\
    \textbf{min}        & 156.000      & 156.000                      & 156.000      & 168.600          & 157.900         & 169.100           & 156.000      \\
    \textbf{Q1}         & 179.300      & 174.200                      & 179.300      & 173.450          & 167.300         & 174.000           & 177.600      \\
    \textbf{mediaan}    & 181.800      & 179.500                      & 181.800      & 176.000          & 175.000         & 184.600           & 179.650      \\
    \textbf{Q3}         & 186.300      & 183.400                      & 186.300      & 179.150          & 180.750         & 188.700           & 181.200      \\
    \textbf{max}        & 198.000      & 198.000                      & 198.000      & 183.300          & 190.800         & 195.900           & 186.300      \\
    \textbf{IQR}        & 7.000        & 9.150                        & 7.000        & 5.700            & 13.450          & 14.700            & 3.600        \\ \bottomrule
  \end{tabular}
  \caption{Overzicht resultaten in oefening \ref{ex:ais} voor de variabele \texttt{ht} (height/lengte), met drie cijfers na de komma. In deeloefening 2 zijn de resultaten zowel gegeven voor de hele groep (roeiers, netballers én tennissers) als opgesplitst (via de functie \texttt{aggregate}).}
  \label{tab:opl-ais-ht}
\end{table}

In deeloefening 1 en 2 nemen we de variabele \texttt{sex} als voorbeeld. Zie tabel \ref{tab:opl-ais-sex} voor een overzicht. Over kwalitatieve variabelen valt minder te zeggen, we geven hier een frequentietabel waaruit we de modus kunnen afleiden.

In deeloefening 3 zijn enkel vrouwen geselecteerd, en voor deze oefening tonen we in tabel \ref{tab:opl-ais-sport} de frequenties van variabele \texttt{sport}.

\begin{table}
  \centering
  \begin{tabular}{@{}l|r|rrrr}
  	\toprule
  	               & \textbf{(1)} &                    \multicolumn{4}{c}{\textbf{(2)}}                     \\
  	               & \textbf{Row} & \textbf{hele groep} & \textbf{Row} & \textbf{Netball} & \textbf{Tennis} \\ \midrule
  	\textbf{f}     &           22 &                  52 &           22 &               23 &               7 \\
  	\textbf{m}     &           15 &                  19 &           15 &                0 &               4 \\
  	\textbf{modus} &            f &                   f &            f &                f &               f \\ \bottomrule
  \end{tabular}
  \caption{Overzicht resultaten in oefening \ref{ex:ais} (1) en (2) voor de variabele \texttt{sex}. Meer bepaald zijn hier de frequenties van de waarden opgegeven, en ook telkens de modus.}
  \label{tab:opl-ais-sex}
\end{table}

\begin{table}
  \centering
  \begin{tabular}{@{}l|l}
  	\toprule
  	                 & Frequenties \\ \midrule
  	\textbf{B\_ball} & 13          \\
  	\textbf{Row}     & 22          \\
  	\textbf{modus}   & Row         \\ \bottomrule
  \end{tabular}
  \caption{Overzicht resultaten in oefening \ref{ex:ais} (3) voor de variabele \texttt{sport}.}
  \label{tab:opl-ais-sport}
\end{table}
\include{4_centrale-limietstelling}
\include{5_toetsingsprocedures}
\chapter{Bivariate analyse}
\label{ch:analyse2var}

In de vorige hoofdstukken hebben we telkens één variabele tegelijkertijd onderzocht. In dit hoofdstuk bestuderen we \emph{verbanden}\index{verband} (ook: \emph{samenhang}\index{samenhang} of \emph{associatie}\index{associatie}) tussen variabelen. We spreken van een verband tussen twee variabelen wanneer de waarde van de ene variabele op een systematische manier verandert ten opzichte van de waarde van de andere. Anders gezegd, wanneer we tot op zekere hoogte een voorspelling kunnen doen over de waarde van een variabele aan de hand van de waarde van een andere.

\section{Leerdoelen}
\label{sec:analyse2var-leerdoelen}

Na dit hoofdstuk moet je in staat zijn om:

\begin{itemize}
  \item De volgende begrippen uit dit hoofdstuk uit te leggen:
  \begin{itemize}
    \item afhankelijke variabele, onafhankelijke variabele;
    \item een verband (samenhang, associatie) tussen twee variabelen, stijgend/dalend verband, lineair verband;
  \end{itemize}
  \item Voor de in dit hoofstuk besproken combinaties van meetniveaus geschikte analysetechnieken te benoemen;
  \item Voor een combinatie van twee kwalitatieve variabelen:
  \begin{itemize}
    \item Een kruistabel op te stellen en de regel van Cochran toe te passen;
    \item De $\chi^2$-statistiek te berekenen;
    \item De $\chi^2$-toets uit te voeren (voor associatie en goodness-of-fit);
    \item Gestandaardiseerde residuën te berekenen en te interpreteren;
    \item Cramér's V te berekenen en de waarde te interpreteren
  \end{itemize}
  \item Voor een combinatie van een kwalitatieve onafhankelijke en kwantitatieve afhankelijke variabele:
  \begin{itemize}
    \item De correcte variant van de $t$-toets voor twee steekproeven (gepaard, onafhankelijk) toe te passen;
    \item De effectgrootte (Cohen's $d$) te berekenen en de waarde te interpreteren;
  \end{itemize}
  \item Voor een combinatie van twee kwantitatieve variabelen:
  \begin{itemize}
    \item het functievoorschrift van de regressierechte te bepalen en die te tekenen;
    \item de covariantie, correlatiecoëfficiënt en determinatiecoëfficiënt te berekenen en de waarde ervan met de correcte verwoordingen te interpreteren;
  \end{itemize}
  \item Voor de in dit hoofstuk besproken combinaties van meetniveaus geschikte visualisatietechnieken te benoemen en toe te passen;
  \item Een gegeven grafiek met twee variabelen te interpreteren, o.a.~het grafiektype benoemen, afleiden of er een verband bestaat, welk soort verband en de mate van associatie (zwak, matig, sterk).
  \item Aan de hand van een situatie kunnen uitleggen dat een verband niet noodzakelijk een noodzakelijk verband inhoudt, en waarom;
\end{itemize}

\section{Inleiding}
\label{sec:analyse2var-inleiding}

Wanneer we een verband beschrijven tussen variabelen, onderscheiden we:

\begin{itemize}
  \item De \emph{afhankelijke variabele}\index{variabele!afhankelijke}, waarover we een voorspelling willen doen;
  \item De \emph{onafhankelijke variabele}\index{variabele!onafhankelijke}, op basis van dewelke we de voorspelling doen.
\end{itemize}

Dus als de onafhankelijke variabele op een bepaalde manier verandert, verwachten we dat de waarde van de afhankelijke variabele op een voorspelbare manier mee verandert.

\begin{example}
  Een voorbeeld waarbij verbanden kunnen gevonden worden tussen variabelen vind je bij Ant Colony Optimization (ACO). Dit is een techniek die gebruikt wordt in verschillende computationele problemen. Men baseert zich hier op hoe mieren voedsel zoeken en  vinden en dat communiceren aan de groep. Mieren verspreiden feromonen als ze op pad gaan op zoek naar eten. Hoe langer het pad, hoe minder feromonen het pad zal bevatten, hoe korter het pad, hoe groter de kans dat er een grote concentratie aan feromonen te vinden is. Mieren worden aangetrokken door deze feromonen en zullen dus proberen de meest bewandelde paden te gebruiken om naar een bepaalde voedselbron te gaan. Nu kan je onderzoeken of de tijd voor het vinden van een pad, afhangt van een aantal variabelen:

  \begin{itemize}
    \item De mate waarin feromonen verspreid worden
    \item De mate waarin een feromoon verdwijnt
    \item Het aantal obstakels tussen het nest en de voedselbron
    \item De vorm van de obstakels tussen nest en voedselbron (vinden ze sneller het pad indien er geen hoeken aan de obstakels zijn bv.)
  \end{itemize}
\end{example}

Voor het onderzoeken of er een verband bestaat tussen twee variabelen bestaan er verschillende analyse- en visualisatietechnieken die afhangen van het meetniveau. Tabel~\ref{tab:analyse2var} geeft een overzicht van geschikte analysetechnieken die in deze cursus besproken worden en Tabel~\ref{tab:visualisatie2var} van geschikte grafiektypes.

Merk trouwens op dat er nog veel meer analysetechnieken bestaan die niet in deze cursus besproken worden! Hier vind je enkel een topje van de ijsberg\ldots Als je in je latere carrière nood hebt aan statistische technieken (en de auteurs van deze cursus zijn optimistisch dat dit ook effectief ooit eens zal gebeuren), kijk je best na of er betere technieken bestaan voor het beantwoorden van jouw specifieke onderzoeksvraag.

\begin{table}
  \begin{tabular}{llll}
    \toprule
    \textbf{Onafhankelijke} & \textbf{Afhankelijke} & \textbf{Toets}                & \textbf{Metriek}      \\
    \midrule
    Kwalitatief             & Kwalitatief           & $\chi^2$-toets                & Cramér's $V$          \\
    Kwalitatief             & Kwantitatief          & $t$-toets voor 2 steekproeven & Cohen's $d$           \\
    Kwantitatief            & Kwantitatief          & ---                           & Regressie, correlatie \\
    \bottomrule
  \end{tabular}
  \caption{Overzicht van analysetechnieken voor het verband tussen twee variabelen. De tabel geeft telkens enerzijds geschikte statistische toetsen en anderzijds metrieken die de mate van het verband uitdrukken.}
  \label{tab:analyse2var}
\end{table}

\begin{table}
  \begin{tabular}{lll}
  	\toprule
  	\textbf{Onafhankelijke} & \textbf{Afhankelijke} & \textbf{Grafiektype}                                  \\
  	\midrule
  	Kwalitatief             & Kwalitatief           & Mozaïekdiagram, rependiagram, geclusterd staafdiagram \\
  	Kwalitatief             & Kwantitatief          & Boxplot, staafdiagram met error bars                  \\
  	Kwantitatief            & Kwantitatief          & Spreidingsdiagram, regressierechte                    \\
  	\bottomrule
  \end{tabular}
  \caption{Overzicht van visualisatietechnieken voor het verband tussen twee variabelen.}
  \label{tab:visualisatie2var}
\end{table}

\section{Kwalitatief--kwalitatief}
\label{sec:kwalitatief-kwalitatief}

In deze sectie bekijken we methoden om na te gaan of er een verband bestaat tussen twee kwalitatieve variabelen. Al deze methoden zijn gebaseerd op een samenvattende tabel van frequenties van de twee variabelen, de kruistabel (zie Sectie~\ref{ssec:kruistabellen}). Daaruit wordt een statistiek berekend die uitdrukt hoe groot de verschillen zijn tussen de twee variabelen, de zogenaamde ``Chi-kwadraat'', notatie $\chi^2$\footnote{Chi, $\chi$ is een letter uit het Griekse alfabet.} (zie Sectie~\ref{ssec:chi-kwadraat}). Met de $\chi^2$-toets (zie Sectie~\ref{ssec:onafhankelijkheidstoets} en~\ref{ssec:aanpassingstoets}) kan je nagaan of de waarde van $\chi^2$ al dan niet aangeeft of er een verband bestaat tussen twee variabelen. Er bestaat ook een metriek, Cramér's $V$ (zie Sectie~\ref{ssec:cramers-v}), die de waarde van $\chi^2$ herleidt tot een getal tussen 0 en 1 waaruit de sterkte van het verband valt af te leiden.

\subsection{Kruistabellen}
\label{ssec:kruistabellen}

\begin{definition}[Kruistabel]
  In een kruistabel\index{kruistabel} (Eng.: \emph{contingency table} of \emph{cross table}; zie bv.~Figuur~\ref{tab:kruistabel0}) worden de frequenties van twee variabelen samengevat.
  
  Elke cel van de laatste kolom bevat de som van de overeenkomstige rij en elke cel van de laatste rij bevat de som van de overeenkomstige kolom. Dit worden de \emph{marginale totalen}\index{totaal!marginaal}\index{marginaal totaal} genoemd.
\end{definition}

\begin{table} \centering
  \begin{tabular}{@{}rrrr}
    \toprule
                & Vrouw & Man & Totaal \\
    \midrule
           Goed &     9 &   8 &     17 \\
      Voldoende &     8 &  10 &     18 \\
    Onvoldoende &     5 &   5 &     10 \\
         Slecht &     0 &   4 &      4 \\
         Totaal &    22 &  27 &     49 \\
    \bottomrule
  \end{tabular}
  \caption{Een kruistabel voor de waardering door mannen en vrouwen van een bepaald assortiment producten.}
  \label{tab:kruistabel0}
\end{table}

Hoe kunnen we nu zien in een kruistabel of er een verband bestaat tussen twee variabelen? Bekijk de kruistabel in Tabel~\ref{tab:kruistabel0}. Die geeft de resultaten weer van een bevraging waar 49 mensen (22 vrouwen en 27 mannen) gevraagd werd om een waardering te geven (goed, voldoende, onvoldoende of slecht). Als er een verband bestaat tussen de twee variabelen, gender (onafhankelijk) en waardering (afhankelijk), dan moeten we zien dat vrouwen en mannen fundamenteel andere waarderingen gegeven hebben. Omgekeerd, als zowel bij de vrouwen als bij de mannen de verhoudingen tussen de verschillende waarderingen (ongeveer) gelijk zijn, dan is er \emph{geen} verband.

In een gewone kruistabel kunnen we geen directe conclusies trekken, aangezien het analyseren of er samenhang bestaat tussen variabelen niet goed gaat op basis van de absolute frequenties. Er zijn immers meer mannen dan vrouwen in de bevraging! Daarom moeten we percenteren, d.w.z.~binnen elke kolom het percentage berekenen hoe vaak elke beoordeling \emph{naar verhouding} voorkomt. De som van alle percentages in een kolom moet dan gelijk zijn aan 100\%. Nog even snel de regel van percenteren:

\begin{itemize}
  \item Om te weten hoeveel procent $x$ is van $y$, deel je $x$ door $y$ en vermenigvuldig je met 100: $p = \frac{x}{y} \times 100$. Bijvoorbeeld: hoeveel procent is 15 van 20? $\frac{15}{20} \times 100 = 75$, dus 75\%.
  \item Om te weten hoeveel $x\%$ is van $y$: $\frac{x \times y}{100}$. Hoeveel is 60\% van 30? $\frac{60 \times 30}{100} = 18$.
\end{itemize}

In Tabel~\ref{tab:kruistabel1} vind je per geslacht de percentages dat elke waardering is voorgekomen in de steekproef. We zien dan bijvoorbeeld dat 41\% van de vrouwen een waardering goed heeft gegeven (want 9 is ongeveer 41\% van 22), tegen dat 30\% van de mannen.

\begin{table} \centering
  \begin{tabular}{@{}rrrrrrr@{}}
  	\toprule
  	            & Vrouw & Man & Totaal & Vrouw \% & Man\% & Totaal \\
  	\midrule
  	       Goed &     9 &   8 &     17 &     41\% &  30\% &   35\% \\
  	  Voldoende &     8 &  10 &     18 &     36\% &  37\% &   37\% \\
  	Onvoldoende &     5 &   5 &     10 &     23\% &  18\% &   20\% \\
  	     Slecht &     0 &   4 &      4 &      0\% &  15\% &    8\% \\
  	     Totaal &    22 &  27 &     49 &    100\% & 100\% &  100\% \\
  	\bottomrule
  \end{tabular}
  \caption{De kruistabel waarbij we de waarden gepercenteerd hebben.}
  \label{tab:kruistabel1}
\end{table}

Nu kunnen we ons de vraag stellen of de waarderingskeuze afhangt van het gender van de respondent. Er zijn verschillen tussen beide kolommen, dus je kan een vermoeden hebben dat dit inderdaad het geval is. Hoe kleiner de verschillen, hoe minder verband er is tussen beide variabelen, hoe groter de verschillen, hoe groter de samenhang.

\subsection{\texorpdfstring{$\chi^{2}$}{chi-kwadraat}}
\label{ssec:chi-kwadraat}

Uit de vorige sectie kunnen we besluiten dat we nood hebben aan een maat die uitdrukt hoe groot het verschil is tussen de verhoudingen in beide kolommen. Een manier om dit uit te drukken wordt de \emph{chi-kwadraat}-statistiek (notatie $\chi^{2}$) genoemd. $\chi^2$ is gelijk aan 0 als er geen verschil is tussen de verhoudingen in de kolommen van een kruistabel, en als er dus ook volstrekt geen samenhang tussen de variabelen is. Als er wel een verschil is, dan is $\chi^2$ een positief getal. Hoe groter de waarde, hoe groter ook de onderlinge verschillen tussen de kolommen en hoe groter dus ook de samenhang.

De procedure om de $\chi^2$ van een kruistabel te berekenen gaat als volgt:

\begin{enumerate}
  \item Stel eerst de kruistabel op samen met marginale totalen (zie tabel \ref{tab:kruistabel0}).
  \item Vervolgens berekenen we voor elke cel de zogenaamde \emph{verwachte frequentie} (notatie $e$ van \emph{expected}). Dat is de absolute frequentie die je in deze cel zou \emph{verwachten} als je veronderstelt dat er helemaal geen samenhang is tussen de variabelen. Deze kan je bereken als volgt:
  
  \begin{equation}
  e = \frac{rijtotaal \times kolomtotaal}{n}
  \end{equation}
  
  met:
  
  \begin{itemize}
    \item $rijtotaal$ totaal van de rij van de betreffende cel
    \item $kolomtotaal$ totaal van de kolom van de betreffende cel
    \item $n$ het aantal observaties
  \end{itemize}
  
  Voor cel$_{1,2}$ (het verwachte aantal mannen dat waardering ``goed'' gegeven heeft) is dit dus $\frac{17 \times 27}{49} \approx 9.37\%$.
  
  \item Dan bereken je het verschil tussen geobserveerde (notatie $o$, \emph{observed}) en verwachte frequentie ($e$). (Zie tabel~\ref{tab:kruistabel2}).
  
  \begin{table} \centering
    \begin{tabular}{@{}rrrrrrr@{}}
    	\toprule
    	            &                       Vrouw &                          Man & Totaal & Vrouw \% &   Man\% &  Totaal \\
    	\midrule
    	       Goed &  $9 -\textcolor{red}{7.63}$ &  $8 - \textcolor{red}{9.36}$ &   $17$ &   $41$\% &  $30$\% &  $35$\% \\
    	  Voldoende & $8 - \textcolor{red}{8.08}$ & $10 - \textcolor{red}{9.91}$ &   $18$ &   $36$\% &  $37$\% &  $37$\% \\
    	Onvoldoende & $5 - \textcolor{red}{4.48}$ &  $5 - \textcolor{red}{5.51}$ &   $10$ &   $23$\% &  $18$\% &  $20$\% \\
    	     Slecht & $0 - \textcolor{red}{1.79}$ &  $4 - \textcolor{red}{2.20}$ &    $4$ &    $0$\% &  $15$\% &   $8$\% \\
    	     Totaal &                        $22$ &                         $27$ &   $49$ &  $100$\% & $100$\% & $100$\% \\
    	\bottomrule
    \end{tabular}
    \caption{De kruistabel waarbij we de verwachte frequentie $e$ (aangeduid in het rood) bepaald hebben voor elke cel en die aftrekken van de geobserveerde frequentie $o$.}
    \label{tab:kruistabel2}
  \end{table}
  
  \item De laatste stap houdt in dat we een berekening gaan maken voor de maat van afwijking voor elke cel. Net zoals bij het berekenen van de variantie van een steekproef (zie Sectie~\ref{sec:varEnSD}) kwadrateren we het verschil zodat het resultaat altijd positief is en zodat grotere afwijkingen zwaarder door zullen wegen in het resultaat.
  
  We gaan ook de afwijking delen door de verwachte frequentie om hen relatief even belangrijk te maken. Bijvoorbeeld: een afwijking van 5 op een verwachte frequentie van 20 is groter dan bv. een afwijking op een verwachte frequentie van 200. Dit geeft dan voor elke cel van de kruistabel de volgende berekening (zie Tabel~\ref{tab:kruistabel3}):
  
  \begin{equation}
  \frac{(o - e)^{2}}{e}
  \end{equation}
  
  \begin{table} \centering
    \begin{tabular}{@{}rrrrrrr@{}}
    	\toprule
    	            &                    Vrouw &                      Man & Totaal & Vrouw \% &   Man\% &  Totaal \\
    	\midrule
    	       Goed & $\textcolor{blue}{0.24}$ & $\textcolor{blue}{0.20}$ &   $17$ &   $41$\% &  $30$\% &  $35$\% \\
    	  Voldoende & $\textcolor{blue}{0.00}$ & $\textcolor{blue}{0.00}$ &   $18$ &   $36$\% &  $37$\% &  $37$\% \\
    	Onvoldoende & $\textcolor{blue}{0.06}$ & $\textcolor{blue}{0.05}$ &   $10$ &   $23$\% &  $18$\% &  $20$\% \\
    	     Slecht & $\textcolor{blue}{1.80}$ & $\textcolor{blue}{1.46}$ &    $4$ &    $0$\% &  $15$\% &   $8$\% \\
    	     Totaal &                     $22$ &                     $27$ &   $49$ &  $100$\% & $100$\% & $100$\% \\
    	\bottomrule
    \end{tabular}
    \caption{De kruistabel waarbij we het verschil gekwadrateerd hebben en gedeeld door de verwachte frequentie, $\frac{(o - e)^2}{e}$ (in het blauw). Merk op dat deze waarden afgerond zijn tot twee cijfers na de komma, en we hier dus een deel van de precisie kwijt spelen.}
    \label{tab:kruistabel3}
  \end{table}
  
  \item Alle resultaten gaan we tenslotte optellen om zo de $\chi^{2}$ te bekomen:
  
  \begin{equation}
  \chi^{2} = \sum \frac{(o - e)^{2}}{e}
  \end{equation}
  
  In het voorbeeld is $\chi^2 \approx 3,8105$.
\end{enumerate}

Nu zegt deze waarde $\chi^2 \approx 3,8105$ op zich nog \emph{steeds} niet zo veel. Onder welke voorwaarden zeggen we dat er al dan niet een verband is tussen beide variabelen? En hoe sterk is dat verband dan? Een en ander zal ook afhangen van de grootte van de tabel en het totaal aantal observaties. In een kruistabel met meer rijen/kolommen, zal je een grotere $\chi^2$ moeten hebben om te besluiten dat er een verband is.

\subsection[Chi-kwadraatverdeling]{$\chi^{2}$ verdeling}
\label{ssec:chi-kwadraatverdeling}

In Sectie~\ref{ssec:onafhankelijkheidstoets} wordt de $\chi^2$-toets gedefinieerd. Deze toets maakt bij het bepalen van de overschrijdingskans of kritieke grenswaarde gebruik van de zgn.~$\chi^2$-verdeling. Deze stochastische verdeling komt niet in de natuur voor en geen verschijnsel kan erdoor gemodelleerd worden. In \emph{deze} context is ze echter \emph{wel} zeer nuttig.

Laat $X_{1}, X_{2}, \dots X_{v}$ onafhankelijk standaardnormale variabelen zijn ($\sim N(0,1)$). De $\chi^{2}$ (chi-kwadraat) variabele wordt als volgt gedefinieerd:

\[ \chi^{2}_{v} = X_{1}^{2} + X_{2}^{2} + \dots + X_{v}^{2} \]

Het getal $v$ noemt men het aantal vrijheidsgraden van de variabele. $\chi^{2}$ is een continue toevalsveranderlijke, die positief is omdat ze de som is van kwadraten. Haar dichtheidsfunctie is de volgende:

\[ f_{n}(x) = \frac{1}{2^{\frac{n}{2}}\Gamma(\frac{n}{2})} x^{\frac{n}{2} -1} e^{\frac{x}{2}} \]

De verwachtingswaarde (= gemiddelde) is $v$ en zijn variantie is $2v$. Zijn modus voor $v \geq 2$ is $v-2$.

Voor het bepalen van de nodige waarden binnen de $\chi^2$-verdeling (bv. rechterstaartkans, kritieke grenswaarde), kunnen we gebruik maken van hetzij een tabel\footnote{Bijvoorbeeld \url{https://people.richland.edu/james/lecture/m170/tbl-chi.html}}, hetzij statistische software zoals R.

\begin{figure}
  \includegraphics[width=\textwidth]{chi-squared-distribution}
  \caption{Dichtheidsfunctie van de $\chi^2$-verdeling voor verschillende vrijheidsgraden $df$.}
  \label{fig:chi-squared-distribution}
\end{figure}

\subsection{Onafhankelijkheidstoets}
\label{ssec:onafhankelijkheidstoets}

Een onafhankelijkheidstoets\index{onafhankelijkheidstoets} gaat na of er een verband bestaat tussen twee kwalitatieve variabelen aan de hand van de $\chi^2$-statistiek. Deze toets wordt ook de $\chi^2$-toets\index{$\chi^2$-toets!}\index{toets!$\chi^2$-} voor associatie genoemd, of de $\chi^2$-kruistabeltoets, en is ontwikkeld door de Engelse wiskundige en statisticus Karl Pearson\index{Pearson!Karl}.

\subsubsection{Toetsingsprocedure}

De toetsingsprocedure verloopt als volgt:

\begin{enumerate}
  \item \textbf{Bepalen hypotheses}
  Als nulhypothese stellen we dat er geen verband is tussen de onafhankelijke en de afhankelijke variabele (en verwachten we dus een kleine $\chi^2$). Als alternatieve hypothese stellen we dat er \emph{wel} een verband is (en dat $\chi^2$ dus groot is).
  \begin{itemize}
    \item $H_{0}$: er is geen verband tussen de variabelen (of: de variabelen zijn onafhankelijk)
    \item $H_{1}$: er is een verband tussen de variabelen
  \end{itemize}

  \item \textbf{Bepalen significantieniveau $\alpha$}
  
  \item \textbf{Bereken de waarde van de toetsingsgrootheid in de steekproef}
  \[ \chi^{2} = \sum_{i} \frac{(o_{i} - e_{i})^{2}}{e_{i}} \]
  
  \item De waarde van $\chi^2$ is gedistribueerd volgens de $\chi^2$-verdeling (zie Sectie~\ref{ssec:chi-kwadraatverdeling}). Deze stochastische verdeling heeft als bijkomende parameter het aantal vrijheidsgraden $df$. Voor een kruistabeltoets is $df = (r - 1) \times (k - 1)$ met $r$ het aantal rijen en $k$ het aantal kolommen.
  
  Aan de hand van deze verdeling kunnen we op twee equivalente manieren een besluit trekken:
  
  \begin{enumerate}
    \item \textbf{Bereken en teken kritiek gebied}. De kritieke grenswaarde $g$ is het getal waarvoor geldt dat $P(\chi^2 > g) = \alpha$. Deze toets is altijd rechtszijdig. Als $\chi^2 < g$, dan zitten we in het \emph{aanvaardingsgebied}, waar we de nulhypothese niet kunnen verwerpen. Als $\chi^2 > g$, dan zitten we in het \emph{verwerpingsgebied} en zullen we de nulhypothese verwerpen.
    \item \textbf{Bereken de overschrijdingskans $p$}. Dit is de rechterstaartkans voor de bekomen $\chi^2$-statistiek in de steekproef, afhankelijk van het aantal vrijheidsgraden. De interpretatie van de $p$-waarde is ``Als we zouden veronderstellen dat er \emph{geen} verband is tussen de twee variabelen, wat is dan de kans dat ik een waarde voor $\chi^2$ tegenkom in een steekproef die minstens zo groot is als de daarnet berekende?'' Als $p > \alpha$ (en dus de kans relatief groot is dat we deze $ \chi^2$-waarde tegenkomen), dan aanvaarden we de nulhypothese, als $p < \alpha$, dan verwerpen we deze.
  \end{enumerate}

  \item Tenslotte formuleren we de conclusie en beantwoorden we de onderzoeksvraag.
\end{enumerate}

Toegepast op het voorbeeld van hierboven:

\begin{enumerate}
  \item \textbf{Bepalen hypotheses:}
  \begin{itemize}
    \item $H_0$ Er is geen verband tussen gender en waardering
    \item $H_1$ Er is een verband tussen gender en waardering
  \end{itemize}
  \item \textbf{Bepalen significantieniveau:} $\alpha = 0,05$
  
  \item \textbf{Bereken de waarde van de toetsingsgrootheid in de steekproef}
  
  $\chi^2 \approx 3,8105$ (zie Sectie~\ref{ssec:chi-kwadraat})
  
  \item Bepaal het aantal vrijheidsgraden $df = (r - 1) \times (k - 1) = (4 - 1) \times (2 - 1) = 3$.
  
  \begin{enumerate}
    \item \textbf{Bepaal de kritieke grenswaarde:} $g \approx 7,815$. In dit geval is $\chi^2 < g$, dus we zitten in het aanvaardingsgebied. We kunnen $H_0$ dus niet verwerpen.
    \item \textbf{Bereken de overschrijdingskans:} $p \approx 0,2827$. Er is dus een kans van ongeveer 28\% dat we deze $\chi^2$-waarde tegenkomen. Dat is vrij groot, zeker al groter dan $\alpha$. Bijgevolg kunnen we $H_0$ niet verwerpen.
  \end{enumerate}

  \item We kunnen dus besluiten dat er op basis van deze steekproef geen reden is om aan te nemen dat er een verband is tussen gender en waardering. Anders gezegd, er zijn geen significante verschillen tussen de waarderingen bij enerzijds vrouwen en anderzijds mannen.
\end{enumerate}


We geven nog een ander voorbeeld van de onafhankelijkheidstoets aan de hand van een studie door \textcite{Doll1954} over de relatie tussen roken en longkanker. Doll en Hill schreven in 1951 alle Britse huisartsen aan met het verzoek om gegevens over hun leeftijd en rookgedrag. Vervolgens hielden ze jarenlang de overlijdensberichten en de doodsoorzaak bij en herhaalden hun periodiek. De eerste uitkomsten, na circa vier jaar, zijn in tabel~\ref{tab:dollhill} samengevat. Uit de tabel kan makkelijk geconcludeerd worden dat er geen relatie is tussen roken en longkanker. In (ruim) vier jaar is slechts $(84 / 24354) * 100 = 0,35\%$ van de Britse artsen aan longkanker overleden en dat met slechts $(83 / 21261) * 100 = 0,39\%$ van de rokers onder hen. Dit is weinig, maar het is wel veel meer dan hetzelfde cijfer voor de niet-rokers $(1 / 3093) * 100 = 0,032\%$.

\begin{table}
  \begin{center}
    \begin{tabular}{@{}lllll@{}}
    	\toprule
    	               & \textbf{Longkanker} & \textbf{Niet} & \textbf{Wel} & \textbf{Totaal} \\
    	\midrule
    	\textbf{Roker} & \textbf{Wel}        & 21178         & 83           & 21261           \\
    	               & \textbf{Niet}       & 3092          & 1            & 3093            \\
    	               & \textbf{Totaal}     & 24270         & 84           & 24354           \\
    	\bottomrule
    \end{tabular}
  \end{center}
  \caption{Resultaten van het onderzoek van~\textcite{Doll1954}}
  \label{tab:dollhill}
\end{table}

We zien in de tabel dat er wel een erg groot verschil is tussen de geobserveerde aantallen rokers die overlijden aan longkanker en de verwachte frequenties in deze cel. Hetzelfde geldt voor het geringe aantal huisartsen dat niet rookt, maar wel aan longkanker overleden is. Deze observatie maakt ons wel wantrouwig of de eerdere tentatieve conclusie wel juist is. We kunnen afrekenen met deze onzekerheid door de toetsingsgrootheid $\chi^{2}$ uit te rekenen. Dat doen we op de vertrouwde manier:

\begin{enumerate}
  \item \textbf{Bepalen hypotheses}
  \begin{itemize}
    \item $H_{0}$: er is geen verband tussen roken en sterven aan longkanker
    \item $H_{1}$: er is een verband tussen roken en sterven aan longkanker
  \end{itemize}
  \item \textbf{Bepalen $\alpha$ en $n$:} $\alpha = 0,05$ en $n = 24354$.
  \item \textbf{Toetsingsgrootheid en waarde ervan in steekproef}:
  \[ \chi^{2} = \sum_{i=1}^{k \times r} \frac{(o_{i} - e_{i})^{2}}{e_{i}} \approx 10,071 \]
  \item \textbf{Bereken en teken kritiek gebied}: Het aantal vrijheidsgraden is $df = (r-1)(k-1) = 1$, dus voor het gegeven significantieniveau is de kritieke grens 3,8415. Onze toetsingsgrootheid ligt in het kritieke gebied dus verwerpen we $H_{0}$.
\end{enumerate}

We moeten derhalve $H_{0}$, dat er geen relatie is tussen beide variabelen, verwerpen ten gunste van $H_{1}$ dat er wel een relatie is tussen beide variabelen: rokers sterven vaker aan longkanker dan niet-rokers.

Maar, is dit nu een bewijs dat zoals zo vaak verondersteld wordt dat roken longkanker \emph{veroorzaakt}? Nee, dat is het absoluut niet. Een paar alternatieve verklaringen: niet alle rokers krijgen longkanker, de rokers zijn ouder dan de niet-rokers, de rokers wonen veelal in de grote steden met meer vervuilde lucht dan de niet-rokers die veelal op het platte land wonen, ook zou er nog een speciale genetische dispositie kunnen zijn, die zowel van invloed is op de verslaving aan tabak, als op de kans om longkanker te krijgen. Voor een causale interpretatie van de gegevens (let wel, het betreft hier immers geen experiment), moeten we op zijn minst de beschikking hebben over een theorie die de relatie tussen roken en longkanker expliciteert.

\begin{remark}[!!]
  \textbf{Correlation is not causation.} Of anders gezegd, een verband tussen twee variabelen impliceert niet dat er ook een \emph{oorzakelijk} verband is. Dit is een erg vaak voorkomende fout die gemaakt wordt wanneer journalisten een artikel schrijven over resultaten van wetenschappelijk onderzoek. Wees hier dus alert voor wanneer je dergelijke artikels leest!
\end{remark}

\subsubsection{De regel van Cochran}

Bij het berekenen van de $\chi^2$-waarde is het belangrijk dat er voor elke cel in de kruistabel voldoende observaties beschikbaar zijn. Als de steekproef te klein is, zullen de resultaten van de berekening onbetrouwbaar worden.

De statisticus \textcite{Cochran1954}\index{Cochran!William G.} heeft hierover een aantal aanbevelingen geformuleerd die we in deze cursus de Regel van Cochran\index{Cochran!regel van} zullen noemen.

Om de $\chi^2$-toets te mogen toepassen moeten specifiek de volgende voorwaarden vervuld zijn:

\begin{enumerate}
  \item Voor alle categorie\"en moet gelden dat de verwachte frequentie $e$ groter is dan 1.
  \item In ten hoogste 20 \% van de categorie\"en mag de verwachte frequentie $e$ kleiner dan 5 zijn.
\end{enumerate}

\subsection{Aanpassingstoets}
\label{ssec:aanpassingstoets}

De $\chi^2$-toets kan ook worden toegepast wanneer je wil nagaan of een bepaalde discrete verdeling (bv. de frequenties van een enkele kwalitatieve variabele) al dan niet overeenkomen met een gekende verdeling. Deze variant noemen we de \emph{aanpassingstoets}\index{aanpassingstoets} of in het Engels de \emph{goodness-of-fit test}\index{goodness-of-fit test}. Deze toets wordt vaak toegepast om na te gaan of de verdeling van een bepaalde kwalitatieve variabele in een steekproef representatief is voor de populatie, in de veronderstelling dat je weet welke frequenties er in de populatie als geheel voorkomen.

Stel, in het onderzoek naar onze superhelden is een steekproef genomen van $n = 400$ observaties. De onderzoekers willen weten of de voorkomende types van superhelden in de steekproef overeenkomen met die in de populatie, m.a.w.~of de steekproef representatief is. Tabel~\ref{tab:frequenties-types-superhelden} geeft een overzicht van de geobserveerde frequenties in de steekproef en de verwachte frequenties in de populatie.

\begin{table}
  \centering
  \begin{tabular}{@{}lcc@{}}
  	\toprule
  	\textbf{Type}   & \textbf{frq steekproef ($o$)} & \textbf{frq populatie ($\pi$)} \\
  	\midrule
  	Mutant          &              127              &              35\%              \\
  	Mens            &              75               &              17\%              \\
  	Alien           &              98               &              23\%              \\
  	God             &              27               &              8\%               \\
  	Demon           &              73               &              17\%              \\
  	\midrule
  	\textbf{Totaal} &              400              &             100\%
  \end{tabular}
  \caption{Absolute frequenties van de types superhelden in de steekproef ($o$) en verwachte relatieve frequenties ($\pi$) in de populatie als geheel.}
  \label{tab:frequenties-types-superhelden}
\end{table}

We willen de frequenties in de steekproef vergelijken met de aantallen die je zou verwachten als de steekproef exact representatief zou zijn naar de types van superhelden. Als deze verschillen relatief groot zijn dan komt de verdeling in de steekproef \emph{niet} overeen met de verdeling in de populaties en zullen we moeten concluderen dat de steekproef niet representatief is. Om te oordelen of deze verschillen relatief groot zijn, voeren we een $\chi^{2}$-toets uit.

Als de steekproef exact representatief is, dan zouden we verwachten dat in de steekproef 35\% van de superhelden een mutant is. Het verwachte aantal of de verwachte frequentie voor deze categorie is dus gelijk aan $0,35 \times 400 = 140$. Er geldt dus:

\[ e = n \times \pi \]

met $e$ de verwachte absolute frequentie in de steekproef, $n$ de steekproefgrootte en $\pi$ de verwachte relatieve frequentie voor de hele populatie. Als de verschillen tussen de geobserveerde en verwachte frequenties $(o - e)$ relatief klein zijn, kunnen ze toegerekend worden aan toevallige steekproeffouten. We kunnen opnieuw $\chi^2$ gebruiken om deze verschillen samen te vatten en te interpreteren:

\[ \chi^{2} = \sum_i \frac{(o_{i} - e_{i})^{2}}{e_{i}} \]

We merken op:

\begin{itemize}
  \item indien de verschillen klein zijn $\Rightarrow$ verdeling is representatief
  \item indien de verschillen groot $\Rightarrow$ verdeling niet representatief
\end{itemize}

We bepalen nu een kritieke grenswaarde $g$ die een $\chi^{2}$ verdeling heeft. Hierbij speelt het aantal vrijheidsgraden ($df$) een rol. Voor deze toets geldt:

\[ df = k - 1 \]

met $k$ het aantal categorie\"en. In ons voorbeeld hebben we $df = 5-1 = 4$. Om de kritieke grenswaarde te bepalen, kan je gebruik maken van een tabel voor de $\chi^2$-verdeling. Voor een gegeven significantieniveau $\alpha$ en vrijheidsgraad $df$ kan je in zo'n tabel de grenswaarde aflezen.

In ons voorbeeld is $\chi^{2} = 3,47$ met grenswaarde $g = 9,49$. Omdat de gevonden toetsingsgrootheid $\chi^2 = 3,47 < g = 9,49$, mogen we besluiten dat de steekproef representatief is.

\subsubsection{Toetsingsprocedure}

We volgen de stappen van een statistische toetsingsprocedure:

\begin{enumerate}
  \item \textbf{Bepalen hypotheses}
  Als nulhypothese formuleren we dat de steekproef representatief is, meer bepaald dat de verdeling in de steekproef gelijk is aan de verdeling over de populatie. Als alternatieve hypothese formuleren we dat de verdelingen verschillend zijn.
  \begin{itemize}
    \item $H_{0}$: steekproef is representatief voor de populatie
    \item $H_{1}$: steekproef is niet representatief voor de populatie
  \end{itemize}
  \item \textbf{Bepalen $\alpha$ en $n$}
  \item \textbf{Waarde van de toetsingsgrootheid in de steekproef}:
  \[ \chi^{2} = \sum_{i=1}^{n} \frac{(o_{i} - e_{i})^{2}}{e_{i}} \]
  \item Bepaal het aantal vrijheidsgraden $df = k - 1$ met $k$ het aantal categorieën
  \begin{enumerate}
    \item \textbf{Bereken en teken kritiek gebied}: de toets is altijd rechtszijdig. Is de toetsingsgrootheid kleiner dan kritieke grenswaarde $\chi^2 < g$, verwerp $H_{0}$ niet. Als $\chi^2 > g$, verwerp $H_{0}$ en aanvaard $H_{1}$.
    \item \textbf{Bereken de overschrijdingskans $p$}. Als $p > \alpha$, dan aanvaarden we de nulhypothese, als $p < \alpha$, dan verwerpen we deze.
  \end{enumerate}
  \item Formuleer tenslotte het antwoord op de onderzoeksvraag.
\end{enumerate}

\subsubsection{Gestandaardiseerde residuen}

We bespreken nog een ander voorbeeld. Beschouw alle gezinnen met precies 5 kinderen in een bepaalde gemeenschap. Wat betreft het aantal jongens/meisjes in zo'n gezin zijn er 6 mogelijkheden:

\begin{enumerate}
  \item 5 jongens
  \item 4 jongens, 1 meisje
  \item 3 jongens, 2 meisjes
  \item 2 jongens, 3 meisjes
  \item 1 jongen, 4 meisjes
  \item 5 meisjes
\end{enumerate}

Stel dat er een onderzoek gebeurd is waar 1022 gezinnen met 5 kinderen hebben aan deelgenomen. In Tabel~\ref{tab:5-kinderen} worden de frequenties gegeven van het aantal jongens in elk gezin. Zijn de waargenomen aantallen in de 6 klassen representatief voor een populatie waar de kans om een jongen te krijgen gelijk is aan de kans om een meisje te krijgen, nl.~0,5?

\begin{table}
  \centering
  \begin{tabular}{@{}cccccccc@{}}
    \toprule
    i       & 0  & 1   & 2   & 3   & 4   & 5  &  \\
    \midrule
    $o_{i}$ & 58 & 149 & 305 & 303 & 162 & 45 &  \\
    \bottomrule
  \end{tabular}
  \caption{Frequenties $o_i$ van het aantal jongens $i$ in gezinnen met vijf kinderen uit een onderzoek met $n = 1022$ gezinnen.}
  \label{tab:5-kinderen}
\end{table}

Indien de veronderstelling waar is, wordt de kans $\pi_{i}$ om $i$ jongens te krijgen bepaald door een binominaalverdeling met parameters $n=5$ en $p=0,5$.

Dit kan je eenvoudig nagaan aan de hand van het voorbeeld. De kans om 2 jongens te krijgen met 5 kinderen is gelijk aan :

\[ (0,5)^{2} \times (1-0,5)^{5-2} \times \binom{5}{2} \]

Algemeen geldt dus:

\[ \pi_{i} = \binom{5}{i}\times 0,5^{i} \times 0,5^{5-i} = \frac{5!}{i!(5-i)!}\times 0,5^{5} \]

Met deze $\pi_{i}$ kunnen we dus de verwachte frequentie $e$ bepalen en de stappen volgen zoals hierboven beschreven. Tabel~\ref{tab:5-kinderen-berekeningen} geeft een overzicht van de nodige berekeningen.

\begin{table}
  \centering
  \begin{tabular}{@{}lrrrrrrr@{}}
  	\toprule
  	$i$                   &     0 &      1 &      2 &      3 &      4 &     5 &  Tot. \\
  	\midrule
  	$o_i$                 &    58 &    149 &    305 &    303 &    162 &    45 &  1022 \\
  	$\pi_i$               &  0,03 &   0,15 &   0,31 &   0,31 &   0,15 & 0,031 &     1 \\
  	$e_i$                 & 31,68 & 159,43 & 318,86 & 318,86 & 159,43 & 31,68 &       \\
  	$\frac{(o-e)^{2}}{e}$ & 21,86 &   0,68 &   0,60 &   0,78 &   0,04 &  5,59 & 29,57 \\
  	$r_i$                 &  4,74 &  -0,89 &  -0,93 & -1,071 &   0,22 &  2,40 &       \\
  	\bottomrule
  \end{tabular}
  \caption{Berekeningen voor de casus van gezinnen met 5 kinderen. $i$ is het aantal jongens in het gezin, $o_i$ de geobserveerde aantallen gezinnen in de steekproef met $i$ jongens. $\pi_i$ is de verwachte kans dat in een gezin van 5 kinderen $i$ jongens voorkomen en $e_i$ de verwachte frequentie. Daaronder wordt nog de berekening van $\chi^2$ getoond en tenslotte de gestandaardiseerde residuën $r_i$.}
  \label{tab:5-kinderen-berekeningen}
\end{table}

\begin{enumerate}
  \item \textbf{Bepalen hypotheses}
  
  \begin{itemize}
    \item $H_{0}$: de steekproef is representatief voor de populatie
    \item $H_{1}$: de steekproef is niet representatief voor de populatie
  \end{itemize}
  \item \textbf{Bepalen $\alpha$ en $n$} : $\alpha = 0,01$ en $n = 1022$.
  \item \textbf{Toetsingsgrootheid en waarde ervan in steekproef}:
  \[ \chi^{2} = \sum_{i=1}^{n} \frac{(o_{i} - e_{i})^{2}}{e_{i}} = 29,5766 \]
  \item \textbf{Bereken en teken kritiek gebied}:  kritieke grens is 15,0863. Onze toetsingsgrootheid ligt dus in het kritieke gebied dus verwerpen we $H_{0}$. 
\end{enumerate}

We vinden dus dat de steekproef \emph{niet} representatief is voor een populatie waar geldt dat de kans op een jongen even groot is als de kans op een meisje.

Nu kunnen we ons de vraag stellen of \emph{elke} klasse afwijkt van de verwachte frequentie, of slechts één of enkele. Welke klassen zijn er onder- of oververtegenwoordigd in de steekproef? Om dit te bepalen, gebruiken we de zgn.~\emph{gestandaardiseerde residuen}\index{residuen!gestandaardiseerde} die aanduiden welke klassen de grootste bijdrage leveren aan de waarde van $\chi^2$. 

\[ r_{i} = \frac{O_{i} - n \pi_{i}}{\sqrt{n \pi_{i}(1-\pi_{i})}} \]

%\begin{exercise}
%  Hoe komen we hier aan de noemer? Waar komt dit mee overeen? Hoe bepaal je de variantie van een binomiale verdeling?
%  
%  Antwoord: $n \times \pi (1-\pi)$
%\end{exercise}

De waarde voor $r_i$ is 0 als $o = e$. Negatieve waarden wijzen er op dat deze klasse ondervertegenwoordigd is in de steekproef, positieve dat ze oververtegenwoordigd is. Er geldt algemeen dat waarden groter dan 2 of kleiner dan $-2$ extreem zijn. We kunnen dus uit Tabel~\ref{tab:5-kinderen-berekeningen} besluiten dat het aantal gezinnen met enkel jongens ($r_5 = 2,4$) of enkel meisjes ($r_0 = 4,74$) groter mag worden genoemd dan verwacht.

\subsection{Cramér's V}
\label{ssec:cramers-v}

Uit de grootte van de $\chi^2$-statistiek is het niet meteen mogelijk om af te leiden of er een verband tussen beide variabelen bestaat. Dit hangt af van de grootte van de kruistabel, meer bepaald het aantal rijen en kolommen.

De Zweedse wiskundige en statisticus Harald Cramér\index{Cramér!Harald} heeft een metriek ontwikkeld die de $\chi^2$ voor gelijk welke kruistabel herleidt tot een waarde tussen 0 en 1, Cramér's V\index{Cramér's V}:

\begin{definition}[Cramér's V]
  \begin{equation}
  V = \sqrt{\frac{\chi^{2}}{n (k-1)}}
  \label{eq:Cramer}
  \end{equation}
  met
  \begin{itemize}
    \item $\chi^{2}$ de berekende chi-kwadraatwaarde.
    \item $n$ het aantal waarnemingen (steekproefgrootte).
    \item $k$ = de kleinste waarde van het aantal kolommen of het aantal rijen van de tabel.
  \end{itemize}
  
\end{definition}

Cramér's V is de $\chi^{2}$, gecorrigeerd voor steekproefomvang en het aantal categorieën in de variabelen. Tabel~\ref{tab:interpretatie-cramers-v} geeft aan hoe je het resultaat kan interpreteren.

\begin{table}
  \centering
  \begin{tabular}{ll}
    $V = 0$ & geen samenhang \\
    $V \approx 0,1$ & zwakke samenhang \\
    $V \approx 0,25$ & redelijk sterke samenhang \\
    $V \approx 0,50$ & sterke samenhang \\
    $V \approx 0,75$ & zeer sterke samenhang \\
    $V = 1$ & volledige samenhang \\
  \end{tabular}
  \caption{Interpretatie van de waarde van Cramérs'V}
  \label{tab:interpretatie-cramers-v}
\end{table}

Voor onze eerdere casus waar werd onderzocht of er een verband was tussen gender en waardering vonden we een $\chi^{2} \approx 3,811$. Cramér's V is dan $\sqrt{\frac{3,811}{49 (2 - 1)}} \approx 0,279$. Dit duidt op redelijk sterke samenhang tussen de variabelen. Met andere woorden, de resultaten van de bevraging geven aan dat er een verschil is in de waardering die vrouwen en mannen geven over het assortiment.

Dit resultaat is opmerkelijk omdat we via de $\chi^2$-toets geen significant verband gevonden hebben tussen beide variabelen. Het is bekend dat Cramér's V de neiging heeft om de mate van associatie te overschatten. Het is dus mogelijk dat dat ook in dit geval gebeurd is.

\begin{example}
  In Tabel~\ref{tab:autovoorkeur} worden de voorkeuren van vrouwen en mannen voor de gegeven automerken opgesomd. We zien dat nog steeds dertig van de honderd respondenten een voorkeur hebben voor de Mercedes, maar dat twee derde van deze dertig vrouwen zijn. We zouden  ook kunnen zeggen dat de helft van de vrouwen een voorkeur heeft voor de Mercedes. Evenzo blijkt dat een derde van de mannen een voorkeur heeft voor een Alfa Romeo, tegenover geen van de vrouwen. Het lijkt alsof de onderscheiden automerken niet gelijkelijk gewaardeerd worden door mannen en vrouwen. Om dit te staven bepalen we $\chi^{2}$ en Cramér's V. Probeer dit zelf, hetzij in R, hetzij met een rekenblad (Excel, Numbers, LibreOffice Calc)! We vinden:
  \[ \chi^{2} = 22,619 \]
  \[ V = \sqrt{\frac{22,619}{100 \times (2-1)}}  = 0,476\]
  
  We vinden dus tussen een redelijk sterke tot sterke samenhang.
\end{example}

\begin{table} \centering
  \begin{tabular}{@{}rrrrrr@{}}
  	\toprule
  	        & Mercedes &  BMW & Porsche & Alfa Romeo & Totaal \\
  	\midrule
  	 Mannen &     $10$ & $10$ &    $20$ &       $20$ &   $60$ \\
  	Vrouwen &     $20$ &  $5$ &    $15$ &        $0$ &   $40$ \\
  	 Totaal &     $30$ & $15$ &    $35$ &       $20$ &  $100$ \\
  	\bottomrule
  \end{tabular}
  \caption{Tabel die uitdrukt hoeveel vrouwen en hoeveel mannen een voorkeur voor een bepaald automerk hebben.}
  \label{tab:autovoorkeur}
\end{table}

\subsection{Visualisatietechnieken}
\label{ssec:kwal-kwal-visualisatie}

\subsection{Geclusterd staafdiagram}
\subsection{Rependiagram}
\subsection{Mozaïekdiagram}

\section{Kwalitatief--kwantitatief}
\label{sec:kwal-kwant}

In deze sectie bekijken we methoden om na te gaan of er een verband bestaat tussen enerzijds een kwalitatieve variabele (onafhankelijk) en anderzijds een kwantitatieve variabele (afhankelijk). Een onderzoek naar genderongelijkheid bij verloning in de ict-sector is hier een goed voorbeeld van. De onafhankelijke variabele \emph{gender} is nominaal (mogelijke waarden bijvoorbeeld M/V/X), de afhankelijke variabele \emph{netto maandloon} is een ratio-variabele.

De benadering die men typisch gebruikt, is om het gemiddelde en/of standaardafwijking van de afhankelijke variabele tussen de verschillende groepen ontstaan uit de onafhankelijke met elkaar te vergelijken. Als de verschillen klein zijn, dan kunnen ze worden toegeschreven aan toevallige steekproeffouten en besluiten we dat er geen verband is. Als de verschillen groot zijn, dan besluiten we dat er \emph{wel} een verband is. Vaak is het bijvoorbeeld nog altijd zo dat mannen significant meer verdienen dan vrouwen voor een gelijkaardige job!

In deze cursus bespreken we de $t$-toets voor 2 steekproeven om te bepalen of er een verschil is tussen twee groepen (zie Sectie~\ref{ssec:t-toets-twee-steekproeven}). Er bestaan statistische toetsen om een groter aantal groepen tegelijk te beschouwen (bijvoorbeeld de ANOVA-toets), maar die vallen buiten het bestek van deze cursus.

Net zoals Cramér's V bij kwalitatieve variabelen bestaan er ook in dit geval metrieken die aangeven hoe sterk het verband is tussen kwalitatieve en kwantitatieve variabelen. In deze context worden die vaak onder de noemer \emph{effectgrootte} genoemd. In deze cursus zien we één definitie van effectgrootte, nl.~Cohen's $d$ (zie Sectie~\ref{ssec:cohens-d}). Opnieuw, er bestaan meerdere vormen van effectgrootte, ook geschikt voor variabelen met andere meetniveaus, maar deze vallen ook buiten het bestek van deze cursus.

\subsection{De \texorpdfstring{$t$}{t}-toets voor twee steekproeven}
\label{ssec:t-toets-twee-steekproeven}

De $t$-toets die geïntroduceerd werd in Sectie~\ref{sec:t-toets} kan ook gebruikt worden om twee steekproeven met elkaar te vergelijken. Je kan er dan mee nagaan of het steekproefgemiddelde van beide steekproeven \emph{significant} verschillend is.

Men maakt onderscheid tussen twee gevallen:

\begin{itemize}
  \item Beide steekproeven zijn onafhankelijk, zijn afzonderlijk genomen. Een voorbeeld is een onderzoek naar een medische behandelingsmethode waar een contolegroep de behandeling \emph{niet} krijgt en een testgroep de behandeling wel krijgt.
  \item De steekproeven zijn afhankelijk, of gepaard. Een voorbeeld is twee metingen uitvoeren op hetzelfde lid van de populatie, zoals de koorts nemen voor en na het innemen van een medicijn om het effect ervan te meten.
\end{itemize}

In R kan je eveneens de functie \texttt{t.test} gebruiken voor het uitvoeren van een toets met twee steekproeven. We geven hieronder twee voorbeelden, één voor elk geval.

\begin{example}
  In een klinisch onderzoek wil men nagaan of een nieuw medicijn als bijwerking een vertraagde (dus hogere) reactiesnelheid heeft~\autocite{Lindquist}.
  
  Zes deelnemers kregen een medicijn toegekend (interventiegroep) en zes anderen een placebo (controlegroep). Vervolgens werd hun reactietijd op een stimulus gemeten (in ms). We willen nagaan of er significante verschillen zijn tussen de interventie- en controlegroep.
  
  \textbf{Opmerking}: De interventiegroep en de controle groep zijn hier toevallig even groot (elk 6 proefpersonen).
  Dit is niet noodzakelijk. Bij een onafhankelijke (niet-gepaarde) steekproeven
  mogen de 2 groepen een verschillende grootte hebben.
  
  \begin{itemize}
    \item Controlegroep: 91, 87, 99, 77, 88, 91 ~~~~~~~~~~~~($\overline{x}=88,83$)
    \item Interventiegroep: 101, 110, 103, 93, 99, 104 ~~($\overline{y}=101,67$)
  \end{itemize}
  
  We noteren $\mu_1$ voor het gemiddelde van de niet behandelde populatie (controlegroep) en $\mu_2$ voor het populatiegemiddelde van de patiënten die het medicijn nemen (interventiegroep).
  
  De hypothesen worden formeel als volgt genoteerd:
  
  $H_0: \mu_1 - \mu_2 = 0$ en $H_1: \mu_1 - \mu_2 < 0$
  
  Als toetsingsgrootheid gebruiken we $\overline{x}-\overline{y}$, met $\overline{x}$ en $\overline{y}$ schattingen voor de \textit{\'echte} waarden $\mu_1$ en $\mu_2$ .
  
  Het gaat hier dus over een linkszijdige test, wat weergegeven wordt door de optie \texttt{alternative = "less"}. In de nulhypothese veronderstellen we dat het verschil tussen de populatiegemiddelden 0 is, wat met de optie \texttt{mu = 0} wordt aangeduid. Merk op dat dit de standaardwaarde is voor deze parameter en dus in principe niet moet worden opgegeven.
  
  \begin{lstlisting}
  controle <-  c(91, 87, 99, 77, 88, 91)
  interventie <- c(101, 110, 103, 93, 99, 104)
  t.test(controle, interventie, alternative="less", mu=0)
  \end{lstlisting}
  
  Het resultaat van de toets:
  
  \begin{verbatim}
  t.test(controle, interventie, alternative="less")
  
  Welch Two Sample t-test
  
  data:  controle and interventie
  t = -3.4456, df = 9.4797, p-value = 0.003391
  alternative hypothesis: true difference in means is less than 0
  95 percent confidence interval:
  -Inf -6.044949
  sample estimates:
  mean of x mean of y 
  88.83333 101.66667
  \end{verbatim}
  
  De teststatistiek $\overline{x}-\overline{y}=-12,833$ komt overeen met een $t$-waarde $t=-3,4456$.
  De parameter $df=9,48$ wordt bepaald door \texttt{t.test()} op basis van
  het aantal elementen in de reeksen $x$ en $y$.
  De berekening hiervan is \textit{niet} triviaal.
  
  De $p$-waarde, 0,003391, ligt duidelijk onder het significantieniveau (niet expliciet opgegeven, dus werd de standaardwaarde $\alpha = 0,05$ gebruikt.)
  
  We mogen dus de nulhypothese verwerpen en besluiten dat volgens de resultaten van deze steekproef het medicijn inderdaad een significant effect heeft op de reactiesnelheid van patiënten.
  
  \textbf{Opmerking}: Vermits $\overline{x}-\overline{y}=-12,833$
  kunnen we met 95\% procent zekerheid zeggen dat het verschil van de \textit{\'echte} gemiddelden ($\mu_1-\mu_2$)
  van een grotere controle- en interventiepopulatie tussen $-\infty$ en $-6.044949$ zal liggen.
  Zie paragraaf \ref{ssec:betrouwbaarheidsinterval-grote-steekproef} (p. \pageref{ssec:betrouwbaarheidsinterval-grote-steekproef})
  en \ref{ssec:betrouwbaarheidsinterval-kleine-steekproef} (p. \pageref{ssec:betrouwbaarheidsinterval-kleine-steekproef})
  over betrouwbaarheidsintervallen.
\end{example}

\begin{example}
  In een studie werd nagegaan of auto's die rijden op benzine met additieven ook een lager verbruik hebben. Tien auto's werden eerst volgetankt met ofwel gewone benzine, ofwel benzine met additieven (bepaald door opgooien van een munt), waarna het verbruik werd gemeten (uitgedrukt in mijl per gallon). Vervolgens werden de auto's opnieuw volgetankt met de andere soort benzine en werd opnieuw het verbruik gemeten. De resultaten worden gegeven in de tabel hieronder.
  
  \begin{center}
    \begin{tabular}{|l|c|c|c|c|c|c|c|c|c|c|}
      \hline 
      Auto & 1 & 2 & 3 & 4 & 5 & 6 & 7 & 8 & 9 & 10 \\ 
      \hline 
      Gewone benzine & 16 & 20 & 21 & 22 & 23 & 22 & 27 & 25 & 27 & 28 \\ 
      \hline 
      Additieven & 19 & 22 & 24 & 24 & 25 & 25 & 25 & 26 & 28 & 32 \\ 
      \hline 
    \end{tabular} 
    %  \caption{Verbruik in mijl per gallon met 2 soorten benzine.}
    %  \label{tab:benzineverbruik-additieven}
  \end{center}
  
  We gaan door middel van een \emph{gepaarde $t$-test} na of auto's significant zuiniger rijden met benzine met additieven.
  
  We kiezen $x$ voor benzine met \texttt{additieven} ($\overline{x}=25,1$ mijl per gallon), en we kiezen $y$ voor \texttt{gewone} bezine ($\overline{y}=23,1$ mijl per gallon).
  
  De nulhypothese $H_0$ is dat je met beiden even veel mijl per gallon kunt rijden ($\mu_{x-y}=0$).
  De alternatieve hypothese $H_1$ dat je verder kunt rijden op benzine met additieven ($\mu_{x-y}>0$).
  
  De optie \texttt{paired=TRUE} geeft aan dat het hier om een gepaarde $t$-toets gaat.
  
  \begin{lstlisting}
  gewone    <- c(16, 20, 21, 22, 23, 22, 27, 25, 27, 28)
  additieven <-c(19, 22, 24, 24, 25, 25, 26, 26, 28, 32)
  t.test(additieven, gewone, alternative="greater", paired=TRUE)
  \end{lstlisting}
  
  Resultaat:
  
  \begin{verbatim}
  Paired t-test
  
  data:  additieven and gewone
  t = 4.4721, df = 9, p-value = 0.0007749
  alternative hypothesis: true difference in means is greater than 0
  95 percent confidence interval:
  1.180207      Inf
  sample estimates:
  mean of the differences 
  2 
  \end{verbatim}
  
  De teststatistiek $\overline{x-y}=2$. Dit komt overeen met een $t$-waarde $t=4,4721$.
  De $p$-waarde, 0,0007749, ligt onder het significantieniveau ($\alpha=0,05$), dus we kunnen de nulhypothese verwerpen. Volgens deze steekproef rijden auto's inderdaad zuiniger met benzine met additieven.
  
  \textbf{Ter info}: Bij 95\% van de ``paren'' van een grotere populatie auto's,
  zal het verschil $x-y$ tussen $1.180207$ en $+\infty$ liggen.
  Dit is het betrouwbaarheidsinterval waarvan sprake in paragraaf \ref{ssec:betrouwbaarheidsinterval-grote-steekproef} (p. \pageref{ssec:betrouwbaarheidsinterval-grote-steekproef})
  en \ref{ssec:betrouwbaarheidsinterval-kleine-steekproef} (p. \pageref{ssec:betrouwbaarheidsinterval-kleine-steekproef}).
\end{example}

\subsection{Effectgrootte - Cohen's \texorpdfstring{$d$}{d}}
\label{ssec:cohens-d}

Met de term \emph{effectgrootte}\index{effectgrootte} bedoelt men een metriek die de impact (of effect) van een gebeurtenis weergeeft. Bij de meeste varianten geeft een grote absolute waarde ook een groter effect aan, een waarde van 0 het ontbreken van een effect.

In deze sectie introduceren we Cohen's $d$\index{Cohen's $d$}, ontwikkeld door de Amerikaanse psycholoog en statisticus Jacob Cohen\index{Cohen, Jacob}. Deze metriek wordt in het bijzonder vaak gebruikt in publicaties over onderzoek naar effecten op leerresultaten in het onderwijs. Een onderzoek in deze context wordt vaak als volgt opgezet:

De onderzoekers wensen het effect te weten van een bepaalde interventie op het leren van studenten/leerlingen. Ze willen bijvoorbeeld een nieuwe lesvorm uitproberen en bepalen of die geschikt is. Er worden testpersonen geselecteerd die typisch aselect verdeeld worden over twee groepen: een controlegroep die een afgebakend stuk leerstof te verwerken krijgt op een ``klassieke'' manier en een interventiegroep die dezelfde leerstof volgens die nieuwe lesvorm krijgt voorgeschoteld. Na afloop van de lessen volgt er dan een toets om te bepalen in hoeverre de studenten van beide groepen de leerstof verworven hebben. We verwachten dan dat de studenten uit de interventiegroep een significant beter resultaat behalen dan de controlegroep. In de wetenschappelijke publicaties die resulteren uit dit soort onderzoek wordt steevast de effectgrootte gepubliceerd.

John \textcite{Hattie2012}\index{Hattie, John} verzamelt al decennia lang publicaties over onderzoek in onderwijs en houdt een lijst bij van effectgroottes van allerlei dingen die impact hebben op studieresultaten bij studenten. Dat gaat niet alleen over lesmethodes, maar ook leerstrategieën van studenten, demografische factoren (gender, sociale status, enz.), school- en klasmanagement, enz. In zijn meta-analyse hebben de meeste bestudeerde factoren een positief effect op de leerresultaten, wat misschien wel een gevolg kan zijn van \emph{publication bias}. 

De gemiddelde gerapporteerde effectgrootte ligt rond $d = 0,4$. De aanbeveling van Hattie is dan ook dat scholen die een positief effect op leerresultaten van studenten willen bekomen, best eerst kijken naar de factoren die resulteren in een effectgrootte van minstens 0,4. Als vuistregel kan je stellen dat een interventie met $d = 1$ als gevolg heeft dat de leerstof die normaal op één jaar gezien wordt, op de helft van de tijd kan verwerkt worden. Tabel~\ref{table:effectgrootte} geeft een overzicht van de interpretatie van verschillende waarden voor Cohen's $d$.

\begin{definition}[Cohen's $d$]
  is gedefinieerd als het verschil tussen twee gemiddelden gedeeld door een standaardafwijking voor de steekproef, meer bepaald:
  \begin{equation}
  d = \frac{\overline{x}_1 - \overline{x}_2}{s}
  \end{equation}
  
  met $\overline{x}_1$ en $\overline{x}_2$ de gemiddelden van beide groepen en $s$ een gecombineerde standaardafwijking voor twee onafhankelijke steekproeven:
  
  \begin{equation}
  s = \sqrt{\frac{(n_1 - 1) s_1^2 + (n_2 - 1) s_2^2}{n_1 + n_2 - 2}}
  \end{equation}
  
  met $s_1^2$ en $s_2^2$ de steekproefvariantie van beide groepen en $n_1$ en $n_2$ het aantal observaties in elke groep.
\end{definition}

\begin{table}
  \centering
  \begin{tabular}{rl}
    \toprule
    \textbf{$|d|$} & \textbf{Effect} \\
    \midrule
    0,01 & Zeer klein \\
    0,20 & Klein \\
    0,50 & Middelmatig \\
    0,80 & Groot \\
    1,20 & Zeer groot \\
    2,00 & Reusachtig \\
    \bottomrule
  \end{tabular}
  \caption{Interpretatie van de absolute waarde van Cohen's $d$. Merk op dat $d$ ook kleiner dan 0 kan zijn, wat wijst op een negatief effect van de interventie.}
  \label{table:effectgrootte}
\end{table}

\subsection{Visualisatietechnieken}
\label{ssec:kwal-kwant-visualisatie}

\subsubsection{Boxplot}

% TODO   Boxplot

\subsubsection{Staafdiagram met error bars}

% TODO  Staafdiagram met error bars

\section{Kwantitatief--kwantitatief}

% TODO: inleiding
% TODO: Spreidingsdiagram

\subsection{Regressie}
\label{sec:regressie}

Bij \index{Regressie} regressie gaan we proberen een consistente en systematische koppeling tussen de variabelen te vinden. Dat betekent concreet: ``als we de waarde van de onafhankelijke variabele kennen, kunnen we dan ook de waarde van de afhankelijke variabele voorspellen?'' We kennen twee soorten verbanden:

\begin{description}
  \item [Monotoon:] een monotoon verband is een verband waarbij de onderzoeker de algemene richting van de samenhang tussen de twee variabelen kan aanduiden, hetzij stijgend, hetzij dalend. De richting van het verband verandert nooit.
  \item [Niet-monotoon:] bij een niet-monotoon verband wordt de aanwezigheid (of afwezigheid) van de ene variabele systematisch gerelateerd aan de aanwezigheid (of afwezigheid) van een andere variabele. De richting van het verband kan echter niet aangeduid worden.
\end{description}

Bij lineaire regressie gaan we ons beperken tot een lineair verband: een rechtlijnige samenhang tussen een onafhankelijke en afhankelijke variabele, waarbij kennis van de onafhankelijke variabele kennis over de afhankelijke variabele geeft.

Bij een lineair verband zijn er drie karakteristieken:

\begin{enumerate}
  \item Aanwezigheid: is er wel een verband tussen de twee variabelen?
  \item Richting: is er een dalend of een stijgend verband?
  \item Wat is de sterkte van het verband: sterk, gematigd of niet-bestaand?
\end{enumerate}

Een voorbeeld van een linear verband $y = \beta_{0} + \beta_{1}x$  vind je bijvoorbeeld in figuur \ref{fig:regressieFig}.

\begin{figure}[t]
  \begin{tikzpicture}
    \begin{axis}[
        axis x line=middle,
        axis y line=middle,
        enlarge y limits=true,
        width=\textwidth, height=8cm,     % size of the image
        grid = major,
        grid style={dashed, gray!30},
        ylabel=$y$,
        xlabel=$x$,
        legend style={at={(0.1,-0.1)}, anchor=north}
      ]
      \addplot[only marks] table  {data/regressie.dat};
      \addplot [no markers, thick, red] table [y={create col/linear regression={y=y}}] {data/regressie.dat};
    \end{axis}
  \end{tikzpicture}
  \caption{Een voorbeeld van een lineair verband}
  \label{fig:regressieFig}
\end{figure}

Zo'n verband kunnen we vinden aan de hand van de \index{Kleinste kwadraten methode} kleinste kwadraten methode van Gauss. Dit wordt als volgt gedaan.

\begin{theorem}

  Een lineair verband wordt weergegeven als volgt:

  \begin{equation}
    y = \beta_{0} + \beta_{1} x
    \label{eq:lineair}
  \end{equation}
  met
  \begin{itemize}
    \item $y$ de afhankelijke
    \item $x$ de onafhankelijke
  \end{itemize}

  We willen hier de som van de kwadraten minimaliseren van de afwijkingen $e_{i} = y_{i} - (\beta_{0} + \beta_{1}x_{i})$. Zo'n afwijking kan ook geschreven worden als (stel $X_{i} = x_{i} - \overline{x}$ en $Y_{i} = y_{i} - \overline{y}$):

  \begin{eqnarray}
    e_{i} & = & y_{i} - \beta_{1} x_{i} - \beta_{0} \\
    e_{i} & = & (y_{i} - \overline{y}) - \beta_{1}(x_{i} - \overline{x}) - (\beta_{0} - \overline{y} + \beta_{1} \overline{x}) \\
    \label{eq:regressie-bewijs}
    e_{i} & = & Y_{i} - \beta_{1} X_{i} - (\beta_{0} - \overline{y} + \beta_{1} \overline{x})
  \end{eqnarray}

  In stap~\ref{eq:regressie-bewijs} doen we eigenlijk $+\overline{x}-\overline{x}+\overline{y}-\overline{y}$, wat een nuloperatie is. Dit is een gedachtensprong die niet meteen voor de hand ligt, maar onthou dat dit een ``shortcut'' is naar de oplossing en dat het ``echte'' bewijs een stuk ingewikkelder is.

  We willen de som van de kwadraten van $e_i$  minimaliseren:

  \begin{eqnarray}
    \sum_{i}^{n} e_{i}^{2} & =& \sum_{i}^{n} (y_{i} - (\beta_{0} + \beta_{1}x_{i}))^{2}\\
    & = & \sum_{i}^{n} ((Y_{i} - \beta_{1} X_{i}) - (\beta_{0} - \overline{y} + \beta_{1}\overline{x}))^{2}\\
    & = & \sum_{i}^{n}(Y_{i} - \beta_{1} X_{i})^2 - 2 \sum_{i}^{n}(Y_i - \beta_1 X_i)(\beta_0 - \overline{y}+ \beta_1\overline{x}) + \sum_{i}^{n}(\beta_{0} - \overline{y} + \beta_{1}\overline{x})^{2} \label{eq:stap1}\\
    & = & \sum_{i}^{n}(Y_{i} - \beta_{1} X_{i})^{2} + n(\beta_{0} - \overline{y} + \beta_{1} \overline{x})^{2} \label{eq:stap2}
  \end{eqnarray}
	
	We kunnen de stap maken van \ref{eq:stap1} naar \ref{eq:stap2} door volgende uit te werken:
	
\[ \sum_{i}^{n}X_i = \sum_{i}^{n} (x_i - \overline{x}) = 0 \]
	en equivalent
\[ \sum_{i}^{n}Y_i = \sum_{i}^{n} (y_i - \overline{y}) = 0 \]
daardoor is
\[ \sum_{i}^{n}(Y_i - \beta_1 X_i) = \sum_{i}^{n}Y_i - \beta_1 \sum_{i}^{n}X_i = 0 \]
en bijgevolg dus ook
\[ 2 \sum_{i}^{n}(Y_i - \beta_1 X_i)(\beta_0 - \overline{y}) \]
	

Nu is $e^{2}_{i}$ geschreven als een som van twee positieve uitdrukkingen. Deze som is minimaal als beide uitdrukkingen minimaal zijn.

  \begin{equation}
    \begin{cases}
      \sum_{i}^{n}( Y_{i} - \beta_{1} X_{i})^{2} \textnormal{ is minimaal.}\\
      n(\beta_{0} - \overline{y} + \beta_{1} \overline{x})^{2} \textnormal{ is minimaal}
    \end{cases}
    \label{eq:vgl}
  \end{equation}
	

Voor de eerste uitdrukking vinden we eigenlijk een kwadratische functie in $\beta_1$.
  \begin{eqnarray}
		& \sum_{i}^{n}( Y_{i} - \beta_{1} X_{i})^{2} \textnormal{ is minimaal.} \label{eq:uitdrukking}\\
		\Leftrightarrow & \sum_i^n (Y_i^2 - 2X_iY_i\beta_1 + X_i^2\beta_1^2) \textnormal{ is minimaal.} \\
		\Leftrightarrow & \beta_1^2 \sum_i^n X_i^2 - 2\beta_1 \sum_i^n X_iY_i + \sum_i^nY_i^2 \textnormal{ is minimaal.} \\
		\Leftrightarrow & \textnormal{is minimaal als } \beta_{1} = \frac{\sum_{i}^{n} X_{i}Y_{i}}{\sum_{i}^{n} X_{i}^{2}}
	\end{eqnarray}

Voor de tweede uitdrukking vinden we

  \begin{eqnarray}
		& n(\beta_{0} - \overline{y} + \beta_{1} \overline{x})^{2} \textnormal{ is minimaal}
		\Leftrightarrow & n(\beta_{0} - \overline{y} + \beta_{1} \overline{x})^{2} = 0 \\
		\Leftrightarrow & \beta_{0} - \overline{y} + \beta_{1} \overline{x} = 0 \\
		\Leftrightarrow & \beta_{0} = \overline{y} - \beta_{1}\overline{x} 
	\end{eqnarray}

	
  met als oplossing

  \begin{equation}
    \begin{cases}
      \beta_{1} = \frac{\sum_{i}^{n} X_{i}Y_{i}}{\sum_{i}^{n} X_{i}^{2}}\\
      \beta_{0} = \overline{y} - \beta_{1}\overline{x}
    \end{cases}
    \label{eq:vgl2}
  \end{equation}

  en dus

  \begin{eqnarray}
    \beta_{1} & = & \frac{\sum_{i}^{n} (x_{i} - \overline{x})(y_{i} - \overline{y})}{\sum_{i}^{n} (x_{i} - \overline{x})^{2}} \\
    \beta_{0} & = & \overline{y} - \beta_{1} \overline{x}
    \label{eq:regressie}
  \end{eqnarray}
\end{theorem}


\begin{table} \centering
  \begin{tabular}{@{}rr@{}} \toprule
    Eiwitgehalte\%& Gewichtstoename (gram)  \\
    \midrule
    0		&	177 \\
    10 	&	231	\\
    20	& 249	\\
    30	& 348 \\
    40	& 361 \\
    50	& 384 \\
    60	& 404 \\
    \bottomrule
  \end{tabular}
  \caption{De data die verzameld geweest is door de kerstman: per eiwitpercentage wordt de gewichtstoename beschouwd.}
  \label{tab:rendieren}
\end{table}



\begin{table} \centering
  \begin{tabular}{@{}llllll@{}}
    \toprule
    $x$   & $y$     & $x-\overline{x}$    & $y - \overline{y}$        & $(x-\overline{x})(y - \overline{y})$       &  $(x-\overline{x})^{2}$    \\ \midrule
    0  & 177 & -30 & -130,71 & 3921,3 & 900  \\
    10 & 231 & -20 & -76,71  & 1534,2 & 400  \\
    20 & 249 & -10 & -58,71  & 587,1  & 100  \\
    30 & 348 & 0   & 40,29   & 0      & 0    \\
    40 & 361 & 10  & 53,29   & 532,9  & 100  \\
    50 & 384 & 20  & 76,29   & 1525,8 & 400  \\
    60 & 404 & 30  & 96,29   & 2888,7 & 900  \\
    &     &     &         & 10990  & 2800 \\ \bottomrule
  \end{tabular}
  \caption{Berekeningen die nodig zijn voor het toepassen van de kleinste kwadratenmethode.}
  \label{tab:rendieren2}
\end{table}

\begin{figure}
  \begin{tikzpicture}
    \begin{axis}[
        axis x line=middle,
        axis y line=middle,
        enlarge y limits=true,
        width=\textwidth, height=8cm,     % size of the image
        grid = major,
        grid style={dashed, gray!30},
        ylabel=gewichtstoename (g),
        xlabel=eiwitgehalte (\%),
        legend style={at={(0.1,-0.1)}, anchor=north}
      ]
      \addplot[only marks] table  {data/santa.txt};
      \addplot [no markers, thick, red] table [y={create col/linear regression={y=y}}] {data/santa.txt};
    \end{axis}
  \end{tikzpicture}

  \caption{Lineair verband tussen eiwitgehalte en gewichtstoename}
  \label{fig:rendierenFiguur}
\end{figure}

\begin{example}
  \label{vb:rendieren}
  We kijken naar het voorbeeld van de Kerstman en zijn rendieren. Hij wil zien of er een lineair verband bestaat tussen het eiwitgehalte van het voeder en de gewichtstoename van de rendieren. Hij voert een aantal proeven uit en bekomt de data in tabel \ref{tab:rendieren}. Door toepassing van de formules die hierboven staan bekomt men (zie tabel \ref{tab:rendieren2}):
  \[ \beta_{1} = \frac{\sum_{i=1}^{n} (x_{i}-\overline{x})(y_{i} - \overline{y})}{\sum_{i=1}^{n} (x_{i}-\overline{x})^{2}} = \frac{10990}{2800} = 3.925 \]
  \[ \beta_{0} = \overline{y} - \beta_{1} \overline{x} = 307.7143 - 3.925 \times 30 = 189.96 \]
  Men heeft dus een lineair verband gevonden die de kwadraten van de residuen minimaliseert. Let wel, er wordt niets gezegd over de sterkte of validiteit van dit verband. Dit verband wordt getekend in figuur \ref{fig:rendierenFiguur}.
\end{example}

Voorbeeld~\ref{vb:rendieren} uitgewerkt in R (met plot van de regressierechte):

\lstinputlisting{data/regressie.R}

\subsection{Correlatie}
\label{sec:correlatie}

\subsubsection{Pearsons product-momentcorrelatiecoëfficiënt}

We kunnen twee statistieken bepalen die de sterkte van een lineair verband uitdrukken. Ook deze zijn---net als de $\chi^2$-toets---ontwikkeld door Karl Pearson\index{Pearson!Karl}.

\begin{definition}[Pearsons product-momentcorrelatiecoëfficiënt]
   Pearsons product momentcorrelatiecoëfficiënt\index{Pearsons product-momentcorrelatiecoëfficiënt} $R$ (of kortweg correlatiecoëfficiënt\index{correlatiecoëfficiënt}) is een maat voor de sterkte van de lineaire samenhang tussen X en Y. De waarde kan vari\"eren van -1 tot 1.

  \begin{itemize}
    \item Een waarde van +1 duidt een positief lineair verband aan.
    \item Een waarde van -1 duidt een negatief lineair verband aan.
    \item Een waarde van 0 wil zeggen dat er totaal geen lineaire samenhang is.
  \end{itemize}
  
  Hoe dichter de correlatiecoëfficiënt bij 1 of -1, hoe beter de kwaliteit van het lineair model.
\end{definition}

\subsubsection{Determinatieco\"effici\"ent}

\begin{definition}
  De \index{Determinatieco\"effici\"ent}determinatieco\"effici\"ent ($R^{2}$) is het kwadraat van de correlatieco\"effici\"ent en verklaart het percentage van de variantie van de waargenomen waarden t.o.v. de regressierechte.

  \begin{itemize}
    \item $R^{2}$ is de verklaarde variantie
    \item $1-R^{2}$ is de onverklaarde variantie
  \end{itemize}
\end{definition}

\begin{figure}[t]
  \begin{tikzpicture}
    \begin{axis}[
        axis x line=middle,
        axis y line=middle,
        enlarge y limits=true,
        width=\textwidth, height=8cm,     % size of the image
        grid = major,
        grid style={dashed, gray!30},
        ylabel=gezinsgrootte moeder,
        xlabel=gezinsgrootte,
        legend style={at={(0.1,-0.1)}, anchor=north}
      ]
      \addplot[only marks] table  {data/families.txt};
      \addplot [no markers, thick, red] table [y={create col/linear regression={y=y}}] {data/families.txt};
    \end{axis}
  \end{tikzpicture}
  \caption{Linear verband tussen grootte van een gezin en de grootte van de familie van de moeder}
  \label{fig:moederVerband}
\end{figure}

\tikzset{small dot/.style={fill=black, circle,scale=0.2}}
\tikzset{every pin/.style={draw=black,fill=yellow!10}}

\begin{figure}[t]%
  \begin{tikzpicture}
    \begin{axis}[
        axis x line=middle,
        axis y line=middle,
        enlarge y limits=true,
        width=\textwidth, height=8cm,     % size of the image
        grid = major,
        grid style={dashed, gray!30},
        ylabel=gezinsgrootte moeder,
        xlabel=gezinsgrootte,
        legend style={at={(0.1,-0.1)}, anchor=north}
      ]
      \draw (axis cs:3,0)--(axis cs:3,8);
      \draw (axis cs:0,4.3)--(axis cs:6,4.3);
      \node[small dot, pin=120:{$III$}] at (axis cs:1.6,7) {};
      \node[small dot, pin=120:{$I$}] at (axis cs:5.5,7) {};
      \node[small dot, pin=120:{$II$}] at (axis cs:1.6,2) {};
      \node[small dot, pin=120:{$IV$}] at (axis cs:5.5,2) {};
      \addplot[only marks] table  {data/families.txt};
    \end{axis}
  \end{tikzpicture}
  \caption{De figuur opgedeeld in 4 kwadranten}%
  \label{fig:kwadranten}%
\end{figure}

\subsubsection{Bepaling van $R$ en $R^{2}$}
\label{sec:determinatiecoef}
Beschouw het voorbeeld in figuur \ref{fig:moederVerband}:  de grootte van een gezin vs. de grootte van het gezin van de moeder. We zien duidelijk dat er een linear verband is. Indien we de gemiddelde berekenen en de figuur in 4 kwadranten (kwadrant $I$, $II$, $III$, $IV$) volgens de gemiddelden verdelen krijgen we de figuur in \ref{fig:kwadranten}.  Dan kunnen we volgende situaties bekijken.

\begin{itemize}
  \item Neem een element uit gebied I. Voor dit element is $x_{i} - \overline{x}$ positief en $y_{i} - \overline{y}$ ook. Dus is hun product. $(x_{i} - \overline{x}) (y_{i} - \overline{y}) > 0$.
  \item Neem een element uit gebied II. Voor dit element is $x_{i} - \overline{x}$ negatief en $y_{i} - \overline{y}$ ook. Dus is hun product. $(x_{i} - \overline{x}) (y_{i} - \overline{y}) > 0$.
  \item Neem een element uit gebied III. Voor dit element is $x_{i} - \overline{x}$ negatief en $y_{i} - \overline{y}$ positief. Dus is hun product. $(x_{i} - \overline{x}) (y_{i} - \overline{y}) < 0$.
  \item Neem een element uit gebied IV. Voor dit element is $x_{i} - \overline{x}$ positief en $y_{i} - \overline{y}$ negatief. Dus is hun product. $(x_{i} - \overline{x}) (y_{i} - \overline{y}) < 0$.
\end{itemize}

Aangezien dat er meer punten in gebieden I en II zijn dan in gebieden III en IV zal de som $\sum_{i} (x_{i} - \overline{x}) (y_{i} - \overline{y})$ een positief getal zijn. Hoe meer punten in I en II, hoe groter het getal. We merken dus een sterk positief lineair verband.

Indien de punten ongeveer gelijk verdeeld zouden zijn over de vier gebieden vinden we dat deze soms dicht bij nul zal zijn. Omgekeerd, indien er een negatief lineair verband zou zijn vinden we een negatief getal.

We hebben dus een maat gevonden om het verband tussen twee variabelen te meten:

\begin{itemize}
  \item Stijgende gecorreleerde verbanden is $\sum_{i} (x_{i} - \overline{x}) (y_{i} - \overline{y})$ positief en groot.
  \item Dalende gecorreleerde verbanden is $\sum_{i} (x_{i} - \overline{x}) (y_{i} - \overline{y})$ negatief en groot (in absolute waarde).
  \item Met niet gecorreleerde variabelen is $\sum_{i} (x_{i} - \overline{x}) (y_{i} - \overline{y})$ klein in absolute waarde.
\end{itemize}

We kunnen deze maat onafhankelijk maken van de grootte van de steekproef door te delen door de steekproefgrootte $n$. Dit noemen we de co-variantie en wordt gedefinieerd als gemeenschappelijke spreiding:

\begin{equation}
  Cov(X,Y) = \frac{\sum_{i}^{n}(x_{i} - \overline{x}) (y_{i} - \overline{y})}{n}
  \label{eq:covariantie}
\end{equation}

Dit geeft ons de gemiddelde afwijking per meetpunt.

Om opnieuw te normaliseren (een variatie in X is niet per se van dezelfde grootteorde als een variatie in Y) gaan we de maatstaf voor het gezamelijk vari\"eren onafhankelijk maken van het aantal waarnemingen en de orde van grootte van de getalswaarden. Zo kunnen we deze waarden universeel vergelijkbaar maken. Daarom delen we de co-variantie door het product van de standaardafwijkingen en noemen we de relatieve co-variantie of Pearson's correlatieco\"effici\"ent ook bekend als product-moment-correlatieco\"effici\"ent of kortweg als correlatieco\"effici\"ent.

\begin{eqnarray}
  R &=&\frac{COV(X,Y)}{\sigma_{x}\sigma_{y}} \\
  &=& \frac{COV(X,Y)}{\sqrt{\frac{\sum(x_{i} - \overline{x})^{2}}{n}} \times \sqrt{\frac{\sum(y_{i} - \overline{y})^{2}}{n}}} \\
  &=& \frac{\sum_{i}^{n}(x_{i}-\overline{x})(y_{i} - \overline{y})}{\sqrt{\sum_{i}^{n} (x_{i}-\overline{x})^{2}} \sqrt{\sum_{i}^{n} (y_{i}-\overline{y})^{2}}}
  \label{eq:relCovar}
\end{eqnarray}

De correlatieco\"effici\"ent is onafhankelijk van de meeteenheid terwijl de covariantie afhankelijk is van de meeteenheid.

\subsubsection{Interpretatie van $R^{2}$}

Als we aannemen dat $x$ niet bijdraagt aan de voorspelling van $y$, dan is de beste voorspelling voor een waarde van $y$ het steekproefgemiddelde $\overline{y}$, dat in figuur \ref{fig:rendierenFiguur3} als een horizontale lijn wordt weergegeven. De verticale lijnstukken zijn de afwijkingen van de waargenomen punten $y$ van deze voorspelling (het steekproefgemiddelde). De som van de kwadraten van deze afwijkingen is:

\[ SS_{yy} = \sum(y_{i} - \overline{y})^{2} \]

Indien we aannemen dat $x$ wel een rol speelt bij de voorspelling van $y$, berekenen we de regressielijn bij dezelfde gegevensverzameling en de afwijkingen van de punten ten opzichte van de lijn zoals in figuur \ref{fig:rendierenFiguur2}.

\[ SSE_{yy} = \sum(y_{i} - \widehat{y})^{2} \]

 Als we nu de afwijkingen vergelijken met elkaar zien we het volgende:
\begin{enumerate}
  \item Als $x$ weinig of niet bijdraagt in de voorspelling zullen de sommen van de kwadraten van de afwijkingen van de twee lijnen nagenoeg dezelfde zijn:
    \[ SS_{yy} = \sum(y_{i} - \overline{y})^{2} \] en
    \[ SSE_{yy} = \sum(y_{i} - \widehat{y})^{2} \]
    waarbij $\widehat{y}$ de voorspelde waarde is.
  \item Als $x$ wel bijdraagt tot de voorspelling van $y$ zal $SSE_{yy}$ kleiner zijn dan $SS_{yy}$. In feite zal
    \[	SSE_{yy} = \sum(y_{i} - \widehat{y})^{2} \]
    gelijk zijn aan nul als alle punten perfect voorspeld worden (en dus op de regressierechte liggen).
\end{enumerate}

De vermindering in de som van de kwadraten die toegeschreven kan worden aan het opnemen van $x$ in het model is dan (uitgedrukt in fractie van $SS_{yy}$)
\[ \frac{SS_{yy} - SSE_{yy}}{SS_{yy}} \]
We noemen $SS_{yy}$ de totale steekproefvariantie van de meetwaarden rond het steekproefgemiddelde $\overline{y}$ en $SSE_{yy}$ de overblijvende niet-verklaarde steekproefvariantie, na het schatten van de lijn $\widehat{y} = \beta_{0} + \beta_{1}x$. Dus dan is $(SS_{yy} - SSE_{yy})$ de verklaarde variantie die toe te schrijven is aan de lineaire relatie met $x$.

Er kan nu worden aangetoond dat bij enkelvoudige lineaire regressie deze fractie
\[ \frac{SS_{yy} - SSE_{yy}}{SS_{yy}} = \frac{\textnormal{verklaarde variantie}}{\textnormal{totale steekproefvariantie}} \]
gelijk is aan het kwadraat van de Pearson correlatieco\"effici\"ent.  (= het deel van de totale variantie dat verklaard wordt door de lineaire rechte).

Tabel~\ref{tab:interpretatie-correlatiecoefficient} geeft een overzicht hoe de waarden van $R$ en $R^2$ kunnen geïnterpreteerd worden, meer bepaald hoe sterk het verband tussen twee variabelen dan is.

\begin{figure}
  \begin{tikzpicture}
    \begin{axis}[
        axis x line=middle,
        axis y line=middle,
        enlarge y limits=true,
        width=\textwidth, height=8cm,     % size of the image
        grid = major,
        grid style={dashed, gray!30},
        ylabel=eiwitgehalte,
        xlabel=gewichtstoename(gram),
        legend style={at={(0.1,-0.1)}, anchor=north}
      ]
      \addplot[only marks] table  {data/santa.txt};
      \addplot [no markers, thick, red] table [y={create col/linear regression={y=y}}] {data/santa.txt};
      \addplot [mark=none, color=red] coordinates {
        (0,177) (0,189.9643)
      };
      \addplot [mark=none, color=red] coordinates {
        (10,231) (10,229.2143)
      };
      \addplot [mark=none, color=red] coordinates {
        (20,249) (20,268.4643)
      };
      \addplot [mark=none, color=red] coordinates {
        (30,348) (30,307.7143)
      };
      \addplot [mark=none, color=red] coordinates {
        (40,361) (40,346.9643)
      };
      \addplot [mark=none, color=red] coordinates {
        (50,384) (50,386.2143)
      };
      \addplot [mark=none, color=red] coordinates {
        (60,404) (60,425.4643)
      };

    \end{axis}
  \end{tikzpicture}
  \caption{Deviaties tot de regressierechte: aanname $x$ geeft extra informatie voor het voorspellen van $y$.}
	\label{fig:rendierenFiguur2}
\end{figure}

\begin{figure}
  \begin{tikzpicture}
    \begin{axis}[
        axis x line=middle,
        axis y line=middle,
        enlarge y limits=true,
        width=\textwidth, height=8cm,     % size of the image
        grid = major,
        grid style={dashed, gray!30},
        ylabel=eiwitgehalte,
        xlabel=gewichtstoename(gram),
      ]
      \addplot[only marks] table  {data/santa.txt};
      \addplot [mark=none, color=black] coordinates {
        (0,307.71) (60,307.71)
      };
      \addplot [mark=none, color=red] coordinates {
        (0,177) (0,307.71)
      };
      \addplot [mark=none, color=red] coordinates {
        (10,231) (10,307.71)
      };
      \addplot [mark=none, color=red] coordinates {
        (20,249) (20,307.71)
      };
      \addplot [mark=none, color=red] coordinates {
        (30,348) (30,307.71)
      };
      \addplot [mark=none, color=red] coordinates {
        (40,361) (40,307.71)
      };
      \addplot [mark=none, color=red] coordinates {
        (50,384) (50,307.71)
      };
      \addplot [mark=none, color=red] coordinates {
        (60,404) (60,307.71)
      };

    \end{axis}
  \end{tikzpicture}
  \caption{Deviaties tot de gemiddelde van y: aanname $x$ geeft geen informatie voor het voorspellen van $y$ ($\overline{y} =307.71$).}
	  \label{fig:rendierenFiguur3}
\end{figure}

\begin{table} \centering \small
  \begin{tabular}{@{}rrrl} \toprule
    $|R|$ & $R^{2}$ & Verklaarde variantie &  Interpretatie \\
    \midrule
    $< 0,3$       & $< 0,1$       & $< 10\%$    & zeer zwak \\
    $0,3 - 0,5$   & $0,1 - 0,25$ & $10 - 25\%$ & zwak \\
    $0,5 - 0,7$   & $0,25 - 0,5$  & $25 - 50\%$ & matig\\
    $0,7 - 0,85$  & $0,5 - 0,75$  & $50 - 75\%$ & sterk\\
    $0,85 - 0,95$ & $0,75 - 0,9$  & $75 - 90\%$ & zeer sterk\\
    $> 0,95$      & $> 0,9$       & $>90\%$     & uitzonderlijk(!)\\
    \bottomrule
  \end{tabular}
  \caption[Interpretatie van $R$ en $R^2$.]{Interpretatie van $R$ en $R^2$.}
  \label{tab:interpretatie-correlatiecoefficient}
\end{table}

\section{Conclusie}

Er bestaan verschillende soorten verbanden tussen variabelen. Wij zijn geïnteresseerd in monotone en lineaire verbanden. We beschikken hier over een correlatieco\"effici\"ent en lineaire regressie. Deze technieken mogen niet met nominale en ordinale variabelen gebruikt worden. Een kleine waarde $(=0)$ voor een maat voor verband betekent alleen dat het overeenkomend verband afwezig is: er kan een ander soort verband aanwezig zijn. Het gebruik van een spreidingsdiagram is dus altijd aan te raden.

Het feit dat twee variabelen gecorreleerd zijn, betekent niet dat de ene de oorzaak is van de andere.

\section{Samenvatting}

In dit hoofdstuk zijn verschillende technieken voorgesteld om na te gaan of er een verband bestaat tussen twee variabelen. De ene variabele noemen we de \emph{onafhankelijke}, de andere de \emph{afhankelijke} variabele. Wat we willen uitzoeken is of de waarde van de onafhankelijke variabele een impact heeft op die van de afhankelijke.

De technieken die we kunnen gebruiken (hetzij rekenkundige, hetzij voor visualisatie), hangen af van het meetniveau van de onderzochte variabelen. Tabel~\ref{tab:overzicht-2-variabelen} geeft een overzicht.

\begin{table}

  \begin{tabular}{llll}
    \toprule
    \multicolumn{2}{c}{\textbf{Meetniveau variabele}}             & \textbf{}                   & \textbf{}                                             \\
    \textbf{Onafhankelijke}       & \textbf{Afhankelijke}         & \textbf{Numeriek}           & \textbf{Visualisatie}                                 \\
    \midrule
    \multirow{3}{*}{Kwalitatief}  & \multirow{3}{*}{Kwalitatief}  & $\chi^2$                    & mozaïekdiagram                                        \\
    &                               & Cramér's V                  & geclusterd staafdiagram                               \\
    &                               &                             & rependiagram                                          \\
    \midrule
    \multirow{2}{*}{Kwalitatief}  & \multirow{2}{*}{Kwantitatief} & t-toets voor 2 steekproeven & boxplot                                               \\
    &                               &                             & \parbox{4.5cm}{(evt. staafdiagram gemiddelde met standaardafwijking)} \\
    \midrule
    \multirow{3}{*}{Kwantitatief} & \multirow{3}{*}{Kwantitatief} & covariantie                 & spreidings-/XY-diagram \\
    &                               & correlatiecoëfficiënt       & regressierechte                                       \\
    &                               & determinatiecoëfficiënt     &                                                      \\
    \bottomrule
  \end{tabular}
  
  \caption{Overzicht technieken voor de analyse van twee variabelen.}
  \label{tab:overzicht-2-variabelen}
\end{table}

\section{Oefeningen}
\label{sec:analyse op 2 variabelen-oefeningen}

De databestanden voor deze oefeningen zijn te vinden op Github (in de directory \emph{oefeningen/datasets}).

\subsection{Kwalitatief--kwalitatief}
\label{ssec:oef-kwal-kwal}

\begin{exercise}
  \label{ex:muziekwijn-analyse} % $\chi^{2}$ - handmatig}
  
  Marktonderzoek toont aan dat achtergrondmuziek in een supermarkt invloed kan hebben op het aankoopgedrag van de klanten. In een onderzoek werden drie methoden met elkaar vergeleken: geen muziek, Franse chansons en Italiaanse hits. Telkens werd het aantal verkochte flessen Franse, Italiaanse en andere wijnen geteld~\autocite{Ryan1998}.
  
  De onderzoeksdata bevindt zich in het csv-bestand MuziekWijn.
  
  Vragen:
  \begin{enumerate}
    \item Stel de correcte kruistabel op. Gebruik hiervoor het R-commando \textit{table} om de frequentietabel te bekomen.
    \item Bepaal de marginale totalen.
    \item Bepaal de verwachte resultaten.
    \item Bereken manueel de $\chi^{2}$ toetsingsgrootheid.  
    \item Bereken manueel de Cramér's V. Wat kan je hieruit besluiten?
  \end{enumerate}
\end{exercise}

\begin{exercise}
  \label{ex:muziekwijn-visualisatie}
  Gebruik dezelfde data.
  \begin{enumerate}
    \item Stel de percentages verkochte wijnen voor in een staafdiagram met de  muziekconditie= Geen.
    \item Stel de percentages verkochte wijnen voor in een geclusterd staafdiagram (clustered bar chart).
    \item Stel de percentages verkochte wijnen voor in rependiagram (stacked bar chart).
  \end{enumerate}
\end{exercise}

\begin{exercise}
  \label{ex:aardbevingen} %$\chi^{2}$ met R}
  Lees het databestand ``Aardbevingen.csv'' in. 	
  \begin{enumerate}
    \item Maak een histogram en een boxplot van de variabele ``Magnitudes''.
    \item Maak een lijngrafiek met het aantal aardbevingen per maand.
    \item Onderzoek of er een verband bestaat tussen de variabelen ``Type'' en ``Source''. Bereken ook de Cramér's V-waarde. Wat is de conclusie?
  \end{enumerate}
\end{exercise}

\begin{exercise}
  \label{ex:chisq-survey}
  Voor deze oefening maken we gebruik van de dataset \texttt{survey} die is meegeleverd met R. De dataset is samengesteld uit een bevraging onder studenten. Om deze te laden, doe het volgende:
  
  \begin{lstlisting}
  library(MASS)
  View(survey)  # Toont de "survey" dataset
  ?survey       # Help-pagina voor deze dataset met uitleg over de inhoud
  \end{lstlisting}
  
  Als je een foutboodschap krijgt bij het laden van de bibliotheek (eerste regel), betekent dit dat de package \texttt{MASS} nog niet geïnstalleerd is. Dit kan je alsnog doen via Tools > Install Packages en het invullen van de package-naam in het tekstveld.
  
  We willen de relatie onderzoeken tussen enkele discrete (nominale of ordinale) variabelen in deze dataset. Voor elke hieronder opgesomde paren, volg deze stappen:
  
  \begin{enumerate}[label=(\alph*)]
    \item Denk eerst eens na welke uitkomst je precies verwacht voor de opgegeven combinatie van variabelen.
    \item Stel een frequentietabel op voor de twee variabelen. De (vermoedelijk) onafhankelijke variabele komt eerst.
    \item Plot een grafiek van de data, bv.~geclusterde staafgrafiek, gestapelde staafgrafiek van relatieve frequenties, of een ``mozaïekgrafiek'' (eenvoudig met \texttt{plot(table(data\$col1, data\$col2))}).
    \item Als je de grafiek bekijkt, verwacht je dan een eerder hoge of eerder lage waarde voor de $\chi^2$-statistiek? Waarom?
    \item Bereken de $\chi^2$-statistiek en de kritieke grenswaarde $g$ (voor significantieniveau $\alpha = 0.05$)
    \item Bereken de $p$-waarde
    \item Moeten we de nulhypothese aanvaarden of verwerpen? Wat betekent dat concreet voor de relatie tussen de twee variabelen?
  \end{enumerate}
  
  Hieronder zijn de te onderzoeken variabelen opgesomd. De vermoedelijke onafhankelijke variabele komt telkens eerst.
  
  \begin{enumerate}
    \item \texttt{Exer} (sporten) en \texttt{Smoke} (rookgedrag)
    \item \texttt{W.Hnd} (de hand waarmee je schrijft) en \texttt{Fold} (de hand die bovenaan komt als je de armen kruist)
    \item \texttt{Sex} (gender) en \texttt{Smoke}
    \item \texttt{Sex} en \texttt{W.Hnd}
  \end{enumerate}
\end{exercise}

\begin{exercise}
  \label{ex:chisq-aids2}
  Laad de dataset \texttt{Aids2} uit package \texttt{MASS} (zie Oefening~\ref{ex:chisq-survey}) die informatie bevat over 2843 patiënten die vóór 1991 in Australië met AIDS besmet werden. Deze dataset werd in detail besproken door~\textcite{Ripley2007}. Onderzoek of er een relatie is tussen de variabele geslacht (\texttt{Sex}) en de manier van besmetting (\texttt{T.categ}).
  
  \begin{enumerate}
    \item Ga op de gebruikelijke manier te werk: visualiseren van de data, $\chi^2$, $g$ en $p$-waarde berekenen ($\alpha = 0,05$), en tenslotte een conclusie formuleren.
    \item Bepaal de gestandaardiseerde residuën om te bepalen welke categorieën extreme waarden bevatten.
  \end{enumerate}
  
\end{exercise}

\begin{exercise}
  \label{ex:chisq-digimeter}
  
  Elk jaar voert Imec (voorheen iMinds) een studie uit over het gebruik van digitale technologieën in Vlaanderen, de Digimeter~\autocite{Vanhaelewyn2016}. In deze oefening zullen we nagaan of de steekproef van de Digimeter 2016 ($n = 2164$) representatief is voor de bevolking wat betreft de leeftijdscategorieën van de deelnemers.
  
  In Tabel~\ref{tab:digimeter2016} worden de relatieve frequencies van de deelnemers weergegeven. De absolute frequenties voor de verschillende leeftijdscategorieën van de Vlaamse bevolking worden samengevat in Tabel~\ref{tab:leeftijd-vlaanderen}. Deze gegevens zijn ook te vinden in bijgevoegd CSV-bestand \texttt{oefeningen/data/bestat-vl-ages.csv}.
  
  \begin{enumerate}
    \item De tabel met leeftijdsgegevens van de Vlaamse bevolking als geheel heeft meer categorieën dan deze gebruikt in de Digimeter. Maak een samenvatting zodat je dezelfde categorieën overhoudt dan deze van de Digimeter. Tip: dit gaat misschien makkelijker in een rekenblad dan in R.
    \item Om de goodness-of-fit test te kunnen toepassen hebben we de absolute frequenties nodig van de geobserveerde waarden in de steekproef. Bereken deze.
    \item Bereken ook de verwachte percentages ($\pi_{i}$) voor de populatie als geheel.
    \item Voer de goodness-of-fit test uit over de verdeling van leeftijdscategorieën in de steekproef van de Digimeter. Is de steekproef in dit opzicht inderdaad representatief voor de Vlaamse bevolking?
  \end{enumerate}
\end{exercise}

\begin{table}
  \caption{Frequenties van de leeftijd van deelnemers aan de iMec Digimeter 2016 en de Vlaamse bevolking.}
  \label{tab:frequenties-leeftijden}
  \centering
  \begin{tabular}{cc}
    \textbf{Leeftijdsgroep} & \textbf{Percentage} \\ \midrule
    15-19 & 6,6\% \\
    20-29 & 14,2\% \\
    30-39 & 15,0\% \\
    40-49 & 16,3\% \\
    50-59 & 17,3\% \\
    60-64 & 7,3\% \\
    64+   & 23,2\% \\
  \end{tabular}
  \subcaption{Percentage van deelnemers aan de Digimeter 2016 van iMec ($n = 2164$), opgedeeld per leeftijdscategorie. \autocite{Vanhaelewyn2016}}
  \label{tab:digimeter2016}
  
  \centering
  \begin{tabular}{cc}
    \textbf{Leeftijdsgroep} & \textbf{Aantal} \\ \midrule
    –5            &     352017      \\
    5-9           &     330320      \\
    10-14          &     341303      \\
    15-19          &     366648      \\
    20-24          &     375469      \\
    25-29          &     387131      \\
    30-34          &     401285      \\
    35-39          &     409587      \\
    40-44          &     458485      \\
    45-49          &     493720      \\
    50-54          &     463668      \\
    55-59          &     413315      \\
    60-64          &     379301      \\
    65-69          &     299152      \\
    70-74          &     279789      \\
    75-79          &     249260      \\
    80-84          &     182352      \\
    85-89          &     104449      \\
    90-94          &      29888      \\
    95-99          &      7678       \\
    100+           &       923
  \end{tabular}
  \subcaption{Absolute frequentie van de Vlaamse bevolking per leeftijdscategorie. Bron: BelStat (\url{https://bestat.economie.fgov.be/bestat/}, C01.1: Bevolking volgens verblijfplaats (provincie), geslacht, positie in het huishouden (C), burgerlijke staat en leeftijd (B)).}
  \label{tab:leeftijd-vlaanderen}
    
\end{table}

\subsection{Kwalitatief--kwantitatief}
\label{ssec:oef-kwal-kwant}

\begin{exercise}
  \label{oef:casus-akin2016-toets}
  
  In Oefening~\ref{oef:casus-akin2016-1var} en volgende hebben we de resultaten van performantiemetingen voor persistentiemogelijkheden in Android geanalyseerd~\autocite{Akin2016}. Er werden experimenten uitgevoerd voor verschillende combinaties van hoeveelheid data (klein, gemiddeld, groot) en persistentietype (GreenDAO, Realm, SharedPreferences, SQLite). Voor elke hoeveelheid data hebben we kunnen bepalen welk persistentietype het beste resultaat gaf.
  
  Nu gaan we uitzoeken of het op het eerste zicht beste persistentietype ook \emph{significant} beter is dan de concurrentie.
  
  Concreet: ga aan de hand van een toets voor twee steekproeven voor elke datahoeveelheid na of het gemiddelde van het best scorende persistentietype \emph{significant lager} is dan het gemiddelde van enerzijds het \emph{tweede} beste en anderzijds het slechtst scorende type.
  
  Kunnen we de conclusie aanhouden dat voor een gegeven datahoeveelheid één persistentietype het beste is, d.w.z.~significant beter is dan gelijk welk ander persistentietype?
\end{exercise}

\begin{exercise}
  Een groot aantal studenten heeft deelgenomen aan een test die in verschillende opeenvolgende sessies werd georganiseerd. Omdat het opstellen van een aparte opgave voor elke sessie praktisch onhaalbaar was, is telkens dezelfde opgave gebruikt. Eigenlijk bestaat er dus het gevaar dat studenten na hun sessie info konden doorspelen aan de groepen die nog moesten komen. De latere groepen hebben dan een voordeel ten opzichte van de eerste. Blijkt dit ook uit de cijfers?
  
  Het bestand \texttt{puntenlijst.csv} bevat alle resultaten van de test. Elke groep wordt aangeduid met een letter, in de volgorde van de sessie.
  
  \begin{itemize}
    \item Dag 1: sessies A, B
    \item Dag 2: sessies C, D, E
    \item Dag 3: sessies F, G, H
  \end{itemize}
  
  Sessies A en B zijn doorgegaan op een andere campus, dus er zou kunnen verondersteld worden dat er weinig tot geen communicatie is met de studenten van de andere sessies.
  
  Als er info met succes doorgespeeld werd, dan verwachten we dat de scores van de groepen die later komen significant beter zijn dan de eerste.
  
  Merk op dat de omgekeerde redenering niet noodzakelijk geldt: als blijkt dat het resultaat van de latere sessies inderdaad significant beter blijkt, dan betekent dat niet noodzakelijk dat de oorzaak (enkel) het doorspelen van informatie is. Er kunnen ook andere oorzaken zijn (bv.~``zwakkere'' klasgroepen zijn toevallig eerder geroosterd).
  
  \begin{enumerate}
    \item Ga op verkenning in de data. Bereken de gepaste centrum- en spreidingsmaten voor de dataset als geheel en voor elke sessie afzonderlijk.
    
    \item Maak een staafgrafiek van de gemiddelde score per sessie. Is dit voldoende om een beeld te vormen van de resultaten? Waarom (niet)?
    
    \item Maak een boxplot van de scores opgedeeld per groep. Vergelijk onderling de hieronder opgesomde sessies. Denk je dat er een significant verschil is tussen de resultaten? Wordt ons vermoeden dat er informatie doorgespeeld wordt bevestigd?
    
    \begin{itemize}
      \item A en B
      \item C, D en E
      \item F, G en H
      \item C en H
      \item A en H
    \end{itemize}
    
    \item Ga door middel van een geschikte statistische toets na of de verschillen tussen de hierboven opgesomde groepen ook \emph{significant} zijn. Kunnen we concluderen dat de latere groepen beter scoren of niet?
  \end{enumerate}
\end{exercise}

\subsection{Kwantitatief--kwantitatief}
\label{ssec:oef-kwant-kwant}

\begin{exercise}
  \label{ex:test-examen}
  In onderstaande tabel vindt men voor elke rij (= persoon) het resultaat van een test en zijn examenscore. Gevraagd:
  \begin{itemize}
    \item Bepaal handmatig de regressierechte $\beta_{0} + \beta_{1} x$. 
    \item Bepaal handmatig de correlatie- en determinatieco\"effici\"ent ($R, R^{2}$) 
    \item Geef uitleg bij de gevonden statistieken.
  \end{itemize}
  
  % \begin{table}
  \centering
  \begin{tabular}{@{}rr@{}} \toprule
    Resultaat Test ($X$) & Examenresultaat ($Y$) \\
    \midrule
    10 & 11 \\
    12 & 14 \\
    8 & 9 \\
    13 & 13 \\
    9 & 9 \\
    10 &  9 \\
    7 & 8 \\
    14 & 14 \\
    11 & 10 \\
    6 & 6  \\
    \bottomrule
  \end{tabular}
  \captionof{table}{Scores test en examen voor aantal personen}
  \label{tab:testExamen}
  % \end{table}	
\end{exercise}

\begin{exercise}
  \label{ex:scatter-correlatiecoeff}
  Gegeven 6 scatterplots in Figuur~\ref{fig:correlaties} en onderstaande correlatieco\"effici\"enten. Match de co\"effici\"enten met de scatterplots. Er is dus één scatterplot waarvan geen correlatie gegeven staat hieronder.
  \begin{itemize}
    \item $r_{1}$ = 0.6
    \item $r_{2}$ = 0
    \item $r_{3}$ = -0.9
    \item $r_{4}$ = 0.9
    \item $r_{5}$ = 0.3
  \end{itemize}
  %\begin{figure}[h!]
  %	\centering
  \includegraphics[width=1.10\textwidth]{correlaties.png}
  \captionof{figure}{Correlaties}
  \label{fig:correlaties}
  %\end{figure}
\end{exercise}

\begin{exercise}
  \label{ex:cats}
  Lees het databestand ``Cats.csv'' in. 
  \begin{enumerate}
    \item Voer een lineaire regressieanalyse uit op de variabelen Lichaamsgewicht (\texttt{Bwt}, afhankelijke variabele) en Gewicht hart (\texttt{Hwt}, onafhankelijke variabele).
    \item Maak een spreidingsdiagram van beide variabelen.
    \item Bereken en teken de regressielijn.
    \item Bereken de correlatie- en de determinatiecoëfficiënt.
    \item Geef een interpretatie van deze resultaten.
  \end{enumerate}
\end{exercise}

\begin{exercise}
  \label{ex:cats-per-geslacht}
  Gebruik dezelfde data als in vorige oefening.
  \begin{enumerate}
    \item Voer een lineaire regressieanalyse uit op de variabelen Lichaamsgewicht (Bwt) en Gewicht hart (Hwt) per geslacht.
    \item Maak een spreidingsdiagram van beide variabelen voor elk van de geslachten.
    \item Bereken en teken telkens de regressielijn.
    \item Bereken de correlatie- en de determinatiecoëfficiënt.
    \item Geef een interpretatie aan deze resultaten.
  \end{enumerate}
\end{exercise}

\begin{exercise}
  \label{ex:pizza}
  Lees het databestand ``Pizza.csv'' in.
  \begin{enumerate}
    \item Voer een volledige lineaire regressieanalyse uit op de variabelen Rating en CostPerSlice. Trek hieruit de juiste conclusies en ga deze ook grafisch na.
    \item Onderzoek een mogelijk verband tussen Rating en Neighbourhood. Welke methode kan je hiervoor gebruiken? Kan je de gegevens van Rating hiervoor in dezelfde vorm gebruiken?
    \item Geef een interpretatie aan deze resultaten.
    \item Stel de kruistabel grafisch voor met een staafdiagram.  Voorzie een legende.
  \end{enumerate}
\end{exercise}


%%%%%%%%%%%%%%%%%%%%%%%%%%%%%%%%%%%%%%%%%%%%%%%%%%%%%%



\subsection{Antwoorden op geselecteerde oefeningen}
\label{ssec:analyse-2-variabelen-oplossingen}

\paragraph{Oefening~\ref{ex:muziekwijn-analyse}}

$\chi^2 \approx 18,2792$, Cramér's $V \approx 0,1939$

\paragraph{Oefening~\ref{ex:chisq-survey}}

\begin{enumerate}
  \item \texttt{Exer}/\texttt{Smoke}: $\chi^2 = 5.4885$, $g = 12.59159$, $p = 0.4828422$
  \item \texttt{W.Hnd}/\texttt{Fold}: $\chi^2 = 1.581399$, $g = 5.9915$, $p = 0.454$
  \item \texttt{Sex}/\texttt{Smoke}: $\chi^2 = 3.554$, $g = 7.8147$, $p = 0.314$
  \item \texttt{Sex}/\texttt{W.Hnd}: $\chi^2 = 0.236$, $g = 3.8415$, $p = 0.627$
\end{enumerate}

\paragraph{Oefening~\ref{ex:chisq-aids2}} $\chi^2 = 1083.372914$, $g = 14.067140$, $p \approx 1.157 \times 10^{-229}$

\paragraph{Oefening~\ref{ex:chisq-digimeter}} $\chi^2 \approx 6.6997$ ($df = 6$), $g \approx 12.5916$, $p \approx 0.3495$


\paragraph{Oefening~\ref{oef:casus-akin2016-toets}}

Tabel~\ref{tab:akin2016-resultaten-ttoets} geeft een overzicht met voor elke datasetgrootte het beste en tweede beste persistentietype (op basis van het steekproefgemiddelde). De conclusie van~\textcite{Akin2016}, dat \emph{Realm} het performantste persistentietype is, blijft overeind, maar voor de kleine datasets is het verschil niet significant.

Merk op dat we hier niet expliciet vooraf een significantieniveau gekozen hebben. Voor $\alpha = 0,1$, $0,05$ of zelfs $0,01$, kunnen we echter dezelfde conclusie trekken.

\begin{table}
  \begin{center}
    \begin{tabular}{llll}
      \toprule
      \textbf{Grootte} & \textbf{Beste} & \textbf{2e beste} & \textbf{$p$-waarde} \\ \midrule
      Small            & Realm          & SharedPreferences & 0.1699     \\
      Medium           & Realm          & GreenDAO          & 0.0002506  \\
      Large            & Realm          & SQLite            & 0.0017     \\ \bottomrule
    \end{tabular}
  \end{center}
  \caption{Resultaten $t$-toets voor de beste en 2e beste persistentietype op basis van steekproefgemiddelde~\autocite{Akin2016}.}
  \label{tab:akin2016-resultaten-ttoets}
\end{table}

\paragraph{Oefening~\ref{ex:test-examen}}

\begin{itemize}
  \item $\beta_{0} \approx 0,6333$, $\beta_{1} \approx 0.9667$
  \item $Cov \approx 6,444$, $R \approx 0,9352$, $R^2 \approx 0,8747$
\end{itemize}

\paragraph{Oefening \ref{ex:cats} en \ref{ex:cats-per-geslacht}}

\begin{center}
  \begin{tabular}{lrrrrr}
  	\toprule
    \textbf{Selectie} & \textbf{$\beta_{0}$} & \textbf{$\beta_{1}$} & \textbf{$Cov$} & \textbf{$R$} & \textbf{$R^2$} \\
    \midrule
  	Hele dataset & -0.3511 & 4.0318 & 0.9496 & 0.8041 & 0.6466 \\
  	Female       &  2.9813 & 2.6364 & 0.1979 & 0.5320 & 0.2831 \\
  	Male         & -1.1768 & 4.3098 & 0.9419 & 0.7930 & 0.6289 \\
    \bottomrule
  \end{tabular}
\end{center}


\include{7_tijdreeksen}

\begin{appendices}
%\chapter{Logistisch regressie}

\section{Inleiding}

In dit onderzoek gaan we een andere vorm van verband zoeken tussen variabelen waarbij de afhankelijke variabele twee waarden kan aannemen. 

\begin{example}
	\label{ex:slagen}
	Stel dat je wil nagaan of het student al dan niet zal slagen voor het examen onderzoekstechnieken. We zijn dus ge\"interesseerd in de voorspelling (door
	onafhankelijke variabelen) van de kans dat een student in de categorie 'examen slagen' of in de categorie 'niet slagen' valt. 
\end{example}

In bovenstaand voorbeeld zal een 'gewone' lineaire regressie analyse 
algemeen wel de juiste richting van de $\beta$-co\"efficiënten opleveren. Maar de schatting is niet helemaal correct, omdat enkele belangrijke regressie assumpties geschonden worden, zoals de normaliteitsassumptie en de assumptie van homoscedasticiteit. Het grootste probleem is evenwel dat de door lineaire regressie voorspelde kansen groter kunnen zijn dan 1 en kleiner dan 0 en dat is niet te interpreteren.

Bij logistische regressie gaan we werken met kansverhoudingen. In voorbeeld \ref{ex:slagen} hebben we een kansverdeling dat een student wel slaagt $(y = 1)$ met kans $p$ gedeeld door de kans om niet te slagen $(y=0)$ met kans $q = 1-p$:
\[ 
	\textnormal{verhouding} = \frac{p}{1-p}
\]

We wensen dat de waarden van de verhouding gaan van $- \infty$ tot $\infty$. Daarom gaan we de natuurlijke logaritme nemen van de verhouding. Om de functie te tekenen van de logaritmische functie kan je onderstaande code gebruiken. 

\lstinputlisting{data/logcurve.R}

Als we de onafhankelijke variabelen $X_1$, $X_2$  \dots $X_n$ noemen,dan ziet het logistische model er in formulevorm als volgt uit:
\[ 
	log(\frac{p}{1-p}) = \beta_0 + \beta_1 X_1 + \cdots + \beta_n X_n 
\]

We kunnen het kansmodel ook herschrijven (afzonderen van de $p$):

\begin{eqnarray}
	p = \frac{e^{\beta_0 + \beta_1 X_1 + \cdots + \beta_n X_n }}{1+ e^{\beta_0 + \beta_1 X_1 + \cdots + \beta_n X_n }}
	\label{eq:prob}
\end{eqnarray} 



We kunnen het kansmodel dan ook herschrijven (afzonderen van de $(1-p)$):
\[ 
1-p = \frac{1}{1+ e^{\beta_0 + \beta_1 X_1 + \cdots + \beta_n X_n }}
\]

Aan deze formules is af te lezen dat de kansen $p$ en $1-p$ bij elkaar opgeteld gelijk zijn aan \'e\'en.
Verder is te zien dat de kansen $p$ en $1-p$ afhankelijk zijn van de variabelen $X_1, X_2 \cdots X_n$, maar dat deze afhankelijkheid niet lineair is. Een logistische regressielijn ziet er dus niet als een rechte lijn
uit, maar als een S-vormige curve. (TODO: hier zou een tekening moeten komen van de sigmo\"ide functie).

Bij logistische regressie gaan we dus op zoek naar goede waarden voor $\beta_0 \cdots \beta_n$ die het model zo goed mogelijk beschrijven zodat we ook voorspellingen kunnen doen. Dit kan in R makkelijk door de methode \texttt{glm}.

Om de logistische functie te tekenen kunnen we gebruik maken van onderstaande code (twee parameters).

\lstinputlisting{data/sigmoid.R}

\subsection{Intu\"itie rond de oplossingsmethode}
Om de waarden van $\beta_0, \beta_1 \cdots \beta_n$ te bepalen gaan we deze keer niet gebruik maken van de kleinste kwadratenmethode (zie sectie \ref{sec:regressie}), maar wel van een meer algemene methode : maximum likelihood methode \index{maximum likelihood}. Hierbij proberen we waarden voor de $\beta_i$ te vinden die ervoor zorgen dat in de trainingsdataset (de dataset die we gebruiken om de parameters $\beta_i$ te bepalen) de elementen die een label 1 krijgen zo goed mogelijk benaderd worden door 1 in vergelijking \ref{eq:prob} en de elementen die een label 0 krijgen zo goed mogelijk benaderd worden door 0. Dit doen we door volgende vergelijking te maximaliseren. 

\begin{equation}
	\Pi_{i: y_i=1} p(x_i) \Pi_{j= y_j = 0} (1 - p(x_j)) 
\end{equation}

De oplossingsmethode wordt ge\"implementeerd in R en is buiten de scope van deze cursus. We refereren de ge\"interesseerde lezer naar \cite{Hastie2009} voor meer informatie rond deze methode. 


\subsection{Performantie van het model}
Er zijn een aantal performantiematen die in rekening moeten gebracht worden wanneer aan logistische regressie gedaan wordt. 

\subsubsection{Akaike Information Criteria}
\index{Akaike Information Criteria}
Dit is een statistiek die wat overeenkomt met $R^2$ vanuit sectie \ref{sec:determinatiecoef}. Het geeft aan hoe goed de opgenomen variabelen in ons model het resultaat weergeven en we wensen die AIC zo laag mogelijk te houden. Het geeft ons dus een inkijk in het gebruik van de variabelen en zorgt ervoor dat we niet te veel variabelen in ons model opnemen. 

De waarde van de AIC is op zichzelf niet van belang, maar wordt vooral gebruikt wanneer de verschillende modellen wilt vergelijken: dan neem je best het model met de laagste AIC. 

\subsubsection{Null deviance}
\index{null deviance}
Dit is een indicatie hoe goed het model de data fit waarbij alleen gebruik gemaakt wordt van de intercept. Hoe lager deze waarde hoe beter.

\subsubsection{Residual deviance}
\index{residual deviance}
Dit is een indicatie hoe goed het model de data fit, waarbij de onafhankelijke variabelen toegevoegd zijn. Hier geldt ook, hoe lager deze waarde hoe beter.

Bij de output in R krijg je bovenstaande waarden. Waar je als onderzoeker vooral ge\"interesseerd in bent is een lage AIC en een een significante daling van the Null Deviance naar de Residual deviance. 


\section{Logistische regressie in R}

We gaan het voorbeeld nemen dan in Kaggle \footnote{\href{https://www.kaggle.com/c/titanic/data}{https://www.kaggle.com/c/titanic/data}} gegeven wordt. Het bevat de informatie rond de mensen die de reis van de titanic ondernomen hebben en het overleefd hebben of niet. De analyse komt uit het blog artikel \cite{michy}, maar is wat aangepast aangezien niet alle conclusies in dit artikel kloppen. 

\subsection{Data cleaning}

Importeer de data, en zorg ervoor dat de juiste types voor de juiste variabelen gekozen zijn (Sex is bijvoorbeeld een \texttt{factor} variabele)

We gaan de data opruimen en kijken welke parameters er in het model kunnen zitten. We gaan dit na door te kijken welke parameters in de dataset niet voldoende aanwezig zijn. 

\begin{lstlisting}
sapply(train,function(x) sum(is.na(x)))
sapply(train, function(x) length(unique(x)))
missmap(train, main = "Missing values vs observed")
\end{lstlisting}
Hierbij zien we dat de variabelen \texttt{cabin} te weinig waarden bevat. Ook \texttt{tickets} laten we vallen aangezien dit weinig invloed zal hebben. 
We nemen bijgevolg een subset van de data en gaan hiermee aan de slag. 
 
\begin{lstlisting}
data <- subset(train,select=c(2,3,5,6,7,8,10,12))
\end{lstlisting} 

We moeten ervoor zorgen dat de andere data elementen die er te kort zijn zinvol ingevuld worden. Je hebt hier verschillende methodieken voor. Je kan vervangen door:
\begin{itemize}
	\item het gemiddelde
	\item de mediaan
	\item de modus
	\item een elementen uit een bepaalde distributie
\end{itemize} 

We gaan voor de optie om de \texttt{NA} elementen te vervangen door hun gemiddelde. 

\begin{lstlisting}
data$Age[is.na(data$Age)] <- mean(data$Age,na.rm=T)
\end{lstlisting}

Voor de nominale en ordinale variabelen kunnen we kijken hoe ze gecodeerd worden door R. 
\begin{lstlisting}
contrasts(data$Sex)
\end{lstlisting}

\subsection{Fitten van de data in R}
We gaan de data opsplitsen in een trainingsset en een testset. We gaan hiervoor de library \texttt{caTools} gebruiken. 

\begin{lstlisting}
install.packages('caTools')
library(caTools)
\end{lstlisting}

Nu kunnen we het model laten opbouwen door R.

\begin{lstlisting}
model <- glm(Survived ~.,family=binomial(link='logit'),data=train)
summary(model)
\end{lstlisting}

Je krijgt volgende output na het uitvoeren van dit commando:
\begin{description}
	\item[Coefficient] De schatting voor de co\"effici\"ent in het model
	\item[Std. error] De standard errors op de co\"effci\"ent. 
	\item[z-statistic] Dit komt overeen met de $\frac{\beta_i}{SE(\beta_i)}$. 
	\item[P-value] De p-waarde geassocieerd met de null-hypothese van de co\"effci\"ent. 
\end{description}
Deze laatste twee getallen hebben wat verduidelijking nodig. Voor elke $\beta_i$ wordt een null-hypothese $H^i_0$ opgesteld. Deze stelt dat 
\[ 
	p(X_i) = \frac{e^{\beta_0 + \cdots \beta_{i-1} + \beta_{i+1} + \cdots \beta_n}}{1+e^{\beta_0 + \cdots \beta_{i-1} + \beta_{i+1} + \cdots \beta_n}}
\]
wat eigenlijk neerkomt dat het model niet afhangt van $X_i$. Wanneer de $|z|$ groot genoeg is en bijgevolg de $p$-waarde klein is mag de $H^i_0$ verworpen worden en kunnen we stellen dat $X_i$ wel degelijk van belang is in het model. 

Om een betrouwbaarheidsinterval te bouwen rond de geschatte parameter $\beta_i$ kan je gewoonweg volgende formule gebruiken:
\[
	\beta_i +z_i  \times SE(\beta_i)
\]

Als output krijgen we:
\begin{lstlisting}
Coefficients:
(Intercept)      Pclass2      Pclass3    Sexfemale          Age       SibSp0       Parch1         Fare    EmbarkedC    EmbarkedQ  
1.36178     -0.96344     -2.19975      2.67728     -0.04503     -0.49519      0.08984      0.00105      0.37631      0.68404  

Degrees of Freedom: 666 Total (i.e. Null);  657 Residual
Null Deviance:	    887.4 
Residual Deviance: 582.5 	AIC: 602.5
\end{lstlisting}

En met \texttt{summary} van het model bekomen we volgende output:

\begin{lstlisting}
Call:
glm(formula = Survived ~ ., family = binomial(link = "logit"), 
data = dresstrain)

Deviance Residuals: 
Min       1Q   Median       3Q      Max  
-2.4971  -0.6377  -0.3730   0.6240   2.5854  

Coefficients:
Estimate Std. Error z value Pr(>|z|)    
(Intercept)  1.361775   0.495174   2.750  0.00596 ** 
Pclass2     -0.963440   0.339019  -2.842  0.00449 ** 
Pclass3     -2.199753   0.335770  -6.551  5.7e-11 ***
Sexfemale    2.677281   0.224722  11.914  < 2e-16 ***
Age         -0.045028   0.009096  -4.951  7.4e-07 ***
SibSp0      -0.495189   0.252907  -1.958  0.05023 .  
Parch1       0.089835   0.304233   0.295  0.76778    
Fare         0.001050   0.002259   0.465  0.64190    
EmbarkedC    0.376309   0.277878   1.354  0.17566    
EmbarkedQ    0.684037   0.364547   1.876  0.06060 .  
---
Signif. codes:  0 '***' 0.001 '**' 0.01 '*' 0.05 '.' 0.1 ' ' 1

(Dispersion parameter for binomial family taken to be 1)

Null deviance: 887.35  on 666  degrees of freedom
Residual deviance: 582.46  on 657  degrees of freedom
AIC: 602.46

Number of Fisher Scoring iterations: 5
\end{lstlisting}

Hieruit kunnen we volgende dingen zeggen:
\begin{itemize}
	\item SibSp, Parch1, Fare, EmbarkedC en EmbarkedQ zijn niet statisch significant. 
	\item We zien dat Sexfemale erg significant. De positieve co\"effici\"ent voor sexFemale toont aan dat vrouw zijn ervoor zorgt dat je meer kans hebt op overleven.  
\end{itemize}

Bij de \texttt{anova} wordt getoond wat het effect is van een variabele een per een toe te voegen aan het model. \textbf{TODO: dit nog eens deftig interpreteren.}


De volledige code kan je hier nog eens bekijken:
\lstinputlisting{data/titanicregression.R}

\section{Oefeningen}

\begin{exercise}
	
	\begin{itemize}
		\item Beschouw de dataset \texttt{Smarker} van de package \texttt{ISLR}. 
		Deze dataset bestaat uit
		het rendement voor de S \& P 500 aandelenindex over 1250 dagen, van 
		begin 2001 tot eind 2005. Voor elke datum hebben we het  retourneer percentage opgenomen voor elk van de vijf vorige handelsdagen (Lag1 t.e.m. Lag5). We hebben ook het Volume opgenomen (het aantal verhandelde aandelen) en het percentage van vandaag. Daarnaast hebben we ook opgenomen of de markt daalde of steeg.
		\item Schrijf de algemene statistieken uit van de verschillende variabelen. 
		\item Probeer eens een plot te maken die aanduid of het volume stijgt of daalt met de jaren. 
		\item We gaan proberen een logistisch model op te stellen dat het stijgen of dalen in functie van lag1 t.e.m. lag5 en volume uitzet. Gebruik hiervoor het commando glm.
		\item Analyseer de co\"effici\"enten. Wat kan je erover zeggen?
		\item Kijk nu eens hoe goed het model de dataset zelf voorspelt. Dit kan je doen door aan het predict commando geen dataset mee te geven. 
		\item Zet de voorspelde probabiliteit om in juiste labels ($\geq 0.5$ up)
		\item  Cree\"er een matrix die de vals positieven en ware positieven e.a. uitzet t.o.v. elkaar. Gebruik hiervoor de methode table. 
		\item Wat kom je hier nu voor uit?
	\end{itemize}
\end{exercise}

\begin{exercise}
	
	\begin{itemize}
		\item Beschouw dezelfde dataset als hierboven, maar train nu de dataset met de elementen van voor 2005 en gebruik als testset de elementen boven 2005. Wat kom je nu uit?
		\item Probeer nu het model aan te passen door de juiste variabelen te kiezen om mee te nemen in het model. 
		\item Wanneer je tevreden bent met het model, probeer dan een voorspelling te doen van een willekeurige dataset.
	\end{itemize}
\end{exercise}



\include{A_notatie}

\clearpage
\addcontentsline{toc}{chapter}{\textcolor{maincolor}{\IfLanguageName{dutch}{Bibliografie}{Bibliography}}}
\printbibliography

\clearpage
\addcontentsline{toc}{chapter}{\textcolor{maincolor}{Index}}
\printindex

\end{appendices}
\end{document}
