\chapter{Aan de slag}
\label{ch:aan-de-slag}

\section{Studiewijzer}

De studiewijzer geeft een overzicht van de belangrijkste informatie over deze cursus, o.a.~leerdoelen, lesmateriaal, weekplanning en leeraanwijzingen. Lees alles aandachtig door!

\subsection{Doel en plaats van de cursus in het curriculum}

Deze cursus is een inleiding op wat tegenwoordig vaak \emph{data science} genoemd wordt. Het doel is om je wegwijs te maken in het correct verzamelen, verwerken en analyseren van numerieke data en daar een onderbouwd onderzoeksverslag over te schrijven.

In de eerste plaats is dit een voorbereiding op de bachelorproef, waar je deze technieken in de praktijk zal moeten omzetten. Maar ook na je afstuderen blijft de kennis die je in deze cursus opdoet waardevol. Succesvolle bedrijven nemen beslissingen, niet op basis van buikgevoel of intuïtie, maar door het verzamelen en analyseren van data. Aan de hand van de technieken die hier toegelicht worden, heb je voldoende achtergrond om vragen te beantwoorden als:

\begin{itemize}
  \item Is een (web)applicatie snel genoeg voor de gebruikers? Is de gebruikerservaring consistent, of zit er grote variatie op responstijden?
  \item Als je twee systemen moet vergelijken, zij het software of hardware, welke van de twee is het meest performant? Is het verschil tussen beide significant, of kunnen verschillen in de metingen te wijten zijn aan het toeval of andere factoren?
  \item Wanneer moeten aankopen van nieuwe apparatuur (bv.~harde schijven, servers, geheugen, enz.) ingepland worden, op basis van historische gebruiksgegevens?
\end{itemize}

De competenties die je in deze cursus verwerft hebben ook buiten de informatica hun nut. Je leert immers kritisch omgaan met data en informatie, en hoe die correct te analyseren en interpreteren. In het politieke en maatschappelijke debat worden moedwillig beweringen gedaan die aantoonbaar fout zijn of die de waarheid proberen ``om te buigen.'' De term die in dat verband vaak de kop opsteekt is `Fake News'. Een manier om je hiertegen te wapenen is kritisch omgaan met de informatie die verspreid wordt. Daardoor kan vaak de achterliggende reden van die desinformatie duidelijk gemaakt worden.

Statistiek  en data science zijn dan ook onontbeerlijk om (i) te data correct te analyseren en tot onderbouwde conclusies te komen en (ii) zelf onderzoeken uit te voeren waarbij je onderbouwde conclusies de wereld kan in sturen. 

\subsection{Leerdoelen en competenties}

\begin{itemize}
  \item Kan begrippen, formules, stellingen en de uitwerking ervan uit de beschrijvende en inductieve statistiek benoemen en verklaren
  \item Kan formules, stellingen uit de beschrijvende en inductieve statistiek in onderzoeksvraagstukken correct toepassen
  \item Kan data analyseren met statistische software
  \item Kan een gestructureerd wetenschappelijk document schrijven en voorzien van referenties in \LaTeX{}
  \item Kan de wetenschappelijke methode vergelijken met niet-wetenschappelijke onderzoeksmethodes en daarbij voor- en nadelen opsommen 
\end{itemize}

Deze vind je ook terug in de studiefiche.

\subsection{Leerinhoud}

Verder in dit hoofdstuk vind je instructies voor het installeren van de nodige software, en een korte inleiding op het werken met R, een programmeertaal voor data-analyse.

Hoofdstuk~\ref{ch:onderzoeksproces} geeft een inleiding op het verloop van een typisch onderzoeksproces en introduceert enkele basisconcepten van data-analyse, waaronder het nemen van een steekproef uit een populatie, variabelen en meetniveaus.

Hoofdstuk~\ref{ch:analyse1var} behandelt de analyse van een enkele variabele, meer bepaald centrum- en spreidingsmaten, samen met geschikte visualisatietechnieken voor elke soort variabele.

Hoofdstuk ~\ref{ch:centrale-limietstelling} herhaalt kort de stof rond kansverdelingen, waarna de centrale limietstelling aan bod komt met een onmiddelijke toepassing via betrouwbaarheidsintervallen. 

Hoofdstuk~\ref{ch:toetsingsprocedures} gaat hierop verder met de algemene werkwijze voor het voeren van statistische toetsen, en specifiek met toetsen voor uitspraken over het gemiddelde van een populatie: de $z$-toets en de $t$-toets.

Waar de vorige hoofdstukken telkens één variabele apart beschouwden, bekijkt Hoofdstuk~\ref{ch:analyse2var} verschillende technieken om verbanden tussen twee variabelen te leggen, afhankelijk van het variabeletype.

Hoofdstuk~\ref{ch:tijdreeksen} geeft een inleiding op het analyseren van hoe de waarde van een variabele evolueert in de tijd aan de hand van wiskundige modellen die onder bepaalde voorwaarden ook toelaten om voorspellingen te doen.

\subsection{Leermateriaal}

Het belangrijkste leermateriaal voor dit opleidingsonderdeel is deze cursus, die ook de oefeningenopgaven bevat. Die wordt ter beschikking gesteld via Chamilo als PDF. Op Chamilo vind je ook de PDF's met de slides gebruikt tijdens de lessen.

Daarnaast krijgen studenten toegang tot een GitHub-repository met de broncode voor:

\begin{itemize}
  \item Deze cursus
  \item De slides gebruikt tijdens de lessen
  \item Broncodevoorbeelden in R voor alle technieken die in de cursus aan bod komen.
\end{itemize}

\textbf{Errata en wijzigingen} aan de cursus worden in GitHub aangebracht. De PDF's op Chamilo zullen niet noodzakelijk bijgewerkt worden. Studenten kunnen zelf de laatste versies van alle documenten met \LaTeX{} genereren.

De software die nodig is voor dit opleidingsonderdeel is gratis/open source. Instructies voor de installatie kan je vinden in Sectie~\ref{sec:installatie-software}.

\subsection{Werkvormen}

\textbf{Studenten afstandsleren} kunnen vragen stellen tijdens de contactmomenten. Dit zijn echter geen lesmomenten! Het rooster vind je in de Chamilo-cursus ``Informatie voor studenten TILE.''

\textbf{Studenten dagonderwijs} krijgen drie uur per week les, waarvan één uur klassikale instructie en hoorcollege, en twee uur oefeningen en begeleiding bij het groepswerk.

\subsection{Werk- en leeraanwijzingen}

Het opleidingsonderdeel \emph{Onderzoekstechnieken} wordt door veel studenten als moeilijk ervaren. Dat is begrijpelijk, want het onderwerp ligt dan ook buiten de comfortzone van de doorsnee informaticastudent en we weten allemaal dat wiskundige vakken niet de populairste van onze opleiding zijn.

Er zijn twee manieren om hier mee om te gaan. Je kan de weg van de minste weerstand nemen: je concentreren op de vakken die je graag doet en een dag voor het examen de cursus doornemen in de hoop dat je voldoende punten bij elkaar sprokkelt om een tien te halen. De ervaring leert dat deze strategie niet succesvol is, wat blijkt uit het lage slagingspercentage in de eerste zittijd (in academiejaar 2016-2017 was dat ca. 35\% voor het dagonderwijs en 10\% voor afstandsleren). In de tweede zittijd zien we vaak een veel hoger slagingspercentage, wat naar onze mening suggereert dat wanneer je voldoende inspanning levert voor dit vak, het zeker haalbaar is.

Enkele tips om w\'el meteen te slagen voor dit vak:

\begin{itemize}
  \item Kom naar de les en \emph{neem actief nota's}~\parencite{Lundin2020};
  \item Werk ook voor dit vak \emph{buiten de contactmomenten}. Herhaal de geziene theorie en werk oefeningen af waarmee je nog niet klaar was. Noteer zaken die je niet begrijpt of waar je vast zit, en stel je vraag tijdens het eerstvolgende college.
  \item Gebruik goede \emph{leertechnieken}. Je vindt een goed overzicht van leertechnieken waarvan het effect wetenschappelijk aangetoond is via de website van \emph{The Learning Scientists}\footnote{\url{http://www.learningscientists.org/}}.
  \begin{itemize}
    \item \emph{Spaced practice:} Studeer in meerdere kleine sessies (minstens één keer per week) en niet in grote blokken. Blokkeer een vast moment in je weekagenda/lesplanning.
    \item \emph{Retrieval practice:} Neem een leeg blad papier en probeer zoveel mogelijk zaken over een bepaald onderwerp op te schrijven vanuit je herinnering (dus zonder in de cursus te kijken). Controleer dit daarna aan de hand van je lesnota's en in de cursus.
    \item \emph{Elaboration:} Stel jezelf vragen over hoe dingen (bv. formules, toetsingsprocedures \ldots) in elkaar zitten en waarom dat zo is. Overleg met medestudenten. Vraag je lector om meer uitleg indien nodig. Leg verbanden tussen verschillende onderwerpen in de cursus (bv. vergelijk toetsingsprocedures).
    \item \emph{Interleaving:} Wissel onderwerpen af tijdens het studeren.
    \item Gebruik \emph{concrete voorbeelden} om abstracte idee\"en te begrijpen. In de cursus worden reeds enkele voorbeelden gegeven, probeer er zelf andere te bedenken. Overleg met medestudenten en vraag eventueel feedback aan je lector.
    \item \emph{Dual coding:} Combineer woord en beeld, probeer de leerstof die je instudeert visueel voor te stellen.
  \end{itemize}
\end{itemize}

Uiteindelijk komt het er op neer dat je voldoende tijd en inspanning investeert om te studeren voor dit vak.

\subsection{Studiebegeleiding en planning}

Studenten \textbf{afstandsleren} die vragen hebben over de leerstof kunnen in de eerste plaats terecht op het forum in Chamilo. Op de contactmomenten voor afstandsleren is er ook gelegenheid voor het stellen van vragen.

Studenten \textbf{dagonderwijs} kunnen vragen stellen tijdens de werkcolleges of  op het forum.

%In Tabel~\ref{tab:weekplanning} vind je een overzicht van de lesplanning voor het dagonderwijs die ook als leidraad kan dienen voor de studieplanning van studenten afstandsleren.

%\begin{table}
%  \begin{center}
%    \begin{tabular}{cll}
%       \hline
%       \textbf{Week} & \textbf{Theorie}     & \textbf{Oefeningen}            \\
%       \hline
%       1  & Intro, Onderzoeksproces         & Software installeren, \LaTeX{} \\
%       2  & Analyse van 1 variabele         & Wetenschappelijk schrijven     \\
%       3  & Steekproefonderzoek             & Analyse van 1 variabele        \\
%       4  & Steekproefonderzoek             & Steekproefonderzoek            \\
%       5  & Toetsingsprocedures ($z$-toets) & Steekproefonderzoek            \\
%       6  & Toetsingsprocedures ($t$-toets) & Toetsingsprocedures            \\
%       7  & Analyse van 2 variabelen        & Toetsingsprocedures            \\
%      --- & \textbf{Paasvakantie}           & ---                            \\
%       8  & Analyse van 2 variabelen        & Analyse van 2 variabelen       \\
%       9  & $\chi^2$-toets                  & Analyse van 2 variabelen       \\
%      10  & Tijdreeksen                     & $\chi^2$-toets                 \\
%      11  & Toelichting bachelorproef       & Tijdreeksen                    \\
%      12  & Herhaling                       & Herhaling                      \\
%      \hline
%    \end{tabular}
%    \caption[Weekplanning]{Weekplanning van de cursus.}
%    \label{tab:weekplanning}
%  \end{center}
%\end{table}

\subsection{Evaluatie}



\begin{itemize}
  \item Eerste examenperiode:
  \begin{itemize}
    \item 70\% periodegebonden evaluatie: schriftelijk examen, bestaande uit een deel gesloten boek (theorie) en een deel met voorbereiding op pc (oefeningen)
    \item 30\% niet-periodegebonden evaluatie, groepsopdracht: het voeren van een mini-onderzoek in groep, bestaande uit een literatuurstudie, opzetten van een reproduceerbaar experiment, verzamelen van meetgegevens en die statistisch analyseren, en er een verslag over schrijven
  \end{itemize}
  \item Tweede examenperiode:
  \begin{itemize}
    \item 70\% periodegebonden evaluatie: schriftelijk examen, bestaande uit een deel gesloten boek (theorie) en een deel met voorbereiding op pc (oefeningen)
    \item 30\% niet-periodegebonden evaluatie: er wordt geen tweede examenkans georganiseerd voor dit onderdeel. Wanneer een student in de eerste examenkans niet geslaagd was voor het opleidingsonderdeel wordt het resultaat voor de groepsopdracht ongewijzigd overgenomen.
  \end{itemize}
\end{itemize}

\section{Installatie software}
\label{sec:installatie-software}

Voor de cursus onderzoekstechnieken maak je gebruik van verschillende softwarepakketten. Hier vind je wat uitleg over de installatie en hoe je er mee aan de slag kan.

\begin{itemize}
  \item Git client (versiebeheersysteem);
  \item \LaTeX{} compiler;
  \item \LaTeX{} editor;
  \item Jabref (bibliografische databank);
  \item R (statistische analysesoftware);
  \item Rstudio (IDE voor R);
  \item Lettertypes voor de HOGENT-huisstijl.
\end{itemize}

Sommige van deze applicaties nemen veel schijfruimte in, dus zorg dat je voldoende ruimte vrij hebt.

In vele andere cursussen rond statistiek of onderzoekstechnieken wordt gebruik gemaakt van commerci\"ele software: SPSS of SAS voor data-analyse, MS Office voor de opmaak van documenten. In deze cursus wordt er expliciet voor gekozen om open source of gratis software te gebruiken. Het grootste voordeel is dat je die ook na je afstuderen nog kan gebruiken zonder dat jij of je bedrijf/organisatie softwarelicenties moet aankopen.

Bovendien zijn de tools die we zullen gebruiken kwalitatief minstens even goed dan hun commerci\"ele tegenhangers. R, een programmeertaal voor statistische analyse, wordt wereldwijd gebruikt in academische én professionele context. De kans is dus niet onbestaande dat je het in je professionele loopbaan nog zal tegenkomen, of het zal kunnen toepassen voor het oplossen van datagerelateerde problemen. Feedback die we kregen van oud-studenten bevestigt dit.

\LaTeX{} is een markuptaal en tekstzetsysteem voor de professionele vormgeving van documenten. De bedoeling is dat de auteur zich vooral moet bezig houden met het logisch structureren van een tekst, en dat het vormgeven op papier wordt overgenomen door de software. Het aanleren van de markuptaal vraagt wat inspanning, maar het is een investering die rendeert wanneer je een lang document (zoals een scriptie) op een professionele, strakke manier wil opmaken. Er zijn in het verleden nog zelden of nooit bachelorproeven ingediend die in MS Word geschreven waren en die een voldoende goede opmaak hadden. Het lijkt veel eenvoudiger om een tekst op te stellen in Word, maar het is zo goed als onmogelijk om in een lang document een consistente en professioneel ogende opmaak te realiseren.

\subsection{Windows}

Omdat het hier toch gaat om een vrij groot aantal applicaties, kunnen Windows-gebruikers beter gebruik maken van de Chocolatey package manager\footnote{\url{https://chocolatey.org/}} in plaats van alles manueel te downloaden en installeren.

Na installatie van Chocolatey\footnote{\url{https://chocolatey.org/install}}, voer je volgende commando's uit als Administrator in een CMD of PowerShell terminal:

\begin{verbatim}
choco install -y git
choco install -y miktex
choco install -y texstudio
choco install -y JabRef
choco install -y r.project
choco install -y r.studio
\end{verbatim}

Wie toch de ``klassieke'' werkwijze wil hanteren, vindt hier de verschillende softwarepakketten:

\begin{itemize}
  \item Git client: \url{https://git-scm.com/download/win}
  \item \LaTeX{} compiler: \url{https://miktex.org/download}
  \item TeXStudio: \url{http://www.texstudio.org/}
  \item Jabref: \url{https://www.fosshub.com/JabRef.html}
  \item R: \url{https://lib.ugent.be/CRAN/}
  \item Rstudio: \url{https://www.rstudio.com/products/rstudio/download/#download}
\end{itemize}

\subsection{macOS}

macOS gebruikers installeren de nodige software best via de Homebrew\footnote{\url{https://brew.sh/}} package manager\footnote{\textbf{Let op!} Deze werkwijze is nog niet getest. Feedback van Mac-gebruikers is welkom!}:

\begin{verbatim}
brew install git
brew cask install mactex
brew cask install texstudio
brew cask install jabref
brew install Caskroom/cask/xquartz
brew install --with-x11 r
brew cask install --appdir=/Applications rstudio
\end{verbatim}

Wie toch alles manueel wil installeren kan de applicaties hier downloaden:

\begin{itemize}
  \item Git client: \url{https://git-scm.com/download/mac}
  \item \LaTeX{} compiler: \url{https://www.tug.org/mactex/mactex-download.html}
  \item TeXStudio: \url{http://www.texstudio.org/}
  \item Jabref: \url{https://www.fosshub.com/JabRef.html}
  \item R: \url{https://lib.ugent.be/CRAN/}
  \item Rstudio: \url{https://www.rstudio.com/products/rstudio/download/#download}
\end{itemize}

\subsection{Linux}
\label{ssec:installatie-linux}

Op RStudio na zijn alle nodige softwarepakketten beschikbaar in de repositories van de meest gebruikte Linux-distributies. We geven hier command-line instructies voor enerzijds Ubuntu (Xenial/16.04) en Debian 9 en anderzijds Fedora.

\paragraph{Ubuntu/Debian} 

Controleer eerst de URL naar de laatste versie van RStudio via de website. Er is een aparte versie voor Debian-gebruikers, zij kopi\"eren dus best de URL van de link op de website i.p.v. deze te gebruiken die hieronder gegeven is.

\begin{verbatim}
sudo apt install biber git jabref r-base texlive-bibtex-extra \
  texlive-extra-utils texlive-fonts-recommended texlive-lang-european \
  texlive-latex-base texlive-latex-extra texlive-latex-recommended \
  texlive-pictures texstudio ttf-mscorefonts
wget https://download1.rstudio.org/desktop/bionic/amd64/rstudio-1.2.5033-amd64.deb
sudo dpkg -i ./rstudio-1.2.5033-amd64.deb
\end{verbatim}

\paragraph{Fedora}

Controleer eerst de link naar de laatste versie van RStudio via de website. Dit is één lang commando:

\begin{verbatim}
sudo dnf install git texstudio R \
  java-1.8.0-openjdk-openjfx texlive-collection-latex \
  texlive-texliveonfly texlive-babel-dutch \
  msttcore-fonts-installer.noarch \
  https://download1.rstudio.org/desktop/fedora28/x86_64/rstudio-1.2.5033-x86_64.rpm
\end{verbatim}

Je kan JabRef ook installeren vanuit de Fedora package repository, maar dan krijg je een verouderde versie. Je kan dan beter de ``Platform Independent Runnable Jar'' downloaden via de projectwebsite\footnote{\url{https://jabref.org/}}. Die kan je dan opstarten vanuit de shell met het commando (hier voorbeeld voor versie 4.3.1):

\begin{verbatim}
java -jar JabRef-4.3.1.jar
\end{verbatim}

\section{Configuratie}

\subsection{Git, GitHub}

Wellicht heb je Git al geconfigureerd voor enkele van je andere vakken. Kijk eventueel alles nog eens na! Als alles ok is, kan je deze sectie overslaan.

\emph{Wij raden aan om Git via de command line te gebruiken.} Zo krijg je het beste inzicht in de werking. Het commando \texttt{git status} geeft op elk moment een goed overzicht van de toestand van je lokale repository en geeft aan met welk commando je een stap verder kan zetten of de laatste stap ongedaan kan maken. Voor wie toch een GUI verkiest, raden we GitKraken~\footnote{\url{https://www.gitkraken.com/}} aan.

Als je nog geen GitHub-account hebt, kies dan een gebruikersnaam die je na je afstuderen nog kan gebruiken (dus bv.~niet je HoGent login). De kans is erg groot dat je tijdens je carrière nog van GitHub gebruik zult maken. Koppel ook je HoGent-emailadres aan je GitHub account (je kan meerdere adressen registreren). Op die manier kan je aanspraak maken op het GitHub Student Developer Pack\footnote{\url{https://education.github.com/pack}}, wat je gratis toegang geeft tot een aantal in principe betalende producten en diensten.

Windows-gebruikers voeren volgende instructies uit via Git Bash, macOS- en Linux-gebruikers via de standaard (Bash) terminal.

\begin{verbatim}
git config --global user.name 'Pieter Stevens'
git config --global user.email 'pieter.stevens.u12345@student.hogent.be'
git config --global push.default simple
\end{verbatim}

Maak ook een SSH-sleutel aan om het synchroniseren met GitHub te vereenvoudigen (je moet dan geen wachtwoord meer opgeven bij push/pull van/naar een private repository).

\begin{verbatim}
ssh-keygen
\end{verbatim}

Volg de instructies op de command-line, druk gewoon ENTER als je gevraagd wordt een wachtwoordzin (pass phrase) in te vullen. In de home-directory van je gebruiker (bv. \verb|c:\Users\Pieter| op Windows, \verb|/Users/pieter| op Mac, \verb|/home/pieter| op Linux) is nu een directory met de naam \verb|.ssh/| aangemaakt met twee bestanden: \verb|id_rsa| (je private key) en \verb|id_rsa.pub| (je public key). Open dit laatste bestand met een teksteditor en kopieer de volledige inhoud naar het klembord. Ga vervolgens naar je GitHub profiel en kies in het menu links voor SSH and GPG keys. Klik rechtsboven op de groene knop met ``New SSH Key'' en plak de inhoud van je publieke sleutel in het veld ``Key''. Bevestig je keuze.

Test nu of je de code van de cursus Onderzoekstechnieken kan downloaden. Ga in de Bash shell naar een directory waar je dit project lokaal wil bijhouden en voer uit:

\begin{verbatim}
git clone git@github.com:HoGentTIN/onderzoekstechnieken-cursus.git
\end{verbatim}

Als dit lukt, is er nu een directory aangemaakt met dezelfde naam als de repository. Je mag indien gewenst de directory verplaatsen en zelfs de naam wijzigen. Doe tijdens het semester regelmatig \texttt{git pull} (binnen deze directory) om de laatste wijzigingen in het cursusmateriaal bij te werken. Pas zelf geen bestanden aan binnen deze repository, dit zal leiden tot conflicten.

\subsection{Lettertypes}

De HOGENT-huisstijl, zoals toegepast in de slides, maakt gebruik van lettertypes die niet standaard geïnstalleerd zijn. Als je een PDF wil genereren van de slides, heb je deze lettertypes nodig.

Gebruikers van Linux downloaden ook best de gekende Microsoft-fonts (Arial, Courier, Times New Roman, enz.). Als je de installatie-instructies in Sectie~\ref{ssec:installatie-linux} gevolgd hebt, dan is dat al in orde.

De benodigde lettertypes zijn:

\begin{itemize}
  \item Montserrat: \url{https://fonts.google.com/specimen/Montserrat}
  \item Code Pro Black: o.a. via \url{https://www.wfonts.com/font/code-pro-black}
  \item Fira Math: \url{https://github.com/firamath/firamath}
  \item Inconsolata: \url{https://fonts.google.com/specimen/Inconsolata}
\end{itemize}

De Google Fonts kan je als volgt downloaden: volg de link naar het lettertype, klik op ``Select this font'' en vervolgens rechtsonder op de zwarte balk met de tekst ``1 Family selected''. In de pop-up zie je rechtsboven een download-icoon. Klik hier op om het lettertype te downloaden.

\subsection{TeXstudio}

Controleer deze instellingen via menu-item \emph{Options > Configure TeXstudio}:

\begin{itemize}
  \item Build:
  \begin{itemize}
    \item Default Compiler: XeLaTeX
    \item Default Bibliography tool: Biber
  \end{itemize}
  \item Commands:
  \begin{itemize}
    \item \texttt{xelatex -synctex=1 -interaction=nonstopmode  -shell-escape \%.tex}
    
    (voeg de optie \texttt{-shell-escape} toe)
  \end{itemize}
  \item Editor:
  \begin{itemize}
    \item Indentation mode: Indent and Unindent Automatically
    \item Replace Indentation Tab by Spaces: Aanvinken
    \item Replace Tab in Text by spaces: Aanvinken
    \item Replace Double Quotes: English Quotes: \verb|``''|
  \end{itemize}

\end{itemize}

Om te testen of TeXstudio goed werkt, kan je het bestand \texttt{cursus/cursus-onderzoekstech\-nie\-ken.tex} openen. Kies \emph{Tools > Build \& View} (of druk F5) om de cursus te compileren in een PDF-bestand. Controleer of er achteraan het document een bibliografie en/of index te vinden is. Indien niet, volg dan nog volgende stappen:

\begin{enumerate}
  \item Kies \emph{Tools > Index} voor het genereren van de index;
  \item Kies \emph{Tools > Bibliography} (of druk F8) voor het genereren van de bibliografie;
  \item Kies \emph{Tools > Index} (geen shortcut) om de zoekindex te genereren;
  \item Voer daarna opnieuw \emph{Build \& View} (F5) uit.
\end{enumerate}

Veel functionaliteiten van \LaTeX{} zitten in aparte packages die niet noodzakelijk standaard geïnstalleerd zijn. De eerste keer dat je een bestand compileert, is het dan ook mogelijk dat er extra packages moeten gedownload worden. MiK\TeX{} zal een pop-up tonen om je toestemming te vragen, bevestig dit. Op Linux is het mogelijk dat je deze packages nog manueel moet installeren. De eerste keer compileren kan enkele minuten duren zonder dat je feedback krijgt over wat er gebeurt. Even geduld, dus!

Indien er zich fouten voordoen bij de compilatie, kan je onderaan in het tabblad Log een overzicht krijgen van de foutboodschappen.

\subsection{JabRef}

JabRef\footnote{\url{http://www.jabref.org/}} is een GUI voor het bewerken van Bib\TeX{}-bestanden, een soort database van bronnen uit de wetenschappelijke- of vakliteratuur voor een \LaTeX{}-document.

Kies in het menu voor \emph{Options > Preferences > General} en kies onderaan voor de optie ``Default bibliography mode'' voor ``biblatex''. Dit maakt de bestandsindeling van de bibliografische databank compatibel met dat van de cursus en het aangeboden \LaTeX{}-sjabloon voor de bachelorproef.

Kies in het \emph{Preferences}-venster voor de categorie \emph{File} en geef een directory op voor het bijhouden van PDFs van de gevonden bronnen onder \emph{Main file directory}. Het is heel interessant om alle gevonden artikels te downloaden en onder die directory bij te houden. Nog beter is om als naam van het bestand de Bib\TeX{} key te nemen (typisch naam van de eerste auteur + jaartal, bv. \texttt{Knuth1998.pdf}). Je kan het bestand dan makkelijk openen vanuit JabRef.

Voor meer gedetailleerde informatie over het bijhouden van bibliografische referenties, zie de bachelorproefgids~\autocite{VanVreckem2017}.

\section{Gebruik van R}

R is een softwarepakket voor het bewerken, analyseren en visualiseren van data. Het heeft onder meer:

\begin{enumerate}
  \item een effectieve gegevensbeheer- en opslagfaciliteit,
  \item een reeks operatoren voor berekeningen op arrays, in het bijzonder matrices,
  \item een grote verzameling van instrumenten voor data-analyse,
  \item grafische faciliteiten voor data-analyse en weergave en
  \item een goed ontwikkelde, eenvoudige en effectieve programmeertaal (genaamd 'S').
\end{enumerate}

R heeft een ingebouwde hulpfaciliteit die vergelijkbaar is met die van UNIX man-pages. Voor meer informatie over elke specifieke functie, bijvoorbeeld \texttt{solve}, kan je volgend commando oproepen

\begin{lstlisting}
> help (solve)
\end{lstlisting}

Een alternatief is
\begin{lstlisting}
> ?solve
\end{lstlisting}

Er is online heel veel informatie terug te vinden over R. Er is een erg levendige en open \textit{community} van mensen wereldwijd die professioneel bezig zijn met R.

Je zal ook merken dat er meerdere manieren zijn om in R eenzelfde taak uit te voeren, bv. een databestand inlezen of een grafiek plotten. Er zijn meer bepaald twee grote ``families'' van werkwijzen die vaak aangeduid worden met enerzijds \textit{Base-R} en anderzijds \textit{the tidyverse}. \textit{Base-R} omvat de commando's en functies die al van oudsher in R aanwezig zijn, maar die door verschillende auteurs kunnen geschreven zijn, met onderling verschillende codeerstijl en API. \textit{The tidyverse} is een verzameling van codebibliotheken met een gemeenschappelijke filosofie en codeerstijl met als doel makkelijk leesbare code en krachtige functionaliteit.

In deze cursus hebben we niet echt een doorgedreven keuze gemaakt voor een van de twee. Voor alle taken die we verwachten dat studenten ze kunnen uitvoeren met R is er echter voorbeeldcode voorzien, daar kan je je op baseren om zelf aan de slag te gaan.

\subsection{Een omgeving opzetten voor bijhouden van oefeningen}

Voor de oefeningen waarvoor je R nodig hebt, kan je best een omgeving opzetten waar je alle databestanden en code kan bijhouden. Het is ook interessant om daar een Git repository van te maken.

\begin{enumerate}
  \item In RStudio, kies voor \textit{File > New Project}
  \item Selecteer \textit{New Directory}
  \item Selecteer \textit{New Project}
  \item Kies een naam voor de directory (bv. \texttt{ozt-oefeningen}) en de directory waaronder je het nieuwe project wil aanmaken (kies je zelf). Vink ook \textit{Create a git repository} aan.
  \item Maak nu een directory \texttt{datasets} aan onder \texttt{ozt-oefeningen} (dat kan via bestandsbeheer of in RStudio rechtsonder in het tabblad \textit{Files}). Kopieer in deze directory alle databestanden uit de Github-repository van de cursus, meer bepaald uit de directory \texttt{oefeningen/datasets} en \texttt{cursus/data}. Uit die laatste directory heb je niet de \texttt{.R}-bestanden nodig, maar wel de \texttt{.csv}, \texttt{.txt} en \texttt{.sav}-bestanden.
  \item Gebruik tenslotte de command-line of de Git-functionaliteit binnen R (bovenaan rechts in het tabblad Git) om een eerste commit aan te maken met de datasets, en de bestanden die door Rstudio zijn aangemaakt: \texttt{.gitignore} en een \texttt{.Rproj}-bestand.
\end{enumerate}

\subsection{Commando's opslaan en output uitvoeren}

Je slaat de code die je schrijft om oefeningen op te lossen best op in een R-script. Dat is een tekstbestand met de extensie \texttt{.R}. Je kan een nieuw R-script aanmaken met \textit{File > New File > R-script (Ctrl+Shift+N)}. Kies zelf zinvolle namen en/of mappenstructuren voor je oefeningen (bv. \texttt{hst2/oef-2-5.R}).

Als je de functionaliteiten van de \textit{tidyverse} wil gebruiken, start dan het script met:

\begin{lstlisting}
library(tidyverse)
\end{lstlisting}

Als de commando's in een extern bestand worden opgeslagen, bv. \texttt{commands.R} in de werkmap, dan kunnen deze in een R-sessie op elk moment uitgevoerd worden op de console (linksonder, tabblad \textit{Console}) met de opdracht:

\begin{lstlisting}
> source("commands.R")
\end{lstlisting}

Dit zal je echter niet zo vaak nodig hebben. Je kan ook op een regel in het script gaan staan en \textit{Ctrl+Enter} indrukken. Dat zal het commando op die regel uitvoeren en verder gaan naar de volgende regel. Door achtereenvolgens \textit{Ctrl+Enter} te blijven indrukken, voer je dus regel per regel het script uit.

\subsection{R omgeving en workspace}

De entiteiten die R cre\"eert en manipuleert staan bekend als objecten. Deze kunnen variabelen zijn, arrays van cijfers, reeksen, functies of meer algemene structuren die uit dergelijke componenten zijn gebouwd. Tijdens een R-sessie worden objecten gemaakt en opgeslagen op naam. Je kan deze in Rstudio rechtsboven in het tabblad \textit{Environment} terugvinden. Hetzelfde R commando

\begin{lstlisting}
> objects()
\end{lstlisting}

geeft een overzicht van alle objecten die gemaakt zijn tot op dat moment.
De verzameling van objecten die momenteel zijn opgeslagen, heet de werkruimte.
Om objecten te verwijderen kan je in het \textit{Environment}-tabblad op het bezem-icoontje klikken. In de console is de functie \texttt{rm} beschikbaar, waarmee je individuele objecten kan verwijderen:

\begin{lstlisting}
> rm (x, y, z, inkt, junk, temp, foo, bar)
\end{lstlisting}

Alle objecten die tijdens een R-sessie zijn aangemaakt, kunnen permanent in een bestand worden opgeslagen voor gebruik in de toekomstige sessies. Wanneer deze optie geactiveerd is, worden de objecten weggeschreven naar een bestand met extensie \texttt{.RData}.

In dit hoofdstuk onderzoeken we hoe je een dataset definieert in R. Er worden slechts twee commando's onderzocht. Het eerste is voor het eenvoudig toewijzen van gegevens, en het tweede is voor het inlezen van een databestand. Er zijn verschillende manieren om gegevens in een R-sessie te lezen, maar we richten ons op slechts twee om het eenvoudig te houden.

\subsection{Toewijzing}

De meest directe manier om een lijst met nummers op te slaan is via een opdracht met behulp van het \texttt{c}-commando. (C staat voor "combineren.") Het idee is dat een lijst met nummers onder een bepaalde naam wordt opgeslagen, en de naam wordt gebruikt om te verwijzen naar de gegevens. Een lijst wordt gespecificeerd met de opdracht \texttt{c}, en de toewijzing wordt geduid met de symbolen "<-". Een andere term die gebruikt wordt om de lijst met nummers te omschrijven is \texttt{vector}.

De cijfers binnen de \texttt{c}-opdracht worden gescheiden door komma's. Als voorbeeld kunnen we een nieuwe variabele maken, genaamd "\texttt{x}".

\begin{lstlisting}
> x <- c(10.4, 5.6, 3.1, 6.4, 21.7)
\end{lstlisting}

Wanneer je dit commando invoert, mag je geen uitvoer zien behalve een nieuwe opdrachtregel. Het commando maakt een lijst met nummers genaamd "x". Om te zien welke elementen zijn opgenomen in x, typ zijn naam en druk op de enter-toets.

Om met \'e\'en van de nummers te werken, kan je toegang krijgen tot de variabele en vervolgens vierkante haakjes noteren die aangeven welk nummer u wilt beschouwen:

\begin{lstlisting}
> x[2]
[1] 5.6
\end{lstlisting}

\subsection{Een csv-bestand lezen}

We gaan ervan uit dat het gegevensbestand een CSV-bestand is (\textit{Comma-Separated Values} of waarden gescheiden door komma's). Dat wil zeggen, elke regel bevat een rij met waarden die getallen of letters kunnen zijn, en elke waarde wordt gescheiden door een komma. We gaan ervan uit dat de eerste rij een lijst met labels bevat. Het idee is dat de labels in de bovenste rij gebruikt worden om te verwijzen naar de verschillende variabelen per rij.

In de \textit{tidyverse} kan je een CSV-bestand inlezen met de functie \texttt{read\_csv()}. \textbf{Let op:} deze functie veronderstelt dat de dataset is ingedeeld volgens Engelstalige conventies: het decimaalteken is de punt (\texttt{.}) en kolommen worden gescheiden door komma's (\texttt{,}). Voor datasets in het Nederlands met komma (\texttt{,}) als decimaalteken en kommapunt (\texttt{;}) als kolomscheidingsteken, is er de functie \texttt{read\_csv2()}.

Het resultaat van de \texttt{read\_csv}-functies is een object van het type \texttt{tibble}\footnote{In \textit{Base R} heb je een gelijkaardige datastructuur, de \texttt{data.table}}. Je kan deze meteen toewijzen aan een variabele, bv.:

\begin{lstlisting}
aardbevingen <- read_csv("datasets/Aardbevingen.csv")
android_persistence
  <- read_csv2("datasets/android_persistence.csv")
\end{lstlisting}

\begin{exercise}
  Probeer een databestand te lezen vanuit RStudio:
  
  \begin{enumerate}
    \item Gebruik de help-functie in Rstudio om na te gaan welke parameters er nog mogelijk zijn in \texttt{read\_csv}.
    \item Importeer het databestand \texttt{computers.csv} en wijs het toe aan de variabele \texttt{computers}. Kijk in het \textit{Environment}-tabblad na wat de waarde is van deze variabele.
  \end{enumerate}
  
Ga met behulp van het help commando na wat de parameters zijn van het commando. Probeer daarna het bestand \texttt{computers.csv} in te lezen. Je vindt het in \texttt{cursus/data}.
\end{exercise}

Het databestand \texttt{computers.csv} komt uit de publicatie van \autocite{Stengos2005}. Deze dataset bevat data van 1993 tot 1995 over de prijzen van computers. Je kan nagaan wat het effect van de toevoeging van een cd-rom-station is op de prijs van de computer, of wat het effect is van de kloksnelheid op de prijs. 

Via het \texttt{names()} commando kan je achterhalen welke kolommen gedefinieerd zijn:

\begin{lstlisting}[breaklines=true]
> names(computers)
[1] "price"   "speed"   "hd"      "ram"     "screen"  "cd"      "multi"   "premium" "ads"     "trend"
\end{lstlisting}

Voor de uitvoering van het commando \texttt{read.csv} gebruikt R een specifiek soort variabele, dat een dataframe heet. Alle gegevens worden opgeslagen in het dataframe als afzonderlijke kolommen. Als u niet zeker weet wat voor variabele u hebt, dan kunt u de opdracht \texttt{attributes} gebruiken. Hiermee worden alle dingen vermeld die R gebruikt om de variabele te beschrijven:

\begin{lstlisting}[breaklines=true]
attributes(computers)
$names
[1] "price"   "speed"   "hd"      "ram"     "screen"  "cd"      "multi"   "premium" "ads"     "trend"  

$class
[1] "tbl_df"     "tbl"        "data.frame"

$row.names
[1]    1    2    3    4    5    6    7    8    9   10   11   12   13   14   15   16   17   18   19   20   21   22   23   24   25   26   27
[28]   28   29   30   31   32   33   34   35   36   37   38   39   40   41   42   43   44   45   46   47   48   49   50   51   52   53   54
...
[ reached getOption("max.print") -- omitted 5259 entries ]

$spec
cols(
price = col_integer(),
speed = col_integer(),
hd = col_integer(),
ram = col_integer(),
screen = col_integer(),
cd = col_character(),
multi = col_character(),
premium = col_character(),
ads = col_integer(),
trend = col_integer()
)

\end{lstlisting}

Met de functie \texttt{glimpse()} kan je een vluchtige blik werpen op de inhoud van de dataset:

\begin{lstlisting}
glimpse(computers)
Observations: 6,259
Variables: 10
$ price   <dbl> 1499, 1795, 1595, 1849, 3295, 3695, 1720, ...
$ speed   <dbl> 25, 33, 25, 25, 33, 66, 25, 50, 50, 50, 33...
$ hd      <dbl> 80, 85, 170, 170, 340, 340, 170, 85, 210, ...
$ ram     <dbl> 4, 2, 4, 8, 16, 16, 4, 2, 8, 4, 8, 8, 4, 8...
$ screen  <dbl> 14, 14, 15, 14, 14, 14, 14, 14, 14, 15, 15...
$ cd      <chr> "no", "no", "no", "no", "no", "no", "yes",...
$ multi   <chr> "no", "no", "no", "no", "no", "no", "no", ...
$ premium <chr> "yes", "yes", "yes", "no", "yes", "yes", "...
$ ads     <dbl> 94, 94, 94, 94, 94, 94, 94, 94, 94, 94, 94...
$ trend   <dbl> 1, 1, 1, 1, 1, 1, 1, 1, 1, 1, 1, 1, 1, 1, ...
\end{lstlisting}

\subsection{Data types}

We kijken naar enkele manieren waarop R gegevens kan opslaan en organiseren. Dit is echter een inleiding dus beschouwen we maar een kleine subset van de verschillende datatypes die door R worden herkend. 

\subsubsection{Numbers}

De meest eenvoudige manier om een nummer op te slaan is om een variabele van een enkel getal te nemen:

\begin{lstlisting}
> a <- 3
\end{lstlisting}

Met deze variable kunnen we enkele basisoperaties doen en opslaan:

\begin{lstlisting}
> b <- sqrt(a*a+3)
> b
[1] 3.464102
\end{lstlisting}

Het \texttt{numeric} commando kan gebruikt worden om een lijst met nummers te initialiseren. Onderstaande opdracht maakt bijvoorbeeld een lijst van 10 nummers. Het \texttt{typeof} commando geeft het type terug van de variabele.

\begin{lstlisting}
> a <- numeric(10)
> a
[1] 0 0 0 0 0 0 0 0 0 0
> typeof(a)
[1] "double"
\end{lstlisting}

\subsubsection{Strings}

Een tekenreeks wordt gespecificeerd door gebruik te maken van aanhalingstekens. Zowel enkelvoudige als dubbele aanhalingstekens zijn mogelijk:

\begin{lstlisting}
> a <- "hello"
> a
[1] "hello"
> b <- c("hello","there")
> b
[1] "hello" "there"
> b[1]
[1] "hello"
\end{lstlisting}

\subsubsection{Factors}

In een volgend hoofdstuk zullen we leren dat elke variabele een zgn.~\textit{meetniveau}\index{meetniveau} heeft (zie Sectie~\ref{sec:onderzoeksproces-basisconcepten}). Eén van deze meetniveaus zijn de zgn.~\textit{kwalitatieve variabelen} die maar een beperkt aantal mogelijke waarden heeft, niet noodzakelijke numeriek. In R wordt dit soort variabelen een \textit{factor}\index{factor} genoemd.

Je geeft aan dat een variabele een factor is met behulp van het \texttt{factor} commando. 

\subsubsection{Data frames}

Data kan worden opgeslagen aan de hand van \textit{data frames}\index{data frame} (\textit{Base R}) of \textit{tibbles}\index{tibble} (\textit{tidyverse}). Beide kunnen meestal door elkaar gebruikt worden en je kan er meestal dezelfde bewerkingen op uitvoeren.

Dit is een manier om verschillende vectoren van verschillende types te nemen en ze op te slaan in dezelfde variabele. De vectoren kunnen van alle soorten zijn. Een dataframe kan bijvoorbeeld verschillende vectoren bevatten en elke lijst kan een vector zijn van factoren, strings of nummers.

Er zijn verschillende manieren om data frames te maken en te manipuleren. In deze cursus zullen we deze meestal aanmaken door een CSV-bestand in te lezen. De meeste andere vallen buiten het bereik van deze inleiding. Ze worden hier alleen genoemd om een meer volledige beschrijving te geven. 

\lstinputlisting{data/dataframe.R}

\subsubsection{Logische variabelen}

Een ander belangrijk gegevenstype is het logische type. Er zijn twee vooraf gedefinieerde variabelen, \texttt{TRUE} en \texttt{FALSE}.

\subsubsection{Tables}

Een andere  manier om informatie op te slaan is in een tabel.  We kijken alleen maar naar het maken en defini\"eren van tabellen. 

\lstinputlisting{data/tables.R}

Als je rijen wilt toevoegen aan de tabel, voeg dan nog een vector toe als argument van de tabelopdracht. In het onderstaande voorbeeld hebben wij twee vragen. In de eerste vraag staan de reacties  'Never', 'Sometimes' of 'Always'. In de tweede vraag staan de reacties 'Yes', 'No' of 'Maybe'. De set van vectoren 'a', en 'b' bevatten het antwoord voor elke meting. Het derde punt in 'a' is hoe de derde persoon op de eerste vraag reageerde en het derde punt in 'b' is hoe de derde persoon op de tweede vraag reageerde.

\lstinputlisting[breaklines=true]{data/twotables.R}

\subsubsection{Matrix}

Een matrix is een verzameling van gegevens die zijn aangebracht in een tweedimensionale rechthoekige indeling. Een voorbeeld van een matrix is bijvoorbeeld als volgt:

\[
\begin{bmatrix}
2 & 3 \\ 
4 & 5  
\end{bmatrix}
\]

\lstinputlisting{data/matrix.R}

\section{Oefeningen}

\begin{exercise}
    In deze oefening werken we met het ingebouwde data frame \texttt{mtcars}. 
    \begin{enumerate}
        \item   Gebruik ingebouwde R-functies om informatie weer te geven over deze dataset
        \item   Geef de waarde terug voor de eerste rij, tweede kolom
        \item   Geef het aantal rijen en het aantal kolommen
        \item   Geef enkel de kolom terug met de definities van de cylinders
    \end{enumerate}

    Om een data frame te bekomen met de twee kolommen \texttt{mpg} en \texttt{hp}, 
    pakken we de kolomnamen in een indexvector in met single square bracket operator. 
    Probeer ook eens op te zoeken hoe je een rijrecord van de ingebouwde data set \texttt{mtcars} bepaalt.
\end{exercise}

\begin{exercise}
  Maak zelf een willekeurige datafile aan in Excel en probeer deze in te lezen in R. Zijn er nog dataformaten die ondersteund worden door R?
\end{exercise}

\begin{exercise}
  Genereer een $4 \times 5$ array en noem die $x$. Geneer daarna een $3 \times 2$ array $i$ waarin de eerste kolom de rij-index kan zijn van $x$ en de tweede kolom een kolomindex voor $x$. Vervang de elementen gedefinieerd door de index in $i$ in $x$ door 0. 
\end{exercise}

\begin{exercise}
  Genereer een vector waar een voornaam en een achternaam in komen. Benoem ook de naam van de kolommen. Geef daarna de voornaam terug van het eerste element van de array. 
\end{exercise}

\begin{exercise}
  Importeer het bestand \texttt{rainforest.csv} in R.
  Je kan dit bestand terugvinden in de Github-repository van deze cursus, in de map \texttt{oefeningen/datasets}. 
  Een beschrijving van dit data frame is terug te vinden in dezelfde map, bestand \texttt{rainforest.html}.
    
  Je kan dit bestand importeren met behulp van volgende code:
  \begin{lstlisting}
rainforest <- read.csv("../path/to/rainforest.csv", sep = ",")
  \end{lstlisting}
  
  Probeer voor deze datafile te tellen hoeveel rijen er zijn per species die volledig en compleet zijn (dus geen n.a. bevatten). 
  Je kan hiervoor \texttt{with}, \texttt{table} en \texttt{complete.cases} gebruiken. 
\end{exercise}

\begin{exercise}
	Genereer een vector met de waarden van $e^x cos(x)$ voor $x= 3, 3.1, 3.2, \dots ,6$
\end{exercise}

\begin{exercise}
	Bereken: $\sum_{i=1}^{100}(i^3 + 4i^2)$
\end{exercise}
