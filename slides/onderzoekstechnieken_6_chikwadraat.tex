\documentclass[aspectratio=169]{beamer}

%==============================================================================
% Aanloop
%==============================================================================

%---------- Packages ----------------------------------------------------------
\usepackage{etex}
\usepackage{graphicx,multicol}
\usepackage{comment,enumerate,hyperref}
\usepackage{amsmath,amsfonts,amssymb}
\usepackage{tikz}
\usepackage[dutch]{babel}
\usepackage{multirow}
\usepackage{eurosym}
\usepackage{listings}
\usepackage{textcomp}
\usepackage{framed}
\usepackage{wrapfig}
\usepackage{pgf-pie}
\usepackage{pgfplots}
\usepackage{booktabs}
\usepackage{pgfplotstable}
\usepackage{changepage}
\usepackage{pst-plot,pst-func}

%---------- Configuratie ------------------------------------------------------

\usetikzlibrary{arrows,shapes,backgrounds,positioning,shadows}
\usetikzlibrary{pgfplots.statistics}

\usetheme{hogent}
\usecolortheme{hgwhite} % witte achtergrond, zwarte tekst

\setbeameroption{show notes}

%---------- Commando-definities -----------------------------------------------

\newcommand{\tabitem}{~~\llap{\textbullet}~~}
\renewcommand{\arraystretch}{1.2}

\pgfmathdeclarefunction{gauss}{2}{%
  \pgfmathparse{1/(#2*sqrt(2*pi))*exp(-((x-#1)^2)/(2*#2^2))}%
}


%---------- Info over de presentatie ------------------------------------------

\title[OZT: chi-kwadraat]{Les 6. De $\chi^{2}$ toets.}
\subtitle{Onderzoekstechnieken}
\author{Jens Buysse \and Wim {De Bruyn} \and Bert {Van Vreckem}}
\date{AJ 2018-2019}

%==============================================================================
% Inhoud presentatie
%==============================================================================

\begin{document}

%---------- Front matter ------------------------------------------------------

\begin{frame}
\maketitle
\end{frame}

\begin{frame}
\frametitle{What's on the menu today?}

\tableofcontents
\end{frame}

%---------- Inhoud ------------------------------------------------------------

\begin{frame}
  \frametitle{Wat weten we nog van vorige les?}

  \begin{itemize}
    \item Wat is een hypothese
    \item Wat zijn de onderdelen van een hypothesetoets
    \item Wat zijn de stappen bij het toetsen?
    \item Welke fouten kunnen er gemaakt worden?
  \end{itemize}
\end{frame}

\section{$\chi^{2}$ toets voor één variabele}

\begin{frame}
  \frametitle{Goodness of fit test}
  \alertbox{Een \textcolor{hgyellow}{goodness of fit test} geeft aan in welke mate een steekproef overeenstemt met een nulhypothese over de verdeling van een kwalitatieve variabele.}

  \begin{columns}
    \begin{column} {0.35\textwidth}

    \begin{figure}
      \centering
        \includegraphics[height=.5\textheight]{img/les6-man.jpg}
    \end{figure}

    \end{column}
    \begin{column} {0.65\textwidth}

    \begin{figure}
      \centering
        \includegraphics[height=.5\textheight]{img/les5-heroes.jpg}
    \end{figure}

    \end{column}
  \end{columns}
\end{frame}

\begin{frame}
\frametitle{Goodness of fit test}
\begin{columns}
  \begin{column} {0.2 \textwidth}
    
    \begin{figure}
      \centering
      \includegraphics[width=\textwidth]{img/les6-man.jpg}
    \end{figure}
    
  \end{column}
  
  \begin{column} { 0.8 \textwidth}
    \begin{table}[h]
      \begin{tabular}{@{}lcc@{}}
      	\toprule
      	\textbf{Type}   & \textbf{\# steekproef} & \textbf{\# populatie} \\ \midrule
      	Mutant          &          127           &         35\%          \\
      	Mens            &           75           &         17\%          \\
      	Alien           &           98           &         23\%          \\
      	God             &           27           &          8\%          \\
      	Demon           &           73           &         17\%          \\ \midrule
      	\textbf{Totaal} &          400           &         100\%         \\
      \end{tabular}
    \end{table}
  \end{column}
\end{columns}
\end{frame}

\begin{frame}
  \frametitle{Goodness of fit test}
   Is de verdeling van de steekproef ($n = 400$) representatief voor de volledige populatie (alle superhelden)?

  \begin{itemize}
    \item Welke aantallen zou je \textit{verwachten} als de steekproef representatief is?
    \item Hoe groot is de afwijking van de \textit{geobserveerde} aantallen?
    \begin{itemize}
      \item klein $\Rightarrow$ verdeling is representatief
      \item groot $\Rightarrow$ verdeling is \textbf{niet} representatief
    \end{itemize}
  \end{itemize}

  \pause
  Zie je een overeenkomst met kruistabellen en Cramer's V?
\end{frame}

\begin{frame}
\frametitle{Notatie}

In de volgende slides is:

\begin{itemize}
  \item $e$ de \textit{verwachte} frequentie voor een categorie
  \item $\pi$ de verwachte \textit{relatieve frequentie} voor een categorie (percentage)
  \item $o$ de \textit{geobserveerde} absolute frequentie
  \item $n$ de steekproefgrootte (zoals steeds)
  \item $i$ een index die een categorie in de frequentietabel aanduidt ($i \in \{1, \ldots, k\}$)
\end{itemize}
\end{frame}

\begin{frame}
  \frametitle{Goodness of fit test}
\begin{itemize}
  \item Exact representatief $\Rightarrow$ 35\% van de superhelden in de steekproef is een mutant
  \item Het verwachte aantal is dus $e = 0.35 \times 400 = 140$.
\end{itemize}
 Er geldt dus:

\[ e = n \times \pi \]

Als de verschillen $o - e$  relatief klein zijn kunnen ze toegerekend worden aan toevallige steekproeffouten.
\end{frame}

\begin{frame}
  \frametitle{Goodness of fit test}
  Beschouw $\chi^{2}$:

\[ \chi^{2} = \sum_{i=1}^{n} \frac{(o_{i} - e_{i})^{2}}{e_{i}} \]

Besluit op basis van de waarde van $\chi^2$:
\begin{itemize}
  \item klein $\Rightarrow$ verdeling representatief
  \item groot $\Rightarrow$ verdeling \textbf{niet} representatief
\end{itemize}

$\chi^{2}$ meet de mate van strijdigheid met de nulhypothese
\end{frame}

\begin{frame}
  \frametitle{Goodness of fit test}
  \begin{columns}
    \begin{column} {0.2 \textwidth}

    \begin{figure}
      \centering
        \includegraphics[width=\textwidth]{img/les6-man.jpg}
    \end{figure}

    \end{column}

    \begin{column} { 0.8 \textwidth}
      % Please add the following required packages to your document preamble:
% \usepackage{booktabs}
\begin{table}[h]
\begin{tabular}{@{}llllll@{}}
\toprule
\textbf{Type superheld} & \textbf{$o$} & \textbf{$\pi$} & \textbf{$e$} & \textbf{$o -e$} & \textbf{$\frac{(o-e)^{2}}{e}$} \\ \midrule
Mutant                  & 127          & 35\%           & 140          & -13             & 1.21                           \\
Mens                    & 75           & 17\%           & 68           & 7               & 0.72                           \\
Alien                   & 98           & 23\%           & 92           & 6               & 0.39                           \\
God                     & 27           & 8\%            & 32           & -5              & 0.78                           \\
Demon                   & 73           & 17\%           & 68           & 5               & 0.37                           \\ \bottomrule
\end{tabular}
\end{table}
    \end{column}
  \end{columns}
\end{frame}


\begin{frame}
  \frametitle{Goodness of fit test}

  \begin{itemize}
    \item De teststatistiek $\chi^{2}$ is verdeeld volgens de $\chi^2$ verdeling.
    \item Kritieke grenswaarde $g$ bij de $\chi^{2}$ verdeling: hierbij speel het aantal vrijheidsgraden een rol ($df$). Er geldt:

      \[ df = k -1 \]

    \item De kritieke grenswaarde $g$ voor een gegeven significantieniveau $\alpha$ en vrijheidsgraden $df$ kan berekend worden met de R-functie `qchisq()`.
    
      \[ P(\chi^2 < g) = 1 - \alpha \]
  \end{itemize}
\end{frame}

\begin{frame}[fragile]
  \frametitle{Goodness of fit test}
  \framesubtitle{Berekening kritieke grenswaarde}

  \begin{itemize}
    \item Stel $\alpha = 0,05$ en $df = 5 - 1 = 4$
    \item \verb|g <- qchisq(0.95, df = 4)|, dus $g = 9,49$
    \item $\chi^{2} = 3,47 < g = 9,49$
    \item Besluit: de steekproef is representatief ($H_0$ wordt aanvaard)
  \end{itemize}
\end{frame}

\begin{frame}[fragile]
\frametitle{Goodness of fit test}
\framesubtitle{Berekening overschrijdingskans}

Je kan ook de overschrijdingskans berekenen:

\[ p = P(X > \chi^2) = 1 - P(X < \chi^2) \]

\begin{itemize}
  \item \verb|p <- 1 - pchisq(3.47, df = 4)|, dus $p = 0,48$
  \item $p = 0,48 < \alpha = 0,05$
  \item Besluit: de steekproef is representatief ($H_0$ wordt aanvaard)
\end{itemize}
\end{frame}

\subsection{Toetsingsprocedure goodness of fit test}

\begin{frame}
  \frametitle{Goodness of fit test}
  \framesubtitle{Toetsingsprocedure}
  
  \begin{enumerate}
  \item \textbf{Bepalen hypotheses}
    \begin{itemize}
      \item $H_{0}$: steekproef is representatief naar populatie
      \item $H_{1}$: steekproef is niet representatief naar populatie
    \end{itemize}
  \item \textbf{Bepalen $\alpha$ en $n$} : $\alpha = 0,05$ en $n = 400$.
\end{enumerate}
\end{frame}

\begin{frame}
\frametitle{Goodness of fit test}
\framesubtitle{Toetsingsprocedure}

\begin{enumerate}
  \item \textbf{Toetsingsgrootheid berekenen}:
  \[ \chi^{2} = \sum_{i=1}^{n} \frac{(o_{i} - e_{i})^{2}}{e_{i}} \]
  \item 
  \begin{enumerate}
    \item \textbf{Kritiek gebied}: Bereken $g$ zodat $P(\chi^2 < g) = 1 - \alpha$
    \item \textbf{Overschrijdingskans}: Bereken $p = 1 - P(X < \chi^2)$
  \end{enumerate}
  
\item Besluit (de toets is altijd rechtszijdig):
  \begin{enumerate}
    \item $\chi^2 < g \Rightarrow$ aanvaard $H_0$, anders verwerp $H_0$
    \item $p > \alpha \Rightarrow$ aanvaard $H_0$, anders verwerp $H_0$
  \end{enumerate}
\end{enumerate}
\end{frame}


\subsection{Voorbeeld}

\begin{frame}
  \frametitle{Voorbeeld gezinnen}
  Beschouw alle gezinnen met 5 kinderen in een bepaalde gemeenschap.
  \pause
  Met betrekking tot samenstelling zijn er 6 mogelijkheden.
\begin{enumerate}
  \item 5 jongens
  \item 4 jongens, 1 meisje
  \item 3 jongens, 2 meisjes
  \item 2 jongens, 3 meisjes
  \item 1 jongen, 4 meisjes
  \item 5 meisjes
\end{enumerate}
Het onderzoek bevat 1022 gezinnen met 5 kinderen
\begin{center}
Zijn de waargenomen aantallen in de 6 klassen representatief voor een populatie waar de kans om een jongen te krijgen = kans om een meisje te krijgen = 0,5?
\end{center}
\end{frame}

\begin{frame}
  \frametitle{Voorbeeld}
  \begin{table}[h]
\begin{tabular}{@{}llllllll@{}}
\toprule
i       & 0  & 1   & 2   & 3   & 4   & 5  &  \\ \midrule
$o_{i}$ & 58 & 149 & 305 & 303 & 162 & 45 &  \\ \bottomrule
\end{tabular}
\end{table}
\pause
Indien de veronderstelling waar is wordt de kans $\pi_{i}$ om $i$ jongens te krijgen bepaald door een binominaalvedeling met parameters $n=5$ en $p=0.5$.
Bv. De kans om 2 jongens te krijgen met 5 kinderen is gelijk aan :

\[ (0.5)^{2} \times (1-0.5)^{5-2} \times \binom{5}{2} \]

Algemeen geldt dus:

\[ \pi_{i} = \binom{5}{i}\times 0.5^{i} \times 0.5^{5-i} = \frac{5!}{i!(5-i)!}\times 0.5^{i} \]
\end{frame}

\begin{frame}
  \frametitle{Voorbeeld}
  \begin{table}[h]
\begin{tabular}{@{}llllllll@{}}
	\toprule
	$i$                   & 0     & 1       & 2      & 3        & 4      & 5     & Tot.  \\ \midrule
	$o_i$                 & 58    & 149     & 305    & 303      & 162    & 45    & 1022  \\
	$\pi_i$               & 0.03  & 0.15    & 0.31   & 0.31     & 0.15   & 0.031 & 1     \\
	$e_i$                 & 31.68 & 159.43  & 318.86 & 318.86   & 159.43 & 31.68 &       \\
	$\frac{(o-e)^{2}}{e}$ & 21.86 & 0.68259 & 0.60   & 0.78     & 0.041  & 5.59  & 29.57 \\
	$r_i$                 & 4.74  & -0.89   & -0.93  & -1.07106 & 0.22   & 2.40  &       \\ \bottomrule
\end{tabular}
\end{table}
\end{frame}

\begin{frame}
  \frametitle{Voorbeeld}
  \begin{enumerate}
  \item \textbf{Bepalen hypotheses}
    \begin{itemize}
      \item $H_{0}$: steekproef is representatief naar populatie
      \item $H_{1}$: steekproef is niet representatief naar populatie
    \end{itemize}
  \item \textbf{Bepalen $\alpha$ en $n$} : $\alpha = 0.01$ en $n = 1022$.
  \item \textbf{Toetsingsgrootheid en waarde ervan in steekproef}:
  \[ \chi^{2} = \sum_{i=1}^{n} \frac{(o_{i} - e_{i})^{2}}{e_{i}} = 29.5766 \]
  \item \textbf{Bereken en teken kritiek gebied}:  kritieke grens is $15.0863$. Onze toetsingsgrootheid ligt dus in het kritieke gebied dus verwerpen we $H_{0}$.
\end{enumerate}
\end{frame}

\subsection{Gestandaardiseerde residuen}
\begin{frame}
  \frametitle{Gestandaardiseerde residuen}
  \alertbox{De \textcolor{hgyellow}{gestandaardiseerde residuen} duiden aan welke klassen de grootste bijdrage leveren aan de waarde van de grootheid. }
  \[ r_{i} = \frac{o_{i} - n \pi_{i}}{\sqrt{n \pi_{i}(1-\pi_{i})}} \]

  \begin{itemize}
    \item Er geldt algemeen: waarden groter dan 2 of kleiner dan $-2$ zijn extreem.
  \end{itemize}
  We kunnen dus besluiten dat het aantal gezinnen waarin alle kinderen hetzelfde geslacht hebben groter mag worden genoemd dan verwacht.

\end{frame}

\begin{frame}
  \frametitle{Voowaarden}
   Om de toets te mogen toepassen dient aan de volgende voorwaarden te zijn voldaan (Regel van Cochran)
\begin{enumerate}
  \item Voor alle categorie\"en moet gelden dat de verwachte waarde $e$ groter is dan 1.
  \item In ten hoogste 20 \% van de categori\"en mag de verwachte waarde $e$ kleiner dan 5 zijn.
\end{enumerate}
\end{frame}

\section{$\chi^{2}$ toets voor twee variabelen}

\begin{frame}
  \frametitle{$\chi^{2}$ toets voor twee variabelen}
  De Chi-kwadraattoets \index{$\chi^{2}$kwadraatkruistabeltoets} laat zich eenvoudig uitbreiden tot een onderzoeksontwerp met twee variabelen, met respectievelijk $r$ en $k$ niveaus.
\end{frame}

\begin{frame}
  \frametitle{Rokersonderzoek}
  In deze studie onderzochten Doll en Hill de relatie tussen roken en longkanker. Doll en Hill schreven in 1951 alle Britse huisartsen aan met het verzoek om gegevens over hun leeftijd en rookgedrag. Vervolgens hielden ze jarenlang de overlijdensberichten en de doodsoorzaak bij en herhaalden dit periodiek. De eerste uitkomsten, na circa vier jaar, zijn in de volgende tabel samengevat.

  \begin{table}[h]
    \begin{tabular}{@{}lllll@{}}
      \toprule
                     & \textbf{Longkanker} & \textbf{Niet} & \textbf{Wel} & \textbf{Totaal} \\ \midrule
      \textbf{Roker} & \textbf{Wel}        & 21178         & 83           & 21261           \\
                     & \textbf{Niet}       & 3092          & 1            & 3093            \\
                     & \textbf{Totaal}     & 24270         & 84           & 24354           \\ \bottomrule
    \end{tabular}
  \end{table}
\end{frame}

\begin{frame}
  \frametitle{Rokersonderzoek}
  \begin{table}[h]
\begin{tabular}{@{}lllll@{}}
\toprule
      & \textbf{Longkanker} & \textbf{Niet} & \textbf{Wel} & \textbf{Totaal} \\ \midrule
Roker & Wel                 & 21178         & 83           & 21261           \\
      & Niet                & 3092          & 1            & 3093            \\
      & Totaal              & 24270         & 84           & 24354           \\ \bottomrule
\end{tabular}
\end{table}

\begin{columns}
  \begin{column}{0.3 \textwidth}

  \begin{figure}
    \centering
      \includegraphics[width=1.00\textwidth]{img/les-6-smoking.jpg}
  \end{figure}

  \end{column}
  \begin{column}{0.7 \textwidth}

  \begin{itemize}
    \item \dots slechts $\frac{84}{ 24354} \times 100 = 0.35\% $ van de Britse artsen aan longkanker overleden
    \item \dots met slechts $\frac{83}{21261} \times 100 = 0.39\%$ van de rokers onder hen
    \item \dots maar  is wel  meer dan hetzelfde cijfer voor de niet-rokers $\frac{1}{3093} * 100 = 0.032\%$.
  \end{itemize}
  \end{column}
\end{columns}
\end{frame}

\begin{frame}
  \frametitle{Rokersonderzoek}
  % Please add the following required packages to your document preamble:
% \usepackage{booktabs}
\begin{table}[h]
\begin{tabular}{@{}lllll@{}}
\toprule
      & \textbf{Longkanker} & \textbf{Niet} & \textbf{Wel} & \textbf{Totaal} \\ \midrule
Roker & Wel                 & 21188         & 73.3         & 21261           \\
      & Niet                & 3082.3        & 10.7         & 3093            \\
      & Totaal              & 24270         & 84           & 24354           \\ \bottomrule
\end{tabular}
\end{table}

\begin{columns}
  \begin{column}{0.3 \textwidth}

  \begin{figure}
    \centering
      \includegraphics[width=1.00\textwidth]{img/les-6-smoking.jpg}
  \end{figure}

  \end{column}
  \begin{column}{0.7 \textwidth}

  \begin{itemize}
    \item $\chi^{2} = 10.35$
    \item We zien in de tabel  dat er wel een erg groot verschil is tussen de geobserveerde aantallen rokers die overlijden aan longkanker en de verwachte waarden in deze cel.
    \item Hetzelfde geldt voor het geringe aantal huisartsen dat niet rookt, maar wel aan longkanker overleden is.
  \end{itemize}
  \end{column}
\end{columns}
\end{frame}

\begin{frame}
  \frametitle{Rokersonderzoek}
  \begin{enumerate}
  \item \textbf{Bepalen hypotheses}
    \begin{itemize}
      \item $H_{0}$: in de populatie is er geen samenhang tussen onafhankelijke en afhankelijke variabele
      \item $H_{1}$: er bestaat wel een samenhang tussen de variabelen in de populatie
    \end{itemize}
  \item \textbf{Bepalen $\alpha$ en $n$} : $\alpha = 0.05$ en $n = 24354$.
  \item \textbf{Toetsingsgrootheid en waarde ervan in steekproef}:
  \[ \chi^{2} = \sum_{i=1}^{n} \frac{(o_{i} - e_{i})^{2}}{E_{i}} = 10.35 \]
  \item \textbf{Bereken en teken kritiek gebied}:  kritieke grens is 3.8415 en aantal vrijheidsgraden $df = (r-1)(k-1)$ Onze toetsingsgrootheid ligt dus in het kritieke gebied dus verwerpen we $H_{0}$.
\end{enumerate}
\end{frame}

\begin{frame}
  \frametitle{Oorzakelijk verband}
  We moeten derhalve $H_{0}$, dat er geen relatie is tussen beide variabelen, verwerpen ten gunste van $H_{1}$ dat er wel een relatie is tussen beide variabelen: rokers sterven vaker aan longkanker dan niet-rokers.
  \begin{columns}
  \begin{column}{0.3 \textwidth}

  \begin{figure}
    \centering
      \includegraphics[width=1.00\textwidth]{img/les-6-smoking2.jpg}
  \end{figure}

  \end{column}
  \begin{column}{0.7 \textwidth}

  \begin{itemize}
    \item  \dots rokers zijn ouder dan de niet-rokers
    \item \dots de rokers wonen veelal in de grote steden met
meer vervuilde lucht dan de niet-rokers
    \item \dots speciale genetische dispositie die zowel van invloed is op de verslaving aan tabak, als op de kans om longkanker te krijgen.
  \end{itemize}
  \end{column}
\end{columns}
Voor een causale interpretatie van de gegevens (het betreft hier immers geen experiment), moeten we op zijn minst de beschikking hebben over een theorie die de relatie tussen roken en longkanker expliciteert.

\end{frame}

\end{document}
