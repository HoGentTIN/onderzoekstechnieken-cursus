% Onderzoekstechnieken/bachelorproef: Software installeren

\documentclass[aspectratio=169]{beamer}

\usepackage{minted}
\setminted{bgcolor=gray!20,gobble=4}

%---------- Stijl -------------------------------------------------------------

\usetheme{hogent}

\usecolortheme{hgwhite} % witte achtergrond, zwarte tekst

%---------- Info over de presentatie ------------------------------------------

\title{Installatie software.}
\subtitle{Onderzoekstechnieken}
\author{Thomas Aelbrecht \and Jens Buysse \and Wim {De Bruyn} \and Pieter-Jan Maenhaut \and Bert {Van Vreckem}}
\date{AJ 2020--2021}

\begin{document}

\begin{frame}
  \maketitle
\end{frame}

\begin{frame}
  \frametitle{What's on the menu today?}

  \tableofcontents
\end{frame}

\begin{frame}
  \frametitle{Voor we beginnen}

  \begin{itemize}
    \item Installatie-instructies voor Windows
    \item Voor Mac/Linux: zie cursus Github-repo, README
    \item Let op! Kopieer geen tekst uit PDF naar console
  \end{itemize}

\end{frame}

\section{Stap 0. Installeer Chocolatey.}

\begin{frame}
  \frametitle{Chocolatey}

  \begin{itemize}
    \item Chocolatey = package manager voor Windows
    \item Ga naar \url{https://chocolatey.org/} > \emph{Get Started}
    \item Volg de instructies!
  \end{itemize}

\end{frame}

\begin{frame}[fragile]
  \frametitle{Onderhoud}

  \begin{itemize}
    \item Zoeken: \verb|choco search ZOEKTERM|
    \begin{itemize}
      \item Of: \url{https://chocolatey.org/packages}
    \end{itemize}
    \item Installeren: \verb|choco install -y PACKAGE|
    \item Lijst geïnstalleerde software: \verb|choco list --localonly|
    \item Updates installeren: \verb|choco upgrade all|
    \item Tip: maak een PowerShell script voor de installatie
  \end{itemize}

\end{frame}

\section{Stap 1. Installeer Git}

\begin{frame}[fragile]
  \frametitle{Git}

  \begin{itemize}
    \item We veronderstellen dat Git (en Git Bash) al geïnstalleerd is!
    \item Zoniet:
  \end{itemize}

  \begin{minted}{powershell}
    PS> choco install -y git
    PS> choco install -y gitkraken
  \end{minted}
\end{frame}

\begin{frame}[fragile]
  \frametitle{Configuratie}

  Bij voorkeur in een Git Bash console:

  \begin{minted}{shell-session}
    $ git config --global user.name "Voornaam Naam"
    $ git config --global user.email "voornaam.naam@hogent.be"
    $ git config --global github.user "GithubUserName"
    $ git config --global push.default simple
    $ git config --global core.autocrlf input
    $ git config --global pull.rebase = true
    $ git config --global rebase.autoStash = true
  \end{minted}

\end{frame}

\section{Stap 2. R en RStudio.}

\begin{frame}
  \frametitle{R, RStudio}

  \begin{description}
    \item[R] (functionele) programmeertaal voor data-analyse
    \item[RStudio] (open source) IDE voor R
  \end{description}

\end{frame}

\begin{frame}[fragile]
  \frametitle{Installatie}

  \begin{minted}{powershell}
    PS> choco install -y r.project
    PS> choco install -y r.studio
  \end{minted}

\end{frame}

\begin{frame}[fragile]
  \frametitle{Configuratie}

  Open RStudio, console:

  \begin{minted}{r}
    > install.packages("tidyverse")
  \end{minted}

\end{frame}

\section{Stap 3. MikTeX, TeXstudio en JabRef.}

\begin{frame}
  \frametitle{Software voor {\LaTeX}}

  \begin{itemize}
    \item {\LaTeX} = tekstzetsysteem voor professioneel opgemaakte documenten
    \item Niet WYSIWYG, maar markuptaal + compiler
    \item Output typisch PDF
  \end{itemize}

  \bigskip

  Aanbevolen software:

  \begin{description}
    \item[MikTeX] {\LaTeX} distributie voor Windows, compilers
    \item[TeXstudio]  IDE voor {\LaTeX}
    \item[JabRef] Bibliografische database voor gebruik in {\LaTeX}
  \end{description}

\end{frame}

\begin{frame}[fragile]
  \frametitle{Installatie}

  \begin{minted}{powershell}
    PS> choco install -y miktex
    PS> choco install -y texstudio
    PS> choco install -y JabRef
  \end{minted}

\end{frame}

\begin{frame}
  \frametitle{Onderhoud Mik{\TeX}}

  \begin{itemize}
    \item Open Mik{\TeX} console als Administrator
    \item Settings:
      \begin{itemize}
        \item Always install missing packages on-the-fly
        \item Default paper format: A4
      \end{itemize}
    \item Updates: Check for updates
      \begin{itemize}
        \item In geval van fouten: installeer updates
      \end{itemize}
    \item Zorg dat Mik{\TeX} niet geblokkeerd wordt door firewall/antivirus
  \end{itemize}
\end{frame}

\begin{frame}[fragile]
  \frametitle{Lettertypes}

  HOGENT huisstijl:

  \begin{itemize}
    \item Montserrat Regular, ExtraBold: \url{https://fonts.google.com/specimen/Montserrat}
    \item Code Pro Black: \url{https://www.dafontfree.net/freefonts-code-pro-black-f62435.htm}
  \end{itemize}

  Code-font met ligaturen (e.g. \(\leftarrow\) i.p.v. \verb|<-|):

  \begin{itemize}
    \item Fira Code: \texttt{choco install firacode}
  \end{itemize}

\end{frame}

\begin{frame}
  \frametitle{Configuratie TeXstudio}

  \begin{itemize}
    \item Options > Configure TeXstudio
    \item Commands:
      \begin{itemize}
        \item XeLaTeX\@: \texttt{xelatex -synctex=1 -interaction=nonstopmode -shell-escape \%.tex}
        \item Latexmk: \texttt{latexmk -xelatex -shell-escape -synctex=1 -interaction=nonstopmode -file-line-error \%}
      \end{itemize}
    \item Build:
      \begin{itemize}
        \item Default Compiler: Latexmk
        \item Default Bibliography tool: Biber
      \end{itemize}
    \item Editor, enz.\ naar eigen voorkeur
  \end{itemize}

\end{frame}

\section{Alternatief: Visual Studio Code.}

\begin{frame}
  \frametitle{VS Code}

  \begin{itemize}
    \item VS Code kan RStudio, TeXstudio en Markdown editor vervangen
    \item Configuratie is wat complexer
  \end{itemize}

\end{frame}

\begin{frame}
  \frametitle{Installatie}

  \begin{itemize}
    \item \texttt{choco install -y vscode}
    \item Extensies:
      \begin{itemize}
        \item {\LaTeX} Workshop (James Yu)
        \item R (Yuki Ueda)
      \end{itemize}
  \end{itemize}

\end{frame}

\section{Stap 4. Repo voor eigen oefeningen, nota's, enz.}

\begin{frame}
  \frametitle{Repo voor eigen oefeningen en nota's}

  \begin{enumerate}
    \item In RStudio, kies \emph{File > New Project}
    \item Selecteer \emph{New Directory}, dan \emph{New Project}
    \item Kies naam voor de directory (bv.~\texttt{ozt-oefeningen})
    \item Vink \emph{Create a git repository} aan
  \end{enumerate}

\end{frame}

\begin{frame}
  \frametitle{Repo voor eigen oefeningen en nota's}

  \begin{enumerate}
    \setcounter{enumi}{4}
    \item Maak directory \texttt{data} aan
    \item Kopieer alle databestanden (\texttt{.csv}, \texttt{.txt}, \texttt{.sav}) uit de cursus-repo (onder \texttt{oefeningen/datasets} en \texttt{cursus/data}) naar deze directory
    \item Maak eerste Git commit
  \end{enumerate}

\end{frame}

\end{document}