%%----------------------------------------------------------------------------
%% Presentatie HoGent Bedrijf en Organisatie
%%----------------------------------------------------------------------------
%% Auteur: Bert Van Vreckem [bert.vanvreckem@hogent.be]

\documentclass{beamer}

%==============================================================================
% Aanloop
%==============================================================================

%---------- Packages ----------------------------------------------------------

\usepackage{graphicx,multicol}
\usepackage{comment,enumerate,hyperref}
\usepackage{amsmath,amsfonts,amssymb}
\usepackage{tikz}
\usepackage[dutch]{babel}
\usepackage[utf8]{inputenc}
\usepackage{multirow}
\usepackage{eurosym}
\usepackage{listings}
\usepackage[T1]{fontenc}
\usepackage{lmodern}
\usepackage{textcomp}

%---------- Configuratie ------------------------------------------------------

\usetikzlibrary{arrows,shapes,backgrounds,positioning,shadows}

\usetheme{hogent}

%---------- Commando-definities -----------------------------------------------

\newcommand{\tabitem}{~~\llap{\textbullet}~~}

%---------- Info over de presentatie ------------------------------------------

\title[Intro]{Onderzoekstechnieken -- Intro}
\author{Anita Bernard, Jens Buysse, Bert {Van Vreckem}}
\date{AJ 2016-2017}

%==============================================================================
% Inhoud presentatie
%==============================================================================

\begin{document}

%---------- Front matter ------------------------------------------------------

% Dia met het HoGent logo
\HoGentLogo

% Titeldia met faculteitslogo
\titleframe

%---------- Inhoud ------------------------------------------------------------

% Dia voor sectiekop, voorbeeld met een afbeelding onderaan de pagina
\section{1}

\sectionframelogo{Onderzoekstechnieken}

\begin{frame}
  \frametitle{Onderzoekstechnieken}

  \scaledimg{img/intro-01.jpg}
\end{frame}

\begin{frame}
  \frametitle{Voorbeelden van slecht onderzoek}

  \scaledimgvert{img/intro-02.png}{img/intro-03.png}
\end{frame}

\begin{frame}
  \frametitle{Voorbeelden van slecht onderzoek}

  \scaledimg{img/intro-04.png}
\end{frame}

\begin{frame}
  \frametitle{Voorbeelden van slecht onderzoek}

  \scaledimg{img/intro-05.jpg}
\end{frame}

\begin{frame}
  \frametitle{Voorbeelden van slecht onderzoek}

  \scaledimg{img/intro-10.jpg}
\end{frame}

\begin{frame}
	\frametitle{Voorbeelden van slecht onderzoek}
	{\tiny \textit{'Het overheidsbeslag is afgelopen jaar gedaald, maar niet zo spectaculair als een grafiek van de N-VA doet uitschijnen. Economieprofessor Tom Verbeke zag dat en wees N-VA met de vinger. 'Als een student dat soort truken toepast, zal hij zich serieus mogen verdedigen.'} }
	\scaledimg{img/intro-nva.jpg}
\end{frame}

\section{2}
\sectionframelogo{Praktisch}

\begin{frame}
  \frametitle{Overzicht leerstof}

\begin{table}[h]
\begin{tabular}{l|l}
  \multirow{2}{*}{\textbf{Inleiding}} &
     \tabitem Inleiding tot het vak \\
   & \tabitem Invoer van gegevens \\

  \hline
  \multirow{2}{*}{\textbf{Analyse op 1 variabele}} &
      \tabitem Enkelvoudige statistieken \\
    & \tabitem Eenvoudige grafieken\\

  \hline
  \multirow{2}{*}{\textbf{Analyse op 2 variabelen}} &
      \tabitem Eenvoudige grafieken \\
    & \tabitem Correlatie en regressie\\

  \hline
  \multirow{3}{*}{\textbf{Steekproeven en kansverdeling}} &
      \tabitem Populatie\\
    & \tabitem Steekproef\\
    & \tabitem Normale verdeling\\

  \hline
  \textbf{Toetsingsprocedure en} & \tabitem Toetsen van hypothesen \\
  \textbf{Chi-kwadraattoets}     & \tabitem $z$-toets, $t$-toets\\
                                 & \tabitem $\chi^{2}$-toets\\

\end{tabular}
\end{table}
\end{frame}

\begin{frame}
  \frametitle{Leermaterialen}

  \begin{itemize}
    \item Slides ter ondersteuning
      \begin{itemize}
        \item bevatten grafieken, tabellen, \ldots die in de les aan bod komen
      \end{itemize}
    \item Theorie aan bord, hoorcollege
      \begin{itemize}
        \item Volg de lessen \textbf{actief} mee!
        \item \textbf{Neem zelf notities!}
      \end{itemize}
    \item Syllabus
  \end{itemize}

  \brightbox{Kom naar de les en \textcolor{HoGentAccent6}{neem nota's}}

  \begin{center}
    \includegraphics[height=3cm]{img/intro-06.jpg}
  \end{center}

\end{frame}

\begin{frame}
    \frametitle{Software}
    
    \begin{itemize}
        \item Git:
        \begin{itemize}
            \item Git client
            \item Account op Github
        \end{itemize}
        \item {\LaTeX}:
        \begin{itemize}
            \item MikTeX/MacTeX/Texlive
            \item TexStudio (of andere editor)
        \end{itemize}
        \item Statistiek: RStudio Desktop
    \end{itemize}

    \centering
    Alle benodigde software is gratis/open source
\end{frame}

\begin{frame}
  \frametitle{Organisatie van de lessen}

  \begin{itemize}
    \item Hoorcollege
      \begin{itemize}
        \item Theorie en grondleggen van de basis
      \end{itemize}
    \item Werkcollege
      \begin{itemize}
        \item Uitdiepen van de kennis
        \item Klassikaal oefeningen maken
        \item De software leren gebruiken
      \end{itemize}
    \item Begeleide zelfstudie
      \begin{itemize}
        \item Klassikaal taak uitwerken
        \item ``Passieve'' begeleiding
      \end{itemize}
  \end{itemize}
\end{frame}

\begin{frame}
  \frametitle{Evaluatie van het vak}

  \begin{itemize}
    \item Eerste examenkans
      \begin{itemize}
        \item Niet-periodegebonden evaluatie (taak): 30\% van het totaal
          \begin{itemize}
            \item Individueel uitschrijven bachelorproefvoorstel
            \item Empirisch onderzoek in groep
          \end{itemize}
        \item Periodegebonden evaluatie (examen): 70\% van het totaal
      \end{itemize}
    \item Tweede examenkans
      \begin{itemize}
        \item Schriftelijk examen: 100\%
      \end{itemize}
  \end{itemize}

  \begin{center}
    \includegraphics[height=3cm]{img/intro-07}
  \end{center}
\end{frame}

\begin{frame}
  \frametitle{NPE: Uitschrijven bachelorproefvoorstel}

  \begin{enumerate}
    \item Bepalen onderzoeksdomein
    \item Literatuurstudie
    \item Vastleggen onderzoeksvraag en formuleren doelstelling(en)
    \item Uitschrijven bachelorproefvoorstel volgens sjabloon
  \end{enumerate}

  Zie Chamilo, onder Opdrachten
\end{frame}

\begin{frame}
  \frametitle{NPE: Casus onderzoeksproces}

  \brightbox{Performantievergelijking tussen databases}

  \begin{enumerate}
    \item Lezen van een wetenschappelijke paper
    \item Experimenten reproduceren
    \item Onderzoeksvraag bijsturen, afbakenen en vastleggen
    \item Uitvoeren van de tests op een gecontroleerde manier.
    \item Resultaten statistisch verantwoord verklaren en neerschrijven
    \item Verslaggeving over het onderzoek
  \end{enumerate}

  Opdrachtbeschrijving op Chamilo, onder Opdrachten
\end{frame}


%---------- Back matter -------------------------------------------------------

\end{document}
