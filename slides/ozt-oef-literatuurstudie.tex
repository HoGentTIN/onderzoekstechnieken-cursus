%%----------------------------------------------------------------------------
%% Onderzoekstechnieken: Literatuurstudie
%%----------------------------------------------------------------------------

\documentclass[aspectratio=169]{beamer}

%==============================================================================
% Aanloop
%==============================================================================

%---------- Vormgeving --------------------------------------------------------

\usetheme{hogent}

\usecolortheme{hgwhite} % witte achtergrond, zwarte tekst

\usepackage{graphicx,multicol}
\usepackage{comment,enumerate,hyperref}
\usepackage{amsmath,amsfonts,amssymb}
\usepackage[dutch]{babel}
\usepackage{multirow}
\usepackage{eurosym}
\usepackage{listings}
\usepackage{textcomp}
\usepackage{framed}
\usepackage{wrapfig}
\usepackage{tabu} %needed for \tabulinesep
\usepackage{wrapfig}
\usepackage{pgf-pie}
\usepackage{pgfplots}
\usepackage{booktabs}
\usepackage{pgfplotstable}
\usepackage{changepage}
\usepackage{ulem} % for \sout{text} (strikethrough)
\usepackage{fancyvrb} % for \begin{Verbatim} (LaTeX controls within verbatim)
\usepackage{csquotes}

%---------- Configuratie ------------------------------------------------------

\pgfplotsset{compat=1.16}
\usetikzlibrary{arrows,shapes,backgrounds,positioning,shadows,calc}
\usetikzlibrary{pgfplots.statistics}

%---------- Commando-definities -----------------------------------------------

\newcommand{\tabitem}{~~\llap{\textbullet}~~}
\newcommand{\alertbox}[2][hgblue]{%
  \setbeamercolor{alertbox}{bg=#1,fg=white}
  \begin{beamercolorbox}[sep=2pt,center]{alertbox}
    \textbf{#2}
  \end{beamercolorbox}
}

%---------- Bibliografie ------------------------------------------------------

\usepackage[backend=biber,style=apa]{biblatex}
\DeclareLanguageMapping{dutch}{dutch-apa}
\addbibresource{ozt-oef-1-latex.bib}

%---------- Info over de presentatie ------------------------------------------

\title{Een literatuurstudie uitvoeren.}
\subtitle{Onderzoekstechnieken}
\author{Thomas Aelbrecht \and Jens Buysse \and Wim {De Bruyn} \and Pieter-Jan Maenhaut \and Bert {Van Vreckem}}
\date{AJ 2020--2021}

%==============================================================================
% Inhoud presentatie
%==============================================================================

\begin{document}

\begin{frame}
  \maketitle
\end{frame}

\begin{frame}
  \frametitle{What's on the menu today?}

  \tableofcontents
\end{frame}

\section{Wat is een literatuurstudie?}

\begin{frame}
  \frametitle{Literatuurstudie.}

  \begin{itemize}
    \item Onderdeel van elk artikel, eindwerk
    \item Inleiding op het onderwerp
    \item Samenvatting van wat auteur gelezen heeft
    \item Verwijzingen naar vakliteratuur
  \end{itemize}

\end{frame}

\begin{frame}
    \frametitle{Doel van de literatuurstudie.}

    \begin{itemize}
      \item Wat is de huidige stand van zaken?
      \item Wat zeggen experts er over?
      \item Onderzoeksvragen verduidelijken, in context plaatsen
      \item Er is een probleem dat een oplossing vraagt
    \end{itemize}

    \bigskip

    \alertbox{\textcolor{hgyellow}{Elke bewering} in een literatuurstudie moet je bewijzen a.h.v.~referenties}
  \end{frame}

\begin{frame}
  \frametitle{Vaak voorkomende fouten.}

  \begin{itemize}
    \item Onvolledige/geen referentielijst
    \item Enkel URLs
    \item Te weinig informatie in referentielijst
    \begin{itemize}
      \item \(\Rightarrow\) bronnen niet terug te vinden
    \end{itemize}
    \item Onaanvaardbare bronnen
    \item Verkeerde opmaak
    \item Geen verwijzingen naar bronnen vanuit de tekst
    \item Opgedeeld per type (boek, web, enz.)
  \end{itemize}
\end{frame}


\begin{frame}[plain]
  \frametitle{Doel van de referentielijst.}

  Lezers toelaten:

  \begin{itemize}
    \item De gerefereerde bronnen op te zoeken
    \item Waarde bronnen zelf te beoordelen
  \end{itemize}

  {\pause}

  Strikte, vastgelegde vorm:

  \begin{itemize}
    \item Vastgelegde regels, afh.~publicatie (bv. IEEE, APA, Chicago Manual of Style, \ldots)
    \item Vaste volgorde (volgorde in de tekst of alfabetisch)
    \item Lijst URLs is onvoldoende!
  \end{itemize}

  {\pause}

  \alertbox{Gebruik \textcolor{hgyellow}{referentie-software} om je referentielijst op te maken!}
\end{frame}

\begin{frame}
  \frametitle{Wanneer refereren naar de literatuur?}

  \begin{itemize}
    \item Definities, eerste vermelding vakterm
    \item Overnemen uit bron van letterlijk citaat, vertaling/parafrase, of afbeelding
    \begin{itemize}
      \item Geen referentie = \alert{plagiaat!}
    \end{itemize}
    \item Aanhalen resultaten vorig onderzoek
    \item Vrijwel elke bewering die je doet over het vakgebied
  \end{itemize}

  \bigskip

  \alertbox{Referenties geven \textcolor{hgyellow}{geloofwaardigheid} aan je literatuurstudie}
\end{frame}

\subsection{Informatie opzoeken en bijhouden.}

\begin{frame}
  \frametitle{Soorten bronnen.}

  \begin{description}
    \item[Primaire] \textbf{Ruwe data} (zelf) verzameld tijdens onderzoek

      Datasets, enquêtes, interviews, \ldots

    \item[Secundaire] \textbf{Publicatie} van kennis, onderzoek, \ldots door anderen

      Artikel in wetenschappelijke journal of vaktijdschrift, presentatie op conferentie, boek, \ldots

    \item[Tertiaire] \textbf{Indexen}

      Zoekmachine, encyclopedie, databank bibliotheek, \ldots

  \end{description}

  \alertbox{Enkel \textcolor{hgyellow}{secundaire bronnen} zijn bruikbaar als referenties.}
\end{frame}

\begin{frame}
  \frametitle{Informatie opzoeken.}

  Start bij \alert{tertiare} bronnen:

  \begin{itemize}
    \item Google Scholar: \url{https://scholar.google.com/}
    \item ScienceDirect: \url{https://www.sciencedirect.com/}
    \item Springer Online Journals: \url{https://link.springer.com/}
    \item Catalogus Bib: \url{https://www.hogent.be/student/bibliotheken/}
    \item Wikipedia (uiteraard\dots)
  \end{itemize}
\end{frame}

\begin{frame}
  \frametitle{Informatie opzoeken.}

  \alertbox{Let op: tertiare bronnen, ihb.~Wikipedia, zijn zelf \textcolor{hgyellow}{niet} aanvaardbaar als referentie}

  {\pause}

  \begin{itemize}
    \item Geen garantie op juistheid
    \item Beweringen niet altijd aangetoond: [citation needed]
    \item \alert{Wél} een goed startpunt (bv.\ referenties onderaan artikel)
  \end{itemize}
\end{frame}

\begin{frame}
  \frametitle{HOGENT tools.}

  \begin{itemize}
    \item<+-> Bezoek website HOGENT \textbf{bib}: \url{https://www.hogent.be/student/bibliotheken/}
    \begin{itemize}
      \item Handleidingen
      \item Kritisch denken en onderzoekscompetenties
      \item Zoeken op Bachelorproeven
    \end{itemize}
    \item<+-> \textbf{Apollox} (\url{https://apollox.hogent.be/})
    \begin{itemize}
      \item start applicaties/zoekmachines vanop HoGent
      \item vb.~SPSS, Endnote, Visio, Office
      \item Online journals en ebooks waar HoGent een abonnement voor heeft (bv.~ScienceDirect, SpringerLink)
    \end{itemize}
    \item<+-> Zet \textbf{VPN} aan tijdens het opzoeken.
  \end{itemize}
\end{frame}

\begin{frame}
  \frametitle{Startpunten.}
  \framesubtitle{``Wetenschappelijke'' literatuur.}

  \begin{itemize}
    \item<+-> \textbf{Google Scholar}
      \begin{itemize}
        \item Gebruik vanop de campus, VPN of via Apollo
        \item Kijk uit naar download-links aan de rechterkant: [PDF] of [fulltext@Hogent]
        \item Referentie in Bib{\TeX}-formaat verkrijgen (via instellingen)
        \item Gebruik zoekopties (bv.~beperken in tijd)
      \end{itemize}
    \item<+-> \textbf{SpringerLink}
      \begin{itemize}
        \item e-boeken over computerwetenschappen
        \item journals, artikels
      \end{itemize}
    \item<+-> \textbf{Elsevier ScienceDirect}: journals
  \end{itemize}
\end{frame}

\begin{frame}
  \frametitle{Startpunten.}
  \framesubtitle{Vakliteratuur ICT.}

  \begin{itemize}
    \item Presentaties \textbf{vakconferenties} (via Youtube, Vimeo, Slideshare, \dots)
    \begin{itemize}
      \item vb. Google IO, WWDC, FOSDEM, Velocity, \dots
      \item Zoeken via Lanyrd (\url{http://lanyrd.com/topics/})
    \end{itemize}
    \item<+-> Technische \textbf{portaalsites} voor ict-gerelateerde onderwerpen
    \begin{itemize}
      \item vb.~dzone.com, infoq.com, TechNet, enz.
    \end{itemize}
  \end{itemize}
\end{frame}

\begin{frame}
  \frametitle{Startpunten.}
  \framesubtitle{Vakliteratuur ICT.}

  \begin{itemize}
    \item<+-> Wie zijn de belangrijkste namen in de \textbf{community}?
    \begin{itemize}
      \item Keynotes op conferenties, auteurs van standaardwerken, enz.
      \item Volg ze op Twitter
      \item Zoek hun blog
    \end{itemize}
    \item<+-> \textbf{Technische blogs} van bedrijven
    \begin{itemize}
      \item Google Developers Blog, Twitter Engineering/Developer Blog, Netflix Tech Blog, \dots
    \end{itemize}
  \end{itemize}
\end{frame}


\begin{frame}
  \frametitle{Bruikbare bronnen.}
  \framesubtitle{Voor een bachelorproef Informatica}

  \begin{description}
    \item[Journal article]<+-> in wetenschappelijk, peer reviewed tijdschrift
    \item[Conference proceedings]<+-> artikel gepresenteerd op wetenschappelijk, peer reviewed congres
    \item[Thesis]<+-> doctoraat (PhD), Master, Bachelor
    \item[Handleiding]<+-> bv.\ van gebruikte of besproken software
    \item[Boek]<+-> let op: iedereen kan een boek uitgeven. Controleer auteur, uitgeverij, doelpubliek (Springer vs.\ ``for dummies'')
    \item[Presentatie]<+-> door erkend vakexpert, bv.\ op vakconferentie (via Youtube, Vimeo, enz.)
    \item[Blogartikel]<+-> indien geschreven door erkend vakexpert
    \item[Vaktijdschrift]<+-> let op: geschreven door journalist (is geen vakexpert)
  \end{description}
\end{frame}

\begin{frame}
  \frametitle{Onbruikbare bronnen.}

  \begin{itemize}
    \item Eender welk werk zonder auteur of publicatiejaar
    \item Wikipedia-artikel
    \item Blogartikel van iemand buiten het vakgebied
    \item ``White papers'' (meestal niet objectief)
    \item Homepage van een besproken product of bedrijf
    \begin{itemize}
      \item Evt.~in de tekst zelf of als voetnoot
    \end{itemize}
    \item \dots
  \end{itemize}
\end{frame}

\begin{frame}
  \frametitle{Checklist kwaliteit bronnen.}
  \framesubtitle{Doe de CRAP test!}

  \begin{description}
    \item[Current] Publicatiejaar? Is het voldoende recent? Is dit nog conform de \textbf{state-of-the-art}?
    \item[Reliable] Is het objectief? Gebalanceerd of eenzijdig?
    Bronvermeldingen?
    \item[Authoritative] Auteur? Is dit een erkend expert? Wordt er elders naar verwezen?
    \item[Purpose/Point of View] Opinie of feiten? Wil de auteur iets verkopen? Is het relevant voor je onderzoeksvraag?
  \end{description}

\end{frame}

\section{Literatuurstudie in {\LaTeX}.}

\begin{frame}[fragile]
  \frametitle{Bronvermelding en referentielijst in {\LaTeX}.}

  Bib{\LaTeX} en Biber

  \vspace{18pt}

  \verb|artikel.tex|: Hoofdtekst\\
  \verb|artikel.bib|: Bibliografische databank (bewerk met bv.~JabRef)

  \vspace{18pt}

  Preamble:

  \begin{verbatim}
  \usepackage[backend=biber,style=apa]{biblatex}
  \DeclareLanguageMapping{dutch}{dutch-apa}
  \addbibresource{artikel.bib}
  \end{verbatim}

\end{frame}

\subsection{Een bibliografische databank aanleggen.}

\begin{frame}
  \frametitle{Bibliografische gegevens in Jabref.}

  Info die \textbf{altijd} ingevuld moet worden:

  \begin{description}
    \item[Author] Familienaam, Voornaam and Familienaam, Voornaam and Familienaam, Voornaam\ldots
    \item[Title] v/h artikel, boek, \ldots
    \item[Year] of datum van publicatie
    \item[Bibtexkey] id van deze bron, gebruikt bij refereren (tip: klik op sleutel-icoon)
  \end{description}
\end{frame}

\begin{frame}[fragile]
  \frametitle{Bibliografische gegevens in Jabref.}
  \framesubtitle{Extra info voor Article}

  \begin{description}
    \item[Journal] Naam van het tijdschrift
    \item[Volume] Jaargang
    \item[Number] Nummer binnen de jaargang (optioneel)
    \item[Pages] \verb|mmm--nnn|
  \end{description}

  \bigskip

  \textbf{Voorbeeld:}

  \bigskip

  \fullcitebib{Anscombe1973}
\end{frame}

\begin{frame}[plain]
  \frametitle{Bibliografische gegevens in Jabref.}
  \framesubtitle{Extra info voor Electronic}

  \begin{description}
    \item[Url] Hyperlink naar de bron
    \item[Urldate] Datum van raadplegen
  \end{description}

  \bigskip

  \textbf{Voorbeeld:}

  \bigskip

  \fullcitebib{Lundin2020}

\end{frame}

\begin{frame}[plain]
  \frametitle{Bibliografische gegevens in Jabref.}
  \framesubtitle{Extra info voor InProceedings}

  \begin{description}
    \item[Booktitle] ``Proceedings of the [naam conferentie]''
    \item[Editor] Redacteur(s) (optioneel)
    \item[Pages] paginanummers (optioneel)
  \end{description}

  \medskip

  \textbf{Voorbeeld:}

  \fullcitebib{vanderLaanEtAl2015}
\end{frame}

\begin{frame}
  \frametitle{Bibliografische gegevens in Jabref.}

  Vul zoveel mogelijk info in (maakt zoeken gemakkelijker):

  \begin{description}
    \item[DOI] Digital Object Identifier: uniek ID voor artikel, hiermee kan je automatisch alle velden invullen
    \item[URL] ook al is het niet echt een Electronic bron
    \item[Keywords] Sleutelwoorden
    \item[File] PDF van de publicatie
    \item[Abstract] Samenvatting
    \item[Comments] Je eigen samenvatting/opmerkingen
  \end{description}

\end{frame}

\subsection{Refereren naar de literatuur.}

\begin{frame}[fragile]
  \frametitle{Bronvermelding en referentielijst in {\LaTeX}.}

  \begin{itemize}
    \item Verwijzingen in de tekst:

    \begin{itemize}
      \item \verb|\textcite{Knuth1998}| \(\Rightarrow\) Knuth (1998)
      \item \verb|\autocite{Knuth1998}| \(\Rightarrow\) (Knuth, 1998)
    \end{itemize}

    \item Literatuurlijst invoegen: \verb|\printbibliography|

    \item Compileren (in TexStudio):

    \begin{enumerate}
      \item Build/Compile (F5): bronnen worden nog niet toegevoegd, ``keys'' van bronnen in het vet aangeduid
      \item Bibliography (F8): selecteert de gerefereerde bronnen en maakt ze klaar
      \item Build/Compile (F5): effectief invoegen verwijzingen en literatuurlijst
    \end{enumerate}
  \end{itemize}

  Zie de voorziene sjablonen of de cursus voor voorbeelden!
\end{frame}

\begin{frame}
  \frametitle{En nu?}

  \begin{itemize}
    \item Verwerk wat je gelezen hebt tot een doorlopende tekst
    \item Stijl/structuur nabootsen van gelezen artikels!
    \item Structureren, bv. ahv Mind Map
      \begin{itemize}
        \item Xmind, Minder, FreeMind, Vym, \ldots
      \end{itemize}
  \end{itemize}
\end{frame}

\begin{frame}
  \frametitle{Meer info}

  De inhoud van deze les is verwerkt in een ``Praktische gids voor de bachelorproef''

  \vspace{12pt}

  \url{https://github.com/HoGentTIN/bachproef-gids/releases}

\end{frame}

\end{document}